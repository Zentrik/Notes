\section{Direct Products and Small Groups}

\subsection{Direct Products}

\begin{definition}[(External) direct product of groups]
  Let $H$ and $K$ be groups. We can construct the (external) \emph{direct product}, $H \times K$, with a set $\{ (h, k): h \in H, k \in K \}$ and an operation:
  \begin{align*}
    (h_1, k_1) * (h_2, k_2) &= (h_1 *_H h_2, k_1 *_K k_2) \\
    &= (h_1 h_2, k_1 k_2)
  \end{align*} i.e. component wise multiplication. 
  Then $(H \times K, *)$ is a group
\end{definition}

\begin{proof} ~
closure: H is a group $\implies h_1 h_2 \in H$, K is a group $\implies k_1 k_2 \in K$

identity: $(e_H, e_K)$

inverse: $(h, k)^{-1} = (h^{-1}, k^{-1})$

associativity since group operation in both $H$ and $K$ are associative.
\end{proof} 

\begin{remark} ~
  \begin{enumerate}
    \item If $H$ and $K$ are both finite, $| H \times K| = |H| |K|$.
    \item $H \times K$ is abelian iff $(h_1, k_1) * (h_2, k_2) = (h_2, k_2) * (h_1, k_1) \quad \forall \; h_1, h_2 \in H,\; k_1, k_2 \in K$ \\
    iff $(h_1 h_2, k_1 k_2) = (h_2 h_1, k_2 k_1)$ \\
    iff $h_1 h_2 = h_2 h_1$ and $k_1 k_2 = k_2 k_1$ \\
    iff $H$ is abelian and $K$ is abelian.
    \mathitem \begin{align*}
      H &\cong \{ (h, e_K) : h \in H \} \leq H \times K \\
      K &\cong \{ (e_H, k) : k \in K \} \leq H \times K
    \end{align*}
  \end{enumerate}
\end{remark} 

\begin{example} ~\vspace*{-1.5\baselineskip}
  \begin{enumerate}
    \item 
      \begin{align*}
        C_2 \times C_2 &= \langle x \rangle \times \langle y \rangle \\
        &= \{ e, x \} \times \{e, y\} \\
        \text{elements :} &\ (e, e),\ (x, e), (e, y),\ (x, y)
      \end{align*} 
      \begin{align*}
        \begin{array}{c | cccc}
          \circ & (e, e) & (x, e) & (e, y) & (x, y) \\
          \hline
          (e, e) & (e, e) & (x, e) & (e, y) & (x, y) \\
          (x, e) & (x, e) & (e, e) & (x, y) & (e, y) \\
          (e, y) & (e, y) & (x, y) & (e, e) & (x, e) \\
          (x, y) & (x, y) & (e, y) & (x, e) & (e, e)
        \end{array}
      \end{align*} 
      This is the called the Klein 4-group and is $\cong$ to \Cref{exm:nine}.\\
      Note $o\left( (x, e) \right) = o\left( (e, y) \right) = o\left( (x, y) \right) = 2$. So $C_2 \times C_2 \ncong C_4$, as there is no element of order $4$.
    %
    \item However, $C_2 \times C_3 \cong C_6$ (Sheet 2, qn 10).
  \end{enumerate} 
\end{example} 

\begin{lemma}
    Let $(h, k) \in H \times K$ where $H, K$ are groups. 
    Then $o\left( (h, k) \right) = \operatorname{lcm} \left( o(h), o(k) \right)$.
\end{lemma} 

\begin{proof}
  Let $n = o\left( (h, k) \right)$ and $m = \operatorname{lcm} \left( o(h), o(k) \right)$.\\
  Then $h^m = e_H, k^m = e_K$ by \Cref{lem:five}.
  \begin{align*}
    \text{So } (h, k)^m &= (h^m, k^m) \\
    &= (e_H, e_K) \\
    \implies n \mkern-5mu &\;\mid m \text{ by \Cref{lem:five}}. \\
    \text{Also, } (h, k)^n &= (h^n, k^n) \\
    &= (e_H, e_K) \\
    \implies o(h) \mkern-5mu &\;\mid n \text{ and } o(k) \mid n \text{ by \Cref{lem:five}}. \\
    \implies m \mkern-5mu &\;\mid n.
  \end{align*} 
\end{proof} 

Thus we know when $C_m \times C_n \cong C_{mn}$ (Sheet 2, qn 10).
Recognising when a group can be written as a direct product of subgroups is trickier.

\begin{proposition}[Direct Product Theorem]
  Let $G$ be a group with subgroups $H$ and $K$, if:
  \begin{enumerate}
    \item each element of $G$ can be written as $hk$, for some $h \in H$, $k \in K$, \label{5.1-1}
    \item $H \cap K = \{ e \}$ \label{5.1-2}
    \item $hk = kh \quad \forall \; h \in H,\ k \in K$ \label{5.1-3}
  \end{enumerate} 
  Then $G \cong H \times K$ and we call $G$ the (internal) direct product of $H$ and $K$.
\end{proposition} 

\begin{proof}
  \begin{align*}
    \text{Let } \theta : H \times K &\to G \\
    (h, k) &\mapsto hk.
  \end{align*} 
  $\theta$ is a homomorphism:
  \begin{align*}
    \theta \left( (h_1, k_1) (h_2, k_2) \right) &= \theta \left( (h_1 h_2, k_1 k_2) \right) \\
    &= h_1 h_2 k_1 k_2 \\
    &= h_1 k_1 h_2 k_2 \text{ by } \Cref{5.1-3} \\
    &= \theta \left( (h_1, k_1) \right) \theta \left( (h_2, k_2) \right)
  \end{align*} 
  $\theta$ is injective:
  \begin{align*}
    \theta \left( (h_1, k_1) \right) &= \theta \left( (h_2, k_2) \right) \\
    h_1 k_1 &= h_2 k_2 \\
    h_2^{-1} h_1 &= k_2 k_1^{-1} \in H \cap K = \{ e \} \text{ by \Cref{5.1-2}} \\
    \implies h_1 = h_2 &\text{ and } k_1 = k_2 \\
    \text{So } (h_1, k_1) &= (h_2, k_2)
  \end{align*} 
  $\theta$ is surjective by \Cref{5.1-1}. \\
  So $\theta$ is an isomorphism as required.
\end{proof} 

\begin{remark} \label{rem:internal2}
  There are alternative equivalent definitions of internal direct product.
  $G$ is the internal direct product of subgroups $H$ and $K$ if:
  \begin{symenum}
    \itemsymbol{'} $H \trianglelefteq G, K \trianglelefteq G$ \label{5.1-1'}
    \itemsymbol{'} $H \cap K = \{ e \}$ \label{5.1-2'}
    \itemsymbol{'} $HK = G$ \label{5.1-3'}
  \end{symenum} 
\end{remark} 

\begin{proof}
  We need to show \Cref{5.1-1}, \ref{5.1-2}, \ref{5.1-3} $\iff$ \Cref{5.1-1'}, \ref{5.1-2'}, \ref{5.1-3'}.

  $(\implies)$ we show $K \trianglelefteq G$.\\
  Let $k \in K$, $g = h_1 k_1 \in G$ by \Cref{5.1-1}.
  \begin{align*}
    g k g^{-1} &= h_1 \underbrace{k_1 k k_1^{-1}}_{\overline{k} \in K} h_1^{-1} \\
    &= \overline{k} \in K \text{ by } \Cref{5.1-3}
  \end{align*} 
  Similarly $H \trianglelefteq G$. \\
  And \Cref{5.1-1} $\implies$ \Cref{5.1-3'}.

  $(\Longleftarrow)$ need to show \Cref{5.1-3}.
  \begin{align*}
    h \in H, k \in K \text{ consider} \\
    h^{-1} \underbrace{k^{-1} h k}_{\in H} &\in H, \text{ since } H \trianglelefteq G \\
    &\in K, \text{ since } K \trianglelefteq G \\
    \implies h^{-1}k^{-1} h k &= H \cap K = \{ e \} \text{ by \Cref{5.1-2'}} \\
    \implies hk &= kh 
  \end{align*} 
\end{proof} 

\begin{example}
$G = \langle a \rangle = C_{15}$.
  \begin{align*}
    C_5 &\cong \langle a^3 \rangle = H \trianglelefteq G \text{ (as $G$ is abelian)}. \\
    C_3 &\cong \langle a^5 \rangle = K \trianglelefteq G. \\
    H \cap K &= a^{15n} = \{ e \} \\
    a^k &= (a^3)^{2k} (a^5)^{-k} \in HK \\
    \implies C_{15} &\cong K \times H \cong C_3 \times C_5.
  \end{align*} 
\end{example} 

\subsection{Small Groups}

Recall $D_{2n}$, the symmetries of a regular n-gon, generated by 
\begin{align*}
  r : z &\mapsto e^{2 \pi i / n} z \\
  t : z &\mapsto \overline{z} 
\end{align*} 
elements of \begin{align*}
  D_{2n} = \{e, \underbrace{r, r^2, \ldots, r^{n-1}}_\text{rotations}, \underbrace{t, rt \ldots, r^{n-1}t}_\text{reflections} \}.
\end{align*}

\begin{lemma}
  Now suppose $G$ is a group, $n \geq 3$, with $|G| = 2n$, and $\exists \; b \in G$ with $o(b) = n$ and $a \in G$, $o(a) = 2$ and $aba = b^{-1}$.
  Then $G \cong D_{2n}$.
\end{lemma} 

\begin{proof}
  Note $\langle b \rangle \trianglelefteq G$ since of index two, \Cref{lem:twelve}. \\
  Also $a \notin \langle b \rangle$, since $aba = aa b = b$ as $a \in \langle b \rangle \implies ab = ba$.
  \begin{align*}
    \text{So } G &= \langle b \rangle \cup \langle b \rangle a \\
    &= \{e, b, \ldots, b^{n-1}, a, ba, \ldots, b^{n-1}a \}
    \intertext{Furthermore}
    ab &= b^{-1} a \\
    \implies ab^k &= (ab) b^{k-1} \\
    &= b^{-1} a b^{k-1} \\
    &= b^{-2} a b^{k - 2} \\
    &\;\;\vdots \\
    &= b^{-k} a \\
    \text{So } (b^k a) (b^k a) &= b^k b^{-k} a a \\
    &= e.
    \intertext{We can check}
    \theta : D_{2n} &\to G \\
    r &\mapsto b \\
    t &\mapsto a 
  \end{align*} is an isomorphism.
\end{proof} 

\subsubsection{Classifying groups of small order}

\begin{lemma} ~\vspace*{-1.5\baselineskip}
  \begin{itemize}
    \item If $|G| = 1$, $G = \{ e \}$
    \item If $|G| = 2 \implies G \cong C_2$, as we have identity and an another element which must be a self inverse or we can prove by \nameref{cor:two}.
    \item If $|G| = 3 \implies G \cong C_3$ by \Cref{cor:three} ($3$ is prime). 
    \item If $|G| = 5 \implies G \cong C_5$ by \Cref{cor:three} ($5$ is prime).
    \item If $|G| = 7 \implies G \cong C_7$ by \Cref{cor:three} ($7$ is prime).
  \end{itemize}
\end{lemma}

\begin{lemma}
  If $|G| = 4$, $G$ is isomorphic to $C_4$ and $C_2 \times C_2$, both abelian.
\end{lemma} 

\begin{proof}
  By \nameref{thm:three}, if $1 \neq g \in G$ then $o(g) \mid 4$.
  \begin{align*}
    \text{If } \exists \; g &\in G \text{ s.t. } o(g) = 4 \\
    \implies G &\cong C_4
    \intertext{Suppose not, then let}
    1 \neq a \in G &\implies o(a) = 2 \\
    &\implies G \text{ is abelian (Sheet 1, Q7)} \\
    &\implies C_2 \cong \langle a \rangle \trianglelefteq G.
  \end{align*}
  Let $b \in G \setminus \langle a \rangle$, then $1 \neq b \in G$ so $C_2 \cong \langle b \rangle \trianglelefteq G$. \\
  Also, $\langle a \rangle \cap \langle b \rangle = \{ e \}$. \\
  Consider the element $ab$: \\
  if $ab = e \implies a = b^{-1} = b$ \Lightning. \\
  if $ab = a \implies b = e$ \Lightning. \\
  if $ab = b \implies a = e$ \Lightning. \\
  So \begin{align*}
    G &= \{ e, a, b, ab \} \\
    &= \langle a \rangle \langle b \rangle \\
    &\cong \langle a \rangle \times \langle b \rangle \text{ by \Cref{rem:internal2}} \\
    &\cong C_2 \times C_2.
  \end{align*}
\end{proof} 

\begin{lemma}
  If $|G| = 6$, $G$ is isomorphic to $C_6$ and $D_6 \cong S_3$.
  Note $C_6 \ncong D_6$ as $C_6$ is abelian and $D_6$ is not.
\end{lemma} 

\begin{proof}
  Let $1 \neq g \in G \implies o(g) = 2, 3 \text{ or } 6$ by \nameref{cor:two}.
  If all non-identity elements have order $2 \implies |G|$ is a $2$-power \Lightning \ (Sheet 1, Q7). \\
  So $\exists \; b \in G, o(b) = 3$ (Note if $o(g) = 6$ then $o(g^2) = 3$). \\
  $C_3 \cong \langle b \rangle \trianglelefteq G$, RHS by \Cref{lem:twelve} (since of index 2).

  Let $a \in G \setminus \langle b \rangle$,
  \begin{align*}
    \implies a^2 \in \langle b \rangle.
  \end{align*} 
  As $a \langle b \rangle \in G / \langle b \rangle$ and $|G : \langle b \rangle| = 2$, the group has order $2$. 
  So $(a \langle b \rangle)^2 = a^2 \langle b \rangle = e = \langle b \rangle$ by \nameref{cor:two}.
  Then $a^2 \in \langle b \rangle$ by \Cref{lem:eleven}.

  If $a^2 = b$ or $b^2$ then $o(a) = 6$, as if $a^2 = b$, $a^3 = ab$ and if $ab = e$, $a = b^{-1} \in \langle b \rangle$ \Lightning. \\
  $o(a) = 6 \implies G \cong C_6$. 

  Suppose $a^2 = 1$.\\
  Also, $a b a^{-1} = aba \in \langle b \rangle$ as $\langle b \rangle \trianglelefteq G$.
  \begin{align*}
    \text{If } aba^{-1} &= e\implies b = e \text{ \Lightning} \\
    &= b \implies ab = ba \\
    &\hspace{3.3ex}\ \implies o(ab) = 6 \\
    &\hspace{3.3ex}\ \implies G \cong C_6. \\
    &= b^2 = b^{-1}.
  \end{align*} 
  So $o(a) = 2$, $o(b) = 3$, $aba^{-1} = b^{-1} \implies G \cong D_6$.
\end{proof} 

\begin{lemma}
  If $|G| = 9$, $G$ is isomorphic to $C_9$ and $C_3 \times C_3$.
\end{lemma} 

\begin{proof}
  We will show later that groups of order $p^2$ (where $p$ is prime) are abelian, so either $G \cong C_9$. \\
  Or all non-identity elements have order 3 by \nameref{cor:two}
  Choose $e \neq a \in G, b \in G \setminus \langle a \rangle$
  \begin{align*}
    G &= \langle a \rangle \cup \langle a \rangle b \cup \langle a \rangle b^2 \\
    &= \langle a \rangle \langle b \rangle \\
    &\cong \langle a \rangle \times \langle b \rangle \\
    &\cong C_3 \times C_3.
  \end{align*} 
\end{proof} 

\begin{lemma}
If $|G| = 10 \implies G \cong C_{10} \text{ or } D_{10}$ (Sheet 2, Q12, use proof similar to order 8 not 6).
\end{lemma} 

\begin{remark}
  There are lots of groups of order $2^k$.
  e.g. 10 of order 16 and approximately 50,000,000,000 of order $2^{10}$.
\end{remark} 

\subsubsection{Groups of order 8}

\begin{lemma}
  If $G$ has order 8, then either $G$ is abelian (i.e.\ $\cong C_8, C_4\times C_2$ or $C_2\times C_2\times C_2$), or $G$ is not abelian and isomorphic to $D_8$ or $Q_8$ (dihedral or quaternion).
\end{lemma}

\begin{proof}
  Consider the different possible cases:
  \begin{itemize}
    \item If $G$ contains an element of order 8, then $G\cong C_8$.
    \item If all non-identity elements have order 2 $\implies G$ is abelian (Sheet 1, Q7). 
    Let $1 \neq a \in G$, then $C_2 \cong \langle a \rangle \trianglelefteq G$, RHS as $G$ is abelian.
    \begin{align*}
      \text{Choose } b &\notin \langle a \rangle, \\
      \langle a, b \rangle &= \{1, a, b, ab \} \\
      &= \langle a \rangle \langle b \rangle \\
      &\cong \langle a \rangle \times \langle b \rangle \text{ by \Cref{rem:internal2}.}
    \end{align*} 
    Choose $c \in G \setminus \langle a, b \rangle$ 
    \begin{align*}
      G &= \langle a, b \rangle \cup \langle a, b \rangle c \\
      &= \langle a, b \rangle \langle c \rangle \\
      &\cong \langle a, b \rangle \times \langle c \rangle \text{ by \Cref{rem:internal2}}\\
      &\cong \langle a \rangle \times \langle b \rangle \times \langle c \rangle \\
      &\cong C_2 \times C_2 \times C_2
    \end{align*} 
    \item Now suppose $\exists \; g \in G, o(g) > 2 \implies \exists \; a \in G, o(a) = 4$ (if we have $o(a) = 8$, then $o(a^2) = 4$). 
    $C_4 \cong \langle a \rangle \trianglelefteq G$, RHS by \Cref{lem:twelve} (since of index 2).
    \begin{align*}
      \text{Let } b &\in G \setminus \langle a \rangle \\
      \implies b^2 &\in \langle a \rangle.
    \end{align*} (Same reasoning as in $n = 6$, alternative proof is $b^2 \in \langle a \rangle$ or $b^2 \in b \langle a \rangle$, if $b^2 = b a^n \implies b = a^n$ \Lightning \ so $b^2 \in \langle a \rangle$).
    %
    \begin{align*}
      \text{If } b^2 &= a \text{ or } a^3 \\
      \implies o(b) &= 8 \implies G \cong C_8. \\ \\
      \text{Else } b^2 &= e \text{ or } a^2 \\
      \text{Now } b a b^{-1} &\in \langle a \rangle \text{ since } \langle a \rangle \trianglelefteq G \\
      &= a^i \text{ for some } i. \\
      \implies b^2 a b^{-2} &= b (b a b^{-1}) b^{-1} \\
      &= b a^i b^{-1} \\
      &= (b a b^{-1})^i \text{ as } (b a b^{-1}) (b a b^{-1}) = ba^2b \\
      &= a^{i^2} \\
      \text{But } b^2 &\in \langle a \rangle \implies b^2 a b^{-2} = a 
      \intertext{ as $b^2 = a^m$ so must commute with $a$.}
      \implies i^2 &\equiv 1 \pmod 4 \\
      \implies i &\equiv \pm 1 \pmod 4.
    \end{align*}

      \begin{itemize}
        \item When $i \equiv 1 \pmod 4$, $bab^{-1} = a \implies ba = ab$. So $G = \langle a \rangle \cup b \langle a \rangle$ is abelian.
          \begin{itemize}
            \item If $b^2 = e$, then $G = \langle a \rangle \langle b \rangle \cong \langle a\rangle \times \langle b\rangle \cong C_4\times C_2$.
            \item If $b^2 = a^2$, then $(ba^{-1})^2 = e$. So $G = \langle a, ba^{-1}\rangle \cong C_4\times C_2$.
          \end{itemize}
        \item If $i \equiv -1 \pmod 4$, then $bab^{-1} = a^{-1}$.
          \begin{itemize}
            \item If $b^2 = e$, then $G = \langle a, b: a^4 = e = b^2, bab^{-1} = a^{-1}\rangle$. So $G\cong D_8$ by definition.
            \item If $b^2 = a^2$, then we have $G\cong Q_8$, a new group called the quaternion group.
          \end{itemize}
      \end{itemize}%\qedhere
  \end{itemize}
  To show all 5 groups are different, $C_8$ has an element of order $8$, $C_4 \times C_2$ does not.
  $C_4 \times C_2$ has an element of order $4$ whilst $C_2 \times C_2 \times C_2$ does not have elements of order 2 or 4.

  $D_8$ and $Q_8$, $Q_8$ has 6 elements of order 4, but $D_8$ only has 2, so non-isomorphic.
\end{proof}

\subsubsection{Realisations of \texorpdfstring{$Q_8$}{Q₈}}

\begin{enumerate}
  \mathitem \begin{align*}
    Q_8 &= \{ \pm 1, \pm i, \pm j, \pm k \} \\
    \text{with } ij &= k, jk = i, ki = j \\
    ji &= -k, kj = -i, ik = -j \\
    i^2 &= j^2 = k^2 = -1 \\
    \text{so } o(i) &= o(j) = o(k) = 4, o(-1) = 2.
  \end{align*} 
  \mathitem \begin{align*}
    \Biggl\{ \begin{pmatrix}
      1&0\\0&1
    \end{pmatrix},
    \begin{pmatrix}
      i & 0\\0&-i
    \end{pmatrix},
    \begin{pmatrix}
      0&1\\-1&0
    \end{pmatrix},
    \begin{pmatrix}
      0&i\\i&0
    \end{pmatrix}
    \\
    \begin{pmatrix}
      -1&0\\0&-1
    \end{pmatrix},
    \begin{pmatrix}
      -i & 0\\0&i
    \end{pmatrix},
    \begin{pmatrix}
      0&-1\\1&0
    \end{pmatrix},
    \begin{pmatrix}
      0&-i\\-i&0
    \end{pmatrix} \Biggl\} \leq \mathrm{SL}_2(\mathbb{C})
  \end{align*}
  \mathitem \begin{align*}
    Q_8 = \langle a, b \mid a^4 = e, b^2 = a^2, b a b^{-1} = a^{-1} \rangle
  \end{align*} 
\end{enumerate} 