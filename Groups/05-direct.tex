\section{Direct Products and Small Groups}

\subsection{Direct Products}

Let $H$ and $K$ be groups.
We construct the (external) \emph{direct product}, $H \times K$, as follows:

elements: $(h, k),\ h \in H, k \in K$.

operation: \begin{align*}
  (h_1, k_1) * (h_2, k_2) &= (h_1 *_H h_2, k_1 *_K k_2) \\
  &= (h_1 h_2, k_1 k_2)
\end{align*} 
i.e. component wise multiplication.
Then $(H \times K, *)$ is a group:

closure: 
H is a group $\implies h_1 h_2 \in H$ 
K is a group $\implies k_1 k_2 \in K$

identity: $(e_H, e_K)$

inverse: $(h, k)^{-1} = (h^{-1}, k^{-1})$

associativity since group operation in both $H$ and $K$ are associative.

\begin{remark}

  ~

  \begin{enumerate}
    \item If $H$ and $K$ are both finite, $| H \times K| = |H| |K|$.
    \item $H \times K$ is abelian iff $(h_1, k_1) * (h_2, k_2) = (h_2, k_2) * (h_1, k_1) \; \forall \; h_1, h_2 \in H \; \forall \; k_1, k_2 \in K$ \\
    iff $(h_1 h_2, k_1 k_2) = (h_2 h_1, k_2 k_1)$ \\
    iff $h_1 h_2 = h_2 h_1$ and $k_1 k_2 = k_2 k_1$ \\
    iff $H$ is abelian and $K$ is abelian.
    \item 
    \begin{align*}
      H &\cong \{ (h, e_K) : h \in H \} \leq H \times K \\
      K &\cong \{ (e_H, k) : k \in K \} \leq H \times K
    \end{align*}
  \end{enumerate}
\end{remark} 

\begin{example}
  
  ~

  i.
  \begin{align*}
    C_2 \times C_2 &= \langle x \rangle \times \langle y \rangle \\
    &= \{ e, x \} \times {e, y} \\
    \text{elements :} &\ (e, e),\ (x, e), (e, y),\ (x, y)
  \end{align*} 
  \begin{align*}
    \begin{array}{c | cccc}
      \circ & (e, e) & (x, e) & (e, y) & (x, y) \\
      \hline
      (e, e) & (e, e) & (x, e) & (e, y) & (x, y) \\
      (x, e) & (x, e) & (e, e) & (x, y) & (e, y) \\
      (e, y) & (e, y) & (x, y) & (e, e) & (x, e) \\
      (x, y) & (x, y) & (e, y) & (x, e) & (e, e)
    \end{array}
  \end{align*} 
  This is the called the Klein 4-group and is $\cong$ to \Cref{exm:nine}.\\
  Note $o\left( (x, e) \right) = o\left( (e, y) \right) = o\left( (x, y) \right) = 2$. So $C_2 \times C_2 \ncong C_4$.

  ii. However, $C_2 \times C_3 \cong C_6$ (Sheet 2, qn 10).
\end{example} 

\begin{lemma}
    Let $(h, k) \in H \times K$ where $H, K$ are groups. 
    Then $o\left( (h, k) \right) = \operatorname{lcm} \left( o(h), o(k) \right)$.
\end{lemma} 

\begin{proof}
  Let $n = o\left( (h, k) \right)$ and $m = \operatorname{lcm} \left( o(h), o(k) \right)$.\\
  Then $h^m = e_H, k^m = e_K$ by \Cref{lem:five}.
  \begin{align*}
    \text{So } (h, k)^m &= (h^m, k^m) \\
    &= (e_H, e_K) \\
    \implies n &\mid m \text{ by } \ref{lem:five}. \\
    \text{Also, } (e_H, e_K) &= (h, k)^n \\
    &= (h^n, k^n) \\
    \implies o(h) &\mid n \text{ and } o(k) \mid n \text{ by } \ref{lem:five}. \\
    \implies m &\mid n.
  \end{align*} 
  Thus we know when $C_m \times C_n \cong C_{mn}$ (Sheet 2, qn 10).
\end{proof} 

Recognising when a group can be written as a direct product of subgroups is trickier.

\begin{proposition}
  Let $G$ be a group with subgroups $H$ and $K$, if:
  \begin{enumerate}
    \item each element of $G$ can be written as $hk$, for some $h \in H$, $k \in K$, \label{5.1-1}
    \item $H \cap  K = \{ e \}$ \label{5.1-2}
    \item $hk = kh \; \forall \; h \in H,\ k \in K$ \label{5.1-3}
  \end{enumerate} 
  Then $G \cong H \times K$ and we call $G$ the (internal) direct product of $H$ and $K$.
\end{proposition} 

\begin{proof}
  \begin{align*}
    \text{Let } \theta : H \times K &\to G \\
    (h, k) &\mapsto hk.
  \end{align*} 
  $\theta$ is a homomorphism:
  \begin{align*}
    \theta \left( (h_1, k_1), (h_2, k_2) \right) &= \theta \left( (h_1 h_2, k_1 k_2) \right) \\
    &= h_1 h_2 k_1 k_2 \\
    &= h_1 k_1 h_2 k_2 \text{ by } \ref{5.1-3} \\
    &= \theta \left( (h_1, k_1) \right) \theta \left( (h_2, k_2) \right)
  \end{align*} 
  $\theta$ is injective:
  \begin{align*}
    \theta \left( (h_1, h_2) \right) &= \theta \left( (h_2, k_2) \right) \\
    h_1 k_1 &= h_2 k_2 \\
    h_2^{-1} h_1 &= k_2 k_1^{-1} \in H \cap K = \{ e \} \text{ by } \ref{5.1-2} \\
    \implies h_1 = h_2 &\text{ and } k_1 = k_2 \\
    \text{So } (h_1, k_1) &= (h_2, k_2)
  \end{align*} 
  $\theta$ is surjective by \ref{5.1-1}. \\
  So $\theta$ is an isomorphism as required.
\end{proof} 

\begin{remark}
  There are alternative equivalent definitions of internal direction product.
  $G$ is the internal direction product of subgroups $H$ and $K$ if:
  \begin{symenum}
    \itemsymbol{'} $H \trianglelefteq G, K \trianglelefteq G$ \label{5.1-1'}
    \itemsymbol{'} $H \cap K = \{ e \}$ \label{5.1-2'}
    \itemsymbol{'} $HK = G$ \label{5.1-3'}
  \end{symenum} 
\end{remark} 

\begin{proof}
  We need to show \ref{5.1-1}, \ref{5.1-2}, \ref{5.1-3} $\iff$ \ref{5.1-1'}, \ref{5.1-2'}, \ref{5.1-3'}.

  $(\implies)$ we show $K \trianglelefteq G$.\\
  Let $k \in K$, $g = h_1 k_1 \in G$ by \ref{5.1-1}.
  $g k g^{-1} = h_1 \underbrace{k_1 k k_1^{-1}}_{\bar{k} \in K} h_1^{-1}$
  \begin{align*}
    g k g^{-1} &= h_1 \underbrace{k_1 k k_1^{-1}}_{\bar{k} \in K} h_1^{-1} \\
    &= \bar{k} \in K \text{ by } \ref{5.1-3}
  \end{align*} 
  Similarly $H \trianglelefteq G$. \\
  And \ref{5.1-1} $\implies$ \ref{5.1-3'}.

  $(\Longleftarrow)$ need to show \ref{5.1-3}.
  \begin{align*}
    h \in H, k \in K \text{ consider} \\
    h^{-1} \underbrace{k^{-1} h k}_{\in H} &\in H, \text{ since } H \trianglelefteq G \\
    &\in K, \text{ since } K \trianglelefteq G \\
    \implies h^{-1}k^{-1} h k &= H \cap K = \{ e \} \ \ \ref{5.1-2'} \\
    \implies hk &= kh 
  \end{align*} 
\end{proof} 

\begin{example}
  $G = \langle a \rangle = C_{15}$.
  \begin{align*}
    C_5 &\cong \angle a^3 \rangle = H \trianglelefteq G \text{ (as $G$ is abelian)}. \\
    C_3 &\cong \angle a^5 \rangle = K \trianglelefteq G. \\
    H \cap K &= a^{15n} = \{ e \} \\
    a^k &= (a^3)^{2k} (a^5)^{-k} \in HK \\
    \implies C_{15} &\cong K \times H \cong C_3 \times C_5.
  \end{align*} 
\end{example} 
