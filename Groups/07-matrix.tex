\section{Matrix Groups}

Let $M_n(\mathbb{R})$ denote the set of all $n \times n$ matrices with entries in $\mathbb{R}$. 
\begin{definition}[General Linear Group] ~\vspace*{-1.5\baselineskip}
    \begin{align*}
        \operatorname{GL}_n(\mathbb{R}) = \{A \in M_n(\mathbb{R}) : \det A \neq 0\}.
    \end{align*} 
\end{definition} 

\begin{proposition} \label{prp:8}
    $\operatorname{GL}_n(\mathbb{R})$ is a group under matrix multiplication.
    It is called the \emph{general linear group}.
\end{proposition} 

\begin{proof}
    closure: $A, B \in \operatorname{GL}_n(\mathbb{R})$, clearly $AB \in \operatorname{M_n}(\mathbb{R})$ and $\det(AB) = \det A \det B \neq 0$ so $AB \in \operatorname{GL}_n(\mathbb{R})$.

    identity: $I_n = \begin{pmatrix}
        1 & & \\
        & \ddots &  \\
        & & 1
      \end{pmatrix} \in \operatorname{GL}_2(\mathbb{R})$

    inverse: $\det A \neq 0 \implies A^{-1}$ exists, $\det A^{-1} = \frac{1}{\det A} \neq 0$.

    matrix multiplication is associative: 
    \begin{align*}
        (A(BC))_{ij} &= A_{ik} (BC)_{kj} \\
        &= A_{ik} B_{kt} C_{tj} \\
        ((AB)C)_{ij} &= (AB)_{ik} C_{kj} \\
        &= A_{it} B_{tk} C_{kj}.
    \end{align*} 
\end{proof} 

\begin{example} ~\vspace*{-1.5\baselineskip}
    \begin{align*}
        \operatorname{GL}_2(\mathbb{R}) &= \left\{ \begin{pmatrix}
        a & b \\
        c & d
        \end{pmatrix} : a, b, c, d \in \mathbb{R},\ ad - bc \neq 0 \right\} \\
        \begin{pmatrix}
            a & b \\
            c & d
            \end{pmatrix}^{-1} &= \frac{1}{ad - bc} \begin{pmatrix}
        d & -b \\
        -c & -a
        \end{pmatrix}.
    \end{align*}
\end{example} 

\begin{proposition}\label{prp:9} ~\vspace*{-1.5\baselineskip}
    \begin{align*}
        \operatorname{Det} : \operatorname{GL}_n(\mathbb{R}) &\to (\mathbb{R} \setminus \{0\}, \times) \\
        A &\mapsto \det A.
    \end{align*} is a surjective group homomorphism.
\end{proposition} 

\begin{proof}
    Note $(\mathbb{R} \setminus \{0\}, \times)$ is a group.
    Det is clearly a map to $(\mathbb{R} \setminus \{0\}, \times)$, we need to check it's a group homomorphism,
    \begin{align*}
        \operatorname{Det}(AB) &= \det(AB) \hspace{0.5cm} \text{($AB$ is multiplication in } \operatorname{GL}_n)\\
        &= \det A \cdot \det B \hspace{0.5cm} \text{(multiplication in } (\mathbb{R} \setminus \{0\}, \times))\\
        &= \operatorname{Det} A \operatorname{Det} B
    \end{align*} 
    and that Det is surjective.
    Let $r \in (\mathbb{R} \setminus \{0\}, \times)$ then 
    \begin{align*}
        A = \begin{pNiceMatrix}
            r   &       & \Block{2-3}<\huge>{0} \\
                &   1   &        &      &       \\
                &       &   1    &      &       \\
            \Block{2-3}<\huge>{0}
                &       &       & \Ddots    &   \\
                &       &       &      &   1   \\
          \end{pNiceMatrix} \in \operatorname{GL}_n(\mathbb{R}) \mapsto \det(A) = r.
    \end{align*}
\end{proof} 

By \nameref{thm:six} $\ker(\text{Det}) \trianglelefteq \operatorname{GL}_n(\mathbb{R})$.
\begin{align*}
    \ker(\text{Det}) &= \{A \in \operatorname{GL}_n(\mathbb{R}) : \det A = 1\} \\
    &= \operatorname{SL}_n(\mathbb{R}) \text{ the special linear group}.
\end{align*} 
Furthermore, by \nameref{thm:six}
\begin{align*}
    \operatorname{GL}_n(\mathbb{R}) / \operatorname{SL}_n(\mathbb{R}) \cong (\mathbb{R} \setminus \{0\}, \times).
\end{align*} 

\begin{remark}
    More generally we can define the general linear group and special linear group over any field. \\
    Examples of fields: \\
    $\mathbb{R}, \mathbb{C}, \mathbb{Q}, \mathbb{F}_p$ where $\mathbb{F}_p = \left( \{0, 1, 2, \ldots, p- 1\}, +_p, \times_p \right)$ and $p$ is prime.
    Note $\operatorname{GL}_n(\mathbb{F}_p)$ and $\operatorname{SL}_n(\mathbb{F}_p)$ are finite groups.
\end{remark} 

What is $|GL_3(\mathbb{F}_p)|$? \\
Non-zero determinant means we have linearly independent columns.
The no. of choices for the first column is $p^3 - 1$ (can't be $\begin{pmatrix}0 & 0 & 0\end{pmatrix}^T$).
The second column is not a multiple of first so $p^3 - p$ (there are $p$ multiples of first column).
Third column not in space spanned by first two columns, this space has size $p^2$ (consider $\alpha c_1 + \beta c_2$ with $\alpha, \beta \in \mathbb{F}_p$), so $p^3 - p^2$. \\
$\implies |GL_3(\mathbb{F}_p)| = (p^3 - 1)(p^3 - p)(p^3 - p^2)$.

We can still consider 
\begin{align*}
    \mathrm{Det} : \operatorname{GL}_3(\mathbb{F}_p) &\to (\mathbb{F}_p \setminus \{0\}, \times) \\
    A &\mapsto \det(A).
\end{align*} 
Note $(\mathbb{F}_p \setminus \{0\}, \times)$ is a group: we have closure, identity $= 1$ and associativity.
Let $a \in \mathbb{F}_p \setminus \{0\}$, by Bezout's Thm $\exists \; x, y$ s.t.
\begin{align*}
    ax + py &= 1 \\
    \therefore ax &\equiv 1 \pmod p \\
    \text{Choose } \bar{x} &\equiv x \pmod p, \\
    1 \leq \bar{x} &\leq p - 1,\ a^{-1} = x
\end{align*} 
Det is a surjective homomorphism to $(\mathbb{F}_p \setminus \{0\}, \times)$ so by \nameref{thm:six}
\begin{align*}
    |\operatorname{GL}_3(\mathbb{F}_p)| / |\operatorname{SL}_3(\mathbb{F}_p)| &= p - 1 \\
    \implies |\operatorname{SL}_3(\mathbb{F}_p)| &= \frac{(p^3 - 1)(p^3 - p)(p^3 - p^2)}{p - 1}.
\end{align*} 

\subsection{Actions of $\operatorname{GL}_n(\mathbb{C})$}

\begin{enumerate}
    \item Let $\mathbb{C}^n$ denote vectors of length $n$ with entries in $\mathbb{C}$:
    \begin{align*}
        \operatorname{GL}_n(\mathbb{C}) \times \mathbb{C}^n &\to \mathbb{C}^n \\
        (A, v) &\mapsto Av.
    \end{align*} 
    Note $Iv = v$, $(AB)v = A(B(v))$.
    This action is faithful: $Av = v \forall \; v \in \mathbb{C}^n \implies A = I_n$ (consider $v = e_i$).
    The action has two orbits
    \begin{align*}
        \operatorname{Orb}_{\operatorname{GL}_n(\mathbb{C})}(\underline{0}) &= \{\underline{0}\} \\
        \operatorname{Orb}_{\operatorname{GL}_n(\mathbb{C})}(v) &= \mathbb{C}^n \setminus \{0\} \text{ for } v \neq \underline{0},
    \end{align*} i.e. given $w \neq 0 \; \exists \; A \in \operatorname{GL}_n(\mathbb{C})$ s.t. $Av = w$.
    \item Conjugation action of $\operatorname{GL}_n(\mathbb{C})$ on $M_n(\mathbb{C})$ (set of all matrices)
    \begin{align*}
        \operatorname{GL}_n(\mathbb{C}) \times M_n(\mathbb{C}) &\to M_n(\mathbb{C}) \\
        (P, A) &\mapsto P A P^{-1}. \\
    \text{Note: } PQ(A) &= PQ A (PQ)^{-1} \\
        &= PQ A Q^{-1} P^{-1} \\
        &= P(Q(A)).
    \end{align*} 
\end{enumerate} 

\begin{remark}
    Matrices $A$ and $B$ are conjugate if they represent the same linear map. 
    If $PAP^{-1} = B$, then $P$ represents a change of basis matrix. (See LA next year)
\end{remark} 

\begin{example} ~\vspace*{-1.5\baselineskip}
    \begin{align*}
        A : e_1 &\mapsto 2 e_1 \\
        e_2 &\mapsto 3e_2 \\
        A &= \begin{pmatrix}
        2 & 0 \\
        0 & 3
        \end{pmatrix} \\
        \text{Let } P : e_1 &\mapsto e_2 \\
        e_2 &\mapsto e_1 \\
        P &= \begin{pmatrix}
        0 & 1 \\
        1 & 0
        \end{pmatrix} \\
        &= P^{-1} \\
        \intertext{$P$ is a change of basis}
        PAP^{-1} &= \begin{pmatrix}
            0 & 1 \\
            1 & 0
            \end{pmatrix} \begin{pmatrix}
                2 & 0 \\
                0 & 3
                \end{pmatrix} \begin{pmatrix}
                0 & 1 \\
                1 & 0
                \end{pmatrix} \\
            &= \begin{pmatrix}
            3 & 0 \\
            0 & 2
            \end{pmatrix} \\
        \text{i.e. } e_2 &\mapsto 3e_2 \\
        e_1 &\mapsto 2 e_1.
    \end{align*}
\end{example} 

We will use the following result from V\&M when investigating M\"obius groups.

\emph{Result}
Let $A \in M_2(\mathbb{C})$ and consider conjugation action of $\operatorname{GL}_2(\mathbb{R})$ on $M_2(\mathbb{C})$.
Then precisely one of the following occurs
\begin{enumerate}
    \item The orbit of $A$ contains a diagonal matrix $\begin{pmatrix}\lambda & 0 \\0 & \mu\end{pmatrix}$ with $\lambda \neq \mu$.
    \item The orbit of $A$ is $\begin{pmatrix}\lambda & 0 \\0 & \lambda\end{pmatrix} = \lambda I$ for some $\lambda$.
    \item The orbit of $A$ contains a matrix $\begin{pmatrix}\lambda & 1 \\0 & \lambda\end{pmatrix}$ for some $\lambda$.
\end{enumerate} 

\begin{proof}
    See V\&M but essentially 
    \begin{enumerate}
        \item In this case $A$ has $2$ distinct eigenvalues $\lambda \neq u$, take a basis consisting of an eigenvector for $\lambda$ and an eigenvector for $\mu$.
        Distinct pairs given distinct orbits.
        \item $PAP^{-1} = \lambda I \implies A = P \lambda I P^{-1} = \lambda I$, eigenvalues $\lambda, \lambda$, 2 linearly independent eigenvectors.
        \item In this case $A$ has a repeated eigenvalue, but just one linearly independent eigenvector.
    \end{enumerate} 
\end{proof} 

\subsection{Orthogonal Group}

\begin{aside}{Aside: Recalling transpose properties}

Recall if $A \in M_n(\mathbb{R})$, $A^T$ is defined by $(A^T)_{ij} = A_{ji}$, i.e. the $ij$-th entry of $A^T$ is the $ji$-th entry of $A$
\begin{align*}
    A &= \begin{pmatrix}2 & 4 \\3 & 5\end{pmatrix} \\
    A^T &= \begin{pmatrix}2 & 3 \\4 & 5\end{pmatrix}
\end{align*} 

\emph{Note}
\begin{enumerate}
    \item
    \begin{align*}
        (AB)^T &= B^T A^T \\
        [(AB)^T]_{ij} &= (AB)_{ji} \\
        &= A_{jk}B_{ki} \\
        [B^T A^T]_{ij} &= B^T_{ik} A^T_{kj} \\
        &= B_{ki} A_{jk} \checkmark
    \end{align*} 
    \item \begin{align*}
        A A^T &= I \\
        \implies 1 &= \det (A^T A) \\
        &= \det A^T \det A \\
        &= (\det A)^2 \\
        \implies \det A &\neq 0
    \end{align*}
    \item \begin{align*}
        A A^T &= I \iff A^T A = I. \\
        \implies A^T A &= A^{-1} \underbrace{A A^T}_I A \\
        &= A^{-1} A \\
        &= I.
    \end{align*} 
    \item
    \begin{align*}
        (A^T)^{-1} &= (A^{-1})^T \\
        \text{Since } I_n &= (A A^{-1})^T \\
        &= (A^{-1})^T A^T
    \end{align*} 
    \item \begin{align*}
        \det A^T &= \det A.
    \end{align*} 
\end{enumerate} 

\end{aside}

\begin{definition}[Orthogonal group] ~\vspace*{-1.5\baselineskip}
    \begin{align*}
        O_n(\mathbb{R}) &= \{ A \in M_n(\mathbb{R}) : A^T A = I\}
    \end{align*} (so columns of $A$ form an orthonormal basis for $\mathbb{R}^n$).  
\end{definition} 

\begin{proposition}\label{prp:10}
    $O_n(\mathbb{R})$ is a subgroup of $\operatorname{GL}_n(\mathbb{R})$ called the \emph{orthogonal group}.
\end{proposition} 

\begin{proof} \mbox{}
    \begin{itemize}
        \item $\det A \neq 0 \implies O_n(\mathbb{R}) \subseteq \operatorname{GL}_n(\mathbb{R})$.
        \item closure: \begin{align*}
            A, B &\in O_n(\mathbb{R}) \\
            (AB)^T (AB) &= B^T A^T A B \\
            &= B^T B = I \\
            \implies AB &\in O_n(\mathbb{R})
        \end{align*} 
        \item $I_n \in O_n(\mathbb{R})$
        \item Associativity is inherited.
        \item inverse: $A^T A = I_n \implies A^T = A^{-1}$ and $A^T \in O_n(\mathbb{R})$ since $(A^T)^T = A$ and $A A^T = I$.
    \end{itemize} 
\end{proof} 

Note $1 = (\det A)^2 \implies \det A = \pm 1$ if $A \in O_n(\mathbb{R})$. \\
So, \begin{align*}
    \text{Det} : O_n(\mathbb{R}) &\to \left( \{ \pm 1 \}, \times \right) \\
    A &\mapsto \det A
\end{align*} is a surjective homomorphism as, $\begin{pNiceMatrix}
    -1 &       & \Block{2-3}<\huge>{0} \\
        &   1   &        &      &       \\
        &       &   1    &      &       \\
    \Block{2-3}<\huge>{0}
        &       &       & \Ddots    &   \\
        &       &       &      &   1   \\
  \end{pNiceMatrix} \in O_n(\mathbb{R})$.
\begin{align*}
    \text{So } \ker (\text{Det}) &= \{ A \in O_n(\mathbb{R}) : \det A = 1\} \\
    &= SO_n(\mathbb{R}) \trianglelefteq O_n(\mathbb{R}).
\intertext{By \nameref{thm:six}:}
    O_n(\mathbb{R}) / SO_n(\mathbb{R}) &\cong C_2.
\end{align*} 

\begin{lemma}\label{lem:20}
    Let $A \in O_n(\mathbb{R})$ and $\underline{x}, \underline{y} \in \mathbb{R}^n$. Then
    \begin{enumerate}
        \item $A \underline{x} \cdot A \underline{y} = \underline{x} \cdot \underline{y}$
        \item $|A \underline{x}| = |\underline{x}|$. \label{07-itm-2}
    \end{enumerate} 
    So $A$ is an isometry (distance preserving map) of Euclidean space $\mathbb{R}^n$.
\end{lemma} 

\begin{proof} \mbox{}
    \begin{enumerate}
        \item \begin{align*}
            A \underline{x} \cdot A \underline{y} &= (A \underline{x})^T (A \underline{y}) \\
            &= \underline{x}^T A^T A \underline{y} \\
            &= \underline{x}^T \underline{y} \\
            &= \underline{x} \cdot \underline{y}.
        \end{align*} 
        \item \begin{align*}
            |A \underline{x}|^2 &= A \underline{x} \cdot A \underline{x} \\
            &= \underline{x} \cdot \underline{x} \\
            &= |\underline{x}|^2
        \end{align*} 
    \end{enumerate} 
\end{proof} 

Note by \Cref{07-itm-2} if $\lambda$ is an eigenvalue of $A$, $A \underline{x} = \lambda \underline{x} \implies |\lambda \underline{x}| = |\underline{x}|$ i.e. $|\lambda| = 1$.

\subsubsection{In 2 dimensions}
Let $A = \begin{pmatrix}a & b \\c & d\end{pmatrix} \in \operatorname{GL}_2(\mathbb{R})$ and
\begin{align*}
    I &= A A^T \\
    &= \begin{pmatrix}a & b \\c & d\end{pmatrix} \begin{pmatrix}a & c \\b & d\end{pmatrix} \\
    \implies 1 &= a^2 + b^2 = c^2 + d^2 \\
    0 &= ac + bd. \\
    I &= A^T A \\
    &= \begin{pmatrix}a & c \\b & d\end{pmatrix}  \begin{pmatrix}a & b \\c & d\end{pmatrix} \\
    \implies 1 &= a^2 + c^2 = b^2 + d^2 \\
    0 &= ab + cd. \\
\end{align*}

For $0 \leq \theta < 2 \pi$ let 
\begin{align*}
    \begin{pmatrix}a \\c\end{pmatrix} &= \begin{pmatrix}\cos \theta \\ \sin \theta\end{pmatrix} \\
    \text{so } \begin{pmatrix}b \\ d\end{pmatrix} &= \begin{pmatrix}\mp \sin \theta \\ \pm \cos \theta\end{pmatrix}. \\
    A &= \begin{pmatrix}
    \cos \theta & -\sin \theta \\
    \sin \theta & \cos \theta 
    \end{pmatrix} \\ 
    \det A &= 1 \text{ so a rotation and all elements of $SO_2(\mathbb{R})$ are of this form}. \\
    \text{Or } A &= \begin{pmatrix}
        \cos \theta & \sin \theta \\
        \sin \theta & - \cos \theta 
        \end{pmatrix} \\
    \det A &= - 1 \\
    A \begin{pmatrix}x \\y\end{pmatrix} &= e^{i \theta} \overline{x + iy} = e^{i \theta} \bar{z}
    \intertext{What are the fixed points?}
    z = e^{i \theta} \bar{z} &\iff e^{- i \theta / 2} z = e^{i \theta / 2} \bar{z} \\
    &\iff e^{- i \theta / 2} z = t \in \mathbb{R} \\
    &\iff z = e^{i \theta / 2} t. \\
    &\implies \text{ a reflection in line } e^{i \theta / 2} \\
    \intertext{All elmenents of $O_2(\mathbb{R}) \setminus SO_2(\mathbb{R})$ of this form.}
    \text{So, } O_2(\mathbb{R}) &= \underbrace{SO_2(\mathbb{R})}_\text{rotations} \mathbin{\dot{\cup}} \underbrace{\begin{pmatrix}1 & 0 \\0 & -1\end{pmatrix} SO_2(\mathbb{R})}_\text{reflections} 
\end{align*} 
Note any element of $O_2(\mathbb{R})$ is a product of at most two reflections.
Since if $A \in SO_2(\mathbb{R})$ then $A = \underbrace{\left( A \begin{pmatrix}1 & 0 \\0 & -1\end{pmatrix} \right)}_\text{reflection} \underbrace{\begin{pmatrix}1 & 0 \\0 & -1\end{pmatrix}}_\text{reflection}$.

\subsubsection{In 3 dimensions}
\begin{proposition}\label{prp:11}
    Let $A \in SO_3(\mathbb{R})$.
    Then $A$ has an eigenvector with eigenvalue $1$.
\end{proposition} 

\begin{proof}
    \begin{align*}
        \det (A - I) &= \det  (A - A A^T) \\
        &= \det A \det (I - A^T) \\
        &= 1 \cdot \det (I - A)^t \\
        &= \det (I - A) \\
        &= (-1)^3 \det (A - I) \\
        &= - \det (A - I) \\
        \implies \det (A - I) &= 0 \text{ and $A$ has an eigenvalue $= 1$}.
    \end{align*} 
\end{proof} 

\begin{proof}[Alternative Proof]
    Consider $\chi_A(x)$ (characteristic poly of $A$), it is a cubic in $\mathbb{R}$ and its roots multiply to 1 ($\det A = 1 = \Pi_i \lambda_i$).
    Thus it has a real root, and $|\lambda| = 1 \implies \lambda = \pm 1$. 
    But the other eigenvalues are either a complex conjugate pair, then $\lambda = 1$ as product of eigenvalues give $\det A = 1$, or all are real so either $1, -1, -1$ or $1, 1, 1$.
\end{proof} 

\begin{theorem} \label{thm:11}
    Let $A \in SO_3(\mathbb{R})$ then $A$ is conjugate (i.e. there is a change of basis) to a matrix of the form $\begin{pmatrix}
    \cos \theta & -\sin \theta & 0 \\
    \sin \theta & \cos \theta & 0 \\
    0 & 0 & 1
    \end{pmatrix}$ for some $\theta \in [0, 2 \pi]$.
    In particular, $A$ is a rotation around an axis through the origin.
\end{theorem} 

\begin{proof}
    By \Cref{prp:11} $\exists \; \underline{v} \in \mathbb{R}^3$ with $A \underline{v} = \underline{v}$, we can assume $|\underline{v}| = 1$.
    Let $\{e_1, e_2, e_3\}$ be the standard orthonormal basis for $\mathbb{R}^3$.
    There exists $P \in SO_3(\mathbb{R})$ s.t. $P \underline{v} = e_3$.
    So $PAP^{-1}(e_3) = e_3$ and for $\Pi$ plane perpendicular to $e_3$ then $PAP^{-1}(\Pi)$ is perpendicular to $e_3$.
    So, 
    \begin{align*}
        PAP^{-1} &= \begin{pNiceArray}{cc|c}[margin]
            \Block{2-2}{\text{action} \\ \text{on $\Pi$}} & & 0 \\
            & & 0 \\
            \hline
            0 & 0 & 0
            \end{pNiceArray} \\
            &= \begin{pNiceArray}{cc|c}[margin]
                \Block{2-2}<\Large>{Q} & & 0 \\
                & & 0 \\
                \hline
                0 & 0 & 0
                \end{pNiceArray} \\
            \det (P A P^{-1}) &= \det A = 1 \\
            \implies \det Q &= 1 \\
            (P A P^{-1}) (P A P^{-1})^T &= I \implies Q Q^T = I.\\
            \text{So, } Q &\in SO_2(\mathbb{R}).
    \end{align*} 
\end{proof}  

Suppose $\underline{r}$ is a reflection in a plane $\Pi$ through 0.
Let $\underline{n}$ be a unit vector perpendicular to $\Pi$.
Then $r(\underline{x}) = \underline{x} - 2 (\underline{x} \cdot \underline{n}) \underline{n}$, $\underline{n} \mapsto - \underline{n}$, $\Pi$ is fixed.
So $\underline{r}$ is conjugate to $\begin{pmatrix}-1 & 0 & 0 \\0 & 1 & 0 \\0 & 0 & 1\end{pmatrix} \in O_3(\mathbb{R})$ by taking basis $\underline{n}$ and two orthogonal unit vectors in $\Pi$.
\begin{align*}
    O_3(\mathbb{R}) = \underbrace{SO_3(\mathbb{R})}_\text{rotations, $\det = 1$} \mathbin{\dot{\cup}} \underbrace{\begin{pmatrix}-1 & 0 & 0 \\0 & 1 & 0 \\0 & 0 & 1\end{pmatrix} SO_3(\mathbb{R})}_{\det = -1}
\end{align*} 
$\begin{pmatrix}-1 & 0 & 0 \\0 & 1 & 0 \\0 & 0 & 1\end{pmatrix} SO_3(\mathbb{R})$ includes
\begin{itemize}
    \item reflections
    \item inversion in origin, $-I_3$
    \item combinations of rotations and reflections
\end{itemize} 

\begin{theorem}\label{thm:13}
    Any element of $O_3(\mathbb{R})$ is a product of at most 3 reflections.
\end{theorem} 

\begin{proof}
    Let $\{e_1, e_2, e_3\}$ be the standard orthonormal basis for $\mathbb{R}^3$.
    Let $A \in O_3(\mathbb{R})$.
    \begin{align*}
        |A e_3| = |e_3| = 1 \text{ since $A$ is an isometry.}
    \end{align*} 
    So $\exists$ a reflection $r_1$ s.t. $r_1 A(e_3) = e_3$.
    Let $\Pi = \langle e_1, e_2 \rangle \mathbin{\bot} e_3$.
    So $r_1 A(\Pi) = \Pi$, as angles are preserved. \\
    $\exists$ a reflection $r_2$ s.t. $r_2(e_3) = e_3,\ r_2(r_1 A (e_2)) = e_2$.
    So $r_2 r_1 A$ fixes $e_2$ and $e_3$. \\
    So $r_2 rq A (e_1) = \pm e_1$, if $e_1 = e_1$, set $r_3 = \text{id}$.
    Else $e_1 = - e_1$, let $r_3$ be the reflection in the plane $\bot$ to $e_1$. \\
    So $r_3 r_2 r_1 A$ fixes $e_1, e_2, e_3$, so $r_3 r_2 r_1 A = \text{id} \implies A = r_1^{-1} r_2^{-1} r_3^{-1} = r_1 r_2 r_3$.
\end{proof} 

Alternatively, any element in $SO_3(\mathbb{R})$ is a product of at most 2 reflections, via 2-dimensional case.
Thus any element of $\begin{pmatrix}-1 & 0 & 0 \\0 & 1 & 0 \\0 & 0 & 1\end{pmatrix} SO_3(\mathbb{R})$ is a product of at most 3 reflections.
Note we do need $3$, e.g. $-I_3$.