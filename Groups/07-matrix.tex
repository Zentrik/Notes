\section{Matrix Groups}

Let $M_n(\mathbb{R})$ denote the set of all $n \times n$ matrices with entries in $\mathbb{R}$. Define
\begin{align*}
    \operatorname{GL}_n(\mathbb{R}) = \{A \in M_n(\mathbb{R}) : \det A \neq 0\}.
\end{align*} 

\begin{proposition} \label{prp:8}
    $\operatorname{GL}_n(\mathbb{R})$ is a group under matrix multiplication.
    It is called the \emph{general linear group}.
\end{proposition} 

\begin{proof}
    closure: $A, B \in \operatorname{GL}_n(\mathbb{R})$, clearly $AB \in \operatorname{M_n}(\mathbb{R})$ and $\det(AB) = \det A \det B \neq 0$ so $AB \in \operatorname{GL}_n(\mathbb{R})$.

    identity: $I_n = \begin{pmatrix}
        1 & & \\
        & \ddots &  \\
        & & 1
      \end{pmatrix} \in \operatorname{GL}_2(\mathbb{R})$

    inverse: $\det A \neq 0 \implies A^{-1}$ exists, $\det A^{-1} = \frac{1}{\det A} \neq 0$.

    matrix multiplication is associative: 
    \begin{align*}
        (A(BC))_{ij} &= A_{ik} (BC)_{kj} \\
        &= A_{ik} B_{kt} C_{tj} \\
        ((AB)C)_{ij} &= (AB)_{ik} C_{kj} \\
        &= A_{it} B_{tk} C_{kj}.
    \end{align*} 
\end{proof} 

\begin{example}
    \begin{align*}
        \operatorname{GL}_2(\mathbb{R}) &= \left\{ \begin{pmatrix}
        a & b \\
        c & d
        \end{pmatrix} : a, b, c, d \in \mathbb{R},\ ad - bc \neq 0 \right\} \\
        \begin{pmatrix}
            a & b \\
            c & d
            \end{pmatrix}^{-1} &= \frac{1}{ad - bc} \begin{pmatrix}
        d & -b \\
        -c & -a
        \end{pmatrix}.
    \end{align*}
\end{example} 

\begin{proposition}\label{prp:9}
    \begin{align*}
        \operatorname{Det} : \operatorname{GL}_n(\mathbb{R}) &\to (\mathbb{R} \setminus \{0\}, \times) \\
        A &\mapsto \det A.
    \end{align*} is a surjective group homomorphism.
\end{proposition} 

\begin{proof}
    Note $(\mathbb{R} \setminus \{0\}, \times)$ is a group.
    Det is clearly a map to $(\mathbb{R} \setminus \{0\}, \times)$, we need to check it's a group homomorphism,
    \begin{align*}
        \operatorname{Det}(AB) &= \det(AB) \hspace{0.5cm} \text{($AB$ is multiplication in } \operatorname{GL}_n)\\
        &= \det A \cdot \det B \hspace{0.5cm} \text{(multiplication in } (\mathbb{R} \setminus \{0\}, \times))\\
        &= \operatorname{Det} A \operatorname{Det} B
    \end{align*} 
    and that Det is surjective.
    Let $r \in (\mathbb{R} \setminus \{0\}, \times)$ then 
    \begin{align*}
        A = \begin{pNiceMatrix}
            r   &       & \Block{2-3}<\huge>{0} \\
                &   1   &        &      &       \\
                &       &   1    &      &       \\
            \Block{2-3}<\huge>{0}
                &       &       & \Ddots    &   \\
                &       &       &      &   1   \\
          \end{pNiceMatrix} \in \operatorname{GL}_n(\mathbb{R}) \mapsto \det(A) = r.
    \end{align*}
\end{proof} 

By \nameref{thm:six} $\ker(\text{Det}) \trianglelefteq \operatorname{GL}_n(\mathbb{R})$.
\begin{align*}
    \ker(\text{Det}) &= \{A \in \operatorname{GL}_n(\mathbb{R}) : \det A = 1\} \\
    &= \operatorname{SL}_n(\mathbb{R}) \text{ the special linear group}.
\end{align*} 
Furthermore, by \nameref{thm:six}
\begin{align*}
    \operatorname{GL}_n(\mathbb{R}) / \operatorname{SL}_n(\mathbb{R}) \cong (\mathbb{R} \setminus \{0\}, \times).
\end{align*} 

\begin{remark}
    More generally we can define the general linear group and special linear group over any field. \\
    Examples of fields: \\
    $\mathbb{R}, \mathbb{C}, \mathbb{Q}, \mathbb{F}_p$ where $\mathbb{F}_p = \left( \{0, 1, 2, \ldots, p- 1\}, +_p, \times_p \right)$ and $p$ is prime.
    Note $\operatorname{GL}_n(\mathbb{F}_p)$ and $\operatorname{SL}_n(\mathbb{F}_p)$ are finite groups.
\end{remark} 

What is $|GL_3(\mathbb{F}_p)|$? \\
Non-zero determinant means we have linearly independent columns.
The no. of choices for the first column is $p^3 - 1$ (can't be $\begin{pmatrix}0 & 0 & 0\end{pmatrix}^T$).
The second column is not a multiple of first so $p^3 - p$ (there are $p$ multiples of first column).
Third column not in space spanned by first two columns, this space has size $p^2$ (consider $\alpha c_1 + \beta c_2$ with $\alpha, \beta \in \mathbb{F}_p$), so $p^3 - p^2$. \\
$\implies |GL_3(\mathbb{F}_p)| = (p^3 - 1)(p^3 - p)(p^3 - p^2)$.

We can still consider 
\begin{align*}
    \mathrm{Det} : \operatorname{GL}_3(\mathbb{F}_p) &\to (\mathbb{F}_p \setminus \{0\}, \times) \\
    A &\mapsto \det(A).
\end{align*} 
Note $(\mathbb{F}_p \setminus \{0\}, \times)$ is a group: we have closure, identity $= 1$ and associativity.
Let $a \in \mathbb{F}_p \setminus \{0\}$, by Bezout's Thm $\exists \; x, y$ s.t.
\begin{align*}
    ax + py &= 1 \\
    \therefore ax &\equiv 1 \pmod p \\
    \text{Choose } \bar{x} &\equiv x \pmod p, \\
    1 \leq \bar{x} &\leq p - 1,\ a^{-1} = x
\end{align*} 
Det is a surjective homomorphism to $(\mathbb{F}_p \setminus \{0\}, \times)$ so by \nameref{thm:six}
\begin{align*}
    |\operatorname{GL}_3(\mathbb{F}_p)| / |\operatorname{SL}_3(\mathbb{F}_p)| &= p - 1 \\
    \implies |\operatorname{SL}_3(\mathbb{F}_p)| &= \frac{(p^3 - 1)(p^3 - p)(p^3 - p^2)}{p - 1}.
\end{align*} 

\subsection{Actions of $\operatorname{GL}_n(\mathbb{C})$}

\begin{enumerate}
    \item Let $\mathbb{C}^n$ denote vectors of length $n$ with entries in $\mathbb{C}$:
    \begin{align*}
        \operatorname{GL}_n(\mathbb{C}) \times \mathbb{C}^n &\to \mathbb{C}^n \\
        (A, v) &\mapsto Av.
    \end{align*} 
    Note $Iv = v$, $(AB)v = A(B(v))$.
    This action is faithful: $Av = v \forall \; v \in \mathbb{C}^n \implies A = I_n$ (consider $v = e_i$).
    The action has two orbits
    \begin{align*}
        \operatorname{Orb}_{\operatorname{GL}_n(\mathbb{C})}(\underline{0}) &= \{\underline{0}\} \\
        \operatorname{Orb}_{\operatorname{GL}_n(\mathbb{C})}(v) &= \mathbb{C}^n \setminus \{0\} \text{ for } v \neq \underline{0},
    \end{align*} i.e. given $w \neq 0 \; \exists \; A \in \operatorname{GL}_n(\mathbb{C})$ s.t. $Av = w$.
    \item Conjugation action of $\operatorname{GL}_n(\mathbb{C})$ on $M_n(\mathbb{C})$ (set of all matrices)
    \begin{align*}
        \operatorname{GL}_n(\mathbb{C}) \times M_n(\mathbb{C}) &\to M_n(\mathbb{C}) \\
        (P, A) &\mapsto P A P^{-1}. \\
    \text{Note: } PQ(A) &= PQ A (PQ)^{-1} \\
        &= PQ A Q^{-1} P^{-1} \\
        &= P(Q(A)).
    \end{align*} 
\end{enumerate} 

\begin{remark}
    Matrices $A$ and $B$ are conjugate if they represent the same linear map. 
    If $PAP^{-1} = B$, then $P$ represents a change of basis matrix. (See LA next year)
\end{remark} 

\begin{example}
    \begin{align*}
        A : e_1 &\mapsto 2 e_1 \\
        e_2 &\mapsto 3e_2 \\
        A &= \begin{pmatrix}
        2 & 0 \\
        0 & 3
        \end{pmatrix} \\
        \text{Let } P : e_1 &\mapsto e_2 \\
        e_2 &\mapsto e_1 \\
        P &= \begin{pmatrix}
        0 & 1 \\
        1 & 0
        \end{pmatrix} \\
        &= P^{-1} \\
        \intertext{$P$ is a change of basis}
        PAP^{-1} &= \begin{pmatrix}
            0 & 1 \\
            1 & 0
            \end{pmatrix} \begin{pmatrix}
                2 & 0 \\
                0 & 3
                \end{pmatrix} \begin{pmatrix}
                0 & 1 \\
                1 & 0
                \end{pmatrix} \\
            &= \begin{pmatrix}
            3 & 0 \\
            0 & 2
            \end{pmatrix} \\
        \text{i.e. } e_2 &\mapsto 3e_2 \\
        e_1 &\mapsto 2 e_1.
    \end{align*}
\end{example} 

We will use the following result from V\&M when investigating M\"obius groups.

\emph{Result}
Let $A \in M_2(\mathbb{C})$ and consider conjugation action of $\operatorname{GL}_2(\mathbb{R})$ on $M_2(\mathbb{C})$.
Then precisely one of the following occurs
\begin{enumerate}
    \item The orbit of $A$ contains a diagonal matrix $\begin{pmatrix}\lambda & 0 \\0 & \mu\end{pmatrix}$ with $\lambda \neq \mu$.
    \item The orbit of $A$ is $\begin{pmatrix}\lambda & 0 \\0 & \lambda\end{pmatrix} = \lambda I$ for some $\lambda$.
    \item The orbit of $A$ contains a matrix $\begin{pmatrix}\lambda & 1 \\0 & \lambda\end{pmatrix}$ for some $\lambda$.
\end{enumerate} 

\begin{proof}
    See V\&M but essentially 
    \begin{enumerate}
        \item In this case $A$ has $2$ distinct eigenvalues $\lambda \neq u$, take a basis consisting of an eigenvector for $\lambda$ and an eigenvector for $\mu$.
        Distinct pairs given distinct orbits.
        \item $PAP^{-1} = \lambda I \implies A = P \lambda I P^{-1} = \lambda I$, eigenvalues $\lambda, \lambda$, 2 linearly independent eigenvectors.
        \item In this case $A$ has a repeated eigenvalue, but just one linearly independent eigenvector.
    \end{enumerate} 
\end{proof} 

\begin{center}\rule{\linewidth}{0.5pt}\end{center}

\emph{Aside - Recalling transpose properties}

Recall if $A \in M_n(\mathbb{R})$, $A^T$ is defined by $(A^T)_{ij} = A_{ji}$, i.e. the $ij$-th entry of $A^T$ is the $ji$-th entry of $A$
\begin{align*}
    A &= \begin{pmatrix}2 & 4 \\3 & 5\end{pmatrix} \\
    A^T &= \begin{pmatrix}2 & 3 \\4 & 5\end{pmatrix}
\end{align*} 

\emph{Note}
\begin{enumerate}
    \item
    \begin{align*}
        (AB)^T &= B^T A^T \\
        [(AB)^T]_{ij} &= (AB)_{ji} \\
        &= A_{jk}B_{ki} \\
        [B^T A^T]_{ij} &= B^T_{ik} A^T_{kj} \\
        &= B_{ki} A_{jk} \checkmark
    \end{align*} 
    \item \begin{align*}
        A A^T &= I \\
        \implies 1 &= \det (A^T A) \\
        &= \det A^T \det A \\
        &= (\det A)^2 \\
        \implies \det A &\neq 0 \\
    \end{align*}
    \item \begin{align*}
        A A^T &= I \iff A^T A = I. \\
        \implies A^T A &= A^{-1} \underbrace{A A^T}_I A \\
        &= A^{-1} A \\
        &= I.
    \end{align*} 
    \item
    \begin{align*}
        (A^T)^{-1} &= (A^{-1})^T \\
        \text{Since } I_n &= (A A^{-1})^T \\
        &= (A^{-1})^T A^T
    \end{align*} 
    \item \begin{align*}
        \det A^T &= \det A.
    \end{align*} 
\end{enumerate} 

\begin{center}\rule{\linewidth}{0.5pt}\end{center}

Define \begin{align*}
    O_n(\mathbb{R}) &= \{ A \in M_n(\mathbb{R}) : A^T A = I\}
\end{align*} (so columns of $A$ form an orthonormal basis for $\mathbb{R}^n$).

\begin{proposition}\label{prp:10}
    $O_n(\mathbb{R})$ is a subgroup of $\operatorname{GL}_n(\mathbb{R})$ called the \emph{orthogonal group}.
\end{proposition} 

\begin{proof}
    \begin{align*}
        \det A &\neq 0 \implies O_n(\mathbb{R}) \subseteq \operatorname{GL}_n(\mathbb{R}).
    \end{align*} 
    Associativity is inherited.
\end{proof} 