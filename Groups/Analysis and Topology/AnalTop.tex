%&../preamble

\def\npart {IB}
\def\nterm {Michaelmas}
\def\nyear {2022}
\def\nlecturer {Dr P. Russell}
\def\ncourse {Analysis and Topology}

\def\encodingdefault{TU}\normalfont
\ifnum 0\ifxetex 1\fi\ifluatex 1\fi=0 % if pdftex
  \usepackage[T1]{fontenc}
  \usepackage[utf8]{inputenc}
  \usepackage{textcomp} % provide euro and other symbols
\else % if luatex or xetex
  % \usepackage{unicode-math}
  % \defaultfontfeatures{Scale=MatchLowercase}
  % \defaultfontfeatures[\rmfamily]{Ligatures=TeX,Scale=1}
  % \DeclareMathAlphabet{\mathcal}{OMS}{cmsy}{m}{n}
  % \let\mathbb\relax % remove the definition by unicode-math
  % \DeclareMathAlphabet{\mathbb}{U}{msb}{m}{n}
\fi

\usetikzlibrary{external}
\tikzset{external/system call={xelatex -fmt=../preamble.fmt \tikzexternalcheckshellescape -halt-on-error -interaction=batchmode -jobname "\image" "\texsource"}} % path is relative to file that includes preamble
\tikzexternalize

\providetoggle{DontSetTitleAuthorDate}

\nottoggle{DontSetTitleAuthorDate}{
  \hypersetup{
    pdftitle={Part \npart\ - \ncourse},
    pdfsubject={Cambridge Maths Notes: Part \npart\ - \ncourse},
    pdfkeywords={Cambridge Mathematics Maths Math \npart\ \nterm\ \nyear\ \ncourse}
  }

  \author{Based on lectures by \nlecturer}
  \date{\nterm\ \nyear}
  \title{Part \npart\ --- \ncourse}
}{}

\tikzsetexternalprefix{figtemp/}
\newcommand{\norm}[1]{\left \lVert #1 \right \rVert}
% \includeonly{00-intro.tex}

% \setcounter{section}{-1}

\begin{document}
    \maketitle
    \tableofcontents

    \part{Generalizing continuity and convergence}
    \section{Three Examples of Convergence}
    \subsection{Convergence in $\mathbb{R}$}
    Let $(x_n)$ be a sequence in $\mathbb{R}$ and $x \in \mathbb{R}$.
    We say $(x_n)$ \textit{converges} to $x$ and write $x_n \to x$ if
    \begin{align*}
        \forall \; \epsilon > 0 \quad \exists \; N \quad \forall \; n \geq N \quad |x_n - x| < \epsilon.
    \end{align*} 
    Useful fact: $\forall \; a, b \in \mathbb{R} \ |a+b| \leq |a| + |b|$ (Triangle Inequality).

    Bolzano-Weierstrass Theorem (BWT)
    A bounded sequence in $\mathbb{R}$ must have a convergent subsequence (Proof by interval bisection).

    Recall: A sequence $(x_n)$ in $\mathbb{R}$ is Cauchy if 
    \begin{align*}
        \forall \; \epsilon > 0 \quad \exists \; N \quad \forall \; m, n \geq N \quad |x_m - x_n| < \epsilon.
    \end{align*} 

    Easy exercise Convergent $\implies$ Cauchy

    General Principle of Convergence (GPC)
    Any Cauchy sequence in $\mathbb{R}$ converges.

    \begin{proof}[Outline]
        If $(x_n)$ Cauchy then $(x_n)$ bounded so by BWT has a convergent subsequence, say $x_{n_j} \to x$.
        But as $(x_n)$ Cauchy, $x_n \to x$.
    \end{proof} 

    \subsection{Convergence in $\mathbb{R}^2$}
    \begin{remark}
        This all works in $\mathbb{R}^n$
    \end{remark} 

    Let $(z_n)$ be a sequence in $\mathbb{R}^2$ and $z \in \mathbb{R}^2$.
    What should $z_n \to z$ mean?

    In $\mathbb{R}$: ``As $n$ gets large, $z_n$ gets arbitrarily close to $z$.''

    What does `close' mean in $\mathbb{R}^2$?

    In $\mathbb{R}$: $a, b$ close if $|a - b|$ small.
    In $\mathbb{R}^2$: Replace $|\cdot|$ by $\left \lVert \cdot \right \rVert $

    Recall: If $z = (x, y)$ then $\left \lVert z \right \rVert = \sqrt{x^2 + y^2}$.

    Triangle Inequality If $a, b \in \mathbb{R}^2$ then $\left \lVert a + b \right \rVert \leq \left \lVert a \right \rVert + \left \lVert b \right \rVert$.

    \begin{definition}
        Let $(z_n)$ be a sequence in $\mathbb{R}^2$ and $z \in \mathbb{R}^2$.
        We say $(z_n)$ \vocab{converges} to $z$ and .. $z_n \to z$ if $\forall \; \epsilon > 0 \ \exists \; N \ \forall \; n \geq N \ \left \lVert z_n - z \right \rVert < \epsilon$. 

        Equivalently, $z_n \to z$ iff $\left \lVert z_n - z \right \rVert \to 0$ (convergence in $\mathbb{R}$).
    \end{definition} 

    \begin{example}
        Let $(z_n), (w_n)$ be sequences in $\mathbb{R}^2$ with $z_n \to z, w_n \to w$. 
        Then $z_n + w_n \to z + w$.
    \end{example} 

    \begin{proof}
        \begin{align*}
            \left \lVert (z_n + w_n) - (z + w) \right \rVert &\leq \left \lVert z_n - z \right \rVert + \left \lVert w_n - w \right \rVert \\
            &\to 0 + 0 = 0 \ \text{(by results from IA)}.
        \end{align*} 
    \end{proof} 

    In fact, given convergence in $\mathbb{R}$, convergence in $\mathbb{R}^2$ is easy:
    \begin{proposition} \label{prop:one}
        Let $(z_n)$ be a sequence in $\mathbb{R}^2$ and let $z \in \mathbb{R}^2$.
        Write $z_n = (x_n, y_n)$ and $z = (x, y)$.
        Then $z_n \to z$ iff $x_n \to x$ and $y_n \to y$.
    \end{proposition} 

    \begin{proof}
        ($\implies$): $|x_n - x|, |y_n - y| \leq \norm{z_n - z}$.
        So if $\norm{z_n - z} \to 0$ then $|x_n - x| \to 0$ and $|y_n - y| \to 0$.

        ($\Longleftarrow$): If $|x_n - x| \to 0$ and $|y_n - y| \to 0$ then $\norm{z_n - z} = \sqrt{(x_n - x)^2 + (y_n - y)^2} \to 0$ by results in $\mathbb{R}$.
    \end{proof} 

    \begin{definition}[Bounded Sequence]
        A sequence $(z_n)$ in $\mathbb{R}^2$ is \vocab{bounded} if $\exists \; M \in \mathbb{R}$ s.t. $\forall \; n \ \norm{z_n} \leq M$.
    \end{definition} 

    \begin{theorem}[BWT in $\mathbb{R}^2$]
        A bounded sequence in $\mathbb{R}^2$ must have a convergent subsequence.
    \end{theorem} 

    \begin{theorem}[GPC for $\mathbb{R}^2$]
        Any Cauchy sequence in $\mathbb{R}^2$ converges.
    \end{theorem} 

    \begin{proof}
        Let $(z_n)$ be a Cauchy sequence in $\mathbb{R}^2$.
        Write $z_n = (x_n, y_n)$.
        For all $m, n, |x_m - x_n| \leq \norm{z_m - z_n}$ so $(x_n)$ is a Cauchy sequence in $\mathbb{R}$, so converges by GPC.
        Similarly, $(y_n)$ converges in $\mathbb{R}$.
        So by \ref{prop:one}, $(z_n)$ converges.
    \end{proof} 

    \underline{Thought for the day} What about continuity?
    Let $f: \mathbb{R}^2 \to \mathbb{R}$.
    What does it mean for $f$ to be continuous?
    (Simple modification of defn for $\mathbb{R} \to \mathbb{R}$).

    What can we do with it?

    Big theorem in IA: If $f: \mathbb{R} \to \mathbb{R}$ is a continuous function on a closed bounded interval then $f$ is bounded and attains its bounds.

    Is there a similar theorem for $\mathbb{R}^2 \to \mathbb{R}$.
    What do we replace `closed bounded interval' by?
    We proved the theorem using BWT.
    Why did it work?
    Why did we need a closed bounded interval to make it work?
    What can we do in $\mathbb{R}^2$?

    \section{Metric Spaces}
    \section{Topological Spaces}
    \part{Generalizing differentiation}
\end{document}