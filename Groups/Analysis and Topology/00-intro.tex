\section{Introduction}
\vocab{Markov chains} are random processes (sequences of rvs) that retain no memory of the past.
So the past is independent of the future

Motivation: extend the law of large numbers to the non iid setting

Kolmegorov in 1930 extended them to continuous time Markov processes.

Brownian motion: fundamental object in modern probability theory.

Why study MC?
They are one of the simplest mathematical models for various random phenomena that evolve in time.
They are simple as they do not rely on the past which makes them amenable to analysis, so we can use tools from probability, analysis, combinatorics.

Applications: population growth (branching processes), mathematical genetics, queuing networks, Monte carlo simulation, \dots

Page-Rank algorithm

Model the ... as a directed graph, G: (V, E)

V: set of vertices