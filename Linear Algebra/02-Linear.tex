\section{Linear maps}
\subsection{Linear maps}
\begin{definition}
    If $V, W$ are $F$-vector spaces, a map $\alpha \colon V \to W$ is \textit{linear} if
    \begin{align*}
        \forall \lambda_1, \lambda_2 \in F, \forall v_1, v_2 \in V, \alpha(\lambda_1 v_1 + \lambda_2 v_2) = \lambda_1 \alpha(v_1) + \lambda_2 \alpha(v_2)
    \end{align*}
\end{definition}
\begin{example}
    Let $M$ be a matrix with $n$ rows and $m$ columns.
    Then the map $\alpha \colon \mathbb R^m \to \mathbb R^n$ defined by $x \mapsto M x$ is a linear map.
\end{example}
\begin{example}
    Let $\alpha \colon \mathcal C([0,1], \mathbb R) \to \mathcal C([0,1], \mathbb R)$ defined by $f \mapsto a(f)(x) = \int_0^x f(t) \dd{t}$.
    This is linear.
\end{example}
\begin{example}
    Let $x \in [a,b]$.
    Then $\alpha \colon \mathcal C([a,b], \mathbb R) \to \mathbb R$ defined by $f \mapsto f(x)$ is a linear map.
\end{example}
\begin{remark}
    Let $U, V, W$ be $F$-vector spaces.
    Then,
    \begin{enumerate}
        \item The identity function $i_V \colon V \to V$ defined by $x \mapsto x$ is linear.
        \item If $\alpha \colon U \to V$ and $\beta \colon V \to W$ are linear, then $\beta \circ \alpha$ is linear.
    \end{enumerate}
\end{remark}

\begin{lemma}
    Let $V, W$ be $F$-vector spaces.
    Let $B$ be a basis for $V$.
    If $\alpha_0 \colon B \to W$ is \textit{any} map (not necessarily linear), then there exists a unique linear map $\alpha \colon V \to W$ extending $\alpha_0$: $\forall v \in B, \alpha_0(v) = \alpha(v)$.
\end{lemma}

\begin{proof}
    Let $v \in V$.
    Then, given $B = (v_1, \dots, v_n)$.
    \begin{align*}
        v = \sum_{i=1}^n \lambda_i v_i
    \end{align*}
    By linearity,
    \begin{align*}
        \alpha(v) = \alpha\qty(\sum_{i=1}^n \lambda_i v_i) = \sum_{i=1}^n \lambda_i \alpha(v_i) = \sum_{i=1}^n \lambda_i \alpha_0(v_i)
    \end{align*}
\end{proof}

\begin{remark}
    This lemma is also true in infinite-dimensional vector spaces.
    Often, to define a linear map, we instead define its action on the basis vectors, and then we `extend by linearity' to construct the entire map.
\end{remark}

\begin{remark}
    If $\alpha_1, \alpha_2 \colon V \to W$ are linear maps, then if they agree on any basis of $V$ then they are equal.
\end{remark}

\subsection{Isomorphism}
\begin{definition}[Isomorphism]
    Let $V, W$ be $F$-vector spaces.
    A map $\alpha \colon V \to W$ is an \vocab{isomorphism} if and only if
    \begin{enumerate}
        \item $\alpha$ is linear
        \item $\alpha$ is bijective
    \end{enumerate}
    If such an $\alpha$ exists, we say that $V$ and $W$ are isomorphic, written $V \cong W$.
\end{definition}

\begin{remark}
    If $\alpha$ in the above definition is an isomorphism, then $\alpha\inv \colon W \to V$ is linear.
    Indeed, if $w_1, w_2 \in W$ with $w_1 = \alpha(v_1)$ and $w_2 = \alpha(v_2)$,
    \begin{align*}
        \alpha\inv (w_1 + w_2) = \alpha\inv (\alpha(v_1) + \alpha(v_2)) = \alpha\inv \alpha (v_1 + v_2) = v_1 + v_2 = \alpha\inv(w_1) + \alpha\inv(w_2)
    \end{align*}
    Similarly, for $\lambda \in F, w \in W$,
    \begin{align*}
        \alpha\inv(\lambda w) = \lambda \alpha\inv(w)
    \end{align*}
\end{remark}

\begin{lemma}
    Isomorphism is an equivalence relation on the class of all vector spaces over $F$.
\end{lemma}

\begin{proof}
    \begin{enumerate}
        \item $i_V \colon V \to V$ is an isomorphism
        \item If $\alpha \colon V \to W$ is an isomorphism, $\alpha\inv \colon W \to V$ is an isomorphism.
        \item If $\beta \colon U \to V, \alpha \colon V \to W$ are isomorphisms, then $\alpha \circ \beta \colon U \to W$ is an isomorphism.
    \end{enumerate}
    The proofs of each part are left as an exercise.
\end{proof}

\begin{theorem}
    If $V$ is an $F$-vector space of dimension $n$, then $V \cong F^n$.
\end{theorem}

\begin{proof}
    Let $B = (v_1, \dots, v_n)$ be a basis for $V$.
    Then, consider $\alpha \colon V \to F^n$ defined by
    \begin{align*}
        v = \sum_{i=1}^n \lambda_i v_i \mapsto \begin{pmatrix}\lambda_1 \\ \vdots \\ \lambda_n \end{pmatrix}
    \end{align*}
    We claim that this is an isomorphism.
    This is left as an exercise.
\end{proof}

\begin{remark}
    Choosing a basis for $V$ is analogous to choosing an isomorphism from $V$ to $F^n$.
\end{remark}

\begin{theorem}
    Let $V, W$ be $F$-vector spaces with finite dimensions $n, m$.
    Then,
    \begin{align*}
        V \cong W \iff n = m
    \end{align*}
\end{theorem}
\begin{proof}
    If $\dim V = \dim W = n$, then there exist isomorphisms from both $V$ and $W$ to $F^n$.
    By transitivity, therefore, there exists an isomorphism between $V$ and $W$.

    Conversely, if $V \cong W$ then let $\alpha \colon V \to W$ be an isomorphism.
    Let $B$ be a basis of $V$, then we claim that $\alpha(B)$ is a basis of $W$.
    Indeed, $\alpha(B)$ spans $W$ from the surjectivity of $\alpha$, and $\alpha(B)$ is free due to injectivity.
\end{proof}

\subsection{Kernel and image}
\begin{definition}
    Let $V, W$ be $F$-vector spaces.
    Let $\alpha \colon V \to W$ be a linear map.
    We define the kernel and image as follows.
    \begin{align*}
        N(\alpha) = \ker\alpha = \qty{v \in V \colon \alpha(v) = 0}
    \end{align*}
    \begin{align*}
        \Im(\alpha) = \qty{w \in W \colon \exists v \in V, w = \alpha(v)}
    \end{align*}
\end{definition}
\begin{lemma}
    $\ker \alpha$ is a subspace of $V$, and $\Im \alpha$ is a subspace of $W$.
\end{lemma}
\begin{proof}
    Let $\lambda_1, \lambda_2 \in F$ and $v_1, v_2 \in \ker \alpha$.
    Then
    \begin{align*}
        \alpha(\lambda_1 v_1 + \lambda_2 v_2) = \lambda_1 \alpha(v_1) + \lambda_2 \alpha(v_2) = 0
    \end{align*}
    Hence $\lambda_1 v_1 + \lambda_2 v_2 \in \ker \alpha$.

    Now, let $\lambda_1, \lambda_2 \in F$, $v_1, v_2 \in V$, and $w_1 = \alpha(v_1), w_2 = \alpha(v_2)$.
    Then
    \begin{align*}
        \lambda_1 w_1 + \lambda_2 w_2 = \lambda_1 \alpha(v_1) + \lambda_2 \alpha(v_2) = \alpha(\lambda_1 v_1 + \lambda_2 v_2) \in \Im \alpha
    \end{align*}
\end{proof}
\begin{remark}
    $\alpha \colon V \to W$ is injective if and only if $\ker \alpha = \{ 0 \}$.
    Further, $\alpha \colon V \to W$ is surjective if and only if $\Im \alpha = W$.
\end{remark}
\begin{theorem}
    Let $V, W$ be $F$-vector spaces.
    Let $\alpha \colon V \to W$ be a linear map.
    Then $\overline \alpha \colon V / \ker \alpha \to \Im \alpha$ defined by
    \begin{align*}
        \overline \alpha (v + \ker \alpha) = \alpha(v)
    \end{align*}
    is an isomorphism.
    \textit{This is the isomorphism theorem from IA Groups.}
\end{theorem}
\begin{proof}
    First, note that $\overline\alpha$ is well defined.
    Suppose $v + \ker \alpha = v' + \ker \alpha$.
    Then $v - v' \in \ker \alpha$, hence
    \begin{align*}
        \alpha(v - v') = 0 \implies \alpha(v) - \alpha(v') = 0
    \end{align*}
    so $\overline\alpha$ is indeed well defined.

    Linearity of $\overline \alpha$ follows from linearity of $\alpha$.

    Now, we show $\overline\alpha$ is injective.
    \begin{align*}
        \overline\alpha(v + \ker \alpha) = 0 \implies \alpha(v) = 0 \implies v \in \ker \alpha
    \end{align*}
    Hence, $v + \ker \alpha = 0 + \ker \alpha$.

    Further, $\overline\alpha$ is surjective as if $w \in \Im \alpha$, $\exists \; v \in V$ s.t. $w = \alpha(v) = \overline \alpha(v + \ker \alpha)$.
\end{proof}

\subsection{Rank and nullity}
\begin{definition}{Rank and nullity}
    The \textit{rank} of $\alpha$ is
    \begin{align*}
        r(\alpha) = \dim\Im \alpha.
    \end{align*}
    The \textit{nullity} of $\alpha$ is
    \begin{align*}
        n(\alpha) = \dim\ker \alpha.
    \end{align*}
\end{definition}

\begin{theorem}[Rank-nullity theorem]
    Let $U, V$ be $F$-vector spaces such that the dimension of $U$ is finite.
    Let $\alpha \colon U \to V$ be a linear map.
    Then,
    \begin{align*}
        \dim U = r(\alpha) + n(\alpha)
    \end{align*}
\end{theorem}

\begin{proof}
    We have proven that $U / \ker \alpha \cong \Im \alpha$.
    Hence, the dimensions on the left and right match: $\dim (U/\ker\alpha) = \dim \Im \alpha$.
    \begin{align*}
        \dim U - \dim \ker \alpha\footnote{by \cref{prp:quotientdim}} = \dim \Im \alpha
    \end{align*}
    and the result follows.
\end{proof}
\begin{lemma}[Characterisation of isomorphisms]
    Let $V, W$ be $F$-vector spaces with equal, finite dimension.
    Let $\alpha \colon V \to W$ be a linear map.
    Then, the following are equivalent.
    \begin{enumerate}
        \item $\alpha$ is injective.
        \item $\alpha$ is surjective.
        \item $\alpha$ is an isomorphism.
    \end{enumerate}
\end{lemma}
\begin{proof}
    Clearly, (iii) follows from (i) and (ii) and vice versa.
    The rest of the proof is left as an exercise, which follows from the rank-nullity theorem.
\end{proof}

\begin{example}
    \begin{align*}
        V &= \qty{\begin{bmatrix}x \\y \\z\end{bmatrix} \in \mathbb{R}^3 : x + y + z = 0} \\
        \alpha : \mathbb{R}^3 &\to \mathbb{R} \\
        \begin{bmatrix}x \\y \\z\end{bmatrix} &\mapsto x + y + z \\
        \implies \ker \alpha &= V \\
        \Im \alpha &= \mathbb{R}.
        \intertext{So by rank nullity}
        3 = n(\alpha) + 1 &\implies \dim V = 2
    \end{align*} 
\end{example} 

\subsection{Space of linear maps}
Let $V$ and $W$ be $F$-vector spaces.
Consider the space of linear maps from $V$ to $W$.
Then $L(V,W) = \qty{\alpha \colon V \to W \text{ linear}}$.
\begin{proposition}[Linear maps form a vector space]
    $L(V,W)$ is an $F$-vector space under the operation
    \begin{align*}
        (\alpha_1 + \alpha_2)(v) &= \alpha_1(v) + \alpha_2(v) \\
        (\lambda \alpha)(v) &= \lambda( \alpha(v) )
    \end{align*}
    Further, if $V$ and $W$ are finite-dimensional, then so is $L(V,W)$ with
    \begin{align*}
        \dim_F L(V,W) = \dim_F V \dim_F W
    \end{align*}
\end{proposition}
\begin{proof}
    Proving that $L(V,W)$ is a vector space is left as an exercise.
    The dimensionality part is proven later, \cref{prp:dimmapspace}.
\end{proof}

\subsection{Matrices}
\begin{definition}[Matrix]
    An $m \times n$ matrix over $F$ is an array with $m$ rows and $n$ columns, with entries in $F$.
\end{definition}
\begin{notation}
    We write $M_{m \times n}(F)$ for the set of $m \times n$ matrices over $F$.
\end{notation} 

\begin{proposition}
    $M_{m \times n}(F)$ is an $F$-vector space under
    \begin{align*}
        ((a_{ij}) + (b_{ij})) = (a_{ij} + b_{ij});
    \end{align*}
    \begin{align*}
        \lambda (a_{ij}) = (\lambda a_{ij})
    \end{align*}
\end{proposition}
\begin{proof}
    Left as an exercise
\end{proof}

\begin{proposition}
    $\dim_F M_{m,n}(F) = m n$.
\end{proposition}

\begin{proof}
    Consider the basis defined by, the `elementary matrix' for all $i,j$:
    \begin{align*}
        e_{pq} = \delta_{ip}\delta_{jq}
    \end{align*}
    Then $(e_{ij})$ is a basis of $M_{m \times n}(F)$, since it spans $M_{m \times n}(F)$\footnote{given $A = (a_{ij}) \in M_{n \times n}(F)$, $A = a_{ij} e_{ij}$} and we can show that it is free.
\end{proof}

\subsection{Linear maps as matrices}
Let $V, W$ be $F$-vector spaces and $\alpha : V \to W$ be a linear map.
Consider bases $B$ of $V$ and $C$ of $W$:
\begin{align*}
    B = (v_1, \dots, v_n);\ C = (w_1, \dots, w_m)
\end{align*}
Then let $v \in V$.
We have
\begin{align*}
    v = \sum_{j=1}^n \lambda_j v_j \equiv [v]_B = \begin{pmatrix}
        \lambda_1 \\ \vdots \\ \lambda_n
    \end{pmatrix} \in F^n
\end{align*}
where the vector given is the coordinates in basis $B$.
\begin{notation}
    $[v]_B$ is the coordinates of $v$ in basis $B$.
\end{notation} 
We can equivalently find $[w]_C$, the coordinates of $w$ in basis $C$.
We can now define a matrix of some linear map $\alpha$ in the $B, C$ basis.
\begin{definition}[Matrix of linear map]
    The matrix representing $\alpha$ wrt $B, C$ basis is
    \begin{align*}
        [\alpha]_{B,C} = \begin{pmatrix}
            [\alpha(v_1)]_C, \dots, [\alpha(v_n)]_C
        \end{pmatrix} \in M_{m\times n}(F)
    \end{align*}
\end{definition}

\begin{note}
    Let $[\alpha]_{B,C} = (a_{ij})$, then by definition
    \begin{align*}
        \alpha (v_j) = \sum_{i=1}^m a_{ij} w_i
    \end{align*}
\end{note} 

\begin{lemma}
    For all $v \in V$,
    \begin{align*}
        [\alpha(v)]_C = [\alpha]_{B,C} \cdot [v]_{B}
    \end{align*}
\end{lemma}

\begin{proof}
    We have
    \begin{align*}
        v = \sum_{i=1}^n \lambda_j v_j
    \end{align*}
    Hence
    \begin{align*}
        \alpha\qty(\sum_{i=1}^n \lambda_j v_j) = \sum_{j=1}^n \lambda_j \alpha(v_j) = \sum_{j=1}^n \lambda_i \sum_{i=1}^m a_{ij} w_i = \sum_{i=1}^m \qty( \sum_{j=1}^n a_{ij} \lambda_j ) w_i
    \end{align*}
\end{proof}

\begin{lemma}
    Let $\beta \colon U \to V$ and $\alpha \colon V \to W$ be linear maps.
    Then, if $A,B,C$ are bases of $U,V,W$ respectively, then
    \begin{align*}
        [\alpha \circ \beta]_{A,C} = [\alpha]_{B,C} \cdot [\beta]_{A,B}
    \end{align*}
\end{lemma}

\begin{proof}
    Let $A = [\alpha]_{B, C}$ and $B = [\beta]_{A, B}$.
    Consider $u_l \in A$ (basis of $U$).
    Then
    \begin{align*}
        (\alpha \circ \beta)(u_l) = \alpha(\beta(u_l))
    \end{align*}
    giving
    \begin{align*}
        \alpha\qty(\sum_j b_{jl} v_j) = \sum_j b_{jl} \alpha(v_j) = \sum_j b_{jl} \sum_i a_{ij} w_i = \sum_i \qty( \sum_j a_{ij} b_{jl} ) w_i
    \end{align*}
    where $a_{ij} b_{jl}$ is the $(i, l)$ element of $AB$ by the definition of the product of matrices.
\end{proof}
\begin{proposition} \label{prp:dimmapspace}
    If $V, W$ are $F$-vector spaces, and $\dim_F V = n, \dim_F W = m$, then
    \begin{align*}
        L(V,W) \cong M_{m \times n}(F)
    \end{align*}
    which implies the dimensionality of $L(V,W)$ in $F$ is $m \times n$.
\end{proposition}
\begin{proof}
    Consider two bases $B, C$ of $V, W$.
    We claim that
    \begin{align*}
        \theta \colon L(V,W) &\to M_{m \times n}(F) \\
        \alpha &\mapsto [\alpha]_{B, C}
    \end{align*}
    is an isomorphism.

    First, note that $\theta$ is linear.
    \begin{align*}
        [\lambda_1 \alpha_1 + \lambda_2 \alpha_2] = \lambda_1 [\alpha_1]_{B, C} + \lambda_2 [\alpha_2]_{B, C}.
    \end{align*} 

    Also, $\theta$ is surjective; consider any matrix $A = (a_{ij})$ and consider $\alpha \colon v_j \mapsto \sum_{i=1}^m a_{ij} w_i$ defined on $B$.
    Then this is certainly a linear map which extends uniquely by linearity to $A$, giving $[\alpha]_{B,C} = (a_{ij}) = A$\footnote{Proving this left as an exercise}.

    Now, $\theta$ is injective since $[\alpha]_{B,C} = 0 \implies \alpha = 0$.
\end{proof}
\begin{remark}
    If $B,C$ are bases of $V,W$ respectively, and $\varepsilon_B \colon V \to F^n$ is defined by $v \mapsto [v]_B$, and analogously for $\varepsilon_C$, then the following diagram \color{blue}commutes\color{black}
    % https://q.uiver.app/?q=WzAsNCxbMCwwLCJWIl0sWzIsMCwiVyJdLFsyLDIsIlxcbWF0aGJie0Z9Xm0iXSxbMCwyLCJcXG1hdGhiYntGfV5uIl0sWzAsMSwiXFxhbHBoYSJdLFsxLDIsIlxcdmFyZXBzaWxvbl9DIl0sWzMsMiwiW1xcYWxwaGFdX3tCLEN9IiwyXSxbMCwzLCJcXHZhcmVwc2lsb25fQiIsMl1d
    \[\begin{tikzcd}
        V && W \\
        \\
        {\mathbb{F}^n} && {\mathbb{F}^m}
        \arrow["\alpha", from=1-1, to=1-3]
        \arrow["{\varepsilon_C}", from=1-3, to=3-3]
        \arrow["{[\alpha]_{B,C}}"', from=3-1, to=3-3]
        \arrow["{\varepsilon_B}"', from=1-1, to=3-1]
    \end{tikzcd}\]

    We can see that
    \begin{align*}
        [\alpha]_{B,C} \circ \varepsilon_B = \varepsilon_C \circ \alpha
    \end{align*}
    so the operations commute.
\end{remark}
\begin{example}
    Let $\alpha \colon V \to W$ be a linear map and $Y \leq V$, where $V, W$ are finite-dimensional.
    Then let $\alpha(Y) = Z \leq W$.
    Consider a basis $B$ of $V$, such that $B' = (v_1, \dots, v_k)$ is a basis of $Y$ completed by $B'' = (v_{k+1}, \dots, v_n)$ into $B = B' \cup B''$.
    Then let $C$ be a basis of W, such that $C' = (w_1, \dots, w_\ell)$ is a basis of $Z$ completed by $C'' = (w_{\ell + 1}, \dots, w_m)$ into $C = C' \cup C''$.
    Then
    \begin{align*}
        [\alpha]_{B,C} = \begin{pmatrix}
            \alpha(v_1) & \dots & \alpha(v_k) & \alpha(v_{k+1}) & \dots & \alpha(v_n)
        \end{pmatrix}
    \end{align*}
    For $1 \leq i \leq k$, $\alpha(v_i) \in Z$ since $v_i \in Y, \alpha(Y) = Z$.
    So the matrix has an upper-left $\ell \times k$ block $A$ which is $\alpha \colon Y \to Z$ on the basis $B', C'$.
    We can show further that $\alpha$ induces a map $\overline{\alpha} \colon V / Y \to W / Z$ by $v + Y \mapsto \alpha(v) + Z$.
    This is well-defined; $v_1 + Y = v_2 + Y$ implies $v_1 - v_2 \in Y$ hence $\alpha(v_1 - v_2) \in Z$ as required.
    The bottom-right block is $[\overline{\alpha}]_{B'', C''}$.
\end{example}

\subsection{Change of basis}
Suppose we have two bases $B = \qty{v_1, \dots, v_n}, B' = \qty{v_1', \dots, v_n'}$ of $V$ and corresponding $C, C'$ for $W$.
If we have a linear map $[\alpha]_{B,C}$, we are interested in finding the components of this linear map in another basis, that is,
\begin{align*}
    [\alpha]_{B,C} \mapsto [\alpha]_{B',C'}
\end{align*}
\begin{definition}[Change of basis matrix]
    The \vocab{change of basis} matrix $P$ from $B'$ to $B$ is
    \begin{align*}
        P = \begin{pmatrix}
            [v_1']_B & \cdots & [v_n']_B
        \end{pmatrix}
    \end{align*}
    which is the identity map in $B'$, written
    \begin{align*}
        P = [I]_{B', B}
    \end{align*}
\end{definition}
\begin{lemma}
    For a vector $v$,
    \begin{align*}
        [v]_B = P [v]_{B'}
    \end{align*}
\end{lemma}
\begin{proof}
    We have
    \begin{align*}
        [\alpha(v)]_C = [\alpha]_{B,C} \cdot [v]_C
    \end{align*}
    Since $P = [I]_{B', B}$,
    \begin{align*}
        [I(v)]_B = [I]_{B', B} \cdot [v]_{B'} \implies [v]_B = P[v]_{B'}
    \end{align*}
    as required.
\end{proof}
\begin{remark}
    $P$ is an invertible $n \times n$ square matrix.
    In particular,
    \begin{align*}
        P\inv = [I]_{B,B'}
    \end{align*}
    Indeed,
    \begin{align*}
        [\alpha \circ \beta]_{A, C} &= [\alpha]_{B, C} [\beta]_{A, B} \\
        \implies I_n &= [I \cdot I]_{B,B} = [I]_{B',B} \cdot [I]_{B,B'}
    \end{align*}
    where $I_n$ is the $n \times n$ identity matrix.
\end{remark}

\begin{warning} ~\vspace*{-1.5\baselineskip}
    \begin{align*}
        P &= ([v_1']_B, \dots, [v_n']_B) \\
        \implies [v]_B &= P [v]_{B'} \\
        \implies [v]_{B'} &= \color{red}P\inv\color{black} [v]_B
    \end{align*} 
\end{warning} 
\begin{proposition}
    If $\alpha$ is a linear map from $V$ to $W$, and $P = [I]_{B',B}, Q = [I]_{C',C}$\footnote{$P, Q$ invertible.}, we have
    \begin{align*}
        A' = [\alpha]_{B',C'} = [I]_{C,C'}[\alpha]_{B,C}[I]_{B,'B} = Q\inv AP
    \end{align*}
    where $A = [\alpha]_{B,C}, A' = [\alpha]_{B',C'}$.
\end{proposition}
\begin{proof}
    \begin{align*}
        [\alpha(v)]_C & = Q [\alpha(v)]_{C'} \\
        &= Q [\alpha]_{B',C'} [v]_{B'} \\
        [\alpha(v)]_C &= [\alpha]_{B,C} [v]_B \\
        &= AP[v]_{B'} \\
        \therefore \forall v,\ QA'[v]_{B'} &= AP[v]_{B'} \\
        \therefore QA' &= AP
    \end{align*}
    as required.
\end{proof}

\subsection{Equivalent matrices}
\begin{definition}[Equivalent matrices]
    Matrices $A, A' \in M_{m, n}(F)$ are called \vocab{equivalent} if
    \begin{align*}
        A' = Q\inv AP
    \end{align*}
    for some invertible $m \times m, n \times n$ matrices $Q, P$.
\end{definition}
\begin{remark}
    This defines an equivalence relation on $M_{m,n}(F)$.
    \begin{itemize}
        \item $A = I_m\inv A I_n$;
        \item $A' = Q\inv AP \implies A = Q A' P\inv$;
        \item $A' = Q\inv AP, A'' = (Q')\inv A'P' \implies A'' = (QQ')\inv A(PP')$.
    \end{itemize}
\end{remark}
\begin{proposition}
    Let $V, W$ be vector spaces over $F$ with $\dim_F V = n$, $\dim_F W = m$.
    Let $\alpha \colon V \to W$ be a linear map.
    Then there exists a basis $B$ of $V$ and a basis $C$ of $W$ such that
    \begin{align*}
        [\alpha]_{B,C} = \begin{pmatrix}
            I_r & 0 \\
            0   & 0
        \end{pmatrix}
    \end{align*}
    so the components of the matrix are exactly the identity matrix of size $r$ in the top-left corner, and zeroes everywhere else.
\end{proposition}
\begin{proof}
    We first fix $r \in \mathbb N$ such that $\dim \ker \alpha = n - r$.
    Then we will construct a basis $\qty{v_{r+1}, \dots, v_n}$ of the kernel.
    We extend this to a basis of the entirety of $V$, that is, $\qty{v_1,\dots,v_n}$.
    Then, we want to show that
    \begin{align*}
        \qty{\alpha(v_1), \dots, \alpha(v_r)}
    \end{align*}
    is a basis of $\Im \alpha$.
    Indeed, it is a generating family:
    \begin{align*}
        v &= \sum_{i=1}^n \lambda_i v_i         \\
        \alpha(v) &= \sum_{i=1}^n \lambda_i \alpha(v_i) \\
        &= \sum_{i=1}^r \lambda_i \alpha(v_i) \text{ as $v_{r + i} \in \ker \alpha$}
    \end{align*}
    Then if $y \in \Im \alpha$, there exists $v$ such that $\alpha(v) = y$.
    So \begin{align*}
        y &= \sum_{i=1}^{r}  \lambda_i \alpha(v_i) \in \genset{\alpha(v_1), \dots, \alpha(v_r)}.
    \end{align*} 

    Further, it is a free family:
    \begin{align*}
        \sum_{i=1}^r \lambda_i \alpha(v_i) &= 0 \\
        \alpha\qty(\sum_{i=1}^r \lambda_i v_i) &= 0 \\
        \sum_{i=1}^r \lambda_i v_i & \in \ker \alpha \\
        \sum_{i=1}^r \lambda_i v_i &= \sum_{i=r+1}^n \lambda_i v_i \text{ as $v_{r + i}$ is a basis of $\ker \alpha$.}\\
        \sum_{i=1}^r \lambda_i v_i - \sum_{i=r+1}^n \lambda_i v_i &= 0
    \end{align*}
    But since $\qty{v_1, \dots, v_n}$ is a basis, $\lambda_i = 0$ for all $i$.

    Hence $\qty{\alpha(v_1), \dots, \alpha(v_r)}$ is a basis of $\Im \alpha$.
    Now, we extend this basis to the whole of $W$ to form
    \begin{align*}
        \qty{\alpha(v_1), \dots, \alpha(v_r), w_{r+1}, \dots, w_n}
    \end{align*}
    Now,
    \begin{align*}
        [\alpha]_{BC} & = \begin{pmatrix}
            \alpha(v_1) & \cdots & \alpha(v_r) & \alpha(v_{r+1}) & \cdots & \alpha(v_n)
        \end{pmatrix} \\
                    & = \begin{pmatrix}
            I_r & 0 \\
            0   & 0
        \end{pmatrix}
    \end{align*}
\end{proof}
\begin{remark}
    This also proves the rank-nullity theorem:
    \begin{align*}
        \rank \alpha + \operatorname{null} \alpha = n
    \end{align*}
\end{remark}
\begin{corollary} \label{cor:equivalence}
    Any $m \times n$ matrix $A$ is equivalent to a matrix of the form
    \begin{align*}
        \begin{pmatrix}
            I_r & 0 \\
            0   & 0
        \end{pmatrix}
    \end{align*}
    where $r = \rank A$.
\end{corollary}

\subsection{Column rank and row rank}
\begin{definition}[Column rank]
    Let $A\footnote{$A = (c_1 \mid \dots \mid c_n),\ c_n \in F^m.$} \in M_{m,n}(F)$.
    Then, the \vocab{column rank} of $A$, here denoted $r_c(A)$, is the dimension of the subspace of $F^n$ spanned by the column vectors.
    \begin{align*}
        r_c(A) = \dim \vecspan \qty{c_1, \dots, c_n}
    \end{align*}
\end{definition}

\begin{definition}[Row rank]
    The \vocab{row rank} is the column rank of $A^\transpose$.
\end{definition} 

\begin{remark}
    If $\alpha$ is a linear map, represented by $A$ with respect to some basis, then:
    \begin{align*}
        \rank \alpha = r_c(A) = \dim \Im \alpha
    \end{align*}
\end{remark}
\begin{proof}
    Proof of $\rank \alpha = r_c(A)$ is left as an exercise.
\end{proof} 

\begin{proposition}
    Two matrices are equivalent if they have the same column rank:
    \begin{align*}
        r_c(A) = r_c(A').
    \end{align*}
\end{proposition}
\begin{proof}
    $(\implies)$ If the matrices are equivalent, then they correspond to the same linear map $\alpha$ in two different basis
    \begin{align*}
        r_c(A) &= \rank \alpha \\
        r_c(A') &= \rank \alpha \\
        \implies r_c(A) &= r_c(A')
    \end{align*} 
    $(\Longleftarrow)$ Conversely, if $r_c(A) = r_c(A') = r$, then $A, A'$ are equivalent to
    \begin{align*}
        \begin{pmatrix}
            I_r & 0 \\
            0   & 0
        \end{pmatrix}
    \end{align*}
    By transitivity, $A, A'$ are equivalent.
\end{proof}
\begin{theorem}
    Column rank $r_c(A)$ and row rank $r_c(A^\transpose)$ are equivalent.
\end{theorem}
\begin{proof}
    Let $r = r_c(A)$.
    Then,
    \begin{align*}
        Q\inv AP = \begin{pmatrix}
            I_r & 0 \\
            0   & 0
        \end{pmatrix}_{m \times n}
    \end{align*}
    Then take the transpose:
    \begin{align*}
        (Q\inv AP)^\transpose &= P^\transpose A^\transpose \qty(Q\inv)^\transpose \\
        &= P^\transpose A^\transpose \qty(Q^\transpose)\inv \\
        &= \begin{pmatrix}
            I_r & 0 \\
            0   & 0
        \end{pmatrix}_{m \times n}^\transpose = \begin{pmatrix}
            I_r & 0 \\
            0   & 0
        \end{pmatrix}_{n \times m}
    \end{align*}
    Then $r_c(A^\transpose) = r = r_c(A)$.

    \begin{note}
        We can swap the transpose and inverse on $Q$ because
        \begin{align*}
            (AB)^\transpose &= B^\transpose A^\transpose           \\
            \qty(QQ\inv)^\transpose &= \qty(Q\inv)^\transpose Q^\transpose \\
            I &= \qty(Q\inv)^\transpose Q^\transpose \\
            \qty(Q^\transpose)\inv &= \qty(Q\inv)^\transpose
        \end{align*}
    \end{note} 
\end{proof}
So we can drop the concepts of column and row rank, and just talk about rank as a whole.

\subsection{Conjugation and similarity}
Consider the following special case of changing basis.
\begin{definition}
    If $\alpha \colon V \to V$ is linear, $\alpha$ is called an \vocab{endomorphism}.
\end{definition} 
If $B = C, B' = C'$ then the special case of the change of basis formula is
\begin{align*}
    [\alpha]_{B',B'} = P\inv [\alpha]_{B,B} P
\end{align*}

\begin{definition}[Similar matrices]
    Let $A, A'$ be $n \times n$ (square) matrices.
    We say that $A$ and $A'$ are \vocab{similar} or \vocab{conjugate} iff there exists $P$ ($n \times n$ square invertible matrix) such that $A' = P\inv A P$.
\end{definition} 
This is a central concept when we will study diagonalisation of matrices, Spectral theory.

\subsection{Elementary operations}
\begin{definition}[Elementary column operation]
    An \vocab{elementary column operation} is
    \begin{enumerate}
        \item swap columns $i, j$ ($i \neq j$)
        \item replace column $i$ by $\lambda$ multiplied by the column ($\lambda \neq 0, \lambda \in F$)
        \item add $\lambda$ multiplied by column $i$ to column $j$ ($i \neq j$)
    \end{enumerate}
\end{definition}
We define analogously the elementary row operations.
Note that these elementary operations are invertible (for $\lambda \neq 0$).
These operations can be realised through the action of \underline{elementary matrices}.
For instance, the column swap operation can be realised using
\begin{align*}
    T_{ij} = \begin{pmatrix}
        I_{i-1} & 0 & 0   \\
        0   & A & 0   \\
        0   & 0 & I
    \end{pmatrix};\quad A = \begin{pmatrix}
        0 & 0   & 1 \\
        0 & I & 0 \\
        1 & 0   & 1
    \end{pmatrix}
\end{align*}
To multiply a column by $\lambda$,
\begin{align*}
    n_{i,\lambda} = \begin{pmatrix}
        I_{i-1} & 0       & 0   \\
        0   & \lambda & 0   \\
        0   & 0       & I
    \end{pmatrix}
\end{align*}
To add a multiple of a column,
\begin{align*}
    c_{ij,\lambda} = I + \lambda E_{ij}
\end{align*}
where $E_{ij}$ is the matrix defined by elements $(e_{ij})_{pq} = \delta_{ip} \delta_{jq}$.

An elementary column (or row) operation can be performed by multiplying $A$ by the corresponding elementary matrix from the right (on the left for row operations).
\begin{proof}
    Left as an exercise.
\end{proof} 

\begin{example} ~\vspace*{-1.5\baselineskip}
    \begin{align*}
        \begin{pmatrix}1 & 2 \\3 & 4\end{pmatrix} \begin{pmatrix}0 & 1 \\1 & 0\end{pmatrix} = \begin{pmatrix}2 & 1 \\4 & 3\end{pmatrix}.
    \end{align*} 
\end{example} 

We can prove \cref{cor:equivalence} constructively:
\begin{proof}
    This will essentially provide a constructive proof that any $m \times n$ matrix is equivalent to
    \begin{align*}
        \begin{pmatrix} I_r & 0 \\ 0 & 0 \end{pmatrix}.
    \end{align*}
    We will start with a matrix $A$.
    If all entries are zero, we are done. \\
    So we will pick $a_{ij} = \lambda \neq 0$, and swap rows $i,1$ and columns $j,1$.
    This ensures that $a_{11} = \lambda \neq 0$. \\
    Now we multiply column 1 by $\frac{1}{\lambda}$ so $a_{11} = 1$ now. \\
    Finally, we can clear out row 1 and column 1 by subtracting multiples of rows or columns (3rd elementary operation).
    Then we can perform similar operations on the $(m-1)\times(n-1)$ matrix in the bottom right block and inductively finish this process.
    We end up with:
    \begin{align*}
        \begin{pmatrix} I_r & 0 \\ 0 & 0 \end{pmatrix} &= \underbrace{E_p' \dots E_1'}_\text{row operations} A \underbrace{E_1 \dots E_c}_{\text{column operations}} \\
        &= Q\inv A P
    \end{align*} 
\end{proof} 

\subsection{Gauss' pivot algorithm}
If \underline{only} row operations are used, we can reach the \vocab{row echelon form} of the matrix, a specific case of an upper triangular matrix.
\begin{align*}
    \begin{pmatrix}0 & \dots & 0 & 1 & \dots & \dots \\ 0 & \dots & \dots & \dots & 1 \dots \\ \vdots &  &  &  & & \vdots \\ 0 & \dots & \dots & \dots & \dots & 0 \end{pmatrix}
\end{align*} 
On each row, there are a number of zeroes until there is a one, called the pivot.

First, we assume that $a_{ij} \neq 0$. \\
We swap rows $i, 1$. \\
Then divide the first row by $\lambda = a_{i1}$ to get a one in the top left. \\
We can use this one to clear the rest of the first column. \\
Then, we can repeat on the next column, and iterate.

This is a technique for solving a linear system of equations.

\subsection{Representation of square invertible matrices}
\begin{lemma}
    If $A$ is an $n \times n$ \underline{square invertible} matrix, then we can obtain $I_n$ using only row elementary operations, or only column elementary operations.
\end{lemma}

\begin{proof}
    We show an algorithm that constructs this $I_n$.
    This is exactly going to invert the matrix, since the resultant operations can be combined to get the inverse matrix.
    We will show here the proof for column operations.

    We argue by \underline{induction} on the number of rows. \\
    Suppose we can make the form
    \begin{align*}
        \begin{pmatrix} I_k & 0 \\ A & B \end{pmatrix}
    \end{align*}
    We want to obtain the same structure with $k+1$ rows.

    We claim that there exists $j > k$ such that $a_{k+1,j} \neq 0$.
    Indeed, otherwise we can show that the vector
    \begin{align*}
        \begin{pmatrix} 0 \\ \vdots \\ 1 \\ \vdots \\ 0 \end{pmatrix} = \delta_{k+1,i}
    \end{align*}
    is not in the span of the column vectors of $A$.\footnote{Left as an exercise to check this.}
    This contradicts the invertibility of the matrix.

    Now, we will swap columns $k+1, j$ and divide this column by $\lambda$.
    We can now use this 1 to clear the rest of the $k+1$ row using elementary operations of type 3.

    The desired results follows from induction.
\end{proof}

\begin{remark}
    Inductively, we have found $A E_1 \dots E_c = I_n$ where $E_c$ are elementary.
    Thus, $A\inv = E_1 \dots E_c$ and so this is an algorithm for computing $A\inv$ and so solving linear systems of equations.
\end{remark} 

\begin{proposition}
    Any invertible square matrix is a product of elementary matrices.
\end{proposition}

\begin{proof}
    The proof is exactly the proof of the lemma above.
\end{proof}