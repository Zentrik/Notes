\section{Bilinear Forms}
\subsection{Introduction}
\begin{definition}[Bilinear Forms]
	Let $U, V$ be $F$-vector spaces.
	Then $\phi \colon U \times V \to F$ is a \vocab{bilinear form} if it is `linear in both components'.
	For example, $\phi$ at a fixed $u \in U$ is a linear form $V \to F$ and an element of $V^\star$; and $\phi$ at a fixed $v \in V$ is a linear form $U \to F$ and an element of $U^\star$
\end{definition}

\begin{example}
	Consider the map $V \times V^\star \to F$ given by
	\begin{align*}
		(v, \theta) \mapsto \theta(v).
	\end{align*}
	You can check this is a bilinear map.
\end{example}

\begin{example}[Scalar Product]
	The scalar product on $U = V = \mathbb R^n$ is given by
	\begin{align*}
		\psi: \mathbb{R}^n \times \mathbb{R}^n &\to \mathbb{R} \\
		(x, y) &\mapsto \sum_{i=1}^n x_i y_i
	\end{align*}
	You can check this is a bilinear map.
\end{example}

\begin{example}
	Let $U = V = C([0,1], \mathbb R)$ and consider
	\begin{align*}
		\phi(f,g) = \int_0^1 f(t)g(t) \dd{t}
	\end{align*}
	You can check this is a bilinear map.
\end{example}

\begin{definition}[Matrix of a bilinear form in a basis]
	If $B = (e_1, \dots, e_m)$ is a basis of $U$ and $C = (f_1, \dots, f_n)$ is a basis of $V$, and $\phi \colon U \times V \to F$ is a bilinear form, then the \vocab{matrix of the bilinear form in this basis} is
	\begin{align*}
		[\phi]_{B, C} = \qty( \underbrace{\phi(e_i, f_j)}_{\in F} )_{1 \leq i \leq m, 1 \leq j \leq n}
	\end{align*}
\end{definition}
\begin{lemma}
	We can link $\phi$ with its matrix in a given basis as follows.
	\begin{align*}
		\phi(u,v) = [u]_B^\transpose [\phi]_{B, C} [v]_C
	\end{align*}
\end{lemma}
\begin{proof}
	Let $u = \sum_{i=1}^m \lambda_i e_i$ and $v = \sum_{j=1}^n \mu_j f_j$.
	Then by linearity:
	\begin{align*}
		\phi(u,v) = \phi\qty( \sum_{i=1}^m \lambda_i e_i, \sum_{j=1}^n \mu_j f_j ) = \sum_{i=1}^m \sum_{j=1}^n \lambda_i \mu_j \phi(e_i, f_j) = [u]_B^\transpose [\phi]_{B,C} [v]_C.
	\end{align*}
	Check these equality signs are correct.
\end{proof}
\begin{remark}
	Note that $[\phi]_{B,C}$ is the only matrix such that $\phi(u,v) = [u]_B^\transpose [\phi]_{B, C} [v]_C$.
\end{remark}
\begin{definition}
	Let $\phi \colon U \times V \to F$ be a bilinear form.
	Then $\phi$ induces two linear maps given by the partial application of a single parameter to the function.
	\begin{align*}
		\phi_L \colon U \to V^\star;\quad \phi_L(u) \colon V \to F;\quad v \mapsto \phi(u,v)
	\end{align*}
	\begin{align*}
		\phi_R \colon V \to U^\star;\quad \phi_R(v) \colon U \to F;\quad u \mapsto \phi(u,v)
	\end{align*}
	In particular,
	\begin{align*}
		\phi_L(u)(v) = \phi(u,v) = \phi_R(v)(u)
	\end{align*}
\end{definition}
\begin{lemma}
	Let $B = (e_1, \dots, e_m)$ be a basis of $U$, and let $B^\star = (\varepsilon_1, \dots, \varepsilon_m)$ be its dual; and let $C = (f_1, \dots, f_n)$ be a basis of $V$, and let $C^\star = (\eta_1, \dots, \eta_n)$ be its dual.
	Let $A = [\phi]_{B,C}$.
	Then
	\begin{align*}
		[\phi_R]_{C, B^\star} = A;\quad [\phi_L]_{B, C^\star} = A^\transpose
	\end{align*}
\end{lemma}
\begin{proof}
	\begin{align*}
		\phi_L(e_i)(f_j) = \phi(e_i, f_j) = A_{ij}
	\end{align*}
	Since $\eta_j$ is the dual of $f_j$,
	\begin{align*}
		\phi_L(e_i) = \sum_i A_{ij} \eta_j
	\end{align*}
	Further,
	\begin{align*}
		\phi_R(f_j)(e_i) = \phi(e_i, f_j) = A_{ij}
	\end{align*}
	and then similarly
	\begin{align*}
		\phi_R(f_j) = \sum_i A_{ij} \varepsilon_i
	\end{align*}
\end{proof}

\begin{definition}[Left/ Right Kernel]
	$\ker \phi_L$ is called the \vocab{left kernel} of $\phi$.
	$\ker \phi_R$ is the \vocab{right kernel} of $\phi$.
\end{definition}

\begin{definition}[Degenerate/ Non-Degenerate Bilinear Form]
	We say that $\phi$ is \vocab{non-degenerate} if $\ker \phi_L = \ker \phi_R = \qty{0}$.
	Otherwise, $\phi$ is \vocab{degenerate}.
\end{definition}

\begin{lemma}
	Let $B$ be a basis of $U$, and let $C$ be a basis of $V$, where $U, V$ are finite-dimensional.
	Let $\phi \colon U \times V \to F$ be a bilinear form.
	Let $A = [\phi]_{B,C}$. \\
	Then, $\phi$ is non-degenerate if and only if $A$ is invertible.
\end{lemma}

\begin{corollary}
	If $\phi$ is non-degenerate, then $\dim U = \dim V$.
\end{corollary}
\begin{proof}
	Suppose $\phi$ is non-degenerate.
	Then $\ker \phi_L = \ker \phi_R = \qty{0}$.
	This is equivalent to saying that $n(\phi_L) = n(\phi_R) = 0$.
	We can use the rank-nullity theorem to state that $r(A^\transpose) = \dim U$ and $r(A) = \dim V$.
	This is equivalent to saying that $A$ is invertible.
	Note that this forces $\dim U = \dim V$ as $r(A^\transpose) = r(A)$.
\end{proof}
\begin{remark}
	The canonical example of a non-degenerate bilinear form is the scalar product $\mathbb R^n \times \mathbb R^n \to \mathbb R$ represented by the identity matrix in the standard basis\footnote{$[\phi]_{B, B} = I$ where $B$ the standard bases as $\phi(e_i, e_j) = \delta_{ij}$}.
\end{remark}

\begin{corollary}
	If $U$ and $V$ are finite-dimensional with $\dim U = \dim V$, then choosing a non-degenerate bilinear form $\phi \colon U \times V \to F$ is equivalent to choosing an isomorphism $\phi_L \colon U \to V^\star$.
\end{corollary}

\begin{definition}[Orthogonals]
	If $T \subset U$, then we define
	\begin{align*}
		T^\perp = \qty{ v \in V \colon \forall t \in T, \phi(t,v) = 0 }\footnote{$\phi : (U, V) \to F$.}
	\end{align*}
	Further, if $S \subset V$, we define
	\begin{align*}
		^\perp S = \qty{ u \in U \colon \forall s \in S, \phi(u,s) = 0 }
	\end{align*}
	These are called the \vocab{orthogonals} of $T$ and $S$.
\end{definition}

\subsection{Change of basis for bilinear forms}
\begin{proposition}[Change of basis for bilinear forms]
	Let $B, B'$ be bases of $U$ and $P = [I]_{B', B}$, let $C, C'$ be bases of $V$ and $Q = [I]_{C', C}$, and finally let $\phi \colon U \times V \to F$ be a bilinear form.
	Then
	\begin{align*}
		[\phi]_{B', C'} = P^\transpose [\phi]_{B,C} Q
	\end{align*}
\end{proposition}
\begin{proof}
	We have $\phi(u,v) = [u]_B^\transpose [\phi]_{B,C} [v]_C$.
	Changing coordinates, we have
	\begin{align*}
		\phi(u,v) = (P [u]_{B'})^\transpose [\phi]_{B,C} (Q [v]_{C'}) = [u]_{B'}^\transpose (P^\transpose [\phi]_{B,C} Q) [v]_{C'}\footnote{There is only one matrix $A$ s.t. $\phi(u, v) = [u]_{B'}^\transpose A [v]_{C'}$, see earlier remark.}
	\end{align*}
\end{proof}

\begin{lemma}
	The \vocab{rank} of a bilinear form $\phi$, denoted $r(\phi)$ is the rank of any matrix representing $\phi$.
	This quantity is well-defined.
\end{lemma}

\begin{proof}
	For any invertible matrices $P, Q$, $r(P^\transpose A Q) = r(A)$.
\end{proof}

\begin{remark}
	$r(\phi) = r(\phi_R) = r(\phi_L)$, since $r(A) = r(A^\transpose)$.
\end{remark}

We will see more applications later in the course, especially when we see scalar products.