\section{Introduction}
Probability theory is the mathematical formulation of randomness.
Examples include the modelling of random experiments like flipping a coin, throwing a die, shuffle a deck, and so on.
What we want to do is to develop a mathematical framework to study randomness.

\begin{example} \label{exm:0}
    Dice: outcomes $1, 2, \dots, 6$.
    \begin{itemize}
        \item $\mathbb{P}(2) = \frac{1}{6}.$
        \item $\mathbb{P}(\text{multiple of }3) = \frac{2}{6} = \frac{1}{3}.$
        \item $\mathbb{P}(\text{not a multiple of }3) = \frac{2}{3}$
        \item $\mathbb{P}(\text{prime}) = \frac{1}{2}.$
        \item \begin{align*}
            \mathbb{P}(\text{prime or multiple of 3}) &= \Ccancel[red]{\frac{1}{3} + \frac{1}{2} = \frac{5}{6}.} \\
            &= \frac{4}{6} = \frac{2}{3}. \\
            \mathbb{P}(\text{prime or multiple of 3}) &= \frac{1}{3} + \frac{1}{2} - \frac{1}{6} = \frac{2}{3}
        \end{align*} 
    \end{itemize} 
\end{example}