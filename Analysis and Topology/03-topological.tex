\section{Topological Spaces}

`Do continuity entirely in terms of open sets without mentioning distance'.

Metric space: set with a distance.

Topological space: set with a collection of open subsets.

\subsection{Definitions and Examples}

\begin{definition}[Topological Space]
    A \vocab{topological space} is a set $X$ endowed with a \vocab{topology} $\tau$ that is a subset $\tau \subset \mathcal{P}(X)$ satisfying
    \begin{enumerate}
        \item $\emptyset \in \tau$ and $X \in \tau$;
        \item If $\sigma \subset \tau$ then $\bigcup_{G \in \sigma} G = \tau$;
        \item If $G_1, \dots, G_n \in \tau$ then $\bigcap_{i = 1}^n G_i \in \tau$.
    \end{enumerate} 
\end{definition} 

\begin{remark}
    Could replace (3) by $G, H \in \tau \implies G \cap H \in \tau$.
    Equivalent to (3) by induction.
\end{remark} 

\begin{notation}
    Sometimes write `$(X, \tau)$ is a topological space'.
    If obvious what the topology is, we might just write `$X$ is a topological space'.
\end{notation} 

\begin{example}
    Let $(X, d)$ be a metric space.
    Let $\tau = \{G \subset X : G \text{ open}\}$.
    Then by \cref{prp:26}, $\tau$ is a topology on $X$.

    We say $\tau$ is the \underline{topology, induced by the metric $d$}.
\end{example} 

We want to define open/ closed/ cts etc for topological spaces.
As metric spaces are top. spaces try to make sure it's `backward compatible', so to say.
The new defns shouldn't contradict the old metric space ones.
So in making new defns, we'll be guided by section 2.4.

\begin{definition}
    Let $(X,\tau)$ be a topological space. We say that $G\subset X$ is \emph{open} if $G\in \tau$. We say $F$ is \emph{closed} if $X \backslash F \in\tau$. We say $\mathcal{N} \subset X$ is a \emph{neighbourhood} of $a\in X$ if $\exists G\subset X$ open with $a\in G\subset \mathcal{N}$.
    
    Let $(Y,\sigma)$ be another topological space, and let $f:X\rightarrow Y$. We say $f$ is \emph{continuous} if whenever $G\subset Y$ is open, then $f^{-1}(G)\subset X$ is open. In other words, $f$ is continuous if $\forall G\in \sigma, f^{-1}(G)\subset \tau$.
    
    We say that $f$ is \emph{continuous at $a \in X$} if whenever $\mathcal{N} \subset Y$ is a neighbourhood of $f(a)$ then $f^{-1}(\mathcal{N})\subset X$ is a neighbourhood of $a$.
    
    We say that $f$ is a \emph{homeomorphism} and $X,Y$ are \emph{homeomorphic} if $f$ is a bijection and both $f,f^{-1}$ are continuous. 
    
    A property is \emph{topological} if it is preserved by homeomorphisms; that is to say, if $X,Y$ are homeomorphic then $X$ has the property iff $Y$ does.
\end{definition}

\begin{remark}
1. if $\tau$ is induced by a metric then this is all consistent with the metric space definitions of these concepts.

2. Given our definition, $G$ open iff $G\in \tau$, often don't need to explicitly name the topology. E.g. `Let $X=\R$ with the usual topology and $G\subset X$ be open\dots'. Other times more convenient to specify $\tau$, write `$G\in \tau$' etc.

3. Homeomorphism is an equivalence `relation'.

4. If $a\in G$ and $G$ open then $G$ is a neighbourhood of $a$, however neighbourhoods need not be open in general. A set $G\subset X$ is open iff $G$ is a neighbourhood of each of its points.
\end{remark}

\begin{proposition}             %3.1
Let $X,Y$ be topological spaces and $f: X\rightarrow Y$. Then $f$ is continuous if and only if for all $a\in X$, $f$ is continuous at $a$.
\end{proposition}      

\begin{proof}
Forward: Suppose $f$ continuous and let $a\in X$. Let $\mathcal{N}\subset Y$ be a neighbourhood of $f(a)$. Then there is an open set $G\subset Y$ with $a\in G\subset \mathcal{N}$. As $f$ is continuous, $f^{-1}(G)\subset X$ is open. Now $a\in f^{-1}(a)\subset f^{-1}(\mathcal{N})$ with $f^{-1}(G)$ open. Thus $f$ is continuous at $a$.

Converse: Suppose for all $a\in X$ we have $f$ continuous at $a$. Let $G\subset Y$ be open. Let $a\in f^{-1}(G).$ Then $f(a)\in G$, but $G$ is open so $G$ is a neighbourhood of $f(a)$. Now $f$ is continuous at $a$ so $f^{-1}(a)$ is a neighbourhood of $a$ in $X$. But $a$ was arbitrary so $f^{-1}(G)$ is a neighbourhood of each of its points, i.e. it is open. Hence $f$ is continuous.
\end{proof}

\begin{proposition}
Let $(X,\tau), (Y,\sigma), (Z,\rho)$ be topological spaces, let $f:X\rightarrow Y$ and $g:Y\rightarrow Z$ be continuous. Then $g\circ f: X\rightarrow Z$ is continuous.
\end{proposition}

\begin{proof}
Let $G\in \rho$. As $g$ is continuous, $g^{-1}(G)\in \sigma$. As $f$ continuous, $f^{-1}(g^{-1}(G))\in \tau$. That is, $(g\circ f)^{-1}(G)\in \tau$. So $g\circ f$ continuous.
\end{proof}

\begin{example}[Some examples of topologies]
1. \emph{The discrete topology.} Let $X$ be any set and $\tau = \mathcal{P}(X)$, where `every set is open'. This is not a new example; it is induced by the discrete metric $d(x,y) = 1$ iff $x=y$. Now in $(X,d)$, for any $x\in X$ then $\{x\} = B_1(x)$ is open and so if $G\subset X$ then $G = \bigcup_{x\in G} \{x\}$ is open.

2. \emph{The indiscrete topology.} Let $X$ be any set and $\tau = \{\emptyset, X\}$. Here, `only open sets are $\emptyset$ and whole space'. This is something we have not seen before: $\tau$ cannot be induced by a metric as long as $|X|\ge 2$. Indeed, suppose $|X|\ge 2$ and that $\tau $ is induced by a metric $d$. Let $x,y\in X$ with $x\neq y$, so let $d(x,y) = \delta>0$. Then $B_\delta(x)$ is open with $x\in B_\delta(x)$ and $y\not\in B_\delta(x)$.

3. \emph{The cofinite topology.} Let $X$ be any infinite set and let $\tau = \{G\subset X : X \backslash G \textup{ is finite} \} \cup \{\emptyset\}$. We will check that this is a topology, and not induced by any metric $d$. 

4. \emph{The cocountable topology.} Let $X$ be any uncountable set and let $\tau = \{G\subset X : X\backslash G \textup{ countable} \} \cup \{\emptyset\}$. Then, very similarly to (3), we can show that this is a topology that is not induced by any metric.
\end{example}

\begin{proof}
\begin{enumerate}
    \item $\emptyset\in\tau$ and $X\backslash X = \emptyset$ is finite so $X\in\tau$.
    \item Let $\sigma \subset\tau$. If $\sigma$ is empty or only contains $\emptyset$ then $\bigcup_{G\in \sigma} G = \emptyset\in \tau$. 
    Otherwise, pick $H\in \sigma$ with $H\neq \emptyset$. Then $X\backslash H$ is finite so $(X\backslash\bigcup_{G\in \sigma} G) = \bigcup_{G\in \sigma} (X\backslash G)\subset X\backslash H$ is finite. So $\bigcup_{G\in \sigma} G \in \tau.$
    \item Let $G,H\in \tau$. If $G = \emptyset$ or $H=\emptyset$ then $G\cap H = \emptyset\in\tau$. Otherwise $X\backslash G, X\backslash H$ are finite then $(X\backslash(G\cap H)) = (X\backslash G)\cup(X\backslash H)$ is finite. So $G\cap H\in \tau$.
\end{enumerate}

Furthermore, it is not induced by a metric $d$. Observe first that if $G,H$ open and non-empty then $G\cap H\neq \emptyset$. Then $d(x,y) = \delta > 0$ so $B_\delta/2(x), B_\delta/2(y)$ are non-empty disjoint open sets. So $d$ doesn't induce $\tau$.
\end{proof}

\subsection{Sequences and Hausdorff spaces}

\begin{definition}
Let $X$ be a topological space, let $(x_n)$ be a sequence in $X$ and let $x\in X$. We say $(x_n)$ \emph{converges} to $x$ and write $x_n\rightarrow x$ if whenever $\mathcal{N}\subset X$ is a neighbourhood of $x$ then $\exists N$, $\forall n \geq N$ we have $x_n \in \mathcal{N}$.
\end{definition}

\begin{question}
Let $X$ be an uncountable set with the cocountable topology. What sequences converge in $X$?
\end{question}

\begin{proof}
Suppose $x_n\rightarrow x$. Then let $\mathcal{N} = (X\backslash \{x_n : n\in \mathbb{N}\})\cup\{x\}$. Then $\mathcal{N}$ open and $x\in \mathcal{N}$ so $\mathcal{N}$ is a neighbourhood of $x$. Thus $\exists N$ such that $\forall n\ge N$, $x_n\in \mathcal{N}$. So $\exists N$, $\forall n\ge N$ we have $x_n=x$.

Conversely, it is obvious if $\exists N$ such that $\forall n\ge N$, $x_n=x$ then $x_n\rightarrow x$. So the only convergent sequences in this space are eventually constant.
\end{proof}

\begin{example}
2. Let $X = \{1,2,3\}$ with the indiscrete topology. Let $x_n =i\in X$ with $i\equiv n \mod 3$. So sequence is 1,2,3,1,2,3,$\dots$.

\begin{claim}
    $x_n\rightarrow 2$
\end{claim} 
\begin{proof}
Let $\mathcal{N}$ be a neighbourhood of $2$. Then $\exists G$ open such that $2\in G\subset \mathcal{N}$. But only open sets are $\emptyset, \{1,2,3\}$ so $G = \{1,2,3\}$. So $\mathcal{N} = \{1,2,3\}$ so $\forall n, x_n\in\mathcal{N}$. 
\end{proof}
Similarly $x_n\rightarrow 1$ and $x_n\rightarrow 3$, and so we arrive at the big revelation for topological spaces: \textbf{\color{red} LIMITS OF CONVERGENT SEQUENCES NEED NOT BE UNIQUE.} Thus we can write $\lim_{n\rightarrow \infty} x_n$ anymore.
\end{example}

\begin{remark}
The above proof shows that in any indiscrete space, every sequences converges to every point of the space.
\end{remark}

\begin{definition}
A topological space $X$ is \emph{Hausdorff} if whenever $x,y\in X$ with $x\neq y$ then there are disjoint open $G,H\subset X$ with $x\in G$ and $y\in H$.
\end{definition}

\begin{example}
1. Metric spaces are Hausdorff. Indeed, if $(X,d)$ metric and $x,y\in X$, $x\neq y$, let $\delta = d(x,y) >0$ and take $G = B_{\delta/2}(x)$ and $H=B_{\delta/2}(y)$.

2. Indiscrete spaces are not Hausdorff (assuming $|X|\ge 2$).

3. The cofinite topology is not Hausdorff. Let $X$ be an infinite set with the cofinite topology and let $x,y\in X$ with $x\neq y$. Let $G,H\subset X$ be open with $x\in G$, $y\in H$. Clearly $G,H\neq \emptyset$ so $X\backslash G$, $X\backslash H$ finite and so $X\backslash (G\cap H) = (X\backslash G)\cup (X\backslash H)$ is finite. In particular $G\cap H = \emptyset$.

Similarly, the cocountable topology is not Hausdorff.
\end{example}

\begin{proposition}                        %3.3(29)
Limits of convergent sequences in Hausdorff spaces are unique.
\end{proposition}          
\begin{proof}
Let $X$ be Hausdorff, let $a,b\in X$, and let $(x_n)$ be a sequence in $X$ with $x_n\rightarrow a$ and $x_n\rightarrow b$. Suppose $a\neq b$. Take open $G,H$ with $a\in G$, $b\in H$ and $G\cap H = \emptyset$. Now $G$ is a neighbourhood of $a$ so there is some $N_1$ such that $\forall n\ge N_1$ we have $x_n\in G$. Similarly there is some $N_2$ such that $\forall n\ge N_2$, $x_n\in H$. Taking $n = \max\{N_1,n_2\}$, we get $x_n\in G\cap H = \emptyset$, contradiction.
\end{proof}

\begin{proposition}[Relationship to continuity]          %3.4(30)
Let $X,Y$ be topological spaces and let $f:X\rightarrow Y$ be continuous at $a\in X$. Let $(x_n)$ be a sequence in $X$ with $x_n\rightarrow a$. Then $f(x_n)\rightarrow f(a)$.
\end{proposition}
\begin{proof}
Let $\mathcal{N}\subset Y$ be a neighbourhood of $f(a)$. As $f$ continuous we know $f\inv (\mathcal{N})$ is a neighbourhood of $a$. As $x_n\rightarrow a$ we can find $N$ such that $\forall n\ge N$, $x_n\in f\inv(\mathcal{N})$. Then $\forall n\ge N$, $f(x_n)\in \mathcal{N}$. So $f(x_n)\rightarrow f(a)$.
\end{proof}

\begin{example}[Converse is not true in general!]
Let $X=Y=\R$, $X$ with cocountable topology, $Y$ with usual topology and let $f:X\rightarrow Y$ be the identity function.

Suppose $x_n\rightarrow 0$ in $X$. Then for sufficiently large $n$, $x_n=0$ and so for sufficiently large $n$, $f(x_n) = x_n = 0 = f(0)$ so $f(x_n)\rightarrow f(0)$ in $Y$.

However $(-1,1)\subset Y$ is open and $0\in (-1,1)$ so $(-1,1)$ is a neighbourhood of $0\in Y$. But $f\inv((-1,1)) = (-1,1)\in X$ is not a neighbourhood of $0$ in $X$. So $f$ is not continuous at $0$.
\end{example}
\begin{example}[Converse is still not true even after imposing condition that spaces are Hausdorff.]
Take example as above but replace topology on $X$ by \[\sigma = \{G\subset \R : (X\backslash G) \textup{ countable or } 0\not\in G \}. \]
Check that this is a topology, and this is Hausdorff: suppose $x,y\in X$ with $x\neq y$. If $x,y\neq 0$ then $\{x\}, \{y\}\in \sigma$. While if $x=0$ say, then $\R\backslash \{y\}, \{y\}\in \sigma$.

Now, neighbourhoods of $0$ in $\sigma$ are exactly same as in the cocountable topology. So exactly as before, $x_n\rightarrow 0$ in $X\implies x_n\rightarrow 0$ in $Y$ but $f$ is not continuous at $0$.
\end{example}
\begin{remark}
In a metric space, the topology is completely determined by convergence of sequences. Not true for a general topology space. Hence we'll tend to concentrate more on continuity than convergence of sequences.
\end{remark}

\subsection{Subspaces}
\vspace{3mm}
\begin{definition}
Let $(X,\tau)$ be a topological space and let $Y\subset X$. The \emph{subspace topology} on $Y$ is \[ \sigma = \{G\cap Y : G\in \tau \}.\] Easy to check that this is a topology.
\end{definition}

Let's check that this is indeed backward compatible with our definition of metric spaces.

\begin{proposition}                        %3.5(31)
Let $(X,d)$ be a metric space with topology $\tau$ be induced by $d$. Let $Y$ be a subspace of the metric space $X$. Then $Y$ has the subspace topology.
\end{proposition}  
\begin{proof}
Let $\sigma$ be the topology on $Y$ induced by the metric $d\mid_{Y^2}$. 

First suppose $G\in \tau$; we want to check that $G\cap Y$ is open in $Y$. Let $y\in G\cap Y$. As $y\in G$ and $G$ is open in $X$ we can find $\delta >0$ such that $\forall x\in X$, we have $d(x,y)<\delta \implies x\in G$. Then $\forall x\in Y,$ we have $d(x,y)<\delta \implies x \in G\cap Y$. Thus $G\cap Y$ is a neighbourhood of $y$. Since $y$ is arbitrary, so $G\cap Y\in \sigma$.

Conversely, suppose $H\in \sigma$. For each $y\in H$ we can find $\delta_y>0$ such that $\forall x\in Y$, $d(x,y)<\delta_y\implies x\in H$. Now consider the open balls\[B_{\delta_y}(y) = \{x\in X : d(x,y) < \delta_y\}. \] Each $B_{\delta_y}(y)$ is open, so for each $y\in H$, $y\in B_{\delta_y}(y)$ and $B_{\delta_y}(y)\cap Y\subset H$. Now let $G = \bigcup_{y\in H}B_{\delta_y}(y)$. Then $G$ is open and $G\cap Y= H$. That is, we've found $G\in \tau$ such that $G\cap Y = H$.
\end{proof}

\begin{proposition}                            %3.6(32)
A subspace of a Hausdorff space is Hausdorff.
\end{proposition}
\begin{proof}
Let $(X,\tau)$ be Hausdorff, $Y\subset X$, and $\sigma$ be the subspace topology on $Y$. Let $x,y\in Y$ with $x\neq y$. As $X$ is Hausdorff we can find $G,H\in \tau$ with $x\in G, y\in H, G\cap H = \emptyset$. Then $G\cap Y, H\cap Y \in \sigma$ with $x\in G\cap Y$, $y\in H\cap Y$ and $(G\cap Y)\cap (H\cap Y) = \emptyset$.
\end{proof}

\subsection{Compactness}
\vspace{3mm}
\begin{definition}
Let $(X,\tau)$ be a topological space. An \emph{open cover} of $X$ is a subset $\mathcal{C}\subset \tau$ such that $X = \bigcup_{G\in\mathcal{C}} G$. A \emph{subcover} of $\mathcal{C}$ is a subset $\mathcal{D}\subset \mathcal{C}$ which is itself an open cover. We say that $X$ is \emph{compact} if every open cover of $X$ has a finite subcover. We say that $X$ is \emph{sequentially compact} if every sequence in $X$ has a convergent subsequence.
\end{definition}
\begin{exercise}
Show that a continuous real-valued function on a sequentially compact topological space is bounded and attains its bounds.
\end{exercise}
\begin{remark}
This was the traditional wording we have been using; here and elsewhere, if no topology is specified $\R$ is generally assumed to have the usual topology, and proof is similar to the metric space case.

We've seen that for a metric space, compact and sequentially compact are equivalent; but \textbf{\color{red} This is not true for general topological space!} In fact, $\exists$ compact space that is not sequentially compact, and $\exists$ sequentially compact space that is not compact. However both examples are beyond the scope of this course.

Observe that compactness and sequential compactness are both topological properties. Given that we don't want to think too much about sequences in a general topological space, we'll be concentrating primarily on compactness rather than sequential compactness.
\end{remark}