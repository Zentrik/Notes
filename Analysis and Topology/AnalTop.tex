%&../preamble

\def\npart {IB}
\def\nterm {Michaelmas}
\def\nyear {2022}
\def\nlecturer {Dr P. Russell}
\def\ncourse {Analysis and Topology}

\def\encodingdefault{TU}\normalfont
\ifnum 0\ifxetex 1\fi\ifluatex 1\fi=0 % if pdftex
  \usepackage[T1]{fontenc}
  \usepackage[utf8]{inputenc}
  \usepackage{textcomp} % provide euro and other symbols
\else % if luatex or xetex
  % \usepackage{unicode-math}
  % \defaultfontfeatures{Scale=MatchLowercase}
  % \defaultfontfeatures[\rmfamily]{Ligatures=TeX,Scale=1}
  % \DeclareMathAlphabet{\mathcal}{OMS}{cmsy}{m}{n}
  % \let\mathbb\relax % remove the definition by unicode-math
  % \DeclareMathAlphabet{\mathbb}{U}{msb}{m}{n}
\fi


\usetikzlibrary{external}
\tikzset{external/system call={xelatex -fmt=../preamble.fmt \tikzexternalcheckshellescape -halt-on-error -interaction=batchmode -jobname "\image" "\texsource"}} % path is relative to file that includes preamble
\tikzexternalize

\hypersetup{
  pdftitle={Part \npart\ - \ncourse},
  pdfauthor={\nauthor},
  pdfsubject={Cambridge Maths Notes: Part \npart\ - \ncourse},
  pdfkeywords={Cambridge Mathematics Maths Math \npart\ \nterm\ \nyear\ \ncourse}
}

\author{Based on lectures by \nlecturer \\\small Notes taken by \nauthor}
\date{\nterm\ \nyear}
\title{Part \npart\ --- \ncourse}

\tikzsetexternalprefix{figtemp/}
% \includeonly{00-intro.tex}

% \setcounter{section}{-1}

\begin{document}
    \maketitle
    \tableofcontents

    \part{Generalizing continuity and convergence}
    \section{Three Examples of Convergence}
    \subsection{Convergence in $\mathbb{R}$}
    Let $(x_n)$ be a sequence in $\mathbb{R}$ and $x \in \mathbb{R}$.
    We say $(x_n)$ \textit{converges} to $x$ and write $x_n \to x$ if
    \begin{align*}
        \forall \; \epsilon > 0 \quad \exists \; N \quad \forall \; n \geq N \quad |x_n - x| < \epsilon.
    \end{align*} 
    Useful fact: $\forall \; a, b \in \mathbb{R} \ |a+b| \leq |a| + |b|$ (Triangle Inequality).

    Bolzano-Weierstrass Theorem (BWT)
    A bounded sequence in $\mathbb{R}$ must have a convergent subsequence (Proof by interval bisection).

    Recall: A sequence $(x_n)$ in $\mathbb{R}$ is Cauchy if 
    \begin{align*}
        \forall \; \epsilon > 0 \quad \exists \; N \quad \forall \; m, n \geq N \quad |x_m - x_n| < \epsilon.
    \end{align*} 

    Easy exercise Convergent $\implies$ Cauchy

    General Principle of Convergence (GPC)
    Any Cauchy sequence in $\mathbb{R}$ converges.

    \begin{proof}[Outline]
        If $(x_n)$ Cauchy then $(x_n)$ bounded so by BWT has a convergent subsequence, say $x_{n_j} \to x$.
        But as $(x_n)$ Cauchy, $x_n \to x$.
    \end{proof} 

    \subsection{Convergence in $\mathbb{R}^2$}
    \begin{remark}
        This all works in $\mathbb{R}^n$
    \end{remark} 

    Let $(z_n)$ be a sequence in $\mathbb{R}^2$ and $z \in \mathbb{R}^2$.
    What should $z_n \to z$ mean?

    In $\mathbb{R}$: ``As $n$ gets large, $z_n$ gets arbitrarily close to $z$.''

    What does `close' mean in $\mathbb{R}^2$?

    In $\mathbb{R}$: $a, b$ close if $|a - b|$ small.
    In $\mathbb{R}^2$: Replace $|\cdot|$ by $\left \lVert \cdot \right \rVert $

    Recall: If $z = (x, y)$ then $\left \lVert z \right \rVert = \sqrt{x^2 + y^2}$.

    Triangle Inequality If $a, b \in \mathbb{R}^2$ then $\left \lVert a + b \right \rVert \leq \left \lVert a \right \rVert + \left \lVert b \right \rVert$.

    \begin{definition}
        Let $(z_n)$ be a sequence in $\mathbb{R}^2$ and $z \in \mathbb{R}^2$.
        We say $(z_n)$ \vocab{converges} to $z$ and .. $z_n \to z$ if $\forall \; \epsilon > 0 \ \exists \; N \ \forall \; n \geq N \ \left \lVert z_n - z \right \rVert < \epsilon$. 

        Equivalently, $z_n \to z$ iff $\left \lVert z_n - z \right \rVert \to 0$ (convergence in $\mathbb{R}$).
    \end{definition} 

    \begin{example}
        Let $(z_n), (w_n)$ be sequences in $\mathbb{R}^2$ with $z_n \to z, w_n \to w$. 
        Then $z_n + w_n \to z + w$.
    \end{example} 

    \begin{proof}
        \begin{align*}
            \left \lVert (z_n + w_n) - (z + w) \right \rVert &\leq \left \lVert z_n - z \right \rVert + \left \lVert w_n - w \right \rVert \\
            &\to 0 + 0 = 0 \ \text{(by results from IA)}.
        \end{align*} 
    \end{proof} 

    In fact, given convergence in $\mathbb{R}$, convergence in $\mathbb{R}^2$ is easy:
    \begin{proposition} \label{prop:one}
        Let $(z_n)$ be a sequence in $\mathbb{R}^2$ and let $z \in \mathbb{R}^2$.
        Write $z_n = (x_n, y_n)$ and $z = (x, y)$.
        Then $z_n \to z$ iff $x_n \to x$ and $y_n \to y$.
    \end{proposition} 

    \begin{proof}
        ($\implies$): $|x_n - x|, |y_n - y| \leq \norm{z_n - z}$.
        So if $\norm{z_n - z} \to 0$ then $|x_n - x| \to 0$ and $|y_n - y| \to 0$.

        ($\Longleftarrow$): If $|x_n - x| \to 0$ and $|y_n - y| \to 0$ then $\norm{z_n - z} = \sqrt{(x_n - x)^2 + (y_n - y)^2} \to 0$ by results in $\mathbb{R}$.
    \end{proof} 

    \begin{definition}[Bounded Sequence]
        A sequence $(z_n)$ in $\mathbb{R}^2$ is \vocab{bounded} if $\exists \; M \in \mathbb{R}$ s.t. $\forall \; n \ \norm{z_n} \leq M$.
    \end{definition} 

    \begin{theorem}[BWT in $\mathbb{R}^2$]
        A bounded sequence in $\mathbb{R}^2$ must have a convergent subsequence.
    \end{theorem} 

    \begin{theorem}[GPC for $\mathbb{R}^2$]
        Any Cauchy sequence in $\mathbb{R}^2$ converges.
    \end{theorem} 

    \begin{proof}
        Let $(z_n)$ be a Cauchy sequence in $\mathbb{R}^2$.
        Write $z_n = (x_n, y_n)$.
        For all $m, n, |x_m - x_n| \leq \norm{z_m - z_n}$ so $(x_n)$ is a Cauchy sequence in $\mathbb{R}$, so converges by GPC.
        Similarly, $(y_n)$ converges in $\mathbb{R}$.
        So by \ref{prop:one}, $(z_n)$ converges.
    \end{proof} 

    \underline{Thought for the day} What about continuity?
    Let $f: \mathbb{R}^2 \to \mathbb{R}$.
    What does it mean for $f$ to be continuous?
    (Simple modification of defn for $\mathbb{R} \to \mathbb{R}$).

    What can we do with it?

    Big theorem in IA: If $f: \mathbb{R} \to \mathbb{R}$ is a continuous function on a closed bounded interval then $f$ is bounded and attains its bounds.

    Is there a similar theorem for $\mathbb{R}^2 \to \mathbb{R}$.
    What do we replace `closed bounded interval' by?
    We proved the theorem using BWT.
    Why did it work?
    Why did we need a closed bounded interval to make it work?
    What can we do in $\mathbb{R}^2$?

    \subsection{Convergence of Functions}
    Let $X \subset \mathbb{R}$\footnote{Mostly can think of $X = \mathbb{R}$ or some interval}, let $f_n : X \to \mathbb{R}$ ($n \geq 1$) and let $f: X \to \mathbb{R}$.
    What does it mean for $f_n$ to converge to $f$.

    Obvious idea:
    \begin{definition}[Pointwise convergence]
        Say $(f_n)$ \vocab{converges pointwise} to $f$ and write $f_n \to f$ pointwise if $\forall \; x \in X \ f_n(x) \to f(x)$ as $n \to \infty$.
    \end{definition} 

    Pros
    \begin{itemize}
        \item Simple
        \item Easy to check
        \item Defined in terms of convergence in $\mathbb{R}$
    \end{itemize} 
    Cons
    \begin{itemize}
        \item Doesn't preserve `nice' properties.
        \item `Doesn't feel right'.
    \end{itemize} 

    In all three examples, have $X = [0, 1], f_n \to f$ pointwise.

    \begin{example}[Every $f_n$ continuous but $f$ not] ~\vspace*{-1.5\baselineskip}
        \begin{align*}
            f_n(x) &= \begin{cases}
                nx & x \leq \frac{1}{n} \\
                1 & x \geq \frac{1}{n}
            \end{cases} \\
            f(x) &= \begin{cases}
                0 & x= 0 \\
                1 & x> 0
            \end{cases} 
        \end{align*} 
        {\par
            \centering 
            \includegraphics[height=5cm]{01-pointwise1} 
        \par}
        Clearly $f_n$ continuous for all $n$ but $f$ not.
        If $x = 0,\ \forall \; n \ f_n(0) = 0 = f(0)$.
        If $x > 0$, for sufficiently large $n$ $f_n(x) = 1 = f(x)$ so $f_n(x) \to f(x)$.
    \end{example} 

    \begin{example}[Every $f_n$ integrable but $f$ not]
        \begin{align*}
            f(x) &= \begin{cases}
                1 & x \in \mathbb{Q} \\
                0 & x \notin \mathbb{Q}
            \end{cases}.
        \end{align*} 
        This is a non integrable\footnote{N.B. As in IA `integrable' means `Riemann integrable'} function so now we want to find $f_n$ such that they converge pointwise to this.
        Enumerate the rationals in $[0, 1]$ as $q_1, q_2, \dots$
        For $n \geq 1$, set $f_n(x) = \mathbbm{1}_{q_1, \dots, q_n}$. 
        $f_n$ integrable as it is nonzero at finitely many points.
    \end{example} 

    \begin{example}[Every $f_n$ and $f$ integrable but $\int_0^1 f_n \not\to \int_0^1 f$]
    Let $f(x) = 0$ for all $x$, so $\int_0^1 f = 0$.
    Define $f_n$ s.t. $\int_0^1 f_n = 1$ for all $n$.
    {\par
        \centering 
        \includegraphics[height=5cm]{01-pointwise2} 
    \par}
    \begin{align*}
        f_n(x) &= \begin{cases}
            n & 0 < x < \frac{1}{n} \\
            0 & \text{otherwise}
        \end{cases}.
    \end{align*} 
    \end{example} 

    Better definition:
    \begin{definition}[Uniform convergence]
        Let $X \subset \mathbb{R}$, $f_n : X \to \mathbb{R}$ ($n \geq 1$), $f: X \to \mathbb{R}$.
        We say $(f_n)$ \vocab{converges uniformly} to $f$ and write $f_n \to f$ uniformly if $\forall \; \epsilon > 0 \ \exists \; N \ \forall \; x \in X \ \forall \; n \geq N \ |f_n(x) - f(x)| < \epsilon$.
    \end{definition} 

    cf $f_n \to f$ pointwise: $\forall \; \epsilon > 0 \ \forall \; x \in X \ \exists \; N \ \forall \; n \geq N \ |f_n(x) - f(x)| < \epsilon$. (We have swapped the $\forall \; x \in x$ and $\exists \; N$).
    Pointwise convergence allows for $N$ to be a function of $x$ whilst uniform convergence requires $N$ to work for all $x$ even the worst case.
    In particular, $f_n \to f$ uniformly $\implies f_n \to f$ pointwise.

    \begin{center}
        \tikzset{every picture/.style={line width=0.75pt}} %set default line width to 0.75pt        
    
        \begin{tikzpicture}[x=0.75pt,y=0.75pt,yscale=-1,xscale=1]
        %uncomment if require: \path (0,300); %set diagram left start at 0, and has height of 300
        
        %Straight Lines [id:da15070589522945865] 
        \draw    (120,42) -- (120,228) ;
        \draw [shift={(120,230)}, rotate = 270] [color={rgb, 255:red, 0; green, 0; blue, 0 }  ][line width=0.75]    (10.93,-3.29) .. controls (6.95,-1.4) and (3.31,-0.3) .. (0,0) .. controls (3.31,0.3) and (6.95,1.4) .. (10.93,3.29)   ;
        \draw [shift={(120,40)}, rotate = 90] [color={rgb, 255:red, 0; green, 0; blue, 0 }  ][line width=0.75]    (10.93,-3.29) .. controls (6.95,-1.4) and (3.31,-0.3) .. (0,0) .. controls (3.31,0.3) and (6.95,1.4) .. (10.93,3.29)   ;
        %Straight Lines [id:da06812433005878815] 
        \draw    (338,200) -- (92,200) ;
        \draw [shift={(90,200)}, rotate = 360] [color={rgb, 255:red, 0; green, 0; blue, 0 }  ][line width=0.75]    (10.93,-3.29) .. controls (6.95,-1.4) and (3.31,-0.3) .. (0,0) .. controls (3.31,0.3) and (6.95,1.4) .. (10.93,3.29)   ;
        \draw [shift={(340,200)}, rotate = 180] [color={rgb, 255:red, 0; green, 0; blue, 0 }  ][line width=0.75]    (10.93,-3.29) .. controls (6.95,-1.4) and (3.31,-0.3) .. (0,0) .. controls (3.31,0.3) and (6.95,1.4) .. (10.93,3.29)   ;
        %Curve Lines [id:da9975807479144934] 
        \draw [color={rgb, 255:red, 74; green, 144; blue, 226 }  ,draw opacity=1 ]   (140,140) .. controls (180,110) and (190,140) .. (230,110) .. controls (270,80) and (295,145) .. (320,120) ;
        %Curve Lines [id:da8386196196350342] 
        \draw [color={rgb, 255:red, 208; green, 2; blue, 27 }  ,draw opacity=1 ] [dash pattern={on 4.5pt off 4.5pt}]  (140,127) .. controls (180,97) and (190,127) .. (230,97) .. controls (270,67) and (295,132) .. (320,107) ;
        %Curve Lines [id:da7136000427009845] 
        \draw [color={rgb, 255:red, 208; green, 2; blue, 27 }  ,draw opacity=1 ] [dash pattern={on 4.5pt off 4.5pt}]  (140,153) .. controls (180,123) and (190,153) .. (230,123) .. controls (270,93) and (295,158) .. (320,133) ;
        %Shape: Brace [id:dp021850826872210183] 
        \draw  [color={rgb, 255:red, 208; green, 2; blue, 27 }  ,draw opacity=1 ] (172,125.25) .. controls (173.71,125.25) and (174.57,124.39) .. (174.57,122.68) -- (174.57,122.68) .. controls (174.57,120.23) and (175.43,119) .. (177.15,119) .. controls (175.43,119) and (174.57,117.77) .. (174.57,115.32)(174.57,116.43) -- (174.57,115.32) .. controls (174.57,113.61) and (173.71,112.75) .. (172,112.75) ;
        %Curve Lines [id:da39763918210258864] 
        \draw    (140,150) .. controls (180,120) and (200,130) .. (240,100) .. controls (280,70) and (271,135.75) .. (290,121) .. controls (309,106.25) and (295,155) .. (320,130) ;
        
        % Text Node
        \draw (177,119) node [anchor=west] [inner sep=0.75pt]  [color={rgb, 255:red, 208; green, 2; blue, 27 }  ,opacity=1 ]  {$\varepsilon $};
        % Text Node
        \draw (322,116.6) node [anchor=south west] [inner sep=0.75pt]  [color={rgb, 255:red, 74; green, 144; blue, 226 }  ,opacity=1 ]  {$f$};
        % Text Node
        \draw (321,130.4) node [anchor=north west][inner sep=0.75pt]  [color={rgb, 255:red, 0; green, 0; blue, 0 }  ,opacity=1 ]  {$f_{n}$};
        
        
        \end{tikzpicture}
        
    
    \end{center}

    Equivalently, $f_n \to f$ uniformly if for sufficiently large $n$ $f_n - f$ is bounded and $\sup_{x \in X} |f_n - f| \to 0$.

    \begin{theorem}[A uniform limit of cts functions is cts] \label{thm:4}
        Let $X \subset \mathbb{R}$, let $f_n: X \to \mathbb{R}$ be continuous ($n \geq 1$) and let $f_n \to f: X \to \mathbb{R}$ uniformly.
        Then $f$ is cts.
    \end{theorem} 

    \begin{proof}
        Let $x \in X$.
        Let $\epsilon > 0$.
        As $f_n \to f$ uniformly, we can find $N$ s.t. $\forall \; n \geq N \ \forall \; y \in X \ |f_n(y) - f(y)| < \epsilon$.
        In particular, $\forall \; y \in X \ |f_N(y) - f(y)| < \epsilon$.
        As $f_N$ is cts, we can find $\delta > 0$ s.t. $\forall \; y \in X,\ |y - x| < \delta \implies |f_N(y) - f_N(x)| < \epsilon$.
        Now let $y \in X$ with $|y - x| < \delta$.
        Then
        \begin{align*}
            |f(y) - f(x)| &\leq |f(y) - f_N(y)| + |f_N(y) - f_N(x)| + |f_N(x) - f(x)|\footnote{The core of this proof is this inequality.} \\
            &< \epsilon + \epsilon + \epsilon = 3\epsilon.
        \end{align*} 
        Hence $f$ is cts.
    \end{proof} 

    \begin{remark}
        This is often called a `$3\epsilon$ proof' (or an $\frac{\epsilon}{3}$ proof).
    \end{remark} 

    \begin{theorem} \label{thm:5}
        Let $f_n: [a, b] \to \mathbb{R}$ ($n \geq 1$) be integrable and let $f_n \to f: [a, b] \to \mathbb{R}$ uniformly.
        Then $f$ is integrable and $\int_a^b f_n \to \int_a^b f$ as $n \to \infty$.
    \end{theorem} 

    \begin{proof}
        As $f_n \to f$ uniformly, we can pick $n$ suff. large s.t. $f_n - f$ is bounded.
        Also $f_n$ is bounded (as integrable).
        So by triangle inequality, $f = (f - f_n) + f_n$ is bounded.
        Let $\epsilon > 0$.
        As $f_n \to f$ uniformly there is some $N$ s.t. $\forall \; n \geq N \ \forall \; x \in [a, b]$ we have $|f_n(x) - f(x)| < \epsilon$. \\
        In particular, $\forall \; x \in [a, b] \ |f_N(x) - f(x)| < \epsilon$.

        By Riemann's criterion, there is some dissection $\mathcal{D}$ of $[a, b]$ for which $S(f_n, \mathcal{D}) - s(f_n, \mathcal{D}) < \epsilon$.
        Let $\mathcal{D} = \{x_0, x_1, x_2, \dots, x_k\}$ where $a = x_0 < x_1 < \dots < x_k = b$.
        Now \begin{align*}
            S(f, \mathcal{D}) &= \sum_{i=1}^{k}  (x_i - x_{i-1}) \sup_{x \in [x_{i-1}, x_i]} f(x) \\
            &\leq \sum_{i=1}^{k}  (x_i - x_{i-1}) \sup_{x \in [x_{i-1}, x_i]} (f_N(x) + \epsilon) \\
            &= \sum_{i=1}^{k}  (x_i - x_{i-1}) \left( \left( \sup_{x \in [x_{i-1}, x_i]} f_N(x) \right) + \epsilon\right) \\
            &= \sum_{i=1}^{k}  (x_i - x_{i-1}) \sup_{x \in [x_{i-1}, x_i]} f_N(x) + \sum_{i=1}^{k} (x_i - x_{i-1}) \epsilon \\
            &= S(f_N, \mathcal{D}) + (b - a)\epsilon.
        \end{align*} 
        That is $S(f, \mathcal{D}) \leq S(f_N, \mathcal{D}) + (b - a)\epsilon$.
        Similarly $s(f, \mathcal{D}) \geq s(f_N, \mathcal{D}) - (b - a)\epsilon$.
        Hence
        \begin{align*}
            S(f, \mathcal{D}) - s(f, \mathcal{D}) &\leq S(f_N, \mathcal{D}) - s(f_N, \mathcal{D}) + 2(b - a) \epsilon \\
            &< (2(b-a) + 1) \epsilon
        \end{align*} 
        But $2(b-a) + 1$ is a constant so $(2(b-a) + 1) \epsilon$ can be made arbitrarily small.
        Hence by Riemann's criterion, $f$ is integrable over $[a, b]$.

        Now, for any $n$ suff. large that $f_n - f$ is bounded, 
        \begin{align*}
            \left| \int_a^b f_n - \int_a^b f \right| &= \left| \int_a^b (f_n - f) \right| \\
            &\leq \int_a^b |f_n - f| \\
            &\leq (b - a) \sup_{x \in [a, b]} |f_n - f| \\
            &\to 0 \text{ as } n \to \infty \text{ since $f_n \to f$ uniformly.}\footnote{Note we said that $f_n \to f$ uniformly if $\sup |f_n - f| \to 0$.}
        \end{align*} 
    \end{proof} 

    What about differentiation?
    Here even uniform convergence isn't enough.

    \begin{example}
        $f_n : (-1, 1) \to \mathbb{R}$, each $f_n$ differentiable, $f_n \to f$ uniformly, $f$ not diff.

        Let $f(x) = |x|$ which is not differentiable at $0$.
        {\par \centering \includegraphics[height=5cm]{01-modx} \par}

        \begin{align*}
            f_n &= \begin{cases}
                |x| & |x| \geq \frac{1}{n} \\
                ax^2 + bx + c & |x| < \frac{1}{n}
            \end{cases}.
        \end{align*} 
        We need $a(\frac{1}{n})^2 + \frac{b}{n} + c = \frac{1}{n}$ for continuity.
        Thus $b = 0$ and $c = \frac{1}{n} - \frac{a}{n^2}$.

        Also need $2a \frac{1}{n} + b = 1$ and $2a (- \frac{1}{n}) = -1$ for differentiability so take $a = \frac{n}{2}$, $c = \frac{1}{n} - \frac{1}{2n} = \frac{1}{2n}$.

        If $|x| \geq \frac{1}{n}$ then $|f_n(x) - f(x)| = 0$.
        If $|x| < \frac{1}{n}$:
        \begin{align*}
            |f_n(x) - f(x)| &= \left| \frac{n}{2} x^2 + \frac{1}{2n} - |x| \right| \\
            &\leq \frac{n}{2} x^2 + \frac{1}{2n} + |x| \\
            &\leq \frac{n}{2} (\frac{1}{n})^2 + \frac{1}{2n} + \frac{1}{n} \\
            &= \frac{1}{2n} + \frac{1}{2n} + \frac{1}{n} \\
            &= \frac{2}{n}
        \end{align*} 
        So $\sup_{x \in (-1, 1)} |f_n(x) - f(x)| \leq \frac{2}{n} \to 0$ as $n \to \infty$.
        So $f_n \to f$ uniformly.
    \end{example} 

    If fact we need uniform convergence of the derivatives.

    \begin{theorem} \label{thm:6}
        Let $f_n : (u, v) \to \mathbb{R}$ ($n \geq 1$) with $f_n \to f : (u, v) \to \mathbb{R}$ pointwise.
        Suppose further each $f_n$ is continuously differentiable and that $f_n' \to g : (u, v) \to \mathbb{R}$ uniformly.
        Then $f$ is differentiable with $f' = g$.
    \end{theorem} 

    \begin{proof}
        Fix $a \in (u, v)$.
        Let $x \in (u, v)$, by FTC we have each $f_n'$ is integrable over $[a, x]$ and $\int_{a}^{x} f_n' = f_n(x) - f_n(a)$.
        But $f_n' \to g$ uniformly so by \cref{thm:5} $g$ is integrable over $[a, x]$ and $\int_{a}^{x} g = \lim_{n \to \infty} \int_{a}^{x} f_n' = f(x) - f(a)$.
        So we have shown that for all $x \in (u, v)$
        \begin{align*}
            f(x) &= f(a) + \int_a^x g.
        \end{align*} 

        By \cref{thm:4}, $g$ is cts so by FTC, $f$ is differentiable with $f' = g$.
    \end{proof} 

    \begin{remark}
        It would have sufficed to assume $f_n(x) \to f(x)$ for a single value of $x$ rather than $f_n \to f$ pointwise.
    \end{remark} 

    GPC?

    \begin{definition}[Uniform Cauchy]
        Let $X \subset \mathbb{R}$ and let $f_n : X \to \mathbb{R}$ for each $n \geq 1$.
        We say $f_n)$ is \vocab{uniformly Cauchy} if $\forall \; \epsilon > 0 \ \exists \; N \ \forall \; m, n \geq N \ \forall \; x \in X \ |f_m(x) - f_n(x)| < \epsilon$
    \end{definition} 

    exercise: uniformly convergence $\implies$ uniformly Cauchy.

    \begin{theorem}[General principle of Uniform Convergence (GPUC)] \label{thm:7}
        Let $(f_n)$ be a uniformly Cauchy sequence of functions $X \to \mathbb{R}$ ($X \subset \mathbb{R}$).
        Then $(f_n)$ is uniformly convergent.
    \end{theorem} 

    \begin{proof}
        Let $x \in X$.
        Let $\epsilon > 0$.
        Then $\exists \; N \ \forall \; m,n \geq N \ \forall \; y \in X \ |f_m(y) - f_n(y)| < \epsilon$.
        In particular, $\forall \; m, n \geq N \ |f_m(x) - f_n(x)| < \epsilon$.
        So $(f_n(x))$ is a Cauchy sequence in $\mathbb{R}$ so by GPC it converges, say $f_n(x) \to f(x)$ as $n \to \infty$. 

        We have now constructed $f : X \to \mathbb{R}$ s.t. $f_n \to f$ pointwise. \\
        Let $\epsilon > 0$.
        Then we can find a $N$ s.t. $\forall \; m, n \geq N \ \forall \; y \in X \ |f_m(y) - f_n(y)| < \epsilon$.
        Fix $y \in X$, keep $m \geq N$ fixed and let $n \to \infty$: $|f_m(y) - f(y)| \leq \epsilon$.
        So we have shown that $\forall \; m \geq N,\ |f_m(y) - f(y)| < \epsilon$.

        But $y$ was arbitrary so $\forall \; x \in X \ \forall \; m \geq N \ |f_m(x) - f(x)| \leq \epsilon$.
        That is $f_n \to f$ uniformly.
    \end{proof} 

    BW?

    \begin{definition}[Pointwise bounded]
        Let $X \subset \mathbb{R}$ and let $f_n : X \to \mathbb{R}$ for each $n \geq 1$.
        We say $(f_n)$ is \vocab{pointwise bounded} if $\forall \; x \ \exists \; M \ \forall \; n \ |f_n(x)| \leq M$.
    \end{definition} 

    \begin{definition}[Uniformly bounded]
        Let $X \subset \mathbb{R}$ and let $f_n : X \to \mathbb{R}$ for each $n \geq 1$.
        We say $(f_n)$ is \vocab{uniformly bounded} if $\exists \; M \ \forall \; x \ \forall \; n \ |f_n(x)| \leq M$.\footnote{Again we have just swapped ... as in convergence.}
    \end{definition} 
    
    What would uniform BW say?
    `If $(f_n)$ is a uniformly bounded sequence of functions that it has a uniformly convergent subsequence'.
    But this is \underline{not} true.

    \begin{example}[Counterexample of BW] ~\vspace*{-1.5\baselineskip}
        \begin{align*}
            f_n : \mathbb{R} &\to \mathbb{R} \\
            f_n(x) &= \begin{cases}
                1 & x = n \\
                0 & x \neq n.
            \end{cases}
        \end{align*} 
        Obviously $(f_n)$ uniformly bounded (by 1).
        However, if $m \neq n$ then $f_m(m) = 1$ and $f_n(m) = 0$ so $|f_m(m) - f_n(m)| = 1$ so no subsequence can be uniformly Cauchy so no subsequence can be uniformly convergent.
    \end{example} 

    \underline{Application to power series}
    Recall that if $\sum a_n x^n$ is a real power series with r.o.c $R > 0$ then we can differentiate/ integrate it term-by-term within $(-R, R)$.

    \begin{definition}
        Let $f_n : X \to \mathbb{R}$ ($X \subset \mathbb{R}$) for each $n \geq 0$.
        We say the series $\sum_{n=0}^{\infty} f_n$ \vocab{uniformly converges} if the sequence of partial sums $(F_n)$ does, where $F_n = \sum_{m=0}^{n} f_m$.
    \end{definition} 
    We can apply \cref{thm:4,thm:5,thm:6} to get e.g. if conditions hold with $f_n$ cts diff and uniform convergence then $\sum f_n$ has derivative $\sum f_n'$.

    \underline{Hope} Prove $\sum a_n x^n$ converges uniformly on $(-R, R)$ then hit it with earlier theorems.

    \underline{Not quite true}:
    \begin{example}
        $\sum_{n=0}^{\infty} x^n$ r.o.c 1.
        This does \underline{not} converge uniformly on $(-1, 1)$.
        Let $f(x) = \sum_{n=0}^{\infty} x^n$ and $F_n(x) = \sum_{m=0}^{n} x^m$.
        Note $f(x) = \frac{1}{1 - x} \to \infty$ as $x \to 1$.
        However, $\forall \; x \in (-1, 1) \ |F_n(x)| \leq n + 1$.

        Fix any $n$.
        We can find a point $x \in (-1, 1)$ where $f(x) \geq n + 2$ and so $|f(x) - F_n(x)| \geq 1$.
        So we don't have uniform convergence. 
    \end{example} 

    \underline{Back-up plan}: It does work if we look at a smaller interval.
    
    New plan: show if $0 < r < R$ then we do have uniform convergence on $(-r, r)$. \\
    Given $x \in (-R, R)$ there's some $r$ with $|x| < r < R$: use uniform convergence on $(-r, r)$ to check everything is nice at $x$.
    `Local uniform convergence of power series'.

    \begin{aside}{Aside}
        In $\mathbb{R}$ $x_n \to 0$ if 
        \begin{enumerate}
            \item $\forall \; \epsilon > 0 \ \exists \; N \ \forall \; n \geq N \ |x_n| < \epsilon$.
            \item Equivalently: $\forall \; \epsilon > 0 \ \exists \; N \ \forall \; n \geq N \ |x_n| \leq \epsilon$.
        \end{enumerate} 
        \begin{proof}
            i $\implies$ ii: obvious \\
            ii $\implies$ ii: Let $\epsilon > 0$.
            Pick $N$ s.t. $\forall \; n \geq N \ |x_n|  \leq \frac{1}{2} \epsilon$.
            Then $\forall \; n \geq N \ |x_n| < \epsilon$.
        \end{proof} 


        Also: $f_n,f : X \to \mathbb{R}$, $f_n \to f$ uniformly.
        \begin{enumerate}
            \item $\forall \; \epsilon > 0\ \exists \; N\ \forall \; x \in X\ \forall \; n \geq N\ |f_n(x) - f(x)| < \epsilon$.
            \item For $n$ suff large $f_n - f$ is bounded and $\forall \; \epsilon > 0\ \exists \; N\ \forall \; n \geq N\ \sup_{x \in X} |f_n(x) - f(x)| < \epsilon$. 
        \end{enumerate} 
        \begin{proof}
            ii $\implies$ i: obvious \\
            i $\implies$ ii: if i holds then $\sup_{x \in X} |f_n(x) - f(x)| \leq \epsilon$.
            But OK by same argument as previously.
        \end{proof} 
    \end{aside} 

    \begin{lemma} \label{lem:8}
        Let $\sum a_n x^n$ be a real power series with r.o.c $R > 0$.
        Let $0 < r < R$.
        Then $\sum a_n x^n$ converges uniformly on $(-r, r)$.
    \end{lemma} 

    \begin{proof}
        Define $f, f_n : (-r, r) \to \mathbb{R}$ by $f(x) = \sum_{n=0}^{\infty} a_n x^n$ and $f_m(x) = \sum_{n=0}^{m} a_n x^n$.
        Recall that $\sum a_n x^n$ converges absolutely for all $x$ with $|x| < R$.

        Let $x \in (-r, r)$.
        Then $f$
        \begin{align*}
            |f(x) - f_m(x)| &= \left| \sum_{n=m+1}^{\infty} a_n x^n \right| \\
            &\leq \sum_{n=m+1}^{\infty} |a_n| |x|^n \\
            &\leq \sum_{n=m+1}^{\infty} |a_n| r^n
        \end{align*} which converges by absolute convergence at $r$.
        Hence if $m$ suff large, $f - f_m$ is bounded and 
        \begin{align*}
            \sup_{x \in (-r, r)} |f(x) - f_m(x)| \leq \sum_{n=m+1}^{\infty} |a_n| r^n \to 0
        \end{align*} as $m \to \infty$ by absolute convergence of $r$.
    \end{proof} 

    \begin{theorem} \label{thm:9}
        Let $\sum a_n x^n$ be a real power series with r.o.c $R > 0$.
        Define $f : (-R, R) \to \mathbb{R}$ by $f(x) = \sum_{n=0}^{\infty} a_n x^n$.
        Then 
        \begin{enumerate}
            \item $f$ is continuous;
            \item for any $x \in (-R, R)$ $f$ is integrable over $[0, x]$ with 
            \begin{align*}
                \int_0^x f = \sum_{n=0}^{\infty} \frac{a_n}{n + 1} x^{n + 1}.
            \end{align*} 
        \end{enumerate} 
    \end{theorem} 

    \begin{proof}
        Let $x \in (-R, R)$.
        Pick $r$ s.t. $|x| < r < R$.
        By \cref{lem:8}, $\sum a_n y^n$ converges uniformly on $(-r, r)$.
        But the partial sum functions $y \mapsto \sum_{n=0}^{m} a_n y^n$ ($m \geq 0$) are all cts functions on $(-r, r)$ (as they are polynomials).
        Hence by \cref{thm:4}, $f \mid_{(-r, r)}$\footnote{$f$ restricted to domain $(-r, r)$} is cts.
        Hence $f$ is cts at $x$, but $x$ was arbitrary so $f$ is a cts fcn on $(-R, R)$.

        Moreover, $[0, x] \subset (-r, r)$ so we also have $\sum a_n y^n$ converges uniformly on $[0, x]$.
        Each partial sum function on $[0, x]$ is a poly so can be integrated with $\int_0^x \sum_{n=0}^{m} a_n y^n \,dy = \sum_{n=0}^{m} \int_{0}^{x} a_n y^n \,dy = \sum_{n=0}^{m} \frac{a_n}{n + 1} x^{n+1}$.
        Hence by \cref{thm:5}, $f$ is integrable over $[0, x]$ with
        \begin{align*}
            \int_{0}^{x} f &= \lim_{m \to \infty} \int_{0}^{x} \sum_{n=0}^{m} a_n y^n \,dy \\
            &= \lim_{m \to \infty} \sum_{n=0}^{m} \frac{a_n}{n + 1} x^{n + 1} \\
            &= \sum_{n=0}^{\infty} \frac{a_n}{n + 1} x^{n+1}.
        \end{align*} 
    \end{proof} 

    For differentiation, need technical lemma:
    \begin{lemma} \label{lem:10}
        Let $\sum a_n x^n$ be a real power series with r.o.c $R > 0$.
        Then the power series $\sum_{n \geq 1} n a_n x^{n - 1}$ has r.o.c at least $R$.
    \end{lemma} 

    \begin{proof}
        Let $x \in \mathbb{R}$ with $0 < x < R$.
        Pick $w$ with $x < w < R$.
        Then $\sum a_n w^n$ is absolutely convergent, so $a_n w^n \to 0$ (terms of a convergent series) so $\exists \; M$ s.t. $\forall \; n,\ |a_n w^n| \leq M$.

        For each $n$, 
        \begin{align*}
            |n a_n x^{n - 1}| &= |a_n w^n| \left| \frac{x}{w} \right|^n \frac{1}{|x|} n.
        \end{align*} 
        Fix $n$.
        Let $\alpha = \left| \frac{x}{w} \right| < 1$.
        Let $c = \frac{M}{|x|}$, a constant.
        Then $|n a_n x^{n - 1}| \leq c n \alpha^n$.
        By comparison test, ETS (enough to show) $\sum n \alpha^n$ converges. \\
        Note $\left| \frac{(n + 1) \alpha^{n + 1}}{n \alpha^n} \right| = (1 + \frac{1}{n}) \alpha \to \alpha < 1$ as $n \to \infty$ so done by ratio test.
    \end{proof} 

    \begin{theorem} \label{thm:11}
        Let $\sum a_n x^n$ be a real power series with r.o.c. $R > 0$.
        Let $f : (-R, R) \to \mathbb{R}$ be defined by $f(x) = \sum_{n=0}^{\infty} a_n x^n$.
        Then $f$ is differentiable and $\forall \; x \in (-R, R)$ $f'(x) = \sum_{n=1}^{\infty} n a_n x^{n - 1}$.
    \end{theorem} 

    \begin{proof}
        Let $x \in (-R, R)$.
        Pick $r$ with $|x| < r < R$.
        Then $\sum a_n y^n$ converges uniformly on $(-r, r)$.
        Moreover, the power series $\sum_{n \geq 1} n a_n y^{n - 1}$ has r.o.c at least $R$ and so also converges uniformly on $(-r, r)$.

        The partial sum functions $f_m(y) = \sum_{n=0}^{m} a_n y^n$ are polys so differentiable with $f_m'(y) = \sum_{n=1}^{m} n a_n y^{n - 1}$.
        We now have $f'_m$ converging uniformly on $(-r, r)$ to the function $g(y) = \sum_{n=1}^{\infty} n a_n y^{n - 1}$.

        Hence by \cref{thm:6}, $f \mid_{(-r, r)}$ is differentiable and $\forall \; y \in (-r, r)$ $f'(y) = g(y)$.

        In particular, $f$ is differentiable at $x$ with $f'(x) = g(x)$.
        Hence $f$ is a differentiable function on $(-R, R)$ with derivative $g$ as desired.
    \end{proof} 

    \subsection{Uniform Continuity}
    Let $X \subset \mathbb{R}$.
    Let $f: X \to \mathbb{R}$.
    (May as well think of $X = \mathbb{R}$ or $X = (a, b)$).

    \begin{definition}[Continuous function]
        $f$ is \vocab{continuous} if 
        \begin{align*}
            \forall \; \epsilon > 0 \ \color{red} \forall \; x \in X \ \exists \; \delta > 0 \color{black} \ \forall \; y \in X \ |x - y| < \delta \implies |f(x) - f(y)| < \epsilon.
        \end{align*} 
    \end{definition} 

    \begin{definition}[Unifomly Continuous function]
        $f$ is \vocab{uniformly continuous} if 
        \begin{align*}
            \forall \; \epsilon > 0 \ \color{red} \exists \; \delta > 0 \ \forall \; x \in X \color{black} \ \forall \; y \in X \ |x - y| < \delta \implies |f(x) - f(y)| < \epsilon.
        \end{align*} 
    \end{definition} 

    \begin{remark}
        Clearly if $f$ is uniformly cts then $f$ is cts.
        We would suspect that $f$ cts doesn't imply $f$ uniformly cts.
    \end{remark} 

    \begin{example}
        A function $f: \mathbb{R} \to \mathbb{R}$ that is cts but not uniformly cts.

        {\par
    \centering 
    \includegraphics[height=5cm]{01-uniformcts} 
        \par}
        We want some function that looks like this, a continuous function which gets steeper as we go to infinity.
        So $f(x) = x^2$ ought to work.
        We know $f$ is cts (as it's a poly).
        Suppose $\delta > 0$.
        Then \begin{align*}
            f(x + \delta) - f(x) &= (x + \delta)^2 - x^2 \\
            &= 2 \delta x + \delta^2 \to \infty \text{ as } x \to \infty.
        \end{align*} 
        So in particular, $\forall \; \delta > 0 \ \exists \; x, y \in \mathbb{R}$ s.t. $|x-y| < \delta$ but $|f(x) - f(y)| \geq 1$.
        So conditions for uniform cty fails for $\epsilon = 1$.
        So $f$ not uniform cty.
    \end{example} 

    \begin{example} \label{exm:1/x}
        Make domain bounded.
        We can still fail, e.g. $f: (0, 1) \to \mathbb{R}$ cts but not uniform cts.

        { \par
        \centering 
        \includegraphics[height=5cm]{01-uniformcty2} 
        \par}

        Let $f(x) = \frac{1}{x}$, clearly cts.
        Proof that its not uniform continuity is left as an exercise to the reader.
    \end{example} 

    \begin{theorem} \label{thm:12}
        A continuous real-valued function on a closed bounded interval is uniformly continuous.
    \end{theorem} 

    \begin{proof}
        Let $f: [a, b] \to \mathbb{R}$ and suppose $f$ is cts but not uniformly cts. Then we can find $\epsilon > 0$ st. $\delta > 0 \ \exists \; x,y \in [a, b]$ with $|x-y| < \delta$ but $|f(x) - f(y)| \geq \epsilon$.

        In particular, taking $\delta = \frac{1}{n}$ we can find sequences $(x_n), (y_n) \in [a, b]$ with, for each $n$, $|x_n - y_n| < \frac{1}{n}$ but $|f(x_n) - f(y_n)| \geq \epsilon$.
        The sequence $(x_n)$ is bounded so by BW\footnote{Bolzano Weierstrass} it has a convergent subsequence $x_{n_j} \to x$.
        And $[a, b]$ is a closed interval so $x \in [a, b]$.
        Then $x_{n_j} - y_{n_j} \to 0$ so $y_{n_j} \to x$.

        But $f$ is cts at $x$ so $\exists \; \delta > 0$ s.t. $\forall \; y \in [a, b]$ $|y-x| < \delta \implies |f(y) - f(x)| < \frac{\epsilon}{2}$. 
        Take such a $\delta$.
        As $x_{n_j} \to x$ we can find $J_1$ s.t. $j \geq J_1 \implies |x_{n_j} - x| < \delta$.
        Similarly we can find $J_2$ s.t. $j \geq J_2 \implies |y_{n_j} - x| < \delta$.
        Now let $j = \max (J_1, J_2)$ then $|x_{n_j} - x|, |y_{n_j} - x| < \delta$ so we have $|f(x_{n_j}) - f(x)|, |f(y_{n_j}) - f(x)| < \epsilon / 2$.
        Then $|f(x_{n_j}) - f(y_{n_j})| \leq |f(x_{n_j}) - f(x)| + |f(y_{n_j}) - f(x)| < \epsilon$ \Lightning.
    \end{proof} 

    \begin{corollary} \label{cor:13}
        A continuous real-valued function on a closed bounded interval is bounded.
    \end{corollary} 

    \begin{proof}
        Let $f:[a, b] \to \mathbb{R}$ be a continuous function, and so uniformly continuous by \cref{thm:12}.
        Then we can find $\delta > 0$ s.t. $\forall \; x, y \in [a, b] \ |x - y| < \delta \implies |f(x) - f(y)| < 1$.

        Let $M = \lceil \frac{b - a}{\delta} \rceil$.
        Let $x \in [a, b]$.
        We can find $a = x_0 \leq x_1 \leq \dots \leq x_M = x$ with $|x_i - x_{i - 1}| < \delta$ for each $i$.
        Hence
        \begin{align*}
            |f(x)| &= \qty| f(a) + \sum_{i=1}^{M} f(x_i) - f(x_{i - 1}) |\\
            &\leq |f(a)| + \sum_{i=1}^{M} |f(x_i) - f(x_{i-1})| \\
            &< |f(a)| + \sum_{i=1}^{M} 1 \\
            &= |f(a)| + M.
        \end{align*} 
    \end{proof}
    
    \begin{remark}
        Referring back to \cref{exm:1/x}, starting at $x = 1$ and going towards $x  = 0$ we can that $\delta$ gets smaller and smaller s.t. you require an infinite number of steps to get $0$.
        So $M = \infty$ essentially.
    \end{remark} 

    \begin{corollary} \label{cor:14}
        A continuous real-valued function on a closed bounded interval is integrable.
    \end{corollary} 

    \begin{proof}
        Let $f:[a, b] \to \mathbb{R}$ be a continuous function, and so uniformly continuous by \cref{thm:12}.
        Let $\epsilon > 0$.
        Then we can find $\delta > 0$ s.t. $\forall \; x, y \in [a, b] \ |x - y| < \delta \implies |f(x) - f(y)| < \epsilon$.
        Let $\mathcal{D} = \{ x_0 < x_1 < \dots < x_n\}$ be a dissection s.t. for each $i$ we have $x_i - x_{i-1} < \delta$.

        Let $i \in \{1, \dots, n\}$.
        Then for any $u, v \in [x_{i-1}, x_i]$ we have $|u - v| < \delta$ so $|f(u) - f(v)| < \epsilon$.
        Hence 
        \begin{align*}
            \sup_{x \in [x_{i-1}, x_i]} f(x) - \inf_{x \in [x_{i-1}, x_i]} f(x) \leq \epsilon.
        \end{align*} 
        Hence:
        \begin{align*}
            S(f, \mathcal{D}) - s(f, \mathcal{D}) &= \sum_{i=1}^{n} (x_i - x_{i-1}) \qty(\sup_{x \in [x_{i-1}, x_i]} f(x) - \inf_{x \in [x_{i-1}, x_i]} f(x)) \\
            &\leq \sum_{i=1}^{n} (x_i - x_{i - 1}) \epsilon \\
            &= \epsilon \sum_{i=1}^{n} (x_i - x_{i - 1}) \\
            &= \epsilon (b - a).
        \end{align*} 
        But $\epsilon(b-a)$ can be made arbitrarily small by taking $\epsilon$ small.
        So by Riemann's criterion $f$ is integrable over $[a, b]$.
    \end{proof} 

    \section{Metric Spaces}
    \subsection{Definitions and Examples}

    \begin{question}
        Can we think about convergence in a more general setting? Convergence seemed similar in our 3 settings.

        What do we really need?
    \end{question} 
    \begin{answer}
        We need a notion of distance.

        In $\mathbb{R}$: distance $x$ to $y$ is $|x-y|$. \\
        In $\mathbb{R}^2$: its $\norm{x - y}$. \\
        For functions: distance $f$ to $g$ is $\sup_{x \in X} |f(x) - g(x)|$ (where this exists, i.e. if $f - g$ bounded).

        The triangle inequality was often important (see the proof of uniqueness of limits).
    \end{answer}
    
    \begin{definition}[Metric]
        A $\vocab{metric}$ $d$ is a function $d : X^2 \to \mathbb{R}$ satisfying:
        \begin{itemize}
            \item $d(x, y) \geq 0$ for all $x, y \in X$ with equality iff $x = y$;
            \item $d(x, y) = d(y, x)$ for all $x, y \in X$.
            \item $d(x, z) \leq d(x, y) + d(y, z)$ for all $x, y, z \in X$.
        \end{itemize} 
    \end{definition} 

    \begin{definition}[Metric Space]
        A $\vocab{metric space}$ is a set $X$ endowed with a metric $d$.
    \end{definition} 
    We could also define a metric space as an ordered pair $(X, d)$.
    If it is obvious what $d$ is, we sometimes write `The metric space $X$ \dots'.

    \begin{example}
        $X = \mathbb{R}$, $d(x, y) = |x - y|$
        `The \underline{usual metric} on $\mathbb{R}$'.
    \end{example} 

    \begin{example}
        $X = \mathbb{R}^n$ with the \underline{Euclidean metric}, $d(x, y) = \norm{x - y} = \sqrt{\sum_{i=1}^{n} (x_i - y_i)^2}$.
    \end{example} 

    Uniform convergence of functions doesn't quite work: we want $d(f, g) = \sup |f - g|$ but this might not exist if $f - g$ is unbounded.
    However, we can do something with appropriate sets of functions.

    \begin{example} \label{exm:2.3}
        Let $Y \subset \mathbb{R}$.
        Take $X = B(Y) = \{f : Y \to \mathbb{R} \mid f \text{ bounded}\}$ with the \underline{uniform metric} $d(f, g) = \sup_{x \in Y} |f - g|$.

        Checking triangle inequality:
        \begin{proof}
            Let $f, g, h \in B(Y)$.
            Let $x \in Y$.
            Then \begin{align*}
                |f(x) - h(x)| &\leq |f(x) - g(x)| + |g(x) - h(x)| \\
                &\leq d(f, g) + d(g, h)
                \intertext{Taking $\sup$ over all $x \in Y$}
                d(f, h) \leq d(f, g) + d(g, h).
            \end{align*} 
        \end{proof} 
    \end{example} 

    \begin{definition}[Subspace]
        Suppose $(X, d)$ a metric space and $Y \subset X$.
        Then $d\mid_{Y^2}$ is a metric on $Y$.
        We say $Y$ with this metric is a \vocab{subspace} of $X$.
    \end{definition} 

    \begin{example}
        Subspaces of $\mathbb{R}$: any of $\mathbb{Q}, \mathbb{Z}, \mathbb{N}, [0, 1], \dots$ with the usual metric $d(x, y) = |x-y|$.
    \end{example} 

    \begin{example}
        Recall that a cts function on a closed bounded interval is bounded.
        Define $C([a, b]) = \{f:[a, b] \to \mathbb{R} \mid f \text{ cts}\}$.
        This is a subspace of $B([a, b])$, \cref{exm:2.3}.
        That is $C([a, b])$ is a metric space with the uniform metric $\mathcal{L}(f, g) = \sup_{x \in [a, b]} |f(x) - g(x)|$
    \end{example} 

    \begin{example}
        The empty metric space $X = \emptyset$ with the empty metric.
    \end{example} 

    Could maybe define different metrics on the same set:
    \begin{example}
        The $\ell_1$ metric on $\mathbb{R}^n$: $d(x, y) = \sum_{i=1}^{n} |x_i - y_i|$.
    \end{example} 

    \begin{example}
        The $\ell_\infty$ metric on $\mathbb{R}^n$: $d(x, y) = \max_{i} |x_i - y_i|$.\footnote{Proof of triangle inequality similar to \cref{exm:2.3}}
    \end{example} 

    \begin{example}
        On $C([a, b])$ we can define the $L_1$ metric: $d(f, g) = \int_a^b |f-g|$.
    \end{example} 

    \begin{example}
        $X = \mathbb{C}$ with 
        \begin{align*}
            d(z, w) = \begin{cases}
                0 & z = w \\
                |z| + |w| & z \neq w.
            \end{cases}.
        \end{align*}

        First two conditions of a metric hold obviously, for triangle inequality we need $d(u, w) \leq d(u, v) + d(v, w)$.

        \begin{enumerate}
            \item If $u = w$, LHS = 0 \checkmark
            \item If $u = v$ or $v = w$ then LHS = RHS \checkmark
            \item If $u, v, w$ distinct:
            \begin{align*}
                LHS &= |u| + |w| \\
                RHS &= |u| + |w| + 2|v| \checkmark
            \end{align*} 
        \end{enumerate} 

        This metric is often called the British Rail metric or SNCF metric, you can think of it as for distinct points you have to travel through the origin.
        {\par 
    \centering 
    \includegraphics[height=5cm]{02-britishrail} 
        \par}
    \end{example} 

    \begin{example}
        Let $X$ be any set.
        Define a metric $d$ on $X$ by
        \begin{align*}
            d(x, y) = \begin{cases}
                0 & x = y \\
                1 & x \neq y.
            \end{cases} 
        \end{align*} 

        Easy to check this works.
        This is called the \underline{discrete metric} on $X$.
    \end{example} 

    \begin{example}
        Let $\mathbb{X} = \mathbb{Z}$.
        Let $p$ be a prime.
        The $p$-adic metric on $\mathbb{Z}$ is the metric $d$ defined by:
        \begin{align*}
            d(x, y) = \begin{cases}
                0 & x = y \\
                p^{-a} & \text{if $x \neq y$ and $x - y = p^a m$ with}
            \end{cases} 
        \end{align*} 
        Two numbers are close if difference is divisible by a large power of $p$.
    \end{example} 

    \section{Topological Spaces}
    \part{Generalizing differentiation}
\end{document}