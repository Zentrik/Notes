\section{Propositional Logic}

We build a language consisting of statements/propositions; \\
We will assign truth values to statements; \\
We build a deduction system so that we can prove statements that are true (and only those).
These are also features of more complicated languages.

\subsection{Languages}

Let $P$ be a set of \vocab{primitive propositions}.
Unless otherwise stated, we let $P = \qty{p_1, p_2, \dots}$ (i.e. countable).
The \vocab{language} $L = L(P)$ is a set of \vocab{propositions} (or \vocab{compound propositions}) and is defined inductively by
\begin{enumerate}
    \item if $p \in P$, then $p \in L$;
    \item $\bot \in L$, where the symbol $\bot$ is read `false'/ `bottom';
    \item if $p, q \in L$, then $(p \implies q) \in L$.
\end{enumerate}

\begin{example}
    $((p_1 \implies p_2) \implies (p_1 \implies p_3)) \in L$.
    $(p_4 \implies \bot) \in L$.

    If $p \in L$ then $((p \implies \bot) \implies \bot) \in L$.
\end{example}

\begin{remark}
    Note that the phrase `$L$ is defined inductively' means more precisely the following.
    Let $L_1 = P \cup \qty{\bot}$, and define $L_{n+1} = L_n \cup \qty{(p \implies q) : p, q \in L_n}$.
    We set $L = \bigcup_{n=1}^\infty L_n$.

    Note that the elements of $L$ are just finite strings of symbols from the alphabet $P \cup \qty{(, ), \implies, \bot}$.
    Brackets are only given for clarity; we omit those that are unnecessary, and may use other types of brackets such as square brackets. \\
    We can prove that $L$ is the smallest (w.r.t. inclusion) subset of the set $\Sigma$ of all finite strings in $P \cup \qty{(, ), \implies, \bot}$ s.t. the properties of a language hold. \\
    Note that $L \subsetneq \Sigma$. E.g. $\implies p_1 p_3 ( \in \Sigma \setminus L$.

    Note that the introduction rules for the language are injective and have disjoint ranges, so there is exactly one way in which any element of the language can be constructed using rules (i) to (iii).

    Every $p \in L$ is uniquely determined by the properties of a language above, i.e. either $p \in P$ or $p = \bot$ or $\exists$ unique $q, r \in L$ s.t. $p = (q \implies r)$.
\end{remark}

We can now introduce the abbreviations $\neg, \wedge, \vee, \top$, which are not, and, or and true/top respectively, defined by
\begin{notation}
    \begin{align*}
        \neg p = (p \implies \bot);\quad p \vee q = \neg p \implies q;\quad p \wedge q = \neg (p \implies \neg q), \top = (\bot \implies \bot)
    \end{align*}
\end{notation}

\subsection{Semantic implication}
\begin{definition}[Valuation]
    A \vocab{valuation} is a function $v \colon L \to \qty{0,1}$ s.t.
    \begin{enumerate}
        \item $v(\bot) = 0$;
        \item If $p, q \in L$ then
        $v(p \implies q) = \begin{cases}
            0 & v(p) = 1 \text{ and } v(q) = 0 \\
            1 & \text{else}
        \end{cases}$
    \end{enumerate}
\end{definition}

\begin{example}
    If $v(p_1) = 1$, $v(p_2) = 0$.
    Then \begin{align*}
        v\qty( \underbracket{(\bot \implies p_1)}_{1} \implies \underbracket{(p_1 \implies p_2)}_{0}) = 0
    \end{align*}
\end{example}

\begin{remark}
    On $\qty{0,1}$, we can define the constant $\bot = 0$ and the operation $\implies$ in the obvious way.
    Then, a valuation is precisely a mapping $L \to \qty{0,1}$ preserving all structure, so it can be considered a homomorphism.
\end{remark}

\begin{proposition}
    Let $v, v' \colon L \to \qty{0,1}$ be valuations that agree on the primitives $p_i$.
    Then $v = v'$.
    Further, any function $w \colon P \to \qty{0,1}$ extends to a valuation $v : L \to \qty{0, 1}$ s.t. $v:_P = w$.
\end{proposition}

\begin{remark}
    This is analogous to the definition of a linear map by its action on the basis vectors.
\end{remark}

\begin{proof}
    Clearly, $v, v'$ agree on $L_1$ as $v(\bot) = v'(\bot) = 0$, the set of elements of the language of length 1.
    If $v, v'$ agree at $p, q \in L_n$, then they agree at $p \implies q$.
    So by induction, $v, v'$ agree on $L_{n+1}$ for all $n$, and hence on $L$.

    Let $v(p) = w(p)$ for all $p \in P$, and $v(\bot) = 0$ to obtain $v$ on the set $L_1$.
    Assuming $v$ is defined on $p, q \in L_n$ we can define it at $p \implies q$ in the obvious way.
    This defines $v$ on $L_{n+1}$, hence $v$ is defined on $\cup L_n = L$.
    By construction, $v$ is a valuation on $L$ and $v :_P = w$.
\end{proof}

\begin{example}
    Let $v$ be the valuation with $v(p_1) = v(p_3) = 1$, and $v(p_n) = 0$ for all $n \neq 1, 3$.
    Then, $v((p_1 \implies p_3) \implies p_2) = 0$.
\end{example}

\begin{definition}[Tautology]
    A \vocab{tautology} is $t \in L$ s.t. $v(t) = 1 \; \forall$ valuations $v$.
    We write $\models t$.
\end{definition}

\begin{example}
    $p \implies (q \implies p)$ (a true statement is implied by any true statement).
    \begin{align*}
\begin{array}{cccc}
        v(p) & v(q) & v(q \implies p) & v(p \implies (q \implies p)) \\
        0 & 0 & 1 & 1 \\
        0 & 1 & 0 & 1 \\
        1 & 0 & 1 & 1 \\
        1 & 1 & 1 & 1
    \end{array}
\end{align*}
    Since the right-hand column is always 1, $\models p \implies (q \implies p)$.
\end{example}

\begin{example}[Law of Excluded Middle]
    $\neg \neg p \implies p$, which expands to $((p \implies \bot) \implies \bot) \implies p$.
    \begin{align*}
\begin{array}{cccc}
        v(p) & v(\neg p) & v(\neg \neg p) & v(\neg \neg p \implies p) \\
        0 & 1 & 0 & 1 \\
        1 & 0 & 1 & 1
    \end{array}
\end{align*}
    Hence $\models \neg \neg p \implies p$.
\end{example}

\begin{example}
    $\neg p \vee p$, which expands to $((p \implies \bot) \vee p)$.
    \begin{align*}
\begin{array}{cccc}
        v(p) & v(\neg p) & v(\neg p \vee p) \\
        0 & 1 & 1 \\
        1 & 0 & 1
    \end{array}
\end{align*}
    Hence $\models \neg p \vee p$.
\end{example}

\begin{example}
    $(p \implies (q \implies r)) \implies ((p \implies q) \implies (p \implies r))$.
    Suppose this is not a tautology.
    Then we have a valuation $v$ s.t. $v(p \implies (q \implies r)) = 1$ and $v((p \implies q) \implies (p \implies r)) = 0$.
    Hence, $v(p \implies q) = 1, v(p \implies r) = 0$, so $v(p) = 1, v(r) = 0$, giving $v(q) = 1$, but then $v(p \implies (q \implies r)) = 0$ contradicting the assumption.
\end{example}

\begin{definition}[Semantic Implication]
    Let $S \subseteq L$ and $t \in L$.
    We say $S$ \vocab{entails} or \vocab{semantically implies} $t$, written $S \models t$, if for every valuation $v$ on $L$, $v(s) = 1 \; \forall s \in S \implies v(t) = 1$.
\end{definition}

\begin{example}
    $\qty{p, p \implies q} \models q$.
\end{example}

\begin{example}
    Let $S = \qty{p \implies q, q \implies r}$, and let $t = p \implies r$.
    Suppose $S \not\models t$, so there is a valuation $v$ s.t. $v(p \implies q) = 1, v(q \implies r) = 1, v(p \implies r) = 0$.
    Then $v(p) = 1, v(r) = 0$, so $v(q) = 1$ and $v(q) = 0$ \Lightning.
\end{example}

\begin{definition}[Model]
    Given $t \in L$, say a valuation \vocab{$v$ is a model for $t$} (or \vocab{$t$ is true in $v$}) if $v(t) = 1$.
\end{definition}

\begin{definition}[Model]
    We say that \vocab{$v$ is a model of $S$} in $L$ if $v(s) = 1$ for all $s \in S$.
\end{definition}

Thus, $S \models t$ is the statement that every model of $S$ is also a model of $t$/ $t$ is true in every model of $S$.

\begin{remark}
    The notation $\models t$ is equivalent to $\varnothing \models t$.
\end{remark}

\subsection{Syntactic implication}
For a notion of proof, we require a system of axioms and deduction rules.
As axioms, we take (for any $p, q, r \in L$),
\begin{enumerate}
    \item $p \implies (q \implies p)$;
    \item $(p \implies (q \implies r)) \implies ((p \implies q) \implies (p \implies r))$;
    \item $((p \implies \bot) \implies \bot) \implies p$.
\end{enumerate}

\begin{remark}
    Sometimes, these three axioms are considered axiom \vocab{schemes}, since they are really a different axiom for each $p, q, r \in L$.

    These are all tautologies.
\end{remark}
For deduction rules, we will have only the rule \vocab{modus ponens (MP)}, that from $p$ and $p \implies q$ one can deduce $q$.

\begin{definition}[Proof]
    Let $S \subseteq L$, $t \in L$.
    A \vocab{proof of $t$ from $S$} is a finite sequence $t_1, \dots, t_n$ of propositions in $L$ s.t. $t_n = t$ and every $t_i$ is either
    \begin{enumerate}
        \item an axiom;
        \item an element of $S$ ($t_i$ is a premise or hypothesis); or
        \item follows by MP, where $t_j = p$ and $t_k = p \implies q$ where $j, k < i$.
    \end{enumerate}
    We say that $S$ is the set of \vocab{premises} or \vocab{hypotheses}, and $t$ is the \vocab{conclusion}.

    We say $S$ \vocab{proves} or \vocab{syntactically implies} $t$, written $S \vdash t$, if there exists a proof of $t$ from $S$.
\end{definition}

\begin{example}
    We will show $\qty{p \implies q, q \implies r} \vdash (p \implies r)$.
    \begin{enumerate}
        \item $q \implies r$ (hypothesis)
        \item $(q \implies r) \implies (p \implies (q \implies r))$ (axiom 1)
        \item $p \implies (q \implies r)$ (modus ponens on lines 1, 2)
        \item $(p \implies (q \implies r)) \implies ((p \implies q) \implies (p \implies r))$ (axiom 2)
        \item $(p \implies q) \implies (p \implies r)$ (modus ponens on lines 3, 4)
        \item $p \implies q$ (hypothesis)
        \item $p \implies r$ (modus ponens on lines 5, 6)
    \end{enumerate}
\end{example}

\begin{definition}[Theorem]
    If $\varnothing \vdash t$, we say $t$ is a \vocab{theorem}, written $\vdash t$.
\end{definition}

\begin{example}
    $\vdash (p \implies p)$.
    \begin{enumerate}
        \item $(p \implies ((p \implies p) \implies p)) \implies ((p \implies (p \implies p)) \implies (p \implies p))$ (axiom 2)
        \item $p \implies ((p \implies p) \implies p)$ (axiom 1)
        \item $(p \implies (p \implies p)) \implies (p \implies p)$ (modus ponens on lines 1, 2)
        \item $p \implies (p \implies p)$ (axiom 1)
        \item $p \implies p$ (modus ponens on lines 3, 4)
    \end{enumerate}
\end{example}

\subsection{Deduction theorem}
\begin{theorem}[Deduction Theorem] \label{thm:ded}
    Let $S \subseteq L$, and $p, q \in L$.
    Then $S \vdash (p \implies q)$ iff $S \cup \qty{p} \vdash q$.
\end{theorem}

% Intuitively, provability corresponds to the implication connective in $L$.
\begin{remark}
    This show `$\implies$' really does behave like implication in formal proofs.
\end{remark}

\begin{proof}
    $(\implies)$: Given a proof of $p \implies q$ from $S$, add the line $p$ to the hypothesis and deduce $q$ from modus ponens, to obtain a proof of $q$ from $S \cup \qty{p}$.

    $(\Leftarrow)$: Suppose we have a proof of $q$ from $S \cup \qty{p}$.
    Let $t_1, \dots, t_n$ be the lines of the proof.
    We will prove that $S \vdash (p \implies t_i)$ for all $i$ by induction.
    \begin{itemize}
        \item If $t_i$ is an axiom, we write $t_i$ (axiom); $t_i \implies (p \implies t_i)$ (axiom 1); $p \implies t_i$ (modus ponens).
        \item If $t_i \in S$, we write $t_i$ (hypothesis); $t_i \implies (p \implies t_i)$ (axiom 1); $p \implies t_i$ (modus ponens).
        \item If $t_i = p$, we write the proof of $\vdash p \implies p$ given above.
        \item Suppose $t_i$ is obtained by modus ponens from $t_j$ and $t_k = t_j \implies t_i$ where $j, k < i$.
        We may assume by induction that $S \vdash p \implies t_j$ and $S \vdash p \implies (t_j \implies t_i)$.
        We write
        \begin{enumerate}
            \item $(p \implies (t_j \implies t_i)) \implies ((p \implies t_j) \implies (p \implies t_i))$ (axiom 2)
            \item $(p \implies t_j) \implies (p \implies t_i)$ (modus ponens)
            \item $p \implies t_i$ (modus ponens)
        \end{enumerate}
        giving $S \vdash p \implies t_i$.
    \end{itemize}
\end{proof}

\begin{example}
    Consider $\qty{p \implies q, q \implies r} \vdash p \implies r$.
    By the \nameref{thm:ded}, it suffices to prove $\qty{p \implies q, q \implies r, p} \vdash r$, which is obtained easily from modus ponens.
\end{example}

\subsection{Soundness}
We aim to show $S \models t$ iff $S \vdash t$.
The direction $S \vdash t$ implies $S \models t$ is called \vocab{soundness}, which is a way of verifying that our axioms and deduction rule make sense.
The direction $S \models t$ implies $S \vdash t$ is called \vocab{adequacy}, which states that our axioms are powerful enough to deduce everything that is (semantically) true.

\begin{proposition}[Soundness Theorem]
    Let $S \subseteq L$ and $t \in L$.
    Then $S \vdash t$ implies $S \models t$.
\end{proposition}

\begin{proof}
    We have a proof $t_1, \dots, t_n$ of $t$ from $S$.
    We aim to show that any model of $S$ is also a model of $t$, so if $v$ is a valuation that maps every element of $S$ to 1, then $v(t) = 1$.

    We show this by induction on the length of the proof.
    $v(p) = 1$ for each axiom $p$ (as axioms are tautologies) and for each $p \in S$.
    Further, $v(t_i) = 1, v(t_i \implies t_j) = 1$, then $v(t_j) = 1$.
    Therefore, $v(t_i) = 1$ for all $i$.
\end{proof}

\subsection{Adequacy}
Consider the case of adequacy where $t = \bot$.
If our axioms are adequate, $S \models \bot$ implies $S \vdash \bot$.
We say $S$ is \vocab{consistent} if $S \not\vdash \bot$ and \vocab{inconsistent} if $S \vdash \bot$.
Therefore, in an adequate system, if $S$ has no models then $S$ is inconsistent; equivalently, if $S$ is consistent then it has a model.

In fact, the statement that consistent axiom sets have a model implies adequacy in general.
Indeed, if $S \models t$, then $S \cup \qty{\neg t}$ has no models, and so it is inconsistent by assumption.
Then $S \cup \qty{\neg t} \vdash \bot$, so $S \vdash \neg t \implies \bot$ by the deduction theorem, giving $S \vdash t$ by axiom 3.

We aim to construct a model of $S$ given that $S$ is consistent.
Intuitively, we want to write
\begin{align*}
v(t) = \begin{cases}
    1 & t \in S \\
    0 & t \not\in S
\end{cases}
\end{align*}
but this does not work on the set $S = \qty{p_1, p_1 \implies p_2}$ as it would evaluate $p_2$ to false.

We say a set $S \subseteq L$ is \vocab{deductively closed} if $p \in S$ whenever $S \vdash p$.
Any set $S$ has a \vocab{deductive closure}, which is the (deductively closed) set of statements $\qty{t \in L : S \vdash t}$ that $S$ proves.
If $S$ is consistent, then the deductive closure is also consistent.
Computing the deductive closure before the valuation solves the problem for $S = \qty{p_1, p_1 \implies p_2}$.
However, if a primitive proposition $p$ is not in $S$, but $\neg p$ is also not in $S$, this technique still does not work, as it would assign false to both $p$ and $\neg p$.

\begin{theorem}[Model Existence Lemma] \label{thm:mod}
    Every consistent set $S \subseteq L$ has a model.
\end{theorem}

\begin{remark}
    We use the fact that $P$ is a countable set in order to show that $L$ is countable.
    The result does in fact hold if $P$ is uncountable, but requires Zorn's Lemma and will be proved in Chapter 3.
    Some sources call this theorem the `completeness theorem'.
\end{remark}

\begin{proof}
    First, we claim that for any consistent $S \subseteq L$ and proposition $p \in L$, either $S \cup \qty{p}$ is consistent or $S \cup \qty{\neg p}$ is consistent.
    If this were not the case, then $S \cup \qty{p} \vdash \bot$, and also $S \cup \qty{\neg p} \vdash \bot$.
    By the deduction theorem, $S \vdash p \implies \bot$ and $S \vdash (\neg p) \implies \bot$.
    But then $S \vdash \neg p$ and $S \vdash \neg\neg p$, so $S \vdash \bot$ contradicting consistency of $S$.

    Now, $L$ is a countable set as each $L_n$ is countable, so we can enumerate $L$ as $t_1, t_2, \dots$.
    Let $S_0 = S$, and define $S_1 = S_0 \cup \qty{t_1}$ or $S_1 = S_0 \cup \qty{\neg t_1}$, chosen s.t. $S_1$ is consistent.
    Continuing inductively, define $\overline S = \bigcup_{i} S_i$. \\
    Then, $\forall t \in L$, either $t \in \overline S$ or $\neg t \in \overline S$. \\
    Note that $\overline S$ is consistent since proofs are finite; indeed, if $\overline S \vdash \bot$, then this proof uses hypotheses only in $S_n$ for some $n$, but then $S_n \vdash \bot$ contradicting consistency of $S_n$. \\
    Note also that $\overline S$ is deductively closed, so if $\overline S \vdash p$, we must have $p \in \overline S$; otherwise, $\neg p \in \overline S$ so $\overline S \vdash \neg p$, giving $\overline S \vdash \bot$ by MP, contradicting consistency of $\overline S$.

    Now, define the function
    \begin{align*}
    v(t) = \begin{cases}
            1 & t \in \overline S \\
            0 & t \not\in \overline S
        \end{cases}
    \end{align*}
    We show that $v$ is a valuation, then the proof is complete as $v(s) = 1$ for all $s \in S$.
    Since $\overline S$ is consistent, $\bot \not\in \overline S$, so $v(\bot) = 0$.

    Suppose $v(p) = 1, v(q) = 0$.
    Then $p \in \overline S$ and $q \not\in \overline S$, and we want to show $(p \implies q) \not\in \overline S$.
    If this were not the case, we would have $(p \implies q) \in \overline S$ and $p \in \overline S$, so $q \in \overline S$ as $\overline S$ is deductively closed.

    Now suppose $v(q) = 1$, so $q \in \overline S$, and we need to show $(p \implies q) \in \overline S$.
    Then $\overline S \vdash q$, and by axiom 1, $\overline S \vdash q \implies (p \implies q)$.
    Therefore, as $\overline S$ is deductively closed, $(p \implies q) \in \overline S$.

    Finally, suppose $v(p) = 0$, so $p \not\in \overline S$, and we want to show $(p \implies q) \in \overline S$.
    We know that $\neg p \in \overline S$, so it suffices to show that $(p \implies \bot) \vdash (p \implies q)$.
    By the deduction theorem, this is equivalent to proving $\qty{p, p \implies \bot} \vdash q$, or equivalently, $\bot \vdash q$.
    But by axiom 1, $\bot \implies (\neg q \implies \bot)$ where $(\neg q \implies \bot) = \neg\neg q$, so the proof is complete by axiom 3.
\end{proof}

\begin{corollary}[Adequacy]
    Let $S \subseteq L$ and let $t \in L$, s.t. $S \models t$.
    Then $S \vdash t$.
\end{corollary}

\begin{proof}
    $S \cup \qty{\neg t} \models \bot$, so \nameref{thm:mod}, $S \cup \qty{\neg t} \vdash \bot$.
    Then by \nameref{thm:ded} $S \vdash \neg \neg t$.
    $\neg \neg t \implies t$ by Axiom 3 and so by MP $S \vdash t$.
\end{proof}

\subsection{Completeness}
\begin{theorem}[Completeness Theorem for Propositional Logic]
    Let $S \subseteq L$ and $t \in L$.
    Then $S \models t$ iff $S \vdash t$.
\end{theorem}

\begin{proof}
    Follows from soundness and adequacy.
\end{proof}

\begin{theorem}[Compactness Theorem]
    Let $S \subseteq L$ and $t \in L$ with $S \models t$.
    Then there exists a finite subset $S' \subseteq S$ s.t. $S' \models t$.
\end{theorem}

\begin{proof}
    Trivial after applying the completeness theorem, since proofs depend on only finitely many hypotheses in $S$.
\end{proof}

\begin{corollary}[Compactness Theorem, Equivalent Form]
    Let $S \subseteq L$.
    Then if every finite subset $S' \subseteq S$ has a model, then $S$ has a model.
\end{corollary}

\begin{proof}
    Let $t = \bot$ in the compactness theorem.
    Then, if $S \models \bot$, some finite $S' \subseteq S$ has $S' \models \bot$.
    But this is not true by assumption, so there is a model for $S$.
\end{proof}

\begin{remark}
    This corollary is equivalent to the more general compactness theorem, since the assertion that $S \models t$ is equivalent to the statement that $S \cup \qty{\neg t}$ has no model, and $S' \models t$ is equivalent to the statement that $S' \cup \qty{\neg t}$ has no model.
\end{remark}

\begin{note}
    The use of the word compactness is more than a fanciful analogy.
    See Sheet 1.
\end{note}

\begin{theorem}[Decidability Theorem]
    Let $S \subseteq L$, $S$ finite and $t \in L$.
    Then, there is an algorithm to decide (in finite time) if $S \vdash t$.
\end{theorem}

\begin{proof}
    Trivial after replacing $\vdash$ with $\models$, and checking all valuations by drawing the relevant truth tables.
\end{proof}
