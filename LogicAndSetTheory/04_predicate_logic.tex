\section{Predicate Logic}
\subsection{Languages}
In Propositional Logic we has a set $P$ of primitive propositions and then we combined them using logical connectives $\implies$, $\bot$, ($\wedge$, $\vee$, $\neg$, $\top$) to form the language $L = L(P)$ of all (compound) propositions.

We attached no meaning to primitive propositions. \\
\underline{Aim}: To develop languages to describe a wide range of mathematical theorems.
We will replace primitive propositions with non mathematical statements.

\begin{example}
    In language of groups:
    \begin{align*}
        m(x, m(y, z)) &= m(m(x, y), z) \\
        m(x, i(x)) = e
    \end{align*}
    In language of posets: $x \leq y$. \\
    This will need variables $(x, y, z, \dots)$, operation symbols  ($m, i, e$ with arities $2, 1, 0$ respectively) and predicates (e.g. $\leq$ with arity $2$).

    We will then combine these to build formulae: \\
    In language of cosets,
    \begin{align*}
        (\forall x) (\forall y) (\forall z) \qty( (x \leq y \wedge y \leq z) \implies (x \leq z) )
    \end{align*}
    In language of groups, $(\forall x) \qty( m(x, i(x)) = e)$.

    Valuations will be replaced by a structure, a set $A$, and ``truth-functions'' $p_A : A^n \to {0, 1}$ for every formula $p$.

    If we have set $S$ of formulae, a model $S$ is a structure satisfying all $p \in S$. \\
    $S \models t$ will be same as in Section 1. \\
    $S \vdash t$ will be same as in Section 1 but more complex.
\end{example}

% Recall that a \vocab{group} is a set $A$ equipped with functions $m \colon A^2 \to A$ of arity 2, and $i \colon A^1 \to A$ of arity 1, and a constant $e \in A$ which can be viewed as a function $A^0 \to A$ of arity 0, s.t. a set of axioms hold.
% A \vocab{poset} is a set $A$ equipped with a relation $(\leq) \subseteq A^2$ of arity 2, s.t. a set of axioms hold.
% Other algebraic structures can be described in the same way.

A language in first-order logic is specified by the disjoint set $\Omega$ (set of operation symbols) and $\Pi$ (set of predication) together with an arity function $\alpha \colon \Omega \cup \Pi \to \mathbb N_0 = \qty{0} \cup \mathbb{N}$.
The language $L = L(\Omega, \Pi, \alpha)$ consists of the following:
\vocab{Variables} a countably infinite sets disjoint of $\Omega$, $\Pi$.
We denote variables as $x_1, x_2, x_3, \dots$ (or $x, y, z, \dots$). \\
\vocab{Terms} are defined inductively by
\begin{enumerate}
    \item each variable is a term;
    \item if $f \in \Omega$ with $\alpha(f) = n$ and terms $t_1, \dots, t_n$, then $f\ t_1\dots\ t_n$ is a term (could write $f(t_1, \dots, t_n)$).
\end{enumerate}
The \vocab{atomic formulae} are defined inductively by
\begin{enumerate}
    \item for terms $s, t$, $(s = t)$ is an atomic formula;
    \item if $\varphi \in \Pi$ with $\alpha(\varphi) = n$ and terms $t_1, \dots, t_n$, then $\varphi(t_1, \dots, t_n)$ is an atomic formula.
\end{enumerate}
The \vocab{formulae} are defined inductively by
\begin{enumerate}
    \item each atomic formula is a formula;
    \item $\bot$ is a formula;
    \item if $p$ and $q$ are formulae then $(p \implies q)$ is a formula;
    \item if $p$ is a formula and the variable $x$ has a \vocab{free occurrence in $p$}, then $(\forall x) p$ is a formula.
\end{enumerate}

The \vocab{language} $L = L(\Omega, \Pi, \alpha)$ is the set of formulae.

\begin{definition}[Constant]
    Every operation symbol of arity $0$ is a term, and called a \vocab{constant}.
\end{definition}

\begin{example}
    In the language of groups, $\Omega = \qty{m, i, e}$ and $\Pi = \varnothing$ with $\alpha(m) = 2, \alpha(i) = 1, \alpha(e) = 0$.
    $m(x_1, x_2), m(x_1, i(x_2)), e, m(e, e), mxmyz, mmxyz, mxix$ are examples of terms of the language.
    $e = m(\ell, e), m(x,y) = m(y,x)$ are atomic formulae.
\end{example}

\begin{example}
    In the language of posets, $\Omega = \varnothing$ and $\Pi = \qty{\leq}$ with $\alpha(\leq) = 2$.
    $x = y, x \leq y$ are atomic formulae.
    Technically, $x \leq y$ is written $\leq(x, y)$.
\end{example}

\begin{example}
    In the language of groups, $(\forall x) (m(x,x) = e)$ is a formula.
    Another formula is $m(x,x) = e \implies (\exists y) (m(y,y) = x)$.
\end{example}

\begin{remark}
    A formula is a certain finite string of symbols from the set of variables, $\Omega$, $\Pi, \qty{(, ), \implies, \bot, =, \forall}$; it has no intrinsic semantics.
    We define $\neg p, p \wedge q, p \vee q$ in the usual way.
    We define $(\exists x) p$ to mean $\neg(\forall x) (\neg p)$.
\end{remark}

A term is \vocab{closed} if it contains no variables.
For example, $e, m(e,i(e))$ are closed in the language of groups, but $m(x,i(x))$ is not closed.

\begin{definition}[Free/ Bound Occurence]
    An occurrence of a variable $x$ in a formula $p$ is always \vocab{free} except if $p = (\forall x) q$ in which the $\forall x$ quantifier \vocab{binds} every free occurrence of $x$ and then such occurrences of $x$ are called \vocab{bound occurences}\footnote{The formal defn is by induction on $L$}.
\end{definition}

\begin{example}
    In the formula $(\forall x)(m(x,x) = e)$, each occurrence of $x$ is bound. \\
    In $m(x,x) = e \implies (\exists y)(m(y,y) = x)$, the occurrences of $x$ are free and the occurrences of $y$ are bound. \\
    In the formula $m(x,x) = e \implies (\forall x)(\forall y)(m(x,y) = m(y,x))$, the occurrences of $x$ on the left hand side are free, and the occurrences of $x$ on the right hand side are bound.

    \begin{enumerate}
        \item $(\exists x) (m(x, x) = y) \implies (\forall z) \neg (mmzzz = y)$. The $x$ (not the $x$ in $\exists x$) are bound, all $y$s are free, the $z$ (not the $z$ in $\forall z$) are bound.
        \item $(\forall x) (\forall y) (\forall z) (mm xyz = mx myz)$, this is the associativity law. There are no free variables.
        \item $(\exists x)(m xx = y) \implies (\forall y) (\forall x) (m yz = m zy)$. The $y$ on the LHS is free, the $y$s on the RHS are bound.
        This is technically a correct formula, but in mathematical practice we avoid this.
        \item In the language of posets: $(\forall x) (\forall y) \qty(\qty((x \leq y) \wedge (y \leq x)) \implies (x = y))$ has no free variables.
    \end{enumerate}
\end{example}

\begin{definition}[Sentence]
    A \vocab{sentence} is a formula with no free variables.
\end{definition}

\begin{definition}[Free]
    A variable $x$ in a formula is \vocab{free} if it has a free occurrence in $p$.
    Let $\FV(p)$ denote the set of free variables in $p$.
\end{definition}

\begin{example}
    For instance, $(\forall x)(m(x,x) = e)$ is a sentence, and $(\forall x)(m(x,x) \implies (\exists y)(m(y,y) = x))$ is a sentence. \\
    In the language of posets, $(\forall x)(\exists y)(x \geq y \wedge \neg(x = y))$ is a sentence.
\end{example}

For a formula $p$, term $t$, and variable $x$, the \vocab{substitution} $p[t/x]$ is obtained from $p$ by replacing every free occurrence of $x$ with $t$.
For example,
\begin{align*}
    p = (\exists y)(m(y,y) = x);\quad p[e/x] = (\exists y)(m(y,y) = e)
\end{align*}

\subsection{Semantic implication}
\begin{definition}[Structure]
    Let $L = L(\Omega, \Pi, \alpha)$ be a first-order language.
    An \vocab{structure} in $L$ or \vocab{$L$-structure} is
    \begin{itemize}
        \item a nonempty set $A$;
        \item for each $f \in \Omega$, a function $f_A \colon A^n \to A$ where $n = \alpha(f)$;
        \item for each $\varphi \in \Pi$, a subset $\varphi_A \subseteq A^n$ where $n = \alpha(\varphi)$\footnote{Equivalently $\phi_A : A^n \to \qty{0, 1}$ by identifying a set with its indicator fcn.}.
    \end{itemize}
\end{definition}

\begin{remark}
    We will see later why the restriction that $A$ is nonempty is given here.
\end{remark}

\begin{example}
    In the language of groups, a structure is a nonempty set $A$ with fcns $m_A \colon A^2 \to A, i_A \colon A \to A, e_A \in A$\footnote{$A^0$ is the singleton set}.

    Such a structure may not be a group, as we have not placed any axioms on $A$.
\end{example}

\begin{example}
    In the language of posets, a structure is a nonempty set $A$ with a relation $(\leq_A) \subseteq A^2$.
    This is not yet a poset.
\end{example}

\underline{Next Step}: to define for a formula $p$ what it means that `$p$ is satisfied in $A$'.

\begin{example}
    $p = (\forall x)(m x i x = e)$ in the language of groups.
    $p$ satisfied in structure $A$ should mean that $\forall a \in A$ we have $m_A(a, i_A(a)) = e_A$.
\end{example}

Let $A$ be an $L$-structure.
A term $t$ in $L$ with $FV(t) \subseteq \qty{x_1, \dots, x_n}$ has \vocab{interpretation} $t_A : A^n \to A$ defined as follows:
\begin{itemize}
    \item If $t = x_i$ for $1 \leq i \leq n$ then $t_A(a_1, \dots, a_n) = a_i$.
    \item If $t = \omega t_1 \dots t_m$ with $\omega \in \Omega$, $m = \alpha(w)$, $t_1, \dots, t_m$ terms, then $t_A(a_1, \dots, a_n) = \omega_A((t_1)_A (a_1, \dots, a_n), \dots, (t_m)_A (a_1, \dots, a_n))$.
\end{itemize}

\begin{example}
    In groups $t = m x_1 m x_2 x_3$, $t_A(a_1, a_2, a_3) = m_A(a_1, m_A(a_2, a_3))$.
\end{example}

Next interpret a formula $p$ with $FV(p) \subseteq \qty{x_1, \dots, x_n}$ as a subset $p_A \subset A^n$ or equivalently a fcn $p_A : A^n \to \qty{0, 1}$.
\begin{itemize}
    \item If $p = (s = t)$, $p_A(a_1, \dots, a_n) = 1 \iff s_A(a_1, \dots, a_n) = t_A(a_1, \dots, a_n)$.
    \item If $p = \phi t_1, \dots, t_m$, with $\phi \in \Pi$, $m = \alpha(\phi)$, $t_1, \dots, t_m$ terms.
    Then $p_A(a_1, \dots, a_n) = 1$ iff $\phi_A((t_1)_A(a_1, \dots, a_n), \dots, (t_m)_A(a_1, \dots, a_n)) = 1$.
    \item $\bot_A \equiv 0$.
    \item If $p = (q \implies r)$, then $p_A(a_1, \dots, a_n) = 0$ iff $q_A(a_1, \dots, a_n) = 1$ and $r_A(a_1, \dots, a_n) = 0$.
    \item If $p = (\forall x_{n+1}) q$ where $FV(q) \subseteq \qty{x_1, \dots, x_{n+1}}$, \\
    $p_A = \qty{(a_1, \dots, a_n) \in A^n : (a_1, \dots, a_{n+1}) \in q_A \; \forall a_{n + 1} \in A}$.
\end{itemize}

\begin{example}
    In groups, $p = mm xyz = mx myz$. \\
    $p_A = \qty{(a, b, c) \in A^3 : m_A(m_A(a, b), c) = m_A(a, m_A(b, c))}$. \\
    $q = (\forall x) (\forall y) (\forall z) p$ then $q_A = 1 \iff p_A = A^3$.
\end{example}

% \begin{remark}
%     For a formula $p$ with free variables, we can define $p_A$ to be the subset of $A^k$ where $k$ is the number of free variables, defined s.t. $x \in p_A$ iff the substitution of $x$ in $p$ is evaluated to 1.
% \end{remark}

\begin{definition}[Satisfied]
    If $p_A \equiv 1$ or $p_A = A^n\footnote{$n$ is the number of free variables in $p$.}$ we say the formula $p$ is \vocab{satisfied} in a $L$-structure $A$. \\
    We also say $p$ \vocab{holds} in $A$, $p$ is \vocab{true} in $A$ or $A$ is a \vocab{model} for $p$.
\end{definition}

\begin{definition}[Theory]
    A \vocab{theory} is a set of sentences in $L$, known as $L$'s \vocab{axioms}.
\end{definition}

\begin{definition}[Model]
    A \vocab{model} for a theory $T$ is an $L$-structure $A$ that is a model $\forall p \in T$.
\end{definition}

\begin{example}[Groups]
    Let $L$ be the language of groups.
    The language is specified by $\Omega = \qty{m, i, e}$, $\Pi = \emptyset$, $\alpha$ is $2, 1, 0$ for $m, i, e$ respectively.

    Let
    \begin{align*}
        T = \{&(\forall x)(\forall y)(\forall z)(m(x,m(y,z)) = m(m(x,y), z)), \\
        &(\forall x)(m(x,e) = x \wedge m(e,x) = x), \\
        &(\forall x)(m(x,i(x)) = e \wedge m(i(x),x) = e)\}
    \end{align*}
    Then, an $L$-structure is a model of $T$ iff it is a group.
    Note that this statement has two assertions; every $L$-structure that is a model of $T$ is a group, and that every group can be turned into an $L$-structure that models $T$.

    We say that $T$ \vocab{axiomatises} the theory of groups or the class of groups.
\end{example}

\begin{example}[Posets]
    Let $L$ be the language of posets.
    $\Omega = \emptyset$, $\Pi = \qty{\leq}$, $\alpha(\leq) = 2$.

    Let
    \begin{align*}
        T = \{&(\forall x) (x \leq x) \\
        &(\forall x) (\forall y) \qty(\qty((x \leq y) \wedge (y \leq x)) \implies (x = y)) \\
        &(\forall x) (\forall y) (\forall z) \qty( \qty( (x \leq y) \wedge (y \leq z)) \implies (x \leq z))\}.
    \end{align*}

    The models are partially ordered sets, i.e. $T$ axiomatises the class of posets.
\end{example}

\begin{example}[Rings with $1$]
    $\Omega = \qty{0, 1, +, \cdot, -}$ with $\alpha(0) = \alpha(1) = 0, \alpha(+) = \alpha(\cdot) = 2, \alpha(-) = 1$ and $\Pi = \emptyset$.
    \begin{align*}
        T = \{
            &(\forall x) (\forall y) (\forall z) ((x + y) + z = x + (y + z)) \\
            &(\forall x) ((x + 0 = x) \wedge (0 + x = x)) \\
            &(\forall x) (x + (-x) = 0) \wedge ((-x) + x = 0) \\
            &(\forall x) (\forall y) (x + y = y + x) \\
            &(\forall x) (\forall y) (\forall z) ((xy)z = x(yz)) \\
            &(\forall x) (1 \cdot x = x \wedge x \cdot 1 = x) \\
            &(\forall x) (\forall y) (\forall z) ((x \cdot (y + z) = xy + xz) \wedge ((x+y) \cdot z = xz + yz))
        \}
    \end{align*}

    The models are rings with $1$.
\end{example}

\begin{example}[Fields]
    We use the same language as in rings with $1$.

    The theory is the same as rings with $1$, plus
    \begin{align*}
        &(\forall x)(\forall y) (x \cdot y = y \cdot x) \\
        &\neg(0 = 1) \\
        &(\forall x) (\neg(x = 0) \implies (\exists y)(x \cdot y = 1))
    \end{align*}

    The models are fields.

    % Then $T$ entails the statement that inverses are unique: $(\forall x)(\neg (x = 0) \implies (\forall y)(\forall z) (y \cdot x = 1 \wedge z \cdot x = 1 \implies y = z))$.
\end{example}

\begin{example}[Graph Theory]
    Let $L$ be the language of graphs, defined by $\Omega = \emptyset$, $\Pi = \qty{a}$ ($a =$ `is adjacent to') and $\alpha(a) = 2$.
    Define $T = \qty{(\forall x)(\neg a(x,x)), (\forall x)(\forall y)(a(x,y) \implies a(y,x))}$.
    Then $T$ axiomatises the class of graphs.
\end{example}

\begin{example}[Propositional Theories]
    $\Omega = \emptyset$, $\Pi$ s.t. $\alpha(p) = 0 \; \forall p \in \Pi$. \\
    A structure is a nonempty set $A$ together with $p_A \subset A^0$ for all $p \in \Pi$ (equivalently $p_A \in \qty{0, 1}$). \\
    A structure is a nonempty set $A$ together with a fcn $v : \Pi \to \qty{0, 1}$.

    Every $p \in \Pi$ is an atomic formula.
    Formula w/o variables are precisely elements of $L(\Pi)$ as defined in section 1, i.e. they are propositions in $\Pi$.
    Interpreting these in a structure $A$ is just a fcn $v : L(\Pi) \to \qty{0, 1}$ obtained from $v : \Pi \to \qty{0, 1}$ as in section 1, i.e. a valuation.

    A \vocab{propositional theory} is a set $S$ of formulae not using variables.
    A model for $S$ is a nonempty set $A$ with a valuation $v : L(\Pi) \to \qty{0, 1}$ s.t. $v(s) = 1 \; \forall s \in S$ (Here $A$ is irrelevant).
\end{example}

\subsection{Semantic Entailment}

\begin{definition}[Semantic Entailment]
    For a set $S$ of sentences (i.e. a theory) and a sentence $t$ (in some first-order language $L$) we say $S$ \vocab{(semantically) entails} $t$ if $t$ is satisfied in every model of $S$.
    We write $S \models t$.
\end{definition}

\begin{example}[Groups]
    Let $S$ be the theory of groups.
    $S \models (\forall x)(x \cdot x = e) \implies (\forall x) (\forall y) (xy = yx)$.
\end{example}

\begin{example}[Fields]
    Let $S$ be the theory of fields.
    $S \models (\forall x) (\neg(x = 0) \implies (\forall y) (\forall z) ((xy = 1 \wedge xz = 1) \implies (y = z)))$.
\end{example}

Next, we want to define $S \models t$ for formulae:
\begin{example}
    Let $T$ be the theory of fields.
    Take $S = T \cup \qty{\neg(x = 0)}$, $t = (\exists y) (xy = 1)$.

    Does $S \models t$?
    Yes.

    Suppose $F$ is a structure in which all members of $S$ are true.
    $F$ is a field and for $u = \neg(x = 0)$, $u_F = \qty{a \in F : a \neq 0_F} = F$ \Lightning.

    Also, we'll soon define ``$S \vdash t$'', then $S \vdash t \iff T \vdash \neg(x = 0) \implies (\exists y)(xy = 1)$.
    This will help motivate our defn.
\end{example}

Let $S$ be a set of formulae and $t$ a formula in a language $L$.
For every variable $x$ that occurs free in $S \cup \qty{t}$, introduce a constant $c_x$ (add it to $\Omega$).
Let $L'$ be our new language. \\
For a formula $p$, let $p'$ be the formula obtained from $p$ by replacing free occurrences of $x$ in $p$ by $c_x$ for every $x$. \\
Let $S' = \qty{s' : s \in S}$.

\begin{definition}[Semantic Entailment]
    Say $S$ \vocab{(semantically) entails} $t$, written $S \models t$, if $S' \models t'$.
\end{definition}

\begin{definition}[Substitution]
    If $x$ occurs free in a formula $p$ and $t$ is a term that contains no variable that occurs bound in $p$, we let the \vocab{substitution} $p[t / x]$ be the formula obtained from $p$ by replacing free occurrences of $x$ in $p$ by $t$.
\end{definition}

\begin{example}
    Let $p = (\exists y)(m(y,y) = x)$ then $p[e/x] = (\exists y)(m(y,y) = e)$.
\end{example}

\begin{example}
    In language of groups:
    $p = (\forall y)(mxx = y)$.
    \begin{itemize}
        \item $t = mzz$, then $p[t / x] = (\forall y)(mmzzmzz = y)$.
        \item $t = mzy$ cannot be used since $y$ is bound in $p$.
        \item $t = mxx$, $p[t / x] = (\forall y)(mmxxmxx = y)$.
    \end{itemize}
\end{example}

\subsection{Syntactic Entailment}
We need to define (logical) axioms and deduction rules in order to construct proofs.

The Axioms (the previous 3, 2 more for ``='', 2 for ``$\forall$'') are:
\begin{enumerate}
    \item $p \implies (q \implies p)$ for formulae $p, q$.
    \item $(p \implies (q \implies r)) \implies ((p \implies q) \implies (p \implies r))$ for formulae $p, q, r$.
    \item $\neg\neg p \implies p$ for each formula $p$.
    \item $(\forall x)(x = x)$ for any variable $x$.
    \item $(\forall x)(\forall y)((x = y) \implies (p \implies p[y/x]))$ for any distinct variables $x, y$ where $y$ is not bound in the formula $p$ and $x \in \FV(p)$.
    \item $((\forall x)p) \implies p[t/x]$ for any variable $x \in \FV(p)$, formula $p$, and term $t$ that has no variable that occurs bound in $p$.
    \item $(\forall x)(p \implies q) \implies (p \implies (\forall x)q)$ for any formulae $p, q$ and variable $x \notin \FV(p)$, $x \in \FV(q)$.
\end{enumerate}

\begin{note}
    Every axiom is a tautology ($t$ is a tautology if $\emptyset \models t$, i.e. $t$ holds in every structure).
\end{note}

We define the following deduction rules.
\begin{enumerate}
    \item \vocab{Modus Ponens (MP)} From $p$, $p \implies q$, we can deduce $q$.
    \item \vocab{Generalisation (Gen)} From $p$ s.t. $x \in FV(p)$, we can deduce $(\forall x)p$ provided that $x$ does not occur free in any premise used in the proof of $p$.
\end{enumerate}

\begin{definition}[Proof]
    Let $S$ be a set of formulae and $p$ a formula.
    A \vocab{proof} of $p$ from $S$ is a finite sequence $t_1, \dots, t_n$ of formulae s.t. $t_n = p$ and $\forall i \; t_i$ is an axiom, a premise or deduced form previous lines (i.e. $\exists j, x < i$ s.t. $t_x = (t_j \implies t_i)$ or $\exists j < i$ s.t. $t_i = (\forall x)t_j$, $x \in \FV(t_j)$ and $\forall k < j$ if $t_k \in S$ then $x \notin \FV(t_k)$). \\
    We say $S$ \vocab{proves} $p$ and write $S \vdash p$.
\end{definition}

\begin{remark}
    Suppose we allow $\emptyset$ as a structure.
    Then $(\forall x) \neg (x = x)$ is satisfied in $\emptyset$ whereas $\bot$ is not.
    So $\qty{(\forall x) \neg (x = x)} \not\models \bot$.
    However, $\qty{(\forall x) \neg (x = x)} \vdash \bot$:
    \begin{enumerate}
        \item $(\forall x) \neg (x = x)$ (premise)
        \item $((\forall x) \neg (x = x)) \implies (x=x)$ (A6)
        \item $\neg (x = x)$ (MP)
        \item $(\forall x) (x = x)$ (A4)
        \item $x = x$ (A6 + MP)
        \item $\bot$ (MP)
    \end{enumerate}
\end{remark}

\begin{example}
    We show $\qty{x = y} \vdash (y = x)$.
    \begin{enumerate}
        \item $(\forall x) (\forall y) (x = y) \implies ((x = z) \implies (y = z))$ (A5)
        \item $(x = y) \implies ((x = z) \implies (y = z))$ (A6 + MP twice)\footnote{Set $p = (\forall y) (x = y) \implies ((x = z) \implies (y = z))$ and $t = x$ so A6 gives $(\forall x) (\forall y) (x = y) \implies ((x = z) \implies (y = z)) \implies (\forall y) (x = y) \implies ((x = z) \implies (y = z))$.
        Then by MP we get $(\forall y) (x = y) \implies ((x = z) \implies (y = z))$.
        Then repeat.}
        \item $x = y$ (premise)
        \item $(x = z) \implies (y = z)$ (MP)
        \item $(\forall z) ((x = z) \implies (y = z))$ (Gen)
        \item $(\forall z) ((x = z) \implies (y = z)) \implies ((x = x) \implies (y = x))$ (A6)
        \item $(x = x) \implies (y = x)$ (MP)
        \item $(\forall x) (x = x)$ (A4)
        \item $x = x$ (A6 + MP)
        \item $y = x$ (MP)
    \end{enumerate}
\end{example}

\begin{example}
    We show $\qty{x = y, x = z} \vdash y = z$ where $x, y, z$ are different variables.
    \begin{enumerate}
        \item $(\forall x)(\forall y)(x = y \implies (x = z \implies y = z))$ (axiom 5)
        \item $\qty((\forall x)(\forall y)(x = y \implies (x = z \implies y = z))) \implies (\forall y)(x = y \implies (x = z \implies y = z))$ (axiom 6)
        \item $(\forall y)(x = y \implies (x = z \implies y = z))$ (modus ponens on lines 1, 2)
        \item $\qty((\forall y)(x = y \implies (x = z \implies y = z))) \implies (x = y \implies (x = z \implies y = z))$ (axiom 6)
        \item $x = y \implies (x = z \implies y = z)$ (modus ponens on lines 3, 4)
        \item $x = y$ (hypothesis)
        \item $x = z \implies y = z$ (modus ponens on lines 5, 6)
        \item $x = z$ (hypothesis)
        \item $y = z$ (modus ponens on lines 7, 8)
    \end{enumerate}
\end{example}

\subsection{Deduction theorem}

\begin{proposition}[Deduction Theorem] \label{prp:predded}
    Let $S$ be a set of formulae, and $p, q$ formulae.
    Then $S \vdash (p \implies q)$ iff $S \cup \qty{p} \vdash q$.
\end{proposition}

\begin{proof}
    $(\implies)$: Write down a proof of $p \implies q$ from $S$, one can establish a proof of $q$ from $S \cup \qty{p} \vdash q$ by writing $p$ and applying modus ponens to the original proof.

    $(\Leftarrow)$:
    Let $t_1, \dots, t_n = q$ be a proof of $q$ from $S \cup \qty{p}$.
    We prove that $S \vdash \qty{p \implies t_i}$ by induction on $p_i$. \\
    \underline{Induction hypothesis} at step $i$: for $j < i$, $S \vdash (p \implies t_j)$ s.t. if the proof of $t_j$ from $S \cup \qty{p}$ did not use any premise in which a variable $x$ occurs free, then the proof of $(p \implies t_j)$ from $S$ does not use any premise where $x$ occurs free.

    To see $S \vdash (p \implies t_i)$ consider the following cases:
    \begin{itemize}
        \item If $t_i \in S$ or an axiom then write
        \begin{enumerate}
            \item $t_i$ (premise or axiom)
            \item $t_i \implies (p \implies t_i)$ (A1)
            \item $p \implies t_i$ (MP)
        \end{enumerate}
        \item If $t_i = p$ write down a proof of $p \implies p$ from $\emptyset$.
        \item If $\exists j, k < i$ s.t. $t_k = (t_j \implies t_i)$ then write
        \begin{enumerate}
            \item $(p \implies (t_j \implies t_i)) \implies ((p \implies t_j) \implies (p \implies t_i))$ (A2)
            \item $p \implies t_k$ (Induction hypothesis)
            \item $(p \implies t_j) \implies (p \implies t_i)$ (MP)
            \item $p \implies t_j$ (Induction hypothesis)
            \item $p \implies t_i$ (MP)
        \end{enumerate}
        \item Finally if $\exists j < i$ s.t. $x \in \FV(t_j)$ and $t_i = (\forall x)t_j$, then the proof of $t_j$ from $S \cup \qty{p}$ did not use any premise where $x$ occurs free.
        There are two cases
        \begin{itemize}
            \item If $x$ occurs free in $p$, $p$ did not occur in the proof of $t_j$ from $S \cup \qty{p}$ so it is a proof of $t_j$ from $S$.
            So by (Gen), $S \vdash (\forall x) t_j$, i.e. $S \vdash t_j$.
            Write lines:
            \begin{enumerate}
                \item $t_i \implies (p \implies t_i)$ (A1)
                \item $p \implies t_i$ (MP)
            \end{enumerate}
            \item If $x$ doesn't occur free in $p$, then we have a proof of $p \implies t_j$ by induction hypothesis which does not use any premise where $x$ occurs free.
            So add the lines
            \begin{enumerate}
                \item $(\forall x) (p \implies t_j)$ (Gen)
                \item $(\forall x) (p \implies t_j) \implies (p \implies (\forall x) t_j)$ (A7)
                \item $p \implies (\forall x) t_j$, i.e. $p \implies t_i$ (MP).
            \end{enumerate}
        \end{itemize}

        In all cases, the condition on free variables in the induction hypothesis remains true.
    \end{itemize}
\end{proof}

\underline{Aim}: Want to show $S \vdash p$ iff $S \models p$.

\subsection{Soundness}
The proofs in this section are non-examinable.

\begin{proposition}[Soundness Theorem]
    Let $S$ be a set of formulae and $p$ a formula.
    If $S \vdash t$ then $S \models t$.
\end{proposition}

\begin{proof}
    We have a proof $t_1, \dots, t_n$ of $p$ from $S$.
    We show that if $A$ is a model of $S$, $A$ is also a model of $t_i$ for each $i$ (interpreting free variables as quantified); this can be shown by induction.
    Hence, $S \models p$.
\end{proof}

\subsection{Adequacy}
The proofs in this section are non-examinable.

We want to show that $S \models p$ implies $S \vdash p$.
Equivalently, $S \cup \qty{\neg p} \models \bot$ implies $S \cup \qty{\neg p} \vdash \bot$.
In other words, if $S \cup \qty{\neg p}$ is consistent, it has a model.

\begin{theorem}[Model Existence Lemma] \label{thm:predmodel}
    Every consistent\footnote{If $S$ a consistent theory then $S \not\vdash \bot$.} theory has a model.
\end{theorem}

% \begin{proof}
%     Idea: We build a model from $L = L(\Omega, \Pi)$.
%     Let $A$ be a set of closed terms in $L$, i.e. terms with no variables.

%     \underline{E.g.} $S$ is theory of fields, $L$ the language of rings with $1$.
%     If $A$ consists of $1+1$, $1+0+0+1$, $1 \cdot 1$, $1 \cdot 0$, $1 + (-1)$, \dots, etc.
%     Then $(1 + 1) +_A (1 + 0) = 1 + 1 + 1 + 0 \in A$\dots
% \end{proof}

We will need a number of key ideas in order to prove this.

\begin{enumerate}
    \item We will construct our model out of the language itself using the closed terms of $L$.
    For instance, if $L$ is the language of fields and $S$ is the usual field axioms, we take the closed terms and combine them with $+$ and $\cdot$ in the obvious way.
    \item However, we can prove $S \vdash 1 + 0 = 1$, but $1 + 0$ and $1$ are distinct as strings.
    We will therefore take the quotient of this set by the equivalence relation defined by $s \sim t$ if $S \vdash s = t$.
    If this set is $A$, we define $[s] +_A [t] = [s + t]$, and this is a well-defined operation.
    \item Suppose $S$ is the set of field axioms with the statement that $1 + 1 = 0 \vee 1 + 1 + 1 = 0$.
    In this theory, $S \not\vdash 1 + 1 = 0$ and $S \not\vdash 1 + 1 + 1 = 0$.
    Therefore, $[1+1] \neq [0]$ and $[1+1+1] \neq [0]$, so our structure $A$ is not of characteristic 2 or 3.
    We can overcome this by first extending $S$ to a maximal consistent theory.
    \item Suppose $S$ is the set of field axioms with the statement that $(\exists x)(x \cdot x = 1 + 1)$.
    There is no closed term $t$ with the property that $[t\cdot t] = [1+1]$.
    The problem is that $S$ lacks \vocab{witnesses} to existential quantifiers.
    For each statement of the form $(\exists x)p \in S$, we add a new constant $c$ to the language and add to $S$ the sentence $p[c/x]$.
    This still forms a consistent set.
    \item The resulting set may no longer be maximal, as we have extended our language with new constants.
    We must then return to step (iii) then step (iv); it is not clear if this process ever terminates.
\end{enumerate}

\begin{proof}
    Let $S$ be a consistent set in a language $L = L(\Omega, \Pi)$.
    Extend $S$ to a maximal consistent set $S_1$, using Zorn's lemma.
    Then, for each sentence $p \in L$, either $p \in S_1$ or $\neg p \in S_1$.
    Such a theory is called \vocab{complete}; each sentence or its negation is proven.
    Now, we add witnesses to $S_1$: for each sentence of the form $(\exists x)p \in S_1$, we add a new constant symbol $c$ to the language, and also add the sentence $p[c/x]$.
    We then obtain a new theory $T_1$ in the language $L_1 = L(\Omega \cup C_1 \Pi)$ that has witnesses for every existential in $S_1$.
    One can check easily that $T_1$ is consistent.

    We then extend $T_1$ to a maximal consistent theory $S_2$ in $L_1$, and add witnesses to produce $T_2$ in the language $L_2 = L(\Omega \cup C_1 \cup C_2, \Pi)$.
    Continue inductively, and let $\overline S = \bigcup_{n \in \mathbb N} S_n$ in the language $\overline L = L\qty(\Omega \cup \bigcup_{n \in \mathbb N} C_n, \Pi)$.

    We claim that $\overline S$ is consistent, complete, and has witnesses for every existential in $\overline S$.
    Clearly $\overline S$ is consistent: if $\overline S \vdash \bot$ then $S_n \vdash \bot$ for some $n$ as proofs are finite, contradicting consistency of $S_n$.
    For completeness, if $p$ is a sentence in $\overline L$, $p$ must lie in $L_n$ for some $n$ as it is a finite string of symbols.
    But $S_{n+1}$ is complete in $L_n$, so $S_{n+1} \vdash p$ or $S_{n+1} \vdash \neg p$, so certainly $\overline S \vdash p$ or $\overline S \vdash \neg p$.
    If $(\exists x)p \in \overline S$, then $(\exists x)p \in S_n$ for some $n$, so $T_n$ provides a witness.

    On the closed terms of $\overline L$, we define the relation $s \sim t$ if $\overline S \vdash s = t$.
    This is clearly an equivalence relation, so we can define $A$ to be the set of equivalence classes of $\overline L$ under $\sim$.
    This is an $\overline L$-structure by defining
    \begin{itemize}
        \item $f_A([t_1], \dots, [t_n]) = [f\ t_1 \dots t_n]$ for each $f \in \Omega \cup \bigcup_{n \in \mathbb N} C_n, \alpha(f) = n, t_i$ closed terms;
        \item $\varphi_A = \qty{([t_1], \dots, [t_n]) \in A^n \mid \overline S \vdash \varphi(t_1, \dots, t_n)}$ for each $\varphi \in \Pi, \alpha(\varphi) = n, t_i$ closed terms.
    \end{itemize}
    We claim that for a sentence $p \in \overline L$, we have $p_A = 1$ iff $\overline S \vdash p$.
    Then the proof is complete, as $S \subseteq \overline S$ so $p_A = 1$ for every $p \in S$, so $A$ is a model of $S$.

    We prove this by induction on the length of sentences.
    First, suppose $p$ is atomic.
    $\bot_A = 0$, as $\overline S \not\vdash \bot$.
    For closed terms $s, t$, $\overline S \vdash s = t$ iff $[s] = [t]$ by definition of $\sim$.
    This holds iff $s_A = t_A$ by definition of the operations in $A$.
    This is precisely the statement that $s = t$ holds in $A$.
    The same holds for relations.

    Now consider $p \implies q$.
    $\overline S \vdash p \implies q$ iff $\overline S \vdash \neg p$ or $\overline S \vdash q$ as $\overline S$ is complete and consistent; if $\overline S \not\vdash \neg p$ and $\overline S \not\vdash q$, then $\overline S \vdash p$ and $\overline S \vdash \neg p$.
    By induction on the length of the formula, this holds iff $p_A = 0$ or $q_A = 1$.
    This is the definition of the interpretation of $p \implies q$ in $A$.

    Finally, consider the existential $(\exists x)p$.
    $\overline S \vdash (\exists x)p$ iff there is a closed term $t$ s.t. $\overline S \vdash p[t/x]$, as $\overline S$ has witnesses to every existential.
    By induction (for example on the amount of quantifiers in a formula), this holds iff $p[t/x]_A = 1$ for some closed term $t$.
    This is true exactly when $(\exists x)p$ holds in $A$, as $A$ is precisely the set of equivalence classes of closed terms.
\end{proof}

\begin{corollary}[Adequacy]
    Let $S$ be a set of formulae and $p$ a formula.
    If $S \models p$ then $S \vdash p$.
\end{corollary}

\begin{proof}
    WLOG $S$ is a theory and $p$ is a sentence.
    Since $S \models p$, we have $S \cup \qty{\neg p} \models \bot$.
    By \nameref{thm:predmodel}, $S \cup \qty{\neg p} \vdash \bot$.
    So $S \vdash \neg \neg p$ by \nameref{prp:predded}, and thus $S \vdash p$ by A3.
\end{proof}

\subsection{Completeness}
\begin{theorem}[G\"odel's Completeness Theorem for First Order Logic]
    If $S$ is a set of formulae and $p$ is a formula, then $S \vdash p$ iff $S \models p$.
\end{theorem}

\begin{proof}
    Follows from soundness and adequacy.
\end{proof}

Note that \vocab{first order} refers to the fact that variables quantify over elements, rather than sets of elements.

\begin{remark}
    If $L$ is countable, or equivalently $\Omega$ and $\Pi$ are countable, Zorn's lemma is not needed in the above proof.
\end{remark}

\begin{theorem}[Compactness Theorem]
    Let $S$ be a first-order theory.
    If every finite subset $S' \subseteq S$ has a model, $S$ has a model.
\end{theorem}

\begin{proof}
    If $S \models \bot$, then $S \vdash \bot$.
    Proofs are finite, so $\exists S' \subseteq S$ s.t. $S' \vdash \bot$.
    Hence $S' \models \bot$ \Lightning.
\end{proof}

There is no decidability theorem for first order logic, as $S \models p$ can only be verified by checking its valuation in every $L$-structure.

\underline{Applications}: Can we axiomatise finite groups?
Does there exists theory $T$ whose models are the finite groups?

For $n \in \mathbb{N}$, let $t_n = (\exists x_1) \dots (\exists x_n) (\forall x) (x = x_1 \vee x = x_2 \vee \dots \vee x = x_n)$.
We want $T$ to be the theory of groups $\cup \qty{t_1 \vee t_2 \dots}$.
But $\qty{t_1 \vee t_2 \dots}$ is not a sentence.

\begin{corollary}
    The class of finite groups is not axiomatisable as a first order theory.
    % in the language of groups: there is no theory $S$ s.t. a group is finite iff each $p \in S$ holds in the group.
\end{corollary}

\begin{proof}
    Assume it is, and let $T$ be such a theory.
    Consider $T' = T \cup \qty{\neg t_1, \neg t_2, \dots}$ where $t_n$ are as above.
    If $\neg t_i$, then the group has at least $i$ elements.
    Every finite subset of $T'$ has a model : $C_N$\footnote{Cyclic group of order $N$.} for some large $N$.
    By compactness, $T'$ has a model \Lightning \ as this implies $T$ has at least $i$ elements for every $i$, so cannot be finite.
\end{proof}

\begin{corollary}
    Let $T$ be a first order theory with arbitrarily large finite models.
    Then $T$ has an infinite model.
\end{corollary}

\begin{proof}
    Consider $T' = T \cup \{(\exists x_1)(\exists x_2)(x_1 \neq x_2),(\exists x_2)(\exists x_2)(\exists x_3)(x_1 \neq x_2 \wedge x_2 \neq x_3 \wedge x_1 \neq x_3), \dots\}$.
    By assumption every finite subset of $T$ has a model, so $T'$ has a model.
    A model of $T'$ is just an infinite model of $T$.
\end{proof}

Finiteness is not a first order property.

\begin{theorem}[Upward L\"owenheim--Skolem Theorem]
    Let $S$ be a first order theory with an infinite model.
    Then $S$ has an uncountable model.
\end{theorem}

\begin{proof}
    Add constants $\qty{c_i : i \in I}$ to the language, where $I$ is uncountable.
    Let $S' = S \cup \qty{\neg (c_i = c_j) : i, j \in I, i \neq j}$.
    Any finite set of sentences in $S'$ has a model: indeed, the infinite model of $S$ suffices.
    By compactness, $S'$ has a model.
    A model of $S'$ is a model $B$ of $S$ together with an injection $I \to B$ so $B$ is countable.
\end{proof}

\begin{remark}
    Similarly, we can prove the existence of models of $S$ that do not inject into $X$ for any fixed set $X$.
    Adding $\gamma(X)$\footnote{From Hartog's Lemma.} constants or $\mathcal P(X)$ constants both suffice.
\end{remark}

\begin{example}
    There is an uncountable field, as there is an infinite field $\mathbb Q$.
    There is also a field that does not inject into $X$ for any fixed set $X$.
\end{example}

\begin{theorem}[Downard L\"owenheim--Skolem theorem]
    Let $S$ be a first order theory in a countable language $L$, or equivalently, $\Omega$ and $\Pi$ are countable.
    Then if $S$ has a model, it has a countable model.
\end{theorem}

\begin{proof}
    $S$ is consistent (by soundness), so the model constructed in the proof of the \nameref{thm:predmodel} is countable.
\end{proof}

\subsection{Peano Arithmetic}

We want to axiomatise $\mathbb{N}$ as a first order theory. \\
Consider the language $L$ given by $\Omega = \qty{0, s, +, \cdot}$ with $\alpha(0) = 0, \alpha(s) = 1, \alpha(+) = \alpha(\cdot) = 2$, and $\Pi = \varnothing$.\\
Axioms of Peano Arithmetic (PA)
\begin{enumerate}
    \item $(\forall x)(s(x) \neq 0)$;
    \item $(\forall x)(\forall y)(s(x) = s(y) \implies x = y)$;
    \item $(\forall x)(x + 0 = x)$;
    \item $(\forall x)(\forall y)(x + s(y) = s(x + y))$;
    \item $(\forall x)(x \cdot 0 = 0)$;
    \item $(\forall x)(\forall y)(x \cdot s(y) = x \cdot y + x)$.
    \item $(\forall y_1)\dots(\forall y_n)\qty[p[0/x] \wedge (\forall x)(p \implies p[s(x)/x]) \implies (\forall x)p]$ for each formula $p$ with free variables $x, y_1, \dots, y_n$;
\end{enumerate}
This is the axiom scheme for induction.

These axioms are sometimes called Peano arithmetic, $\mathsf{PA}$, or formal number theory.
The $y_i$ in (7) are called \vocab{parameters}.
Without the parameters, we would not be able to perform induction on sets such as $\qty{x : x \geq y}$ if $y$ is a variable.

\begin{remark}
    Let $p$ be the formula $x + (y + z) = (x + y) + z$.
    Then you can prove in PA that $(\forall x) (\forall y) (\forall z) p$ by induction on $z$ with $x, y$ parameters.
    You prove $(\forall x) (\forall y) \qty(p\qty[0/z] \wedge (\forall z) (p \implies p [\frac{sz}{z}]))$.
\end{remark}

Note that $\mathbb{N}_0$ is a model of $\mathsf{PA}$ (so is $\mathbb{N}$).
So by the upward L\"owenheim--Skolem theorem, it has an uncountable model. \\
Didn't we learn $\mathbb{N}_0$ is uniquely determined by its properties? \\
Yes, but \underline{true} induction says $(\forall A \subseteq \mathbb{N}_0)(((0 \in A) \wedge (\forall x)(x \in A \implies sx \in A)) \implies A = \mathbb{N}_0)$.
In first order theory we cannot quantify over subsets of structures.
% The axiom scheme for induction captures only countably many sub
Axiom (7) applies only to countably many formulae $p$, and therefore only asserts that induction holds for countably many subsets of $\mathbb N_0$.

\begin{definition}[Definable]
    A subset $A \subseteq \mathbb N_0$ is \vocab{definable} in the language of $\mathsf{PA}$ if there is a formula $p$ with a free variable $x$ s.t. $p_{\mathbb{N}_0} = A$, i.e. $\qty{a \in \mathbb{N}_0 : a \text{ satisfies } p} = A$.
\end{definition}

Only countably many formulae exist, so only countably many sets are definable.

\begin{example}
    The set of squares is definable, as it can be defined by the formula $(\exists y)(y\cdot y = x)$. \\
    The set of primes is also definable by $x \neq 0 \wedge x \neq 1 \wedge (\forall y)((y \mid x) \implies y = 1 \wedge y = x)$, where $y \mid x$ is defined to mean $(\exists z)(z \cdot y = x)$. \\
    The set of powers of 2 can be defined by $(\forall y)((y \text{ is prime} \wedge y \mid x) \implies y = 2)$. \\
    The set of powers of 4 and the set of powers of 6 are also definable.
\end{example}

\begin{theorem}[G\"odel's Incompleteness Theorem]
    $\mathsf{PA}$ is not complete.
\end{theorem}

This theorem shows that there is a sentence $p$ s.t. $\mathsf{PA} \not\vdash p$ and $\mathsf{PA} \not\vdash \neg p$.
However, one of $p, \neg p$ must hold in $\mathbb N_0$, so there is a sentence $p$ that is true in $\mathbb N_0$ that $\mathsf{PA}$ does not prove.
This does not contradict the completeness theorem, which is that if $p$ is true in \vocab{every} model in $\mathsf{PA}$ then $\mathsf{PA} \vdash p$.