\section{Well-Orderings}
\subsection{Definition}

\begin{definition}[Linear Order]
    A \vocab{linear order} or \vocab{total order} is a pair $(X, <)$ where $X$ is a set, and $<$ is a relation on $X$ s.t.
    \begin{itemize}
        \item (irreflexivity) $\forall \; x \in X$, $\neg(x < x)$;
        \item (transitivity) $\forall \; x, y, z \in X$, $(x < y \wedge y < z) \implies (x < z)$;
        \item (trichotomy) $\forall \; x, y \in X$, either $x < y$, $y < x$, or $x = y$.
    \end{itemize}
    We say $X$ is linearly ordered by $<$, or simply say $X$ is a linearly ordered set.
\end{definition}

\begin{note}
    In trichotomy, exactly one holds, e.g. if $x < y$ and $y < x$, then $x < x$ by transitivity contradicting irreflexivity.
\end{note}

If $X$ is linearly ordered by $<$, we use the obvious notation $x > y$ to denote $y < x$.
In terms of the $\leq$ relation, we can equivalently write the axioms of a linear order as
\begin{itemize}
    \item (reflexivity) $\forall \; x \in X$, $x \leq x$;
    \item (transitivity) $\forall \; x, y, z \in X$, $(x \leq y \wedge y \leq z) \implies (x \leq z)$;
    \item (antisymmetry) $\forall \; x, y \in X$, if $(x \leq y \wedge y \leq x) \implies (x = y)$.
    \item (trichotomy, or totality) $\forall \; x, y \in X$, either $x \leq y$ or $y \leq x$.
\end{itemize}

\begin{example}
    \begin{enumerate}
        \item $(\mathbb N, \leq)$ is a linear order.
        \item $(\mathbb Q, \leq)$ is a linear order.
        \item $(\mathbb R, \leq)$ is a linear order.
        \item $(\mathbb N^+, |)$ is not a linear order, where $|$ is the divides relation, since $2$ and $3$ are not related.
        \item $(\mathcal P(S), \subseteq)$ is not a linear order if $\abs{S} > 1$, since it fails trichotomy.
    \end{enumerate}
\end{example}

\begin{note}
    If $X$ is linearly ordered by $<$, then any $Y \subset X$ is linearly ordered by $<$ (more precisely the restriction of $<$ to $Y$).
\end{note}

\begin{definition}[Well-Ordering]
    A linear order $(X, <)$ is a \vocab{well-ordering} if every nonempty subset $S \subseteq X$ has a least element.
    \begin{align*}
       \forall \; S \subseteq X,\, S \neq \varnothing \implies \exists \; x \in S,\, \forall \; y \in S,\, x \leq y
    \end{align*}
    We say $X$ is well-ordered by $<$, or simply say $X$ is a well-ordered set.
\end{definition}

\begin{note}
    This least element is unique by antisymmetry.
\end{note}

\begin{example}
    \begin{enumerate}
        \item $(\mathbb N, <)$ is a well-ordering.
        \item $(\mathbb Z, <)$ is not a well-ordering, since $\mathbb Z$ has no least element.
        \item $(\mathbb Q, <)$ is not a well-ordering.
        \item $(\mathbb R, <)$ is not a well-ordering.
        \item $[0,1] \subset \mathbb R$ with the usual order is not a well-ordering, since $(0,1]$ has no least element.
        \item $\qty{\frac{1}{2}, \frac{2}{3}, \frac{3}{4}, \dots} \subset \mathbb R$ with the usual order is a well-ordering.
        \item $\qty{\frac{1}{2}, \frac{2}{3}, \frac{3}{4}, \dots} \cup \qty{1}$ with the usual order is also a well-ordering.
        \item $\qty{\frac{1}{2}, \frac{2}{3}, \frac{3}{4}, \dots} \cup \qty{2}$ with the usual order is another example.
        \item $\qty{\frac{1}{2}, \frac{2}{3}, \frac{3}{4}, \dots} \cup \qty{1 + \frac{1}{2}, 1 + \frac{2}{3}, 1 + \frac{3}{4}, \dots}$ is another example.
    \end{enumerate}
\end{example}

\begin{note}
    Every subset of a well-ordered set is well-ordered.
\end{note}

\begin{remark}
    Let $(X, <)$ be a linear order.
    $(X, <)$ is a well-ordering iff there is no infinite decreasing sequence $x_1 > x_2 > \dots$.
    Indeed, if $(X, <)$ is a well-ordering, then the set $\qty{x_1, x_2, \dots}$ has no minimal element, contradicting the assumption.
    Conversely, if $S \subseteq X$ has no minimal element, then we can construct an infinite decreasing sequence by arbitrarily choosing points $x_1 > x_2 > \dots$ in $S$, which exists as $S$ has no minimal element.
\end{remark}

\begin{definition}[Order-Isomorphism]
    Linear ordered sets $X, Y$ are \vocab{order-isomorphic} if there $\exists$ bijection $f : X \to Y$ which is \vocab{order-preserving}: $\forall \; x < y$ in $X$, $f(x) < f(y)$.
    Such an $f$ is an \vocab{order-isomorphism} and $f\inv$ is also an order-isomorphism.
\end{definition}

\begin{note}
    If linearly ordered sets $X, Y$ are order-isomorphic and $X$ is well-ordered, then so is $Y$.
\end{note}

Examples (1) and (6) are isomorphic, and (7) and (8) are isomorphic.
Examples (1) and (7) are not isomorphic, since example (7) has a greatest element and (1) does not.
Example (9) is not isomorphic to (6) or (7).

\begin{example}
    \begin{enumerate}
        \item $\mathbb{N}, \mathbb{Q}$ are not order-isomorphic.
        \item $\mathbb{Q}, \mathbb{Q} \setminus \qty{0}$ are.
    \end{enumerate}
\end{example}

\begin{definition}[Initial Segment]
    A subset $I$ of a totally ordered set $X$ is an \vocab{initial segment} (i.s.) if $x \in I$ implies $y \in I$ for all $y < x$.
\end{definition}

\begin{example}
    $\qty{1, 2, 3, 4}$ is an i.s. of $\mathbb{N}$.
    $\qty{1, 2, 3, 5}$ is not.
\end{example}

\begin{remark}
    In any linear ordering $X$ and element $x \in X$, the set $\qty{y : y < x}$ is an initial segment by transitivity.

    Not every initial segment is of this form, for instance $\qty{x : x \leq 3}$ in $\mathbb R$, or $\qty{x : x > 0, x^2 < 2}$ in $\mathbb Q$.
\end{remark}

\begin{remark}
    In a well-ordering, every proper initial segment $I \neq X$ is of this form.
    Indeed, letting $I_x = \qty{y : y < x}$ where $x$ is the least element of $X \setminus I$ we see $I_x = I$. \\
    If $y \in I_x$ then $y < x$ so $y \in I$ by choice of $x$, i.e. $I_x \subseteq I$.
    If $y \in I$ and $y \geq x$, then $x \in I$ as $I$ is an i.s. \Lightning \ so $y < x$, i.e. $y \in I_x$ and $I \subseteq I_x$.
\end{remark}

\begin{lemma} \label{lem:1}
    Let $X, Y$ be well-ordered sets, $I$ an i.s. of $Y$ and $f : X \to Y$ be an order-isomorphism between $X$ and $I$. \\
    Then $\forall \; x \in X$, $f(x)$ is the least element of $Y \setminus \qty{f(t) : t < x}$.
\end{lemma}

\begin{proof}
    The set $A = Y \setminus \qty{f(t) : t < x}$ is non-empty, e.g. $f(x) \in A$.
    Let $a$ be the least element of $A$.
    Then $a \leq f(x)$ and $f(x) \in I$ and so $a \in I$.
    Thus $a = f(z)$ for some $z \in X$.
    Note that $z > x$ implies that $a = f(z) > f(x)$ \Lightning, so $z \leq x$.
    If $z < x$ then $a = f(x) \in {f(t) : t < x}$ \Lightning \ as $a \in A$.
    So $z = x$ and $a = f(z) = f(x)$.
\end{proof}

\begin{proposition}[Proof by Induction]
    Let $X$ be a well-ordered set, and let $S \subseteq X$ be s.t. for every $x \in X$
    \begin{align*}
    (\forall \; y < x,\, y \in S) \implies x \in S
    \end{align*}
    Then $S = X$.
\end{proposition}

\begin{remark}
    Equivalently, if $p(x)$ is a property s.t. if $p(y)$ is true for all $y < x$ then $p(x)$, then $p(x)$ holds for all $x$.

    Formally, if $S$ is given by a property $p$, $S = \qty{x \in X : p(x)}$. \\
    $(\forall \; x \in X)\qty((\forall \; y < x, p(y)) \implies p(x)) \implies (\forall \; x \in X, p(x))$ (base case is included).
\end{remark}

\begin{proof}
    Suppose $S \neq X$.
    Then $X \setminus S$ is nonempty, and therefore has a least element $x$.
    But all elements $y < x$ lie in $S$, and so by the property of $S$, we must have $x \in S$, contradicting the assumption.
\end{proof}

\begin{proposition}
    Let $X, Y$ be order-isomorphic well-orderings.
    Then there is exactly one order-isomorphism between $X$ and $Y$.
\end{proposition}

Note that this does not hold for general linear orderings, such as $\mathbb Q$ to itself or $[0,1]$ to itself by $x \mapsto x$ or $x \mapsto x^2$.

\begin{proof}
    Let $f, g \colon X \to Y$ be order-isomorphisms.
    We show that $f(x) = g(x)$ for all $x$ by induction on $x$.
    Suppose $f(y) = g(y)$ for all $y < x$.
    We must have that $f(x) = a$, where $a$ is the least element of $Y \setminus \qty{f(y) : y < x}$.
    Indeed, if not, we have $f(x') = a$ for some $x' > x$ by bijectivity, contradicting the order-preserving property.
    Note that the set $Y \setminus \qty{f(x) : y < x}$ is nonempty as it contains $f(x)$.
    So $f(x) = a = g(x)$, as required.
\end{proof}

\begin{remark}
    Induction proves things.
    We need a tool to construct things.
\end{remark}

\subsection{Initial segments}

\begin{note}
    A function from a set $X$ to a set $Y$ is a subset of $f$ of $X \times Y$ s.t.
    \begin{enumerate}
        \item $\forall \; x \in X \; \exists \; y \in Y \ (x, y) \in f$;
        \item $\forall \; x \in X \; \forall \; y, z \in Y \ ((x, y) \in f \wedge (x, z) \in f) \implies (y = z)$.
    \end{enumerate}

    Of course we write $y = f(x)$ instead of $(x, y) \in f$.
    Note that $f \in \mathcal{P}(X \times Y)$.

    For $Z \subseteq X$, the restriction of $f$ to $Z$ is $\eval{f}_Z = \qty{(x, y) \in f ; x \in Z}$.
    $\eval{f}_Z$ is a fcn $Z \to Y$, so $\eval{f}_Z \subseteq Z \times Y \subseteq X \times Y$ so $f_Z \in \mathcal{P}(Z \times Y)$.
\end{note}

\begin{theorem}[Definition by Recursion]
    Let $X$ be a w.o. set and $Y$ be any set.
    Then for any fcn $G \colon \mathcal{P}(X \times Y) \to Y$ there's a unique fcn $f : X \to Y$ s.t. $f(x) = G(\eval{f}_{I_x})$ for every $x \in X$.
\end{theorem}

\begin{remark}
    What this means in defining $f(x)$, we may use the value of $f(y)$ for all $y < x$.
\end{remark}

\begin{proof}
    For uniqueness, we apply induction on $x$.
    If $f, f'$ agree below $x$, then they must agree at $x$ since $f(x) = G\qty(\eval{f}_{I_x}) = G\qty(\eval{f'}_{I_x}) = f'(x)$.

    We say that $h$ is an \vocab{attempt} to mean that $h \colon I \to Y$ where $I$ is some i.s. of $X$, s.t. $\forall \; x \in I,\ h(x) = G\qty(\eval{h}_{I_x})$ (note $I_x \subseteq I$).

    Let $h, h'$ be attempts.
    We show that $\forall \; x \in X$ if $x \in \operatorname{dom}(h) \cap \operatorname{dom}(h')$ then $h(x) = h'(x)$ ($\operatorname{dom}(h)$ is the domain of h, i.e. $I$ above).
    Fix $x \in \operatorname{dom}(h) \cap \operatorname{dom}(h')$ and assume $h(y) = h'(y)$ for every $y < x$ (note $y < x$ implies $y \in \operatorname{dom}(h) \cap \operatorname{dom}(h')$).
    Then $\eval{h}_{I_x} = \eval{h'}_{I_x}$ so $h(x) = G(\eval{h}_{I_x}) = G(\eval{h'}_{I_x}) = h'(x)$.
    Done by induction.

    Now we need to show that $\forall \; x \in X \; \exists$ attempt $h$ s.t. $x \in \operatorname{dom}(h)$.
    We prove this by induction.
    Fix $x \in X$ and assume that for $y < x$ there's an attempt defined at $y$, and let $h_y$ be the unique attempt with domain $\qty{z \in X : z \leq y} = I_y \cup \qty{y}$.
    Then $h = \bigcup_{y < x} h_y$ is a well defined fcn on $I_x$ and it is an attempt since for $y < x$, $h(y) = h_y(y) = G(\eval{h_y}_{I_y}) = G(\eval{h}_{I_y})$.

    The attempt $h' = h \cup \qty{(x, G(h))}$ is an attempt with domain $I_x \cup \qty{x}$.
    Therefore, there is an attempt defined at each $x$, so we can define $f \colon X \to Y$ by $f(x) = h(x)$ where $h$ is some attempt defined at $x$.
    This is well defined by above and $f(x) = h(x) = G(\eval{h}_{I_x}) = G(\eval{f}_{I_x})$.
\end{proof}

\begin{proposition}[Subset Collapse] \label{prp:sbs}
    Let $Y$ be a w.o. set where $X \subseteq Y$.
    Then $X$ is order-isomorphic to a unique initial segment of $Y$.
\end{proposition}
This is not true for general linear orderings, such as $\qty{1, 2, 3} \subset \mathbb Z$, or $\mathbb Q$ in $\mathbb R$.
\begin{proof}
    WLOG $X \neq \emptyset$.

    Uniquness: Assume $f : X \to I$ is an o.i. where $I$ is an i.s. of $Y$.
    By \cref{lem:1}, $f(x) = \min(Y \setminus \qty{f(y) : y < x, y \in X})$.
    So by induction, $f$ and hence $I$ are uniquely determined.

    Existence:
    If $f$ is some such isomorphism, we must have that $f(x)$ is the least element of $X$ not of the form $f(y)$ for $y < x$.
    We define $f$ in this way by recursion, and this is an isomorphism as required.
    Note that this is always well-defined as $f(y) \leq y$, so there is always some element of $X$ (namely, $x$) not of the form $f(y)$ for $y < x$.
\end{proof}

\begin{remark}
    A w.o. set $X$ cannot be isomorphic to a proper i.s. by uniqueness as it is isomorphic to itself.
\end{remark}

\subsection{Relating well-orderings}
\begin{definition}[Less than or equal]
    For well-ordered sets $X, Y$, we will write $X \leq Y$ if $X$ is o.i. to an i.s. of $Y$.
\end{definition}

$X \leq Y$ iff $X$ is o.i. to some subset of $Y$.

\begin{example}
    $\mathbb N \leq \qty{\frac{1}{2}, \frac{2}{3}, \dots}$.

\end{example}
\begin{proposition}
    Let $X, Y$ be well-ordered sets.
    Then either $X \leq Y$ or $Y \leq X$.
\end{proposition}

\begin{proof}
    Assume $Y \nleq X$.
    Then in particular, $Y \neq \emptyset$.
    Fix $y_0 \in Y$ and define by recursion $f \colon X \to Y$ by
    \begin{align*}
        f(x) = \begin{cases}
            \min (Y \setminus \qty{f(y) : y < x}) & \text{if exists} \\
            y_0 & \text{else}
        \end{cases}
    \end{align*}
    If the `otherwise' clause ever arises, then let $x$ be the least element of $X$ for which this happens.
    Then $f(I_x) = Y$ and for $y < x$ the `otherwise' clause does not occur.
    It follows as in the proof of \nameref{prp:sbs} that $f$ is an o.i. from $I_x$ to $Y$, so $Y \leq X$ \Lightning.

    Hence, the `otherwise' clause never arises, and so it follows as in the proof of \nameref{prp:sbs} that $f$ is an o.i. from $X$ to an i.s. of $Y$.
\end{proof}

\begin{proposition}
    Let $X, Y$ be well-ordered sets s.t. $X \leq Y$ and $Y \leq X$.
    Then $X$ is o.i. to $Y$.
\end{proposition}

\begin{proof}
    Let $f \colon X \to Y$ and $g \colon Y \to X$ be o.i.s to i.s. of $Y$ and $X$ respectively.
    Then $g \circ f$ is an o.i. from $X$ to some i.s. of $X$.
    So by uniqueness in \nameref{prp:sbs}, $g \circ f = \eval{\operatorname{id}}_X$.
    Similarly, $f \circ g = \eval{\operatorname{id}}_Y$, so $f$ and $g$ are inverses.
\end{proof}

\begin{remark}
    This shows that $\leq$ is a linear-order (reflexive, antisymmetric, transitive and trichotomous) provided we identified w.o. sets that are o.i. to each other. \\
\end{remark}

\subsection{Constructing larger well-orderings}

\begin{definition}[Less than]
    For w.o. sets $X, Y$, we write $X < Y$ if $X \leq Y$ and $X$ not o.i. to $Y$.
\end{definition}
So $X < Y \iff X$ o.i. to a proper i.s. of $Y$.

\begin{question}
    Do the w.o. sets form a set? If so, is it a w.o. set?
\end{question}

\begin{answer}
    First we construct new w.o. sets from old.
    ``There is always another'': Let $X$ be w.o. and let $x_0 \not\in X$. \\
    $X^+ = X \cup \qty{x_0}$ is w.o. by setting $x < x_0$ for all $x \in X$.
    This is unique up to o.i. and $X < X^+$.
\end{answer}

\underline{Upper Bounds}: Given set $\qty{X_i : i \in I}$ of w.o. sets.
We seek a w.o. set $X$ s.t. $X_i \leq X \; \forall \; i \in I$.

\begin{definition}[Extends]
    For well-orderings $(X, <_X), (Y, <_Y)$, we say that $(Y, <_Y)$ \vocab{extends} $(X, <_X)$ if $X \subseteq Y$, $\eval{<_Y}_{X} =\, <_X$, and $X$ is an i.s. of $Y$. \\
    Then $\qty{X_i : i \in I}$ is \vocab{nested} if $\forall \; i, j \in I$ either $X_i$ extends $X_j$ or $X_j$ extends $X_i$.
\end{definition}

\begin{proposition} \label{prp:8}
    Let $\qty{X_i : i \in I}$ be a nested set of w.o. sets.
    Then, $\exists$ w.o. set $X$ s.t. $X_i \leq X \; \forall \; i \in I$.
\end{proposition}

\begin{proof}
    Let $X = \bigcup_{i \in I} X_i$ with $x < y$ iff $\exists \; i \in I$ s.t. $x, y \in X_i$ and $x <_i y$ where $<_i$ the well-ordering of $X_i$.
    Since the $X_i$'s are nested, this is a well-defined linear order s.t. each $X_i$ is an i.s. of $X$.

    We show that this is a well-ordering.
    Let $S \subseteq X$ be a nonempty set.
    Since $S = \bigcup_{i \in I} (S \cap X_i)$, $\exists \; i \in I$ s.t. $S \cap X_i \neq \emptyset$.
    Let $x$ be a least element of $S \cap X_i$ (since $X$ is w.o.).
    Then $x$ is a least element of $S$ since $X_i$ is an i.s. and if $y < x$, $y \in X_i$.
\end{proof}

\begin{remark}
    The proposition holds without the nestedness assumption (see Section 5).
\end{remark}

\subsection{Ordinals}
\begin{definition}[Ordinal]
    An \vocab{ordinal} is a w.o. set, where we regard two ordinals as equal if they are o.i.
\end{definition}

\begin{remark}
    We cannot construct ordinals as equivalence classes of well-orderings, due to Russell's paradox.
    Later, we will see a different construction that deals with this problem in Section 5.
\end{remark}

\begin{definition}[Order Type]
    The \vocab{order type} of a w.o. set $X$ is the unique ordinal $\alpha$ o.i. to $X$.
    Let $X$ be a well-ordering corresponding to an ordinal $\alpha$.
\end{definition}

\begin{notation}
    Write ``$\alpha$ is the O.T. of $X$''.
\end{notation}

\begin{example}
    For $k \in \mathbb{N}_0$, we let $k$ be the O.T. of a w.o. set of size $k$ (this is unique). \\
    Let $\omega$ be the O.T. of $\mathbb{N}$ (also of $\mathbb{N}_0$).
\end{example}

\begin{example}
    In the reals, the set $\qty{-2, 3, -\pi, 5}$ has order type 4.
    The set $\qty{\frac{1}{2}, \frac{2}{3}, \frac{3}{4}, \dots}$ has order type $\omega$.
\end{example}

\begin{note}
    For ordinals $\alpha, \beta$ write $\alpha \leq \beta$ if $X \leq Y$ where $X$ is a w.o. set with O.T. $\alpha$ and $Y$ has O.T. $\beta$.
    This does not depend on the choice of representative $X$ or $Y$.

    We define $\alpha < \beta$ for $X < Y$. \\
    Let $\alpha^+$ be the O.T. of $X^+$.
\end{note}

\begin{remark}
    Note that $\leq$ is a linear order; if $\alpha \leq \beta, \beta \leq \alpha$ then $\alpha = \beta$.
\end{remark}

\begin{theorem} \label{thm:9}
    Let $\alpha$ be an ordinal.
    Then the set of ordinals less than $\alpha$ form a w.o. set of O.T. $\alpha$.
\end{theorem}

\begin{proof}
    Let $X$ be a w.o. set with O.T. $\alpha$.

    Then, w.o. sets less than $X$ are the proper i.s. of $X$, up to o.i..
    Let $\widetilde{X} = \qty{Y \subset X : Y \text{ a proper i.s. of } X}$.
    Then $<$ (for w.o. sets) is a linear order on $\widetilde{X}$.

    Note the fcn $X \to \widetilde{X}$ defined by $x \mapsto I_x$ is an o.i.
    So $\widetilde{X}$ is a w.o. set of O.T. $\alpha$.
    So $\qty{\operatorname{O.T.}(Y) : Y \in \widetilde{X}}$ is a set of ordinals $< \alpha$, and $Y \mapsto \operatorname{O.T.}(Y)$ is an o.i. from $\widetilde{X}$ to this set.
\end{proof}

\begin{notation}
    We define $I_\alpha = \qty{\beta : \beta < \alpha}$, which is a nice example of a w.o. set of O.T. $\alpha$.
    This is often a convenient representative to choose for an ordinal.
\end{notation}

\begin{proposition} \label{prp:10}
    Every nonempty set $S$ of ordinals has a least element.
\end{proposition}

\begin{proof}
    Let $\alpha \in S$.
    Suppose $\alpha$ is not the least element of $S$.
    Then $S \cap I_\alpha$ is nonempty.
    But $I_\alpha$ is w.o., so $S \cap I_\alpha$ has a minimal element $\beta$.
    Then $\beta$ is a least element of $S$, as if $\gamma \in S$ s.t. $\gamma < \alpha$, then $\gamma \in I_\alpha \cap S$ and so $\beta \leq \gamma$.
\end{proof}

\begin{theorem}[Burali-Forti paradox]
    The ordinals do not form a set.
\end{theorem}

\begin{proof}
    Suppose $X$ is the set of all ordinals.
    Then $X$ is a w.o.., so it has an order type, say $\alpha$.
    Then $X$ is o.i. to $I_\alpha$, which is a proper i.s. of $X$. \Lightning
\end{proof}

\begin{remark}
    Let $S = \qty{\alpha_i \colon i \in I}$ be a set of ordinals.
    Then by \cref{prp:8}, the nested set $\qty{I_{\alpha_i} : i \in I}$ has an upper bound.
    So $\exists$ ordinal $\alpha$ s.t. $\alpha_i \leq \alpha \; \forall \; i \in I$.
    By \cref{thm:9}, $I_\alpha$ is w.o., so we can the least such $\alpha$: \\
    Take the least element of $\qty{\beta \in I_\alpha \cup \qty{\alpha} : \forall \; i \in I, \alpha_i \leq \beta}$. \\
    We denote by ``$\sup S$'' the \vocab{least upper bound on $S$}.

    Note if $\alpha = \sup S$, then $I_\alpha = \cup_{i \in I} I_{\alpha_i}$.
\end{remark}

\begin{example}
    $\sup \qty{2, 4, 6, \dots} = \omega$.
\end{example}

\subsection{Some ordinals}
\begin{align*}
       0, 1, 2, 3, \dots, \omega
    \end{align*}
Write $\alpha + 1$ for the successor $\alpha^+$ of $\alpha$.
\begin{align*}
       \omega + 1, \omega + 2, \omega + 3, \dots, \omega + \omega = \omega \cdot 2
    \end{align*}
where $\omega + \omega = \omega \cdot 2$ is defined by $\sup \qty{\omega + n : n < \omega}$.
\begin{align*}
       \omega \cdot 2 + 1, \omega \cdot 2 + 2, \dots, \omega \cdot 3, \omega \cdot 4, \omega \cdot 5, \dots, \omega \cdot \omega = \omega^2
    \end{align*}
where we define $\omega \cdot \omega = \sup \qty{\omega \cdot n : n < \omega}$.
\begin{align*}
       \omega^2 + 1, \omega^2 + 2, \dots, \omega^2 + \omega, \dots, \omega^2 + \omega \cdot 2, \dots, \omega^2 + \omega^2 = \omega^2 \cdot 2
    \end{align*}
Continue in the same way.
\begin{align*}
       \omega^2 \cdot 3, \omega^2 \cdot 4, \dots, \omega^3
    \end{align*}
where $\omega^3 = \sup\qty{\omega^2 \cdot n : n < \omega}$.
\begin{align*}
       \omega^3 + \omega^2 \cdot 7 + \omega \cdot 4 + 13, \dots, \omega^4, \omega^5, \dots, \omega^\omega
    \end{align*}
where $\omega^\omega = \sup \qty{\omega^n : n < \omega}$.
\begin{align*}
       \omega^\omega \cdot 2, \omega^\omega \cdot 3, \dots, \omega^\omega \cdot \omega = \omega^{\omega + 1}
    \end{align*}
\begin{align*}
       \omega^{\omega + 2}, \dots, \omega^{\omega \cdot 2}, \omega^{\omega \cdot 3}, \dots, \omega^{\omega^2}, \dots, \omega^{\omega^3}, \dots, \omega^{\omega^\omega}, \dots, \omega^{\omega^{\omega^\omega}}, \dots, \omega^{\omega^{\omega^{\dots}}} = \varepsilon_0
    \end{align*}
where $\varepsilon_0 = \sup\qty{\omega, \omega^\omega, \omega^{\omega^\omega}, \dots}$.
\begin{align*}
       \varepsilon_0 + 1, \varepsilon_0 + \omega, \varepsilon_0 + \varepsilon_0 = \varepsilon_0 \cdot 2, \dots, \varepsilon_0^2, \varepsilon_0^3, \dots, \varepsilon_0^{\varepsilon_0}
    \end{align*}
where $\varepsilon_0^{\varepsilon_0} = \sup\qty{\varepsilon_0^\omega, \varepsilon_0^{\omega^\omega}, \dots}$.
\begin{align*}
       \varepsilon_0^{\varepsilon_0^{\varepsilon_0^{\dots}}} = \varepsilon_1
    \end{align*}
All of these ordinals are countable, as each operation only takes a countable union of countable sets.

\subsection{Uncountable ordinals}
\begin{question}
    Can $\exists$ an uncountable ordinal/ w.o. set?
    Can we well order $\mathbb{R}$?
\end{question}

\begin{answer}
    The reals cannot be explicitly well-ordered.
\end{answer}

\begin{theorem} \label{thm:12}
    There exists an uncountable ordinal.
\end{theorem}

\underline{Idea}: Assume $\alpha$ an uncountable ordinal.
Then there is a least such $\alpha$: \\
$\qty{\beta \in I_\alpha \cup \qty{\alpha} : \beta \text{ uncountable}} \neq \emptyset$, so has a least element, say $\gamma$.
So $I_\gamma$ is exactly the set of all countable ordinals.

If $X$ is a countable w.o. set, then $\exists$ injection $f : X \to \mathbb{N}$.
Then $Y = f(X)$ is w.o. by $f(x) < f(y) \iff x < y$ in $X$.
Then $Y$ is an o.i. to $X$.

\begin{proof}
    Let $A = \qty{(Y, <) \in \mathcal{P}(\mathbb{N}) \times \mathcal{P}(\mathbb{N} \times \mathbb{N}) : Y \text{ is a w.o. by } <}$.
    Let $B = \qty{\operatorname{O.T.}(Y, <) : (Y, <) \in A}$.
    By above, $B$ is exactly the set of all countable ordinals.

    Let $\omega_1 = \sup B$.
    If $\omega_1 \in B$, then $\omega_1^+ \in B$ \Lightning \ as $\omega$ countable $\implies \omega^+$ countable.
\end{proof}

\begin{remark}
    Without introducing $A$, it would be difficult to show that $B$ was in fact a set.
\end{remark}

\begin{remark}
    Another ending to the proof above is as follows.
    $B$ cannot be the set of all ordinals, since the ordinals do not form a set by the Burali-Forti paradox, so there exists an uncountable ordinal.
    In particular, there exists a least uncountable ordinal.
\end{remark}

The ordinal $\omega_1$ has a number of remarkable properties.
\begin{enumerate}
    \item It is the least uncountable ordinal.
    \item $\omega_1$ is uncountable, but $\qty{\beta : \beta < \alpha}$ is countable for all $\alpha < \omega_1$, i.e. every proper i.s. of $\omega_1$ is countable.
    \item There exists no sequence $\alpha_1, \alpha_2, \dots$ in $I_{\omega_1}$ with supremum $\omega_1$, as $\sup{\alpha_i}$ is the O.T of $\bigcup_{i \in \mathbb{N}} I_{\alpha_i}$ which is countable.
\end{enumerate}

\begin{theorem}[Hartog's Lemma]
    For every set $X$, $\exists$ an ordinal $\alpha$ that does not inject into $X$.
\end{theorem}

\begin{proof}
    Repeat proof of \cref{thm:12} with $X$ instead of $\mathbb{N}$.
\end{proof}

\begin{remark}
    We write $\gamma(X)$ for the least ordinal that does not inject into $X$.
    For example $\gamma(\omega) = \omega_1$.

    $0, 1, \dots, \omega, \dots, \epsilon_0 = \omega^{\omega^{\omega^{\dots}}}, \dots, \epsilon_1, \dots, \epsilon_{\epsilon_{\epsilon}}, \dots, \omega_1, \dots, \omega_1 \cdot 2, \dots, \omega_2 = \gamma(\omega), \dots$
\end{remark}

\subsection{Successors and limits}

Let $\alpha$ be an ordinal, consider whether $\alpha$ has a greatest element (i.e. if $X$ has O.T. $\alpha$, does $X$ have a greatest element).
\begin{definition}[Successor]
    If $\exists$ greatest element of $I_\alpha$, say $\beta$, then $I_\alpha = I_\beta \cup \qty{\beta}$.
    So $\alpha = \beta^+$ and $\alpha = (\sup I_\alpha)^+$.
    We call such $\alpha$ a \vocab{successor}.

    Else, $I_\alpha = \sup I_\alpha$. i.e. $\alpha = \sup \qty{\beta : \beta < \alpha}$.
    Say $\alpha$ is a \vocab{limit}.
\end{definition}

\begin{example}
    $1 = 0^+$ is a successor.
    5 is a successor.
    $\omega + 2 = (\omega^+)^+$ is a successor.
    $\omega = \sup \qty{n < \omega}$ is a limit as it has no greatest element.
    $\omega_1$ is a limit.
    0 is a limit.
\end{example}

\subsection{Ordinal arithmetic}
Let $\alpha, \beta$ be ordinals.
We define $\alpha + \beta$ by induction on $\beta$ with $\alpha$ fixed, by
\begin{itemize}
    \item $\alpha + 0 = \alpha$;
    \item $\alpha + \beta^+ = (\alpha + \beta)^+$;
    \item $\alpha + \lambda = \sup\qty{\alpha + \gamma : \gamma < \lambda}$ for $\lambda \neq 0$ a limit ordinal.
\end{itemize}

\begin{remark}
    As the ordinals do not form a set, we must technically define addition $\alpha + \gamma$ by induction on the set $\qty{\gamma : \gamma \leq \beta}$.
    The choice of $\beta$ does not change the definition of $\alpha + \gamma$ as defined for $\gamma \leq \beta$.
    This gives a well-defined ``$+$'' by uniqueness in the recursion thm.

    Similarly, we can prove things by induction:
    Let $P(\alpha)$ be a statement for each ordinal $\alpha$, then
    \begin{align*}
        (\forall \; \alpha) \qty((\forall \; \beta) \qty[(\beta < \alpha) \implies P(\beta)] \implies P(\alpha)) \implies (\forall \; \alpha) P(\alpha).
    \end{align*}
    If not, then $\exists \; \alpha$ s.t. $P(\alpha)$ is false.
    Then $\exists$ least such $\alpha$ ($\qty{\beta \leq \alpha : P(\beta) \text{ false}} \neq \emptyset$).
    By \cref{prp:10}, $\alpha$ is the least element.
    So $P(\beta)$ is true $\forall \; \beta < \alpha$.
    By assumption $P(\alpha)$ is true.
\end{remark}

\begin{example}
    For any $\alpha$, $\alpha + 1 = \alpha + 0^+ = (\alpha + 0)^+ = \alpha^+$. \\
    If $m < \omega$, then we have $m + 0 = m$ and for $n < \omega$, $m + (n + 1) = m + n^+ = (m + n)^+ = (m+n) + 1$.

    So on $\omega$, ordered addition is the normal addition.

    $\omega + 2 = \omega + 1^+ = (\omega + 1)^+ = (\omega^+)^+$. \\
    $\omega + \omega = \sup \qty{\omega + n : n < \omega} = \sup\qty{\omega + 1, \omega + 2, \dots}$

    $1 + \omega = \sup\qty{1 + \gamma : \gamma < \omega} = \sup\qty{1, 2, 3, \dots} = \omega \neq \omega + 1$. \\
    Therefore, ``$+$'' is noncommutative.
\end{example}

\begin{proposition} \label{prp:14}
    $\forall \; \alpha, \beta, \gamma$ ordinals, $\beta \leq \gamma \implies \alpha + \beta \leq \alpha + \gamma$.
\end{proposition}

\begin{proof}
    We prove this by induction on $\gamma$, with $\alpha, \beta$ fixed.

    \underline{$\gamma = 0$}: If $\beta \leq \gamma$, then $\beta = 0$, so the result is true.

    \underline{$\gamma = \delta^+$}: If $\beta \leq \gamma$, then \underline{either} $\beta = \gamma$ and we are done.
    \underline{Or} $\beta \leq \delta$ and so $\alpha + \beta \leq \alpha + \delta$ as $\delta < \gamma$ and induction hypothesis.
    Further $\alpha + \delta < (\alpha + \delta)^+ = \alpha + \delta^+ = \alpha + \gamma$.

    \underline{$\gamma \neq 0$ limit}: If $\beta \leq \gamma$, then wlog $\beta < \gamma$, so $\alpha + \beta \leq \sup \qty{\alpha + \delta : \delta < \gamma} = \alpha + \gamma$.
\end{proof}

\begin{remark}
    From \cref{prp:14}, we get $\beta < \gamma \implies \alpha + \beta < \alpha + \gamma$. \\
    Indeed, $\alpha + \beta < (\alpha + \beta)^+ = \alpha + \beta^+ \leq \alpha + \gamma$ since $\beta^+ \leq \gamma$ (from \cref{prp:14}).

    Note that $1 < 2$ but $1 + \omega = 2 + \omega = \omega$.
\end{remark}

\begin{lemma} \label{lem:15}
    Let $\alpha$ be an ordinal and $S$ a non-empty set of ordinals.
    Then $\alpha + \sup S = \sup \qty{\alpha + \beta : \beta \in S}$.
\end{lemma}

\begin{proof}
    If $\beta \in S$, then $\alpha + \beta \leq \alpha + \sup S$ (\cref{prp:14}).
    Hence $\sup \qty{\alpha + \beta : \beta \in S} \leq \alpha + \sup S$.

    For the reverse inequality, consider two cases.
    If $S$ has greatest element, $\beta$ say, then $\alpha + \sup S = \alpha + \beta$.
    $\forall \; \gamma \in S, \gamma \leq \beta$, so by \cref{prp:14}, $\alpha + \gamma \leq \alpha + \beta$.
    It follows that $\sup\qty{\alpha + \gamma : \gamma \in S} = \alpha + \beta$. \\
    If $S$ has no greatest element, then $\lambda = \sup S$ is a $\neq 0$ limit ordinal (If $\lambda = \gamma^+$, then $\gamma < \lambda$ so $\exists \; \delta \in S$ s.t. $\gamma < \delta$ then $\lambda = \gamma^+ \leq \delta$ so $\lambda = \delta \in S$ \Lightning).
    So $\alpha + \sup S = \sup \qty{\alpha + \beta : \beta < \lambda}$ by defn. \\
    If $\beta < \lambda$, then $\exists \; \delta \in S$ s.t. $\beta < \delta$.
    By \cref{prp:14}, $\alpha + \beta \leq \alpha + \delta$.
    It follows that $\sup \qty{\alpha + \beta : \beta < \lambda} \leq \sup \qty{\alpha + \delta : \delta \in S}$.
\end{proof}

\begin{proposition} \label{prp:16}
    $\forall \; \alpha, \beta, \gamma$, $(\alpha + \beta) + \gamma = \alpha + (\beta + \gamma)$.
\end{proposition}

\begin{proof}
    By induction on $\gamma$.

    \underline{$\gamma = 0$}: $(\alpha + \beta) + 0 = \alpha + \beta = \alpha + (\beta + 0)$.

    \underline{$\gamma = \delta^+$}: $(\alpha + \beta) + \delta^+ = \qty((\alpha + \beta) + \delta)^+ = (\alpha + (\beta + \delta))^+ = \alpha + (\beta + \delta)^+ = \alpha + (\beta + \delta^+) = \alpha + (\beta + \gamma)$.

    \underline{$\gamma \neq 0$ limit}:
    \begin{align*}
        (\alpha + \beta) + \gamma &= \sup \qty{(\alpha + \beta) + \delta : \delta < \gamma} \\
        &= \sup \qty{\alpha + (\beta + \delta) : \delta < \gamma} \\
        &= \alpha + \sup\qty{\beta + \delta : \delta < \gamma} \text{ by \cref{lem:15}} \\
        &= \alpha + (\beta + \gamma).
    \end{align*}
\end{proof}

The above is the \vocab{inductive} definition of addition; there is also a \vocab{synthetic} definition of addition.
We can define $\alpha + \beta$ to be the order type of $\alpha \sqcup \beta$, where every element of $\alpha$ is taken to be less than every element of $\beta$.

For instance, $\omega + 1$ is the order type of $\omega$ with a point afterwards, and $1 + \omega$ is the order type of a point followed by $\omega$, which is clearly isomorphic to $\omega$.
Associativity is clear, as $(\alpha + \beta) + \gamma$ and $\alpha + (\beta + \gamma)$ are the order type of $\alpha \sqcup \beta \sqcup \gamma$.
\begin{proposition}
    The inductive and synthetic definitions of addition coincide.
\end{proposition}
\begin{proof}
    We write $+'$ for synthetic addition, and aim to show $\alpha + \beta = \alpha +' \beta$.
    We perform induction on $\beta$.

    For $\beta = 0$, $\alpha + 0 = \alpha$ and $\alpha +' 0 = \alpha$.
    For successors, $\alpha + \beta^+ = (\alpha + \beta)^+ = (\alpha +' \beta)^+$, which is the order type of $\alpha \sqcup \beta \sqcup \qty{\star}$, which is equal to $\alpha +' \beta^+$.

    Let $\lambda$ be a nonzero limit.
    We have $\alpha + \lambda = \sup\qty{\alpha + \gamma : \gamma < \lambda}$.
    But $\alpha + \gamma = \alpha +' \gamma$ for $\gamma < \lambda$, so $\alpha + \lambda = \sup\qty{\alpha +' \gamma : \gamma < \lambda}$.
    % As the sets $\alpha \sqcup \gamma$ are nested, their supremum is their union $\cup_{\gamma < \lambda} (\alpha \sqcup \gamma) = \alpha \sqcup \lambda$ which has OT $\alpha +' \lambda$.
    As the set $\qty{\alpha +' \gamma : \gamma < \lambda}$ is nested, it's sup is equal to its union, which is $\alpha +' \lambda$.
\end{proof}
Synthetic definitions can be easier to work with if such definitions exist.
However, there are many definitions that can only easily be represented inductively, and not synthetically.

We define multiplication inductively by
\begin{itemize}
    \item $\alpha 0 = 0$;
    \item $\alpha \beta^+ = \alpha\beta + \alpha$;
    \item $\alpha \lambda = \sup\qty{\alpha \gamma : \gamma < \lambda}$ for $\lambda$ a nonzero limit.
\end{itemize}
\begin{example}
    $\omega 2 = \omega 1 + \omega = \omega 0 + \omega + \omega = \omega + \omega$.
    Similarly, $\omega 3 = \omega + \omega + \omega$.
    $\omega \omega = \sup\qty{0, \omega 1, \omega 2, \dots} = \qty{0, \omega, \omega + \omega, \dots}$.
    Note that $2 \omega = \sup\qty{0, 2, 4, \dots} = \omega$.
    Multiplication is noncommutative.
    One can show in a similar way that multiplication is associative.
\end{example}
We can produce a synthetic definition of multiplication, which can be shown to coincide with the inductive definition.
We define $\alpha \beta$ to be the order type of the Cartesian product $\alpha \times \beta$ where we say $(\gamma, \delta) < (\gamma', \delta')$ if $\delta < \delta'$ or $\delta = \delta'$ and $\gamma < \gamma'$.
For instance, $\omega 2$ is the order type of two infinite sequences, and $2 \omega$ is the order type of a sequence of pairs.

Similar definitions can be created for exponentiation, towers, and so on.
For instance, $\alpha^\beta$ can be defined by
\begin{itemize}
    \item $\alpha^0 = 1$;
    \item $\alpha^{(\beta^+)} = \alpha^\beta \alpha$;
    \item $\alpha^\lambda = \sup\qty{\alpha^\gamma : \gamma < \lambda}$ for $\lambda$ a nonzero limit.
\end{itemize}
For example, $\omega^2 = \omega^1 \omega = \omega^0 \omega \omega = \omega \omega$.
Further, $2^\omega = \sup\qty{2^0, 2^1, \dots} = \omega$, which is countable.
