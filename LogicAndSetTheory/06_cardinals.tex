\section{Cardinals}

\subsection{Definitions}
We will study the possible sizes of sets in $\mathsf{ZFC}$.
Write $x \equiv y$ if there exists a bijection from $x$ to $y$, i.e. $(\exists f)(f : x \to y \wedge \text{`$f$ is a bijection'})$.
This is an equivalence relation class, the equivalence classes are proper classes (except for $\qty{\emptyset}$). \\
How do we pick a representative form each equivalence class? \\
We seek for each set $x$, a set $\card(x)$ s.t. $(\forall x)(\forall y)(\card x = \card y \iff x \equiv y)$.

In $\mathsf{ZFC}$, this is easy: given a set $x$, $x$ can be well-ordered, so $x \equiv OT(x)$, i.e. $x \equiv \alpha$ for some $\alpha \in ON$.
We can define $\card x = $ least $\alpha \in ON$ s.t. $x \equiv \alpha$.

In $\mathsf{ZF}$ (due to D.S. Scott), we can define the \vocab{essential rank} as follows:
$\operatorname{ess\ rank}(x) = $ least $\alpha$ s.t. $\exists y \subseteq V_\alpha$ with $y \equiv x$, i.e. least $\alpha$ s.t. $y \equiv x$ and $\rank y = \alpha$.
Note that $\operatorname{ess\ rank}(x) \leq \rank(x)$ (take $\alpha = \rank(x)$ and $y = x$).
Define $\card x = \qty{y  \subseteq V_{\operatorname{ess\ rank}(x)} : y \equiv x}$.

\subsection{The hierarchy of alephs}

\begin{definition}[Cardinal]
    A set $m$ is a \vocab{cardinal} if $m = \card x$ for some set $x$, in which case we say $m$ is the \vocab{cardinality} of $x$.
\end{definition}

\begin{definition}[Initial Ordinal]
    Say that an ordinal $\alpha$ is an \vocab{initial ordinal} if $(\forall \beta < \alpha)(\beta \not\equiv \alpha)$.
\end{definition}
I.e. $\alpha$ is initial if there is no bijection from $\alpha$ to any smaller ordinal.

\begin{example}
    For any set $x$, $\gamma(x)$ (from Hartogs' lemma) is an initial ordinal. \\
    So $0, 1, 2, 3, \dots$ are initial ordinals (easy $\omega$-induction). \\
    $\omega$ is also an initial ordinal. \\
    $\omega^2$ is not initial as $\omega^2 \equiv \omega$ as countable. \\
    $\epsilon_0 = \omega^{\omega^{\omega^{\dots}}}$ is not as $\epsilon_0 \equiv \omega$. \\
    $\omega_1$ is initial.
\end{example}

We define $\omega_\alpha$ for each ordinal $\alpha$ by recursion.
\begin{itemize}
    \item $\omega_0 = \omega$;
    \item $\omega_{\alpha + 1} = \gamma(\omega_\alpha)$;
    \item $\omega_\lambda = \sup\qty{\omega_\alpha : \alpha < \lambda}$ for a nonzero limit ordinal $\lambda$.
\end{itemize}
Each of these ordinals is initial, and every initial ordinal $\beta$ is of the form $\omega_\alpha$.

\begin{proposition}
    The ordinals $\omega_\alpha$ are exactly the infinite initial ordinals.
\end{proposition}

\begin{proof}
    First we prove that every $\omega_\alpha$ is initial, by induction: \\
    $\alpha = 0$ $\checkmark$ \\
    $\alpha = \beta^+$ $\checkmark$ as $\forall x$ $\gamma(x)$ is initial \\
    $\alpha \neq 0$ a limit, Fix $\beta < \omega_\alpha$.
    Need that $\beta \not\equiv \omega_\alpha$.
    Since $\beta < \omega_\alpha$, we  have $\beta < \omega_\gamma$ for some $\gamma < \alpha$.
    Then by IH, $\omega_\gamma$ is initial, so $\beta \not\equiv \omega_\gamma$.
    If $\beta \equiv \omega_\alpha$, then $\omega_\gamma \hookrightarrow \omega_\alpha \equiv \beta$ i.e. $\exists$ injection $\omega_\gamma \hookrightarrow \beta$.
    By Schr\"oder-Bernstein, $\beta \equiv \omega_\gamma$ \Lightning.

    Conversely, let $\delta$ be an infinite initial ordinal.
    We need $\delta = \omega_\alpha$ for some $\alpha$. \\
    Easy induction show that $\alpha \leq \omega_\alpha \; \forall a$ so $\delta < \omega_{\delta + 1}$.
    Take the least $\alpha$ s.t. $\delta < \omega_\alpha$.
    Then $\alpha \neq 0$ and $a$ is not a limit, o/w $\delta < \omega$ for some $\gamma < \alpha$ \Lightning. \\
    So $\alpha = \beta^+$ for some $\beta$.
    So we have $\omega_\beta \leq \delta < \omega_{\beta^+} = \gamma(\omega_\beta)$.
    Hence $\delta \hookrightarrow \omega_\beta$ and $\omega_\beta \hookrightarrow \delta$.
    So by Schr\"oder-Bernstein, $\omega_\beta \equiv \delta$, so $\delta = \omega_\beta$ as $\delta$ is initial.
\end{proof}

\begin{definition}[Aleph Numbers]
    We write $\aleph_\alpha$ for $\card(\omega_\alpha)$ where $\alpha \in ON$.
    The \vocab{alephs} are the cardinalities of the infinite initial ordinals.
\end{definition}

\begin{example}
    $\aleph_0 = \card \omega$, $\aleph_1 = \card \omega_1$.
\end{example}

In $\mathsf{ZF}$ without $\mathsf{AC}$, the $\aleph_\alpha$ are the cardinalities of the well-orderable sets.

\subsection{Cardinal Arithmetic}
We use $m, n, p$ etc. for cardinalities and $M, N, P$ etc. for sets with cardinalities $m, n, p$ respectively.

We write $m \leq n$ if there exists an injection from $M$ to $N$ where $\mathrm{card}(M) = m, \mathrm{card}(N) = n$\footnote{This is well-defined, if $M \equiv M'$ and $N \equiv N'$ then $M \hookrightarrow N$ iff $M' \hookrightarrow N'$.}. \\
Similarly, we write $m < n$ if $m \leq n$ and $m \neq n$. \\
For example, $\mathrm{card}(\omega) < \mathrm{card}(\mathcal P(\omega))$.
By the Schr\"oder--Bernstein theorem, if $m \leq n$ and $n \leq m$, then $m = n$.
Hence, $\leq$ is a partial order on cardinals.
This is in fact a total order in $\mathsf{ZFC}$, since we can well-order the two sets in question, and one injects into the other; alternatively, the $\aleph$ numbers are clearly totally ordered.

Let $m, n$ be cardinals.
Then,
\begin{enumerate}
    \item $m + n = \mathrm{card}\qty(M \amalg N)$;
    \item $m \cdot n = \mathrm{card}(M \times N)$;
    \item $m^n = \mathrm{card}(M^N)$;
\end{enumerate}
where $m = \mathrm{card}(M), n = \mathrm{card}(N)$, and $M^N$ is the set of functions $N \to M$.
This is well-defined.

\underline{Some Properties} \\
$m + n = n + m$ ($M \sqcup N \equiv N \sqcup M$) \\
$m \cdot n = n \cdot m$ ($M \times N \equiv N \times M$) \\
$m (n + p) = mn + mp$ ($M \times (N \sqcup P) \equiv M \times N \sqcup M \times P$)
$(m^n)^p = m^{np}$, $m^n m^p = m^{n + p}$, $m^p \cdot n^p = (mn)^p$ (not true for ordinals though)

$m \leq n \implies m + p \leq n + p$, $mp \leq np$, $m^p \leq n^p$ etc.

\begin{note}
    For all cardinals $m$, $m < 2^m$, i.e. $M \not\equiv \mathcal{P}(M)$.
    In fact, there is no surjection from $M$ to $\mathcal{P}(M)$ by Cantor's diagonal argument.
\end{note}

% \begin{example}
%     In particular, $\aleph_0 < 2^{\aleph_0}$ which contrasts with the fact that $\omega = 2^{\omega}$ for ordinal exponentiation. \\
%     $\aleph_0 \cdot \aleph_0 = \aleph_0$ but $\omega^2 \neq \omega$.
% \end{example}

\begin{example}
    $\mathbb R, \mathcal P(\omega), \qty{0,1}^\omega$ biject.
    Hence, $\mathrm{card}(\mathbb R) = \mathrm{card}(\mathcal P(\omega)) = 2^{\aleph_0}$.
    In particular, cardinal exponentiation and ordinal exponentiation do not coincide, as $2^\omega = \omega$.

    The cardinality of the set of sequences of reals is
    \[ \mathrm{card}(\mathbb R^\omega) = (2^{\aleph_0})^{\aleph_0} = 2^{\aleph_0 \cdot \aleph_0} = 2^{\aleph_0} \]
    Note that this statement requires that addition and multiplication are commutative, $\aleph_0 \cdot \aleph_0 = \aleph_0$ as $\omega \times \omega$ bijects with $\omega$, and that $(m^n)^p = m^{np}$.
    The latter holds as $(M^N)^P$ is the set of functions $P \to (N \to M)$, and $M^{N \times P}$ is the set of functions $N \times P \to M$.
\end{example}

\begin{theorem}
    $\aleph_\alpha \cdot \aleph_\alpha = \aleph_\alpha$ for any $\alpha \in ON$.
\end{theorem}

A consequence of this result is that addition and multiplication of alephs is easy.

\begin{proof}
    We show this by induction on $\alpha$.
    Define a well-ordering of $\omega_\alpha \times \omega_\alpha$ by `going up in squares':
    \begin{align*}
        (x,y) < (z,w) \iff &(\max(x,y) < \max(z,w)) \ \vee \\
        &(\max(x,y) = \max(z,w) = \beta \\
        &\quad\quad\wedge (y < \beta, z < \beta \vee x = z = \beta, y < w \vee y = w = \beta, x < z))
    \end{align*}
    For any $\delta \in \omega_\alpha \times \omega_\alpha$, $\delta \in \beta \times \beta$ for some $\beta < \omega_\alpha$, as $\omega_\alpha$ is a limit ordinal (e.g. if $\delta = (x, y)$ then $\beta = \max\qty{x, y}^+$ will do).
    By induction, we can assume $\beta \times \beta$ bijects with $\beta$ (or $\beta$ is finite).
    Hence, the initial segment $I_\delta$ is contained in $\beta \times \beta$ and hence has cardinality at most $\mathrm{card}(\beta \times \beta) < \mathrm{card}(\omega_\alpha)$.

    So every proper initial segment of $\omega_\alpha \times \omega_\alpha$ has order type $ < \omega_\alpha$ and hence $\omega_\alpha \times \omega_\alpha$ has order type $\leq \omega_\alpha$.
    It follows that $\omega_\alpha \times \omega_\alpha$ injects into $\omega_\alpha$ and so $\aleph_\alpha \cdot \aleph_\alpha \leq \aleph_\alpha$.

    Since $\aleph_\alpha = \aleph_\alpha \cdot 1 \leq \aleph_\alpha \cdot \aleph_\alpha$ and so the result follows.
\end{proof}

\begin{corollary} \label{cor:6-3}
    For any ordinals $\alpha < \beta$, we have $\aleph_\alpha + \aleph_\beta = \aleph_\alpha \cdot \aleph_\beta = \aleph_\beta$.
\end{corollary}

\begin{proof}
    \[ \aleph_\beta \leq \aleph_\alpha + \aleph_\beta \leq 2 \cdot \aleph_\beta \leq \aleph_\alpha \aleph_\beta \leq \aleph_\beta^2 = \aleph_\beta \]
\end{proof}

\begin{note}
    In ZFC one can define more general infinite sums and products of cardinals. In the definitions below, as earlier, lower-case letters denote cardinals and upper-case letters denote sets with cardinality the corresponding lower-case letter.
\end{note}

\begin{definition}
    Let $I$ be a set, and for each $i \in I$ let $m_i$ be a cardinal. Then
    \begin{align*}
        \sum_{i \in I} m_i=\operatorname{card}\qty(\bigsqcup_{i \in I} M_i) \text { and } \prod_{i \in I} m_i=\operatorname{card}\qty(\prod_{i \in I} M_i)
    \end{align*}
    where $\bigsqcup_{i \in I} M_i=\bigcup_{i \in I} M_i \times\{i\}$ and
    \begin{align*}
        \prod_{i \in I} M_i=\qty{f: I \rightarrow \bigcup_{i \in I} M_i: f(i) \in M_i \text { for all } i \in I}.
    \end{align*}
\end{definition}

\begin{note}
    We need AC in these definitions twice. Firstly, we need to make a choice of sets $M_i$ with cardinality $m_i$. Secondly, when we show that these definitions don't depend on the choice of the $M_i$, we need to make a choice of bijections $f_i: M_i \rightarrow M_i^{\prime}$ where $M_i^{\prime}$ is another set with cardinality $m_i$.
\end{note}

\begin{example}
    It is possible to show results similar to \cref{cor:6-3}. For example, if $m_i \leqslant \aleph_\alpha$ for all $i \in I$ and $\operatorname{card}(I) \leqslant \aleph_\alpha$, then $\sum_{i \in I} m_i \leqslant \aleph_\alpha$.
\end{example}

Infinite products of cardinals relate to cardinal exponentiation which is hard. We can achieve some reduction in the problem of studying cardinal exponentiation. For example, if $\alpha \leqslant \beta$, then
\begin{align*}
    2^{\aleph_\beta} \leqslant \aleph_\alpha^{\aleph_\beta} \leqslant\qty(2^{\aleph_\alpha})^{\aleph_\beta}=2^{\aleph_\alpha \cdot \aleph_\beta}=2^{\aleph_\beta}
\end{align*}

So it is of interest to study cardinals of the form $2^{\aleph_\beta}$. We know that $\aleph_\beta<2^{\aleph_\beta}$ but very little else is known. For example, a natural question is whether $2^{\aleph_0}$ is equal to $\aleph_1$. Since $2^{\aleph_0}$ is the cardinality of $\mathbb{R}$, this became known as the Continuum Hypothesis (or $\mathrm{CH}$ for short):
\begin{align*}
    2^{\aleph_0}=\aleph_1
\end{align*}
P. Cohen proved in the 1960 s that if $\mathrm{ZFC}$ is consistent, then so are $\mathrm{ZFC}+\mathrm{CH}$ and $\mathrm{ZFC}+\neg \mathrm{CH}$. So $\mathrm{CH}$ is independent of $\mathrm{ZFC}$.

Hence, for example, $X \amalg X$ bijects with $X$ for any infinite set $X$.

Cardinal exponentiation is not as simple as addition and multiplication.
For instance, in $\mathsf{ZF}$, $2^{\aleph_0}$ need not even be an aleph number, for instance if $\mathbb R$ is not well-orderable.
In $\mathsf{ZFC}$, the statement $2^{\aleph_0} = \aleph_1$ is independent of the axioms; this is called the \vocab{continuum hypothesis}.
$\mathsf{ZFC}$ does not even decide if $2^{\aleph_0} < 2^{\aleph_1}$.
Even today, not all implications about cardinal exponentiation (such as $\aleph_\alpha^{\aleph_\beta}$) are known.