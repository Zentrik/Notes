\section{Quantum computation}

\subsection{Classical computation}
A \vocab{computational task} takes an input bit string and produces an output bit string.

A decision problem is a computational task that produces an output of length 1.
Let $B = B_1 = \qty{0,1}$ and denote $B_n = \qty{0,1}^n$.
Define $B^\star = \bigcup_{n \geq 1} B_n$.
A \vocab{language} is a subset $L \subseteq B^\star$.
A decision problem corresponds to the problem of checking whether a word $w \in B^\star$ lies in a language $L$.
For example, the set of primes, expressed in binary, forms a language $P \subseteq B^\star$, and there is a corresponding decision problem to check if a given binary string represents a prime.

More generally, the output of a computational task can be of any length.
For example, the task $\mathsf{FACTOR}(x)$ takes the input $x$ and produces a bit string containing a factor of $x$, or 1 if $x$ is prime.

There are various models of computation, but we restrict to the \vocab{circuit} or \vocab{gate array} model.
In this model, we have an input $x = b_1 \dots b_n \in B_n$, and extend it with some trailing zeroes to add scratch space to perform computations.
We then perform some computational steps, an application of designated Boolean gates $f \colon B_n \to B_m$ on preassigned bits.
For each $n$, we have a circuit $C_n$, which is a prescribed sequence of computational steps that performs a given task for all inputs of size $n$.
The output to the computation is a designated subsequence of the extended bit string.

Suppose that, in addition to extending the input bit string with zeroes, we also add $k$ random bits, which have values set to 0 or 1 uniformly at random.
The output of the computation will now be probabilistic.
The probability that the output is $y$ is $a2^{-k}$, where $a$ is the number of bit strings $r$ that produce the desired outcome.
We typically require that the output is correct with some prescribed probability.

\subsection{Classical complexity}
The \vocab{time complexity} is a measure of the amount of computational steps required for a particular algorithm for an input of size $n$.
In the circuit model, we define $T(n)$ to be the total number of gates in the circuit $C_n$, known as the \vocab{size} of the circuit or \vocab{runtime} of the algorithm.

For a positive function $T(n)$, we write $T(n) = O(f(n))$ if there exist positive constants $c, n_0$ such that for all $n > n_0$, we have $T(n) \leq cf(n)$.
If $T(n) = O(n^k)$ for some $k > 0$, we say that $T(n)$ is $O(\mathrm{poly}(n))$, and the corresponding algorithm is a \vocab{poly-time} algorithm.
The class of languages for which the membership problem has a classical poly-time algorithm is called $\mathsf{P}$.
The class of languages for which the membership problem has a randomised classical poly-time algorithm that gives the correct answer with probability at least $\frac{2}{3}$ is called $\mathsf{BPP}$, short for \vocab{bounded-error probabilistic poly-time}.
The problem $\mathsf{FACTOR}(M,N)$ which determines if there is a nontrivial factor of $N$ that is at most $M$ does not lie in $\mathsf{BPP}$.
The best known runtime is $T(n) = \exp\qty(O\qty(n^{\frac{1}{3}}\qty(\log n)^\frac{2}{3}))$.

A black box promise problem is a computational task where the input is a \vocab{black box} or \vocab{oracle} which can compute a Boolean function $f \colon B_m \to B_n$, and there is an \vocab{a priori promise} on $f$ restricting the possible values of $f$.
For example, the black box promise problem for constant vs.\ balanced functions takes a function $f \colon B_n \to B$ such that $f$ is constant or \vocab{balanced}, in which case $f$ is equal to zero for exactly half of the $2^n$ possible inputs.

The corresponding complexity is called \vocab{query complexity}, which counts the amount of times we need to query the black box.
We typically wish to minimise the query complexity.

\subsection{Quantum circuits}
In a quantum circuit, we have qubit inputs $\ket{b_1} \dots \ket{b_n} \ket{0} \dots \ket{0}$ analogously to the classical case.
The input size $n$ is the number of qubits.
The addition of randomness to classical computation needs no analogue in the quantum case, since randomness is obtained by measurement.
For instance, if we have a qubit $\ket{0}$, we can generate a uniform Bernoulli random variable by sending the qubit through a Hadamard gate and then measuring in the computational basis.

The computational steps are gates or unitary operators, which act on a prescribed set of qubits, constituting a quantum circuit $C_n$.
The output is obtained by performing a measurement on a prescribed set of qubits.
One can show that any circuit involving arbitrarily many measurements is equivalent to a circuit that only performs a single measurement at the end of the computation.

\subsection{Quantum oracles}
Note that all quantum gates are invertible, as they are represented with unitary operators, but not all classical gates are invertible.
Any $f \colon B_m \to B_n$ can be expressed in an equivalent invertible form $\widetilde f \colon B_{m+n} \to B_{m+n}$ by defining $\widetilde f(b,c) = (b, c \oplus f(b))$.
If we can compute $f$ we can also compute $\widetilde f$, and conversely given $\widetilde f$ we can find $f(b) = \widetilde f(b, 0)$.
This is self-inverse.
\[ \widetilde f(\widetilde f(b,c)) = \widetilde f(b, c \oplus f(b)) = (b, c \oplus f(b) \oplus f(b)) = (b, c) \]
A quantum oracle for a function $f \colon B_m \to B_n$ is the quantum gate $U_f$ acting on $m + n$ qubits such that $U_f \ket{x} \ket{y} = \ket{x} \ket{y \oplus f(x)}$ for $\ket{x}, \ket{y}$ states in the computational basis.
In other words, its action on the computational basis is $\widetilde f$.
We say that $\ket{x}$ is the \vocab{input register} and $\ket{y}$ is the \vocab{output register}.

One can show that $U_f$ is always a unitary operator.
We can show this directly by considering $U_f \ket{x'}\ket{y'} = \ket{x'}\ket{y' \oplus f(x')}$, and we can take the inner product with $U_f \ket{x} \ket{y} = \ket{x} \ket{y \oplus f(x)}$.
An easier way to show this is to consider $\widetilde f \colon B_k \to B_k$ as a permutation on $B_k$ where $m + n = k$.
We can write $U_f \ket{x} \ket{y} = U_f \ket{i_1 \dots i_k} = \ket{\widetilde f(i_1 \dots i_k)}$.
Since $\widetilde f$ is a permutation, $U_f$ is therefore represented by a permutation matrix, which has a single 1 in each row and column.
All permutation matrices are unitary.

In contrast to a classical oracle, a quantum oracle can act on a superposition of input registers.
Let $f \colon B_m \to B_n$, and consider the \vocab{equal superposition} state $\ket{\varphi_m} = \frac{1}{\sqrt{2^m}} \sum_{x \in B_m} \ket{x}$.
We can find
\[ U_f \ket{\varphi_m} \ket{y} = U_f \qty( \frac{1}{\sqrt{2^m}} \sum_{x \in B_m} \ket{x}) \ket{y} = \frac{1}{\sqrt{2^m}} \sum_{x \in B_m} U_f \ket{x} \ket{y} = \ket{\psi_f} \]
In a single use of the oracle, we obtain a final state which depends on the value of $f$ for all possible inputs.
This allows us to obtain ``global'' information about the function $f$ (e.g. determine some joint properties of all the values) with just one run of $U_f$, whereas classically this may require many evaluations of $f$.

One can easily create such an equal superposition state $\ket{\varphi_m}$ by sending the $m$-qubit state $\ket{0} \dots \ket{0}$ through $m$ Hadamard gates $H \otimes \dots \otimes H$.
We have
\begin{align*}
    H \otimes \dots \otimes H (\ket{0} \dots \ket{0}) &= \frac{1}{\sqrt{2^m}} (\ket{0} + \ket{1}) \dots (\ket{0} + \ket{1}) \\
    &= \frac{1}{\sqrt{2^m}} \sum_{x_1, \dots, x_m} \ket{x_1} \dots \ket{x_m} \\
    &= \frac{1}{\sqrt{2^m}} \sum_{x \in B_m} \ket{x}.
\end{align*}
This creates a superposition of exponentially many terms using a linear amount of Hadamard gates.

\subsection{Deutsch--Jozsa algorithm}
Consider the black box problem for balanced vs.\ constant functions.
This is when we have a function $f : B_n \to B$ which is either constant or ``balanced'', i.e. $|f\inv(0)| = |f\inv(1)| = 2^{n-1}$.
The problem is to determine whether a $f$ is balanced or constant.

Classically, one needs $2^{n-1} + 1$ queries to solve the problem in the worst case.
This amount of queries is clearly sufficient; even if $f$ is balanced, the first $2^{n-1}$ queries could have equal outcomes, but the subsequent query must have a different outcome.
Suppose that there exists an algorithm that can solve the problem in $2^{n-1}$ queries.
An adversary that controls the oracle can respond with 0 for every query, and subsequently choose a function $f$ that agrees with the earlier query results but is balanced or constant as required to cause the algorithm to produce an error.
Therefore, classically we require a query complexity of $O(\exp(n))$.

Suppose we have a quantum oracle $U_f$ with $U_f \ket{x} \ket{y} = \ket{x} \ket{y \oplus f(x)}$, where $\ket{x}$ is an $n$-qubit state and $\ket{y}$ is a 1-qubit state.
Set each qubit to state $\ket{0}$, then act by $H^{\otimes n} \otimes (H \cdot X)$ on $\ket{x} \ket{y}$.
We then obtain the state $\ket{A} = \frac{1}{\sqrt{2^n}} \sum_{x \in B_n} \ket{x} \ket{-}$.
Send this state through the oracle to obtain $U_f \ket{A} = \frac{1}{\sqrt{2^n}} U_f \sum_{x \in B_n} \ket{x} \ket{-}$.
Note that
\begin{align*}
    U_f \ket{x} \ket{-} &= \frac{1}{\sqrt{2}} U_f \qty(\ket{x}\ket{0} - \ket{x}\ket{1}) \\
    &= \frac{1}{\sqrt{2}} \qty(\ket{x}\ket{f(x)} - \ket{x}\ket{f(x)^c}) \\
    &= \begin{cases}
        \frac{1}{\sqrt{2}}\ket{x}(\ket{0} - \ket{1}) = \ket{x}\ket{-} & \text{if } f(x) = 0 \\
        \frac{1}{\sqrt{2}}\ket{x}(\ket{1} - \ket{0}) = -\ket{x}\ket{-} & \text{if } f(x) = 1
    \end{cases} \\
    &= (-1)^{f(x)} \ket{x}\ket{-}
\end{align*}
The method of encoding all information into a phase is called \vocab{phase kickback}.
Hence,
\[ U_f \ket{A} = \frac{1}{\sqrt{2^n}} U_f \sum_{x \in B_n} \ket{x} \ket{-} = \frac{1}{\sqrt{2^n}} \qty(\sum_{x \in B_n} (-1)^{f(x)} \ket{x}) \ket{-} \]
We can then easily discard the last qubit, as it is now in a product state.
We obtain
\[ \ket{f} = \frac{1}{\sqrt{2^n}} \sum_{x \in B_n} (-1)^{f(x)} \ket{x} \]
If $f$ is constant,
\[ \ket{f} = \pm \frac{1}{\sqrt{2^n}} \sum_{x \in B_n} \ket{x} = \pm \qty(H \ket{0})^{\otimes n} \]
If we apply $H^{\otimes n}$ to $\ket{f}$, we obtain $\pm \ket{0}^{\otimes n}$.
If $f$ is balanced, writing $\ket{\varphi_n} = \frac{1}{\sqrt{2^n}} \sum_{y \in B_n} \ket{y}$,
\[ \ip{f}{\varphi_n} = \frac{1}{2^n} \sum_{x, y \in B_n} (-1)^{f(x)} \ip{y}{x} = \frac{1}{2^n} \sum_{x \in B_n} (-1)^{f(x)} = 0 \]
In this case, $\ket{f}$ is orthogonal to $\ket{\varphi_n}$.
Applying $H^{\otimes n}$ to $\ket{f}$, we have that $H^{\otimes n} \ket{f}$ is orthogonal to $H^{\otimes n} \ket{\varphi_n} = \ket{0}^{\otimes n}$.

After obtaining $\ket{f}$, we apply $H^{\otimes n}$ and measure in the computational basis.
If $f$ is constant, we measure $0\dots 0$ with probability 1, and if $f$ is balanced, we measure $0\dots 0$ with probability 0.
This allows us to infer whether $f$ is constant or balanced with probability 1.
\begin{center}
    \leavevmode
    \Qcircuit{
        \lstick{\ket{0}_1} & \qw & \gate{H} \qw & \multigate{4}{U_f} & \gate{H} & \meter \qw & \rstick{x_1} \cw \\
        \lstick{\ket{0}_2} & \qw & \gate{H} \qw & \ghost{U_f} & \gate{H} & \meter \qw & \rstick{x_2} \cw \\
        \vdots & & & & & & \vdots \\
        \lstick{\ket{0}_n} & \qw & \gate{H} \qw & \ghost{U_f} & \gate{H} & \meter \qw & \rstick{x_n} \cw \\
        \lstick{\ket{0}} & \gate{X} \qw & \gate{H} \qw & \ghost{U_f} & \rstick{\text{discard}} \qw
    }
\end{center}
For this algorithm, we use one query and $3n+2$ further operations.

Suppose we permit a probability $\varepsilon > 0$ of error.
In the quantum case, we only need one query.
In the classical case, there is a randomised algorithm which solves the problem with a constant number $O\qty(\log \frac{1}{\varepsilon})$ of queries for all $n$.
Choose $k$ inputs each chosen uniformly at random, and evaluate $f(x)$ for each $x$ in this set.
If $f(x)$ is constant for all of these $k$ inputs, we infer $f$ is constant; otherwise we infer it is balanced.
An error can only occur when the function is balanced but we infer it is constant.
The probability of error is $\frac{2}{2^k} = 2^{-k+1}$.
Hence, we can take $\varepsilon < 2^{-k+1}$, so $k = O\qty(\log \frac{1}{\varepsilon})$.

\subsection{Simon's algorithm}
Consider a function $f \colon B_n \to B_n$ with the promise that either $f$ is injective, or $f(x) = f(y)$ if and only if $y = x$ or $y = x \oplus \xi$ for a fixed $0 \neq \xi \in B_n$.
The problem is to determine with bounded error whether $f$ is in the 1-1 form or the 2-1 form, and in the latter case, to find the constant $\xi$.
Note that $f(x \oplus \xi) = f(x)$ is the statement that $f$ has period $\xi$.

Classically, the query complexity is $O(\exp(n))$.
In order to solve the problem, we need to find two distinct $x, y$ inputs for which $f(x) = f(y)$, or show that this is not possible.
However, there is a quantum algorithm with query complexity $O(n)$.

\subsection{Quantum Fourier transform}
Let $\mathcal V_N$ be a state space, and $\mathcal B_N = \qty{\ket{0}, \ket{1}, \dots, \ket{N-1}}$ be an orthonormal basis for $\mathcal V_N$.
Write $\mathbb Z_N$ for integers modulo $N$, and let $\omega = e^{\frac{2\pi i}{N}}$.
For $\ket{k} \in \mathcal B_N$, we define
\[ QFT_N\ket{k} = \frac{1}{\sqrt{N}} \sum_{\ell=0}^{N-1} e^{\frac{2\pi i}{N}k\ell} \ket{\ell} = \frac{1}{\sqrt{N}} \sum_{\ell=0}^{N-1} \omega^{k\ell} \ket{\ell} \]
The quantum Fourier transform can be viewed as a generalisation of the Hadamard operator, as $QFT_2 = H$.

We show that this is a unitary operator.
\[ (QFT)_{jk} = \bra{j}QFT\ket{k} = \frac{1}{\sqrt{N}} \sum_{\ell=0}^{N-1} \omega^{k\ell} \ip{j}{\ell} = \frac{1}{\sqrt{N}} \omega^{jk} \]
\[ QFT = \frac{1}{\sqrt{N}} \begin{pmatrix}
    1 & 1 & 1 & 1 & \cdots \\
    1 & \omega & \omega^2 & \omega^3 & \cdots \\
    1 & \omega^2 & \omega^4 & \omega^6 & \cdots \\
    1 & \omega^3 & \omega^6 & \omega^9 & \cdots \\
    \vdots & \vdots & \vdots & \vdots & \ddots
\end{pmatrix} \]
Let $S_j$ be the sum of the $j$th row or column.
If $j = 0$, $S_j = \frac{1}{\sqrt{N}} N$.
Otherwise,
\[ S_j = \frac{1}{\sqrt{N}} (1 + \omega^j + \dots + \omega^{j(N-1)}) = \frac{1}{\sqrt{N}} \cdot \frac{1 - \omega^{jN}}{1 - \omega^j} = 0 \]
We can use this to prove that $(QFT^\dagger QFT)_{jk} = \delta_{jk}$, so it is a unitary operator.

\subsubsection{Application: Periodicity Determination}

Suppose we have a periodic function $f \colon \mathbb Z_N \to Y$, where typically $Y = \mathbb Z_M$ for some $M$.
Let $r$ be the smallest integer in $\mathbb Z_N$ for which $f(x+r) = f(x)$ for all $x \in \mathbb Z_N$, so $f$ is periodic with period $r$.
Suppose further that $f$ is injective in each period.
We wish to find $r$ with a particular probability of error.

There is a classical algorithm with query complexity $O(\sqrt{N}) = O\qty(2^{\log N^{\frac{1}{2}}}) = O\qty(2^{\frac{1}{2} \log N})$.
In the quantum case, for any error probability $\varepsilon \in (0,1)$, there is an algorithm with query complexity $O(\log \log N)$, which provides an exponential speed increase.

We first describe an attempt to construct such an algorithm without using the quantum Fourier transform.
Begin with the uniform superposition state $\ket{\psi_N} = \frac{1}{\sqrt{N}} \sum_{x=0}^{N-1} \ket{x}$.
Consider the quantum oracle $U_f$ corresponding to $f \colon \mathbb Z_N \to \mathbb Z_M$, defined by $U_f \ket{x} \ket{y} = \ket{x} \ket{y + f(x)}$, where addition is performed modulo $M$.
Set the output register $\ket{y}$ to $\ket{0}$, and then compute $\ket{f} = U_f \ket{\psi_N} \ket{0}$.
We obtain
\[ \ket{f} = U_f \ket{\psi_N} \ket{0} = \frac{1}{\sqrt{N}} \sum_{x=0}^{N-1} U_f \ket{x} \ket{0} = \frac{1}{\sqrt{N}} \sum_{x=0}^{N-1} \ket{x} \ket{f(x)} \]
Since $r$ is the period, we have $r \mid N$, so let $A = \frac{N}{r} \in \mathbb N$ be the number of periods.
We now measure the second register, giving an outcome $y = f(x_0)$ for some $x_0 \in \qty{0, \dots, r-1}$.
Note that $y = f(x_0 + jr)$ for any $j \in \qty{0, \dots, A-1}$.
The terms in $\ket{f}$ which contribute to the outcome $y = f(x_0)$ are
\[ \frac{1}{\sqrt{N}} \sum_{j=0}^{A-1} \ket{x_0 + jr} \ket{f(x_0)} \]
Hence, the probability of obtaining a particular outcome $f(x_0)$ is $\frac{A}{N} = \frac{1}{r}$.
Then, the post-measurement state of the input register is
\[ \ket{\text{per}} = \frac{1}{\sqrt{A}} \sum_{j=0}^{A-1} \ket{x_0 + jr} \]
The state $\ket{\text{per}}$ is periodic.
If we measure the input register, we obtain $\ket{x_0 + j_0 r}$ for some $j_0 \in \qty{0, \dots, A-1}$, selected uniformly at random.
The probability that the outcome of this second measurement is $x_0 + j_0 r$ is $\frac{1}{A}$.
Therefore, no information about $r$ is obtained.

We resolve this issue by utilising the quantum Fourier transform.
Instead of measuring the input register, we act on $\ket{\text{per}}$ by $QFT_N$.
Since
\[ QFT_N\ket{x} = \frac{1}{\sqrt{N}}\sum_{y=0}^{N-1}\omega^{xy} \ket{y} \]
we find
\begin{align*}
    QFT_N \ket{\text{per}} &= \frac{1}{\sqrt{A}}\sum_{y=0}^{N-1} QFT_N \ket{x_0 + jr} \\
    &= \frac{1}{\sqrt{A}} \frac{1}{\sqrt{N}} \sum_{j=0}^{A-1} \sum_{y=0}^{N-1} \omega^{(x_0 + jr)y} \ket{y} \\
    &= \frac{1}{\sqrt{NA}} \sum_{y=0}^{N-1} \omega^{x_0 y} \underbrace{\qty[\sum_{j=0}^{A-1} \qty(\omega^{ry})^j]}_S \ket{y}
\end{align*}
Note that $Ar = N$ and
\[ S = \begin{cases}
    A & \text{if } \omega^{ry} = 1 \\
    \frac{1 - \omega^{ryA}}{1 - \omega^{ry}} = 0 & \text{otherwise}
\end{cases} \]
Note that $\omega^{ry} = 1$ if $y = kA = \frac{kN}{r}$ for $k \in \qty{0, \dots, r - 1}$.
Hence, we obtain
\[ QFT_N \ket{\text{per}} = \frac{A}{\sqrt{NA}} \sum_{k=0}^{r-1} \omega^{x_0 \frac{kN}{r}} \ket{\frac{kN}{r}} = \frac{1}{\sqrt{r}} \sum_{k=0}^{r-1} \omega^{x_0 \frac{kN}{r}} \ket{\frac{kN}{r}} \]
The value of $x_0$ is no longer present in a ket, and has been converted into phase information.
It therefore does not affect measurement outcomes.
The periodicity in $r$ has been inverted into periodicity in $\frac{1}{r}$.
The resulting state is still periodic, but each period begins at 0 instead of $x_0$.

Now, when measuring this register, the outcome is $c = \frac{k_0N}{r}$ for some $k_0 \in \qty{0, \dots, r-1}$.
Each outcome occurs with probability $r\inv$.
Note that $\frac{k_0}{r} = \frac{c}{N}$, and $\frac{c}{N}$ is known after performing the measurement; we wish to know the value of $r$.

Suppose first that $k_0$ is coprime to $r$.
In this case, we can cancel $\frac{c}{N}$ to its lowest form, then the denominator is $r$.
If $k_0$ is not coprime to $r$, the denominator $\widetilde r$ will instead be a factor of $r$.
To solve this, we can compute the reduced denominator and then evaluate $f(0), f(\widetilde r)$; if they are equal, $\widetilde r = r$, and otherwise, $\widetilde r \mid r$.
We would like to know the probability that a randomly chosen $k_0$ is coprime to the true periodicity $r$.
\begin{theorem}[Coprimality Theorem]
    Let $\varphi(r)$ denote the number of integers less than $r$ that are coprime to $r$.
    Then there exist $c > 0, r_0 > 0$ such that for all $r \geq r_0$, $\varphi(r) \geq c \frac{r}{\log \log r}$.
    In particular, $\varphi(r) = \Omega\qty(\frac{r}{\log \log r})$.
\end{theorem}
This theorem implies that since $k_0$ is chosen uniformly at random, the probability that $k_0$ is coprime to $r$ is $O\qty(\frac{1}{\log \log r})$.
We claim that if we repeat this process $O(\log \log r)$ times, we will obtain an outcome $c$ such that after cancellation, $\frac{c}{N} = \frac{k_0}{r}$ where $k_0$ is coprime to $r$ in at least one case, with a constant probability.
This claim follows from the following lemma.
\begin{lemma}
    Suppose that a single trial has success probability $p$, and the trial is repeated $M$ times independently, for any $\varepsilon \in (0,1)$, the probability of at least one success is greater than $1 - \varepsilon$ if $M = \frac{-\log \varepsilon}{p}$.
\end{lemma}
Therefore, to achieve a constant probability $1 - \varepsilon$ of success, we need $O\qty(\frac{1}{p})$ trials.
In the algorithm above, $p = O\qty(\frac{1}{\log \log r})$, so we need $O(p) = O(\log \log r) < O(\log \log N)$ trials to achieve the desired result.

In each invocation of the algorithm, we query $f$ three times: once to construct the state $\ket{f}$, and twice to check if $\widetilde r$ is the true periodicity.
We also need to apply the quantum Fourier transform $QFT_N$, which has implementations in $O((\log N)^2)$ steps.
We must also perform standard arithmetic operations such as to cancel denominators, which are computable in $O(\mathrm{poly}(\log N))$ steps.
Therefore, we succeed in determining the period with any constant probability of success $1 - \varepsilon$ with $O(\log \log N)$ queries and $O(\mathrm{poly}(\log N))$ additional steps.

\subsection{Efficient implementation of quantum Fourier transform - Non Examinable}
We can implement a quantum Fourier transform using $O(\mathrm{poly}(\log N))$ gates if $N = 2^n$.
In this case, $QFT_N$ acts on $n$ qubits.
If $N \neq 2^n$, we do not have an efficient implementation; in this case, we approximate $N$ by $2^k$ for some $k \in \mathbb Z$.
In the case $N = 2^n$, we demonstrate a quantum circuit of size $O(n^2)$.

If $x \in \mathbb Z_n = \qty{0, \dots, 2^n - 1}$, note that
\[ QFT_N \ket{x} = \frac{1}{\sqrt{N}} \sum_{y=0}^{N-1} \omega^{xy} \ket{y} \]
We can represent $x$ and $y$ by $n$-bit strings.
\[ x = (x_0, x_1, \dots, x_{n-1});\quad x = \sum_{i=0}^{n-1} 2^i x_i \]
Now, $\omega^{xy} = \exp\qty[\frac{2\pi i}{2^n} xy]$.
\[ \frac{xy}{2^n} = \frac{1}{2^n} \qty[(x_0 + 2x_1 + \dots + 2^{n-1} x_{n-1})(y_0 + 2y_1 + \dots + 2^{n-1} y_{n-1})] \]
Retaining only the fractional terms of $\frac{xy}{2^n}$, as integral parts do not contribute to the final result, we obtain
\[ y_{n-1}(.x_0) + y_{n-2}(.x_1 x_0) + \dots + y_0(.x_{n-1} \dots x_0) \]
where for instance $.x_1 x_0 = \frac{x_1}{2} + \frac{x_0}{2^2}$.
Hence,
\begin{align*}
    QFT \ket{x} &= \frac{1}{\sqrt{2^n}} \sum_{y_0, \dots, y_{n-1} \in \qty{0,1}} \exp\qty[\frac{2\pi i xy}{2^n}] \ket{y_{n-1}} \dots \ket{y_0} \\
    &= \qty(\frac{1}{\sqrt{2}} \sum_{y_{n-1} \in \qty{0,1}} \exp\qty[2 \pi i y_{n-1} (.x_0)] \ket{y_{n-1}}) \cdots \qty(\frac{1}{\sqrt{2}} \sum_{y_0 \in \qty{0,1}} \exp\qty[2 \pi i y_0 (.x_{n-1}\dots x_0)] \ket{y_0}) \\
    &= \frac{1}{\sqrt{2}} \qty(\ket{0} + e^{2\pi i (.x_0)} \ket{1}) \dots \frac{1}{\sqrt{2}} \qty(\ket{0} + e^{2\pi i (.x_{n-1} \dots x_0)} \ket{1})
\end{align*}
To implement the quantum Fourier transform, we will use the Hadamard gate, the 1-qubit phase gate, and the 2-qubit controlled phase gate.
Note that we can write
\[ H\ket{x} = \frac{1}{\sqrt{2}} \qty[\ket{0} + e^{2\pi i(.x)} \ket{1}] \]
For any $d \in \mathbb Z_+$, the phase gate is given by
\[ R_d = \begin{pmatrix}
    1 & 0 \\
    0 & \exp[\frac{i\pi}{2^d}]
\end{pmatrix} = \begin{pmatrix}
    1 & 0 \\
    0 & \exp[2\pi i(.\underbrace{0\dots 0}_{\mathclap{d \text{ zeroes}}}1)]
\end{pmatrix} \]
Note that $R_d \ket{0} = \ket{0}$ and $R_d \ket{1} = e^{2\pi i(.0\dots 01)} \ket{1}$.
In the case $d = 1$, we obtain $R_1 \ket{1} = e^{2 \pi i(.01)} \ket{1} = i \ket{1}$.
The two-qubit controlled phase gate, denoted $CR_d$, is drawn
\begin{center}
    \leavevmode
    \Qcircuit{
        \lstick{\ket{\psi}} & \gate{R_d} & \qw \\
        \lstick{\ket{\varphi}} & \ctrl{-1} & \qw
    }
\end{center}
If $\ket{\varphi} = \ket{0}$, $CR_d\ket{0}\ket{\psi} = \ket{0}\ket{\psi}$.
If $\ket{\varphi} = \ket{1}$, $CR_d\ket{1}\ket{\psi} = \ket{1}R_d\ket{\psi}$.
We will now describe the quantum circuit for $QFT_8$, so $N = 8$ and $n = 3$.
\begin{center}
    \leavevmode
    \Qcircuit{
        \lstick{\ket{x_2}} & \gate{H} & \gate{R_1} & \gate{R_2} & \qw & \qw & \qw & \rstick{\ket{y_0}} \qw \\
        \lstick{\ket{x_1}} & \qw & \ctrl{-1} & \qw & \gate{H} & \gate{R_1} & \qw & \rstick{\ket{y_1}} \qw \\
        \lstick{\ket{x_0}} & \qw & \qw & \ctrl{-2} & \qw & \ctrl{-1} & \gate{H} & \rstick{\ket{y_2}} \qw
    }
\end{center}
Applying the given gates to $\ket{x_2}$, we obtain
\begin{align*}
    \ket{x_2} &\xrightarrow{H} \frac{1}{\sqrt{2}}\qty[\ket{0} + e^{2\pi i (.x_2)} \ket{1}] \\
    &\xrightarrow{R_1} \frac{1}{\sqrt{2}}\qty[\ket{0} + e^{2\pi i (.x_2)} e^{2\pi i(.0x_1)} \ket{1}] \\
    &\xrightarrow{R_2} \frac{1}{\sqrt{2}}\qty[\ket{0} + e^{2\pi i(.x_2)} e^{2\pi i(.0x_1)} e^{2\pi i(.00x_0)} \ket{1}] \\
    &= \frac{1}{\sqrt{2}}\qty[\ket{0} + e^{2\pi i(.x_2 x_1 x_0)} \ket{1}] = \ket{y_0}
\end{align*}
as required.
Typically, after applying the above circuit, we will swap the states $\ket{y_0}, \ket{y_1}, \ket{y_2}$ to be in reverse order; this takes $O(n)$ gates.

In this implementation, we used 3 Hadamard gates, and $2 + 1 = 3$ controlled phase gates.
If $N = 2^n$, we need $n$ Hadamard gates and $\frac{n(n-1)}{2} = O(n^2)$ controlled phase gates.

\subsection{Grover's algorithm}
Suppose we have a large unstructured database of $N$ items, in which we aim to locate a particular `good' item.
Suppose that given an item, we can easily check if it is the `good' item.
We wish to construct an algorithm to locate this good item with success probability at least $1 - \varepsilon$.
Each access to the database is considered a query.

In the classical case, we need $O(N)$ queries: if we find a bad item, it gives us no information about the location of the good item.
The probability that any item is good is $\frac{1}{N}$.
Given $M$ queries, the probability of success is $\frac{M}{N} \geq 1 - \varepsilon$, so $M \geq (1-\varepsilon)N$ gives $M = O(N)$.
In the quantum case, $O(\sqrt{N})$ queries are necessary and sufficient.
This is not an exponential speedup but a quadratic speedup.

Let $\mathcal V$ be a vector space, and let $\ket{v} \in \mathcal V$.
We define the rank 1 projection operator $\Pi_{\ket{\alpha}} = \op{\alpha}{\alpha}$, and the reflection operator $I_{\ket{\alpha}} = I - 2 \op{\alpha}{\alpha}$.
Note that $I_{\ket{\alpha}} \ket{\alpha} = - \ket{\alpha}$.
Let $\ket{\psi} \in \mathcal S_{\ket{v}}^\perp = \vecspan\qty{\ket{\beta} \in \mathcal V \mid \ip{\alpha}{\beta} = 0}$ .
Then $I_{\ket{\alpha}} \ket{\psi} = \ket{\psi} - 2\ket{\alpha}\ip{\alpha}{\psi} = \ket{\psi}$.

For any unitary operator $U$ acting on $\mathcal V$, we have $U \Pi_{\ket{\alpha}} U^\dagger = U \op{\alpha}{\alpha} U^\dagger = \Pi_{U\ket{\alpha}}$.
Note also that $U I_{\ket{\alpha}} U^\dagger = U(I - 2 \op{\alpha}{\alpha}) U^\dagger = I - 2 \op{U\alpha}{U\alpha} = I_{U\ket{\alpha}}$.

If $\mathcal V = \mathbb C^2$, for all $\ket{\alpha} \in \mathcal V$, let $\ket{\alpha^\perp}$ be orthogonal to $\ket{\alpha}$.
For all $\ket{v} \in \mathcal V$, we can write $\ket{v} = a \ket{\alpha} + b \ket{\alpha^\perp}$, so $\Pi_{\ket{\alpha}} \ket{v} = a \ket{\alpha}$ and $I_{\ket{\alpha}} \ket{v} = -a\ket{\alpha} + b \ket{\alpha^\perp}$.

Let $N = 2^n$, so we can label each item in the database with an $n$-bit binary string.
We will convert the search problem into a black-box promise problem.
The database corresponds to the Boolean function $f \colon B_n \to B$ where $f(x_0) = 1$ for a particular $x_0 \in B_n$, and $f(x) = 0$ otherwise.
The corresponding quantum oracle is $U_f \ket{x} \ket{y} = \ket{x} \ket{y \oplus f(x)}$, where $\ket{x} \in (\mathbb C^2)^{\otimes n}$ and $\ket{y} \in \mathbb C^2$.
The fact that the database is unstructured corresponds to the fact that the quantum oracle $U_f$ is a black box.
We will use the operator $I_{x_0}$, which has the following action on the basis vectors.
\[ I_{x_0} \ket{x} = \begin{cases}
    +\ket{x} & \text{if } x \neq x_0 \\
    -\ket{x} & \text{if } x = x_0
\end{cases} \]
If $x_0 = 0\dots 0 \in B_n$, we define $I_0 = I_{x_0}$.
Note that $I_{x_0}$ can be implemented using $U_f$; indeed,
\begin{align*}
    U_f \ket{x} \ket{-} &= \frac{1}{\sqrt{2}} U_f \ket{x} \qty(\ket{0} - \ket{1}) \\
    &= \frac{1}{\sqrt{2}}\qty(\ket{x} \ket{f(x)} - \ket{x} \ket{f(x)^c}) \\
    &= \begin{cases}
        \frac{1}{\sqrt{2}} \ket{x} \qty(\ket{0} - \ket{1}) & \text{if } x \neq x_0 \\
        \frac{1}{\sqrt{2}} \ket{x} \qty(\ket{1} - \ket{0}) & \text{if } x = x_0
    \end{cases} \\
    &= \begin{cases}
        +\ket{x}\ket{-} & \text{if } x \neq x_0 \\
        -\ket{x}\ket{-} & \text{if } x = x_0
    \end{cases}
\end{align*}
Hence, $U_f \ket{x} \ket{-} = (I_{x_0} \ket{x})\ket{-}$.
So if $\ket{\psi} \in (\mathbb C^2)^{\otimes n}$, $\ket{\psi} = a_0 \ket{x_0} + \sum_{x \neq x_0} a_x \ket{x}$ gives $U_f \ket{\psi}\ket{-} = (I_{x_0} \ket{\psi}) \ket{-} = -a_0 \ket{x_0} + \sum_{x \neq x_0} a_x \ket{x}$.

Given a black box which computes $I_{x_0}$ for some $x_0 \in B_n$, we wish to determine $x_0$ with the least amount of queries.
We will now describe Grover's algorithm.
We begin with the equal superposition state $\ket{\psi_0} = \frac{1}{\sqrt{2^n}} \sum_{x \in B_n} \ket{x}$.
Consider \vocab{Grover's iteration operator} $Q = -H_n I_0 H_n I_{x_0} = -I_{\psi_0} I_{x_0}$\footnote{$H_n I_0 H_n = I_{H \ket{0}} = I_{\psi_0}$.} where $H_n = H^{\otimes n}$.
Note that $Q$ is real-valued, so acts geometrically on the real-valued vector $\ket{\psi_0}$ in real Euclidean space.
It has the following properties.
\begin{enumerate}
    \item In the plane $\mathcal P(x_0)$ spanned by $\ket{x_0}$ and $\ket{\psi_0}$, $Q$ acts as a rotation through an angle $2\alpha$ where $\sin \alpha = \frac{1}{\sqrt{2^n}}$.
    \item In the plane orthogonal to $\mathcal P(x_0)$, $Q$ acts as $-I$.
\end{enumerate}
We repeatedly apply $Q$ to $\ket{\psi_0}$ to obtain the rotated vector $\ket{\psi_0'}$, and then measure in the computational basis.
\[ \ket{\psi_0'} = a_0 \ket{x_0} + \sum_{x_i \neq x_0} \sum a_i \ket{x_i} \]
Hence, the probability that the outcome is $x_0$ is $\abs{a_0}^2 = \abs{\ip{x_0}{\psi_0'}}^2 = \abs{\cos \delta}^2 \approx 1$ where $\delta$ is the angle between $\ket{\psi_0'}$ and $\ket{x_0}$.

If $n$ is large, $\ket{\psi_0}$ is almost orthogonal to $\ket{x_0}$, with $\ip{x_0}{\psi_0} = \frac{1}{\sqrt{2^n}} = \cos \beta$.
By property (1), $Q$ acting on $\ket{\psi_0}$ rotates the state by $2\alpha$, where $\sin\alpha = \frac{1}{\sqrt{2^n}}$.
Let $m$ be the number of iterations needed to rotate $\ket{\psi_0}$ close to $\ket{x_0}$.
Then
\[ m = \frac{\beta}{2\alpha} = \frac{\arccos\qty(\frac{1}{\sqrt{2^n}})}{2\arcsin\qty(\frac{1}{\sqrt{2^n}})} \]
Since $\sin \alpha \approx \alpha$, this implies that $2\alpha \approx 2\sin\alpha = \frac{2}{\sqrt{2^n}}$.
Then $2 \alpha m \approx \frac{\pi}{2}$, so $m \approx \frac{\pi}{4\alpha} = \frac{\pi}{4} \sqrt{N}$.
The number of iterations is independent of $\ket{x_0}$; it depends only on $n$.
\begin{example}
    Consider a database with four items, so $n = 2, N = 4$.
    Here, $\sin \alpha = \frac{1}{2}$, so $\alpha = \frac{\pi}{6}$.
    $Q$ causes a rotation through $2\alpha = \frac{\pi}{3}$.
    The initial state is
    \[ \ket{\psi_0} = \ket{++} = \frac{1}{2}\qty(\ket{00} + \ket{01} + \ket{10} + \ket{11}) \]
    For any $x_0 \in B_2$, we have $\cos\beta = \ip{x_0}{\psi_0} = \frac{1}{2}$ so $\beta = \frac{\pi}{3}$.
    Therefore, we need precisely one iteration, which rotates $\ket{\psi_0}$ to $\ket{x_0}$ exactly.
    Performing a measurement in the computational basis, we obtain $x_0$ with certainty.
\end{example}
We now prove the geometric properties of $Q$.
First, note that $Q = -H_n I_0 H_n I_{x_0} = -I_{\ket{\psi_0}} I_{\ket{x_0}}$.
If $\ket{\alpha}, \ket{v} \in \mathcal V$ and $\ket{v} \in \mathcal P(x_0)$, we have
\begin{itemize}
    \item $I_{\ket{x_0}} \ket{v} = \ket{v} - 2\ip{x_0}{v} \ket{x_0}$;
    \item $I_{\ket{\psi_0}} \ket{v} = \ket{v} - 2\ip{\psi_0}{v} \ket{\psi_0}$.
\end{itemize}
These operators are reflections about lines perpendicular to $\ket{x_0}$ and $\ket{\psi_0}$ respectively.
Thus, $\mathcal P(x_0)$ is stable under the action of $I_{\ket{x_0}}$ and $I_{\ket{\psi_0}}$.

Let $M_1, M_2$ be lines in the Euclidean plane, intersecting at $O$.
Let $\theta$ be the angle between $M_1$ and $M_2$.
Then, reflection about $M_1$ then $M_2$ acts as an anticlockwise rotation by $2\theta$ about $O$.

In our case, the angle between the lines perpendicular to $\ket{x_0}$ and $\ket{\psi_0}$ is $\beta$.
Therefore, $I_{\ket{\psi_0}} I_{\ket{x_0}}$ is an anticlockwise rotation by an angle of $2\beta$.
For any real unit vector $v \in \mathbb R^2$, we have $-I_v = I_{v^\perp}$ where $v_\perp$ is a unit vector orthogonal to $v$.
Hence, $-I_{\ket{\psi_0}} I_{\ket{x_0}} = I_{\ket{\psi_0^\perp}} I_{\ket{x_0}}$, which is an anticlockwise rotation by an angle of $2\alpha$, as $\alpha + \beta = \frac{\pi}{2}$.
This proves property (1).

Now consider $\ket{\xi} \in \mathcal P(x_0)^\perp$ perpendicular to $\ket{\psi_0}$ and to $\ket{x_0}$.
Clearly $I_{\ket{x_0}} \ket{\xi} = \ket{\xi}$ and $I_{\ket{\psi_0}} \ket{\xi} = \ket{\xi}$.
So $Q \ket{\xi} = -\ket{\xi}$, giving property (2).

Grover's algorithm achieves an unstructured search for a unique good item in approximately $\frac{\pi}{4}\sqrt{N}$ queries, and there is no algorithm that has smaller asymptotic query complexity.
Any quantum algorithm that achieves this search in an unstructured database of size $N$ must use $O(\sqrt{N})$ queries.
Moreover, it can be shown that $\frac{\pi}{4} (1 - \varepsilon) \sqrt{N}$ queries are insufficient for each $\varepsilon$, so Grover's algorithm is tight.

\subsection{Grover's algorithm for multiple items}
Consider the case where there are $r \geq 1$ good items, and $r$ is known.
Here, $f(x_i) = 1$ if $i = 1, \dots, r$, and $f(x) = 0$ otherwise, where $x_1, \dots, x_r$ are the binary labels for the good items.
We want to find any of the good items.
Then, define
\[ I_G \ket{x} = I - 2\sum_{i=1}^r \op{x_i}{x_i} = \begin{cases}
    +\ket{x} & x \not\in \qty{x_1, \dots, x_r} \\
    -\ket{x} & x \in \qty{x_1, \dots, x_r}
\end{cases} \]
Note that $I_G$ is not of the form $I_{\ket{v}}$ for a single vector $\ket{v}$.
Now, define $Q_G = -H_n I_0 H_n I_G = -I_{\ket{\psi_0}} I_G$.
Let $\ket{\psi_G} = \frac{1}{\sqrt{r}} \sum_{i=1}^r \ket{x_i}$ be an equal superposition of the good states, and $\ket{\psi_B} = \frac{1}{\sqrt{N-r}} \sum_{i=r+1}^N \ket{x_i}$ be an equal superposition of the bad states.
Note that $\ip{\psi_G}{\psi_B} = 0$.
Begin with the equal superposition state.
\[ \ket{\psi_0} = (H\ket{0})^{\otimes N} = \frac{\sqrt{r}}{\sqrt{N}} \ket{\psi_G} + \frac{\sqrt{N-r}}{\sqrt{N}} \ket{\psi_B} \]
Consider the plane $\mathcal P_G$ spanned by $\ket{\psi_G}$ and $\ket{\psi_0}$, which contains $\ket{\psi_B}$.
Let $\alpha$ be the angle between $\ket{\psi_G}$ and $\ket{\psi_0^\perp}$.

We show that in the plane $\mathcal P_G$, $Q_G$ acts as a rotation through an angle $2\alpha$ where $\sin \alpha = \ip{\psi_0}{\psi_G} = \frac{\sqrt{r}}{\sqrt{N}}$.
The states $\ket{\psi_G}, \ket{\psi_B}$ form an orthonormal basis for $\mathcal P_G$.
We find $I_G (a \ket{\psi_G} + b \ket{\psi_B}) = -a \ket{\psi_G} + b \ket{\psi_B}$; indeed, restricting to the plane $\mathcal P_G$, the action of $I_G$ is precisely the action of $I_{\ket{\psi_G}}$.
Hence, as before, $Q_G$ causes the desired rotation through $2\alpha$ in this plane.
The probability of finding a single good item is $\abs{\ip{\psi}{\psi_G}}^2$, as $\ket{\psi} = a \ket{\psi_G} + b \ket{\psi_B}$.

\begin{aside}{$r$ unknown - Non Examinable}
    Suppose now that $r$ is unknown.
    In this case, we start with $\ket{\psi_0}$ and repeatedly apply $Q$ to rotate $\ket{\psi_0}$ to $\ket{\psi_G}$ as before.
    However, we do not know how many iterations of $Q$ to apply, since this depends on $r$.

    If $r \ll N$, we choose $K$ uniformly at random in $\qty(0, \frac{\pi}{4}\sqrt{N})$, and apply $K$ iterations of $Q$.
    We measure the final state $\ket{\psi^K}$ to obtain $x$, and check if $f(x) = 1$ or not.
    Note that each iteration causes a rotation of $2\alpha$ where $\sin \alpha = \frac{\sqrt{r}}{\sqrt{N}}$ so $2\alpha \approx 2\frac{\sqrt{r}}{\sqrt{N}}$.
    Choosing $K$ therefore implicitly chooses a random angle in the range $\qty(0, \frac{\pi}{2} \sqrt{r})$.
    Now, if the final rotated state $\ket{\psi}$ makes an angle within $\pm \frac{\pi}{4}$ with $\ket{\psi_G}$, the probability of locating a good item is $\abs{\ip{\psi}{\psi_G}}^2 \geq \cos^2 \frac{\pi}{4} = \frac{1}{2}$.
    Since for every quadrant in the plane $\mathcal P_G$, half of the angles are within $\pm \frac{\pi}{4}$ from the $y$-axis, the randomised procedure using $O(\sqrt{N})$ queries will locate a good item with probability approximately $\frac{1}{4}$.
    The procedure can then be repeated to reduce the error probability to an acceptable level.
\end{aside}

\subsection{\texorpdfstring{$\mathsf{NP}$}{NP} problems}
A \vocab{verifier} $V$ for a language $L$ is a computation with two inputs $w, c$ such that
\begin{enumerate}
    \item if $w \in L$, there exists a \vocab{certificate of membership} $c$ such that $V(w,c)$ halts in an accepting state; and
    \item if $w \not\in L$, for any $c$, $V(w,c)$ halts in a rejecting state.
\end{enumerate}
$V$ is a \vocab{poly-time} verifier if for all inputs $w, c$, the algorithm $V$ runs in polynomial time in $n$, where $n$ is the size of the input $w$.
A problem in the \vocab{non-deterministic polynomial-time} complexity class $\mathsf{NP}$ is easy to verify, but may be hard to solve.
More precisely, a language $L$ is in $\mathsf{NP}$ if it has a polynomial time verifier $V$.

Alternatively, consider a computer operating non-deterministically; at each binary choice, the computer duplicates itself and performs both branches in parallel.
We require that all possible paths eventually halt with either an accepting or rejecting state.
The running time of a given algorithm is the length of the longest path.
The computation is defined to accept its input if at least one path accepts it, and rejects its input if all paths reject it.
One can check that $\mathsf{NP}$ is precisely the class of languages that are decided by a non-deterministic computation with polynomial running time.

Let $f \colon B_n \to B$ be a Boolean formula.
The \vocab{Boolean satisfiability problem} $\mathsf{SAT}$ seeks an assignment of the variables $x_1, \dots, x_n$ such that $f(x_1, \dots, x_n) = 1$.
Any such assignment is called a \vocab{satisfying assignment}.
This problem clearly lies in $\mathsf{NP}$; if $f$ is satisfiable, then $c$ is any assignment for which $V(f,c) = 1$ where $V(f,c) = f(c)$.
Brute-force methods have $O(2^n)$ runtime.

Searching for arbitrarily many good items in an unstructured database corresponds to $\mathsf{SAT}$.
Assuming that there are few satisfying assignments, we can run the randomised Grover's algorithm to give a quantum algorithm for solving $\mathsf{SAT}$ in $O(\sqrt{2^n})$ time with low probability of error.
Any $\mathsf{NP}$ problem can be converted into an application of $\mathsf{SAT}$; we say $\mathsf{SAT}$ is \vocab{$\mathsf{NP}$-complete}.
Grover's algorithm can hence be applied to any $\mathsf{NP}$ problem to provide a quadratic speedup.

\subsection{Shor's algorithm}
Suppose $N$ is a positive integer and $n = \ceil{\log N}$ is the number of bits in a binary representation of $N$.
We wish to factorise $N$.
We will describe an algorithm which, given $N$ and a fixed acceptable probability of error, outputs a factor $1 < k < N$, or outputs $N$ if $N$ is prime.
This algorithm runs in polynomial time in $n$; there is no classical algorithm with this property.

We first use results from number theory to convert the problem into a periodicity determination problem.
Then, we apply the quantum period-finding algorithm using the quantum Fourier transform.

Choose an integer $1 < a < N$ uniformly at random, and compute $b = \mathrm{gcd}(a,N)$.
If $b > 1$, then $b \mid N$ so is a factor; in this case we simply output $b$.
If $b = 1$, then $a, N$ are coprime.
\begin{theorem}[Euler's theorem]
    Let $a, N$ be coprime.
    Then there exists $1 < r < N$ s.t. $a^r \equiv 1$ mod $N$.
    A minimal such $r$ is called the \vocab{order} of $a$ modulo $N$.
\end{theorem}
Consider the \vocab{modular exponentiation function} $f \colon \mathbb Z \to \faktor{\mathbb Z}{N\mathbb Z}$ s.t. $f(k) = a^k$ mod $N$.
This function satisfies $f(k_1 + k_2) = f(k_1)f(k_2)$.
$f$ is periodic with period $r$, and is injective within each period as $r$ minimal.

Suppose that we can find $r$, and suppose $r$ is even.
Then $a^r - 1 \equiv (a^{\frac{r}{2}}+1)(a^{\frac{r}{2}}-1) \equiv 0$ mod $N$.
Note that $N \nmid (a^{\frac{r}{2}}-1)$ since $r$ was minimal s.t. $a^r \equiv 1$ mod $N$.
If $N \nmid (a^{\frac{r}{2}}+1)$, then $N$ must have some prime factors in $(a^{\frac{r}{2}}+1)$ and some in $(a^{\frac{r}{2}}-1)$.
We can use Euclid's algorithm to compute $\mathrm{gcd}(a^{\frac{r}{2}}+1, N)$ and $\mathrm{gcd}(a^{\frac{r}{2}}-1, N)$, which are factors of $N$.
Thus, we find factors of $N$ provided $r$ is even and $a^{\frac{r}{2}} + 1 \not\equiv 0$ mod $N$.

Consider $N = 15, a = 7$.
Then $f(k) = 7^k$ mod $15$ takes values $1, 7, 4, 13$, so has period $r = 4$.
This is even, so we can write $a^r - 1 = (a^{\frac{r}{2}} + 1)(a^{\frac{r}{2}} - 1) = 50 \cdot 48$.
$N = 15$ does not divide $50$, so $\gcd(50, N) = 5$ is a factor, and $\gcd(48, 15) = 3$ is a factor.
\begin{theorem}
    Let $N$ be odd and not a prime power.
    Then, choosing $a$ uniformly at random s.t. $\gcd(a,N) = 1$, the probability that $r$ is even and $(a^{\frac{r}{2}} + 1) \not\equiv 0$ mod $N$ is at least $\frac{1}{2}$.
\end{theorem}
This implies that if $N$ is odd and not a prime power, we obtain a factor of $N$ with probability at least $\frac{1}{2}$.
We repeat this process until the probability of not finding a factor is acceptably low.
If $N$ is even, we simply output 2 as a factor.
\begin{lemma}
    Let $N = c^\ell$ for some $c, \ell \in \mathbb N$.
    There is a classical polynomial-time algorithm that computes $c$.
\end{lemma}
% If the previous tests fail, we can run this classical algorithm to test if $N$ is a prime power.
Shor's algorithm can be summarised as follows.
\begin{enumerate}
    \item Test if $N$ is even; if so, output 2 and halt.
    \item Run the classical algorithm to test if $N$ is of the form $c^\ell$ with $\ell > 1$; if so, output $c$ and halt.
    \item Choose $1 < a < N$ uniformly at random and compute $b = \gcd(a,N)$.
    If $b > 1$, output $b$ and halt.
    \item Find the period $r$ of the modular exponentiation function $f(k) = a^k$ mod $N$.
    If this fails, return to step (3).
    \item If $r$ is even and $(a^{\frac{r}{2}} + 1) \not\equiv 0$ mod $N$, compute $t = \gcd(a^{\frac{r}{2}} + 1, N)$; if $1 < t < N$, output $t$ and halt.
    Otherwise, return to step (3).
\end{enumerate}
We now describe the method to compute the period of the modular exponentiation function.
Note that $f \colon \mathbb Z \to \mathbb Z_N$, not $\mathbb Z_N \to \mathbb Z_M$; we therefore cannot directly use the algorithm discussed previously.
We must first truncate the domain $\mathbb Z$ to some $\mathbb Z_M$.
If $r$ is unknown, $f$ will not necessarily be periodic on $\mathbb Z_M$.
However, if $M$ is $O(N^2)$, the single incomplete period has a negligible effect on the periodicity determination.
We will define $M = 2^m$ for some $m$ and use $QFT_M$.

Consider a finite domain $D = \qty{0, \dots, 2^m - 1}$, where $m$ is the smallest integer s.t. $2^m > N^2$.
Suppose $2^m = Br + b$ where $0 \leq b < r$, so $B = \floor*{\frac{2^m}{r}}$.
We start with the equal superposition state $\ket{\psi_m} = \frac{1}{\sqrt{2^m}} \sum_{x \in D} \ket{x}$.
Consider the quantum oracle $U_f$ corresponding to the modular exponentiation function $f$.
Then
\begin{align*}
    \ket{\Psi} &= U_f \ket{\psi_m} \ket{0} \\
    &= \frac{1}{\sqrt{2^m}} \sum_{x \in D} \ket{x} \ket{f(x)} \\
    &= \frac{1}{\sqrt{2^m}} \sum_{x_0 = 0}^{b-1} \sum_{j=0}^B \ket{x_0 + jr} \ket{f(x_0)} + \frac{1}{\sqrt{2^m}} \sum_{x_0 = b}^r \sum_{j=0}^{B-1} \ket{x_0 + jr}\ket{f(x_0)}
\end{align*}
Measuring the second register, we obtain an outcome $y = f(x_0)$.
In the case $x_0 < b$, $f(x_0) = f(x_0 + jr)$ for $j \in \qty{0, \dots, B}$.
If $x_0 \geq b$, $f(x_0) = f(x_0 + jr)$ for $j \in \qty{0, \dots, B - 1}$.

If $y = f(x_0)$ for $x_0 < b$, the probability of measuring $y$ is $\frac{B+1}{2^m}$.
The post-measurement state of the first register is $\ket{\mathrm{per}} = \frac{1}{\sqrt{B+1}} \sum_{j=0}^B \ket{x_0 + jr}$.
In the case $x_0 \geq b$, we have $\ket{\mathrm{per}} = \frac{1}{\sqrt{B}} \sum_{j=0}^{B-1} \ket{x_0 + jr}$ with prob $\frac{B}{2^m}$.
In both cases,
\[ \ket{\mathrm{per}} = \frac{1}{\sqrt{A}} \sum_{j=0}^{A-1} \ket{x_0 + jr} \]
where $A = B+1$ if $y = f(x_0)$ with $x_0 < b$ and $A = B$ if $y = f(x_0)$ with $x_0 \geq b$.
We act on $\ket{\mathrm{per}}$ by $QFT_{2^m}$ to obtain
\begin{align*}
    QFT_{2^m} \ket{\mathrm{per}} &= \frac{1}{\sqrt{A}} \frac{1}{\sqrt{2^n}} \sum_{j = 0}^{A-1} \sum_{c=0}^{2^m - 1} \omega^{(x_0 + jr)c}\ket{c} \\
    &=  \frac{1}{\sqrt{A}} \frac{1}{\sqrt{2^n}} \sum_{c=0}^{2^m - 1} \omega^{x_0 c} \underbrace{\qty[\sum_{j = 0}^{A-1} (\omega^{cr})^j]}_{S} \ket{c}
\end{align*}
where $\omega = 2^{\frac{2\pi i}{M}}$ where $M = 2^m$.
$S$ is a geometric series with $\alpha = \omega^{cr}$.
If $\frac{M}{r} \not\in \mathbb Z$, $\alpha^A \neq 1$.
We claim that a measurement on $QFT_{2^m} \ket{\mathrm{per}}$ yields an integer $c$ which is close to a multiple of $\frac{M}{r}$ with high probability.

Consider $k\frac{2^m}{r}$ for $k = 0, \dots, r-1$.
Each of these multiples is within $\frac{1}{2}$ of a unique integer; indeed, $2^m = Br + b$ so $r < 2^m$, giving that $k\frac{2^m}{r}$ cannot be a half integer.
Consider the values of $c$ s.t. $\abs{c - k \frac{2^m}{r}} < \frac{1}{2}$ for $k = 0, \dots, r-1$.
% Note that $\omega^{cr} = 1$ if $e^{\frac{2\pi i cr}{M}} = 1$.
\begin{theorem}
    Suppose that $QFT_{2^m} \ket{\mathrm{per}} = \sum_{c=0}^{2^m - 1} g(c) \ket{c}$, and that we measure the state and receive an outcome $c$.
    Let $c_k$ be the unique integer s.t. $\abs{c_k - k\frac{2^m}{r}} < \frac{1}{2}$.
    Then $\prob{c = c_k} > \frac{\gamma}{r}$ for a fixed constant $\gamma$ (which can be shown to be $\frac{4}{\pi^2}$).
    Moreover, the probability that $k, r$ are coprime is $\Omega\qty(\frac{1}{\log \log r})$ by the coprimality theorem.
\end{theorem}
Thus, with $O(\log \log N) > O(\log \log r)$ repetitions, we obtain a good $c$ value with high probability.
Suppose that we measure $c$ s.t. $\abs{c - k \frac{2^m}{r}} < \frac{1}{2}$, so $\abs{\frac{c}{2^m} - \frac{k}{r}} < \frac{1}{2^{m+1}}$.
Recall that $r < N$ and $m$ is minimal s.t. $2^m > N^2$.
Then $\abs{\frac{c}{2^m} - \frac{k}{r}} < \frac{1}{2N^2}$.
Note that $\frac{c}{2^m}$ is known.

We show that there is at most one fraction $\frac{k}{r}$ with denominator $r < N$ s.t. $\abs{\frac{c}{2^m} - \frac{k}{r}} < \frac{1}{2N^2}$.
Suppose $\frac{k'}{r'}, \frac{k''}{r''}$ both satisfy this requirement.
Then
\[ \abs{\frac{k'}{r'} - \frac{k''}{r''}} = \frac{\abs{k'r'' - k''r'}}{r'r''} \geq \frac{1}{r'r''} > \frac{1}{N^2} \]
But $\abs{\frac{c}{2^m} - \frac{k'}{r'}}, \abs{\frac{c}{2^m} - \frac{k'}{r'}} < \frac{1}{2N^2}$, contradicting the triangle inequality.
This result is the reason for choosing $m$ minimal s.t. $2^m > N^2$.
Therefore, we have with high probability that $\frac{c}{2^m}$ is close to a unique fraction $\frac{k}{r}$.
\begin{example}
    Let $N = 39$ and choose $a = 7$; note that 7 and 39 are coprime.
    Let $r$ be the period of $f(k) = a^k$ mod 39.
    Note that $2^{10} < N^2 < 2^{11}$, so set $m = 11$.
    Suppose that $QFT_{2^{11}} \ket{\mathrm{per}}$ gives a measurement of $c$.
    Then $\abs{c - k \frac{2^{11}}{r}} < \frac{1}{2}$ with probability $\frac{\gamma}{r}$.

    Suppose $c = 853$.
    One can explicitly check all fractions of the form $\frac{a}{b}$ to find one that satisfies $\abs{\frac{a}{b} - \frac{853}{2048}} < \frac{1}{2^{12}}$.
    This is consistent with $\frac{a}{b} = \frac{5}{12}, \frac{10}{24}$; as we are constrained by coprimality we must choose $r = 12$.
    One can check that $7^{12} \equiv 1$ mod 39, hence $r = 12$.
    Note that $O(N^2) = O(\exp(n))$ computations are needed for this calculation; there is a more efficient way to compute $a, b$ using continued fractions.
\end{example}
A rational number $\frac{s}{t}$ can be written in the form of a continued fraction
\[ \frac{s}{t} = \frac{1}{a_1 + \frac{1}{a_2 + \frac{1}{\dots + \frac{1}{a_\ell}}}} = [a_1, \dots, a_\ell] \]
where $a_1, \dots, a_\ell$ are positive integers.
We can write $\frac{s}{t} = \frac{1}{\frac{t}{s}} = \frac{1}{a_1 + \frac{s_1}{t_1}}$, and so on.
For example, if $\frac{s}{t} = \frac{13}{35}$, we can find $a_1 = 2, a_2 = 2, a_3 = 1, a_4 = 1, a_5 = 2$ and $\ell = 5$.
Since the sequence $t_k$ is decreasing, the expansion will always terminate.
For each $k = 1, \dots, \ell$, we can truncate the computation at level $k$.
This gives the sequence of rational numbers
\[ \frac{p_1}{q_1} = [a_1], \frac{p_2}{q_2} = [a_1, a_2], \dots, \frac{p_\ell}{q_\ell} = [a_1, \dots, a_\ell] = \frac{s}{t} \]
$\frac{p_k}{q_k}$ is the $k$th \vocab{convergent} of the continued fraction $\frac{s}{t}$.
\begin{lemma}
    Let $a_1, \dots, a_\ell$ be positive reals, and let $p_0 = 0, q_0 = 1, p_1 = 1, q_1 = a_1$.
    Then,
    \begin{enumerate}
        \item $[a_1, \dots, a_k] = \frac{p_k}{q_k}$ where $p_k = a_k p_{k-1} + p_{k-2}$ and $q_k = a_k q_{k-1} + q_{k-2}$;
        \item if the $a_k$ are integers, then so are the $p_k$ and $q_k$, with $q_k p_{k-1} - p_k q_{k-1} = (-1)^k$ for $k \geq 1$, and moreover $\gcd(p_k, q_k) = 1$.
    \end{enumerate}
\end{lemma}
\begin{theorem}
    Consider a continued fraction $\frac{s}{t} = [a_1, \dots, a_\ell]$, and let $\frac{p_k}{q_k}$ be the $k$th convergent.
    If $s$ and $t$ are given by $m$-bit integers, then the length $\ell$ of the continued fraction is $O(m)$, and the continued fraction and its convergents can be computed in $O(m^3)$ time.
\end{theorem}
\begin{proof}[Proof sketch]
    We have $a_k \geq 1$ and $p_k, q_k \geq 1$.
    Part (i) of the above lemma implies that $(p_k)$ and $(q_k)$ are increasing sequences.
    If $k$ is even, $p_k \geq 2p_{k-2}$ and $q_k \geq 2q_{k-2}$ hence $p_k, q_k \geq 2^{\frac{k}{2}}$.
    Thus, in general, $p_k, q_k \geq 2^{\floor*{\frac{k}{2}}}$.
    We therefore need at most $\ell = O(m)$ iterations to obtain $\frac{s}{t}$ exactly, since $q_k, p_k$ are coprime and each are at least $2^{\floor*{\frac{k}{2}}}$.
    The computation of each successive $a_k$ value involves division of $O(m)$-bit integers and converting it into an integer and remainder term; these computations can be performed in $O(m^2)$ time.
    Hence, the entire computation requires only $O(m^3)$ time.
\end{proof}
\begin{theorem}
    Let $x \in \mathbb Q$ with $0 < x < 1$.
    Let $\frac{p}{q} \in \mathbb Q$ s.t. $\abs{x - \frac{p}{q}} < \frac{1}{2q^2}$.
    Then $\frac{p}{q}$ is a convergent of the continued fraction expansion of $x$.
\end{theorem}
In our situation, we have $c$ s.t.
\[ \abs{\frac{c}{2^m} - \frac{k}{r}} < \frac{1}{2N^2};\quad r < N \]
In particular, $\abs{\frac{c}{2^m} - \frac{k}{r}} < \frac{1}{2r^2}$, and we have seen that there is at most one fraction $\frac{k}{r}$ s.t. this holds.
Note that $0 < c < 2^m$, so $0 < \frac{c}{2^m} < 1$.
Hence, $\frac{k}{r}$ is a convergent of $\frac{c}{2^m}$.
Note that $2^m > N^2 > 2^{m-1}$, so $c, 2^m$ are $O(m)$-bit integers, and hence the sequence of convergents (and in particular $\frac{k}{r}$) can be computed in $O(m^3)$ time.
We can then explicitly check for each convergent $\frac{k}{r}$ if $\abs{\frac{c}{2^m} - \frac{k}{r}} < \frac{1}{2N^2}$ and $r < N$ hold.
\begin{example}
    Consider again $N = 39$ and $2^m = 2^{11} = 2048$.
    Suppose $c = 853$.
    Then one can explicitly compute
    \[ \frac{c}{2^m} = \frac{853}{2048} = [2, 2, 2, 42, 4] \]
    Its convergents are
    \[ \frac{1}{2};\quad \frac{2}{5};\quad \frac{5}{12};\quad \frac{212}{509};\quad \frac{853}{2048} \]
    Only $\frac{5}{12}$ satisfies $\abs{\frac{c}{2^m} - \frac{k}{r}} < \frac{1}{2N^2}$ and $r < N$.
    So $r = 12$ is the period.
\end{example}
% time complexity nonexaminable
A classical factoring algorithm takes $O\qty(\exp(n^{\frac{1}{3}}))$ time; we analyse the time complexity of Shor's algorithm.
Consider the case when $N$ is odd and not a prime power, and let $n = \log N$.
The modular exponentiation function requires $O(m) = O(n)$ multiplications, each of which take $O(m^2) = O(n^2)$ time, so this algorithm takes $O(n^3)$ time.
The construction of the equal superposition state requires $m = O(n)$ Hadamard gates, and applying the quantum oracle gives the state $\frac{1}{2^m} \sum_{x \in B_m} \ket{x} \ket{f(x)}$ in $O(n^3)$ steps.
We measure the second register which contains $O(n)$ qubits, hence requiring $O(n)$ single-qubit measurements.
The first register is then in state $\ket{\mathrm{per}}$.
We then apply the quantum Fourier transform $QFT_{2^m}$, which can be implemented in $O(m^2) = O(n^2)$ steps.
We then measure the first register to obtain $c$, requiring $O(n)$ single-qubit measurements.
Then, we find $r$ from $c$ using the continued fraction algorithm, requiring $O(n^3)$ steps.
A good $c$ value is obtained with probability $1 - \varepsilon$ with $O(\log \log N) = O(\log n)$ repetitions.
Then, $t = \gcd(a^{\frac{r}{2}} + 1, N)$ is computed using Euclid's algorithm, taking $O(n^3)$ steps.
If $r$ is odd or is even but $t = 1$, then we return to the start, and the case where $r$ is even and $t \neq 1$ occurs with probability at least $1 - \varepsilon$ if we perform $\log \frac{1}{\varepsilon}$ repetitions.
