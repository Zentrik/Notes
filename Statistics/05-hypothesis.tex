\section{Hypothesis testing}

\subsection{Hypotheses}
\begin{definition}[Hypothesis]
	A \vocab{hypothesis} is an assumption about the distribution of the data $X$.
\end{definition}

Scientific questions are often phrased as a decision between two hypotheses.

\begin{definition}[Null Hypothesis]
	The \vocab{null hypothesis} $H_0$ is usually a basic hypothesis, often representing the simplest possible distribution of the data.
\end{definition} 

\begin{definition}[Alternative Hypothesis]
	The \vocab{alternative hypothesis} $H_1$ is the alternative, if $H_0$ were found to be false.
\end{definition} 

\begin{example}
	Let $X = (X_1, \dots, X_n)$ be i.i.d.\ Bernoulli random variables with parameter $\theta$.
	We could take, for example, $H_0 \colon \theta = \frac{1}{2}$ and $H_1 \colon \theta = \frac{3}{4}$.
	Alternatively, we could take $H_0 \colon \theta = \frac{1}{2}$ and $H_1 \colon \theta \neq \frac{1}{2}$.
\end{example}

\begin{example}
	Suppose $X_1, \dots, X_n$ takes values in $\{0\} \cup \mathbb{N}$.
	We can take $H_0 \colon X_i \overset{\mathrm{iid}}{\sim} \mathrm{Poi}(\lambda)$ for some $\lambda > 0$, and $H_1 \colon X_i \overset{\mathrm{iid}}{\sim} f_1$ for some other distribution $f_1$.
	This is known as a \vocab{goodness of fit} test, which checks how well the model used for the data fits.
\end{example}

\begin{example}
	Say $X$ has pdf $f(\cdot \mid \theta)$ with $\theta \in \Theta$.
	We could say $H_0 : \theta \in A \subset \Theta$ and $H_1 : \theta \notin A$.
\end{example} 

\begin{definition}
	A \vocab{simple hypothesis} is a hypothesis which fully specifies the p.d.f.\ or p.m.f.\ of the data.
	A hypothesis that is not simple is called \vocab{composite}.
\end{definition}

\begin{example}
	In the first example above, $H_0 \colon \theta = \frac{1}{2}$ is simple, and $H_1 \colon \theta \neq \frac{1}{2}$ is composite.
	In the second example, $H_0 \colon X_i \overset{\mathrm{iid}}{\sim} \mathrm{Poi}(\lambda)$ is composite since $\lambda$ was not fixed.
	In the last example, $H_0$ is simple only if $|A| = 1$.
\end{example}

\subsection{Testing hypotheses}
\begin{definition}
	A \vocab{test} of the null hypothesis $H_0$ is defined by a \vocab{critical region} $C \subseteq \mathcal X$.
	When $X \in C$, we \textit{reject} the null hypothesis.
	This is a positive result.
	When $X \not\in C$ we \textit{fail to reject} the null hypothesis, or find \textit{no sufficient evidence against} the null hypothesis.
	This is the negative result.
\end{definition}

\begin{definition}[Type Errors]
	A \vocab{type I} error, or a \textit{false positive}, is the error made by rejecting the null hypothesis when it is true.
	A \vocab{type II} error, or a \textit{false negative}, is the error made by failing to reject the null hypothesis when it is false.
	When $H_0, H_1$ are simple, we define
	\begin{align*}
		\alpha &= \psub{H_0}{H_0 \text{ is rejected}} = \psub{H_0}{X \in C} \\
		\beta &= \psub{H_1}{H_0 \text{ is not rejected}} = \psub{H_1}{X \not\in C}
	\end{align*}
	The \vocab{size} of a test is $\alpha$, which is the probability of a type I error.
	The \vocab{power} of a test is $1 - \beta$, which is the probability of not finding a type II error.
\end{definition}

\begin{remark}
	There is typically a tradeoff between $\alpha$ and $\beta$.
	Often, statisticians will choose an `acceptable' value for the probability of type I errors $\alpha$, and then maximise the power with respect to this fixed $\alpha$.
	Computing the size of a test is typically simpler since it does not depend on $H_1$.
\end{remark} 

\subsection{Neyman-Pearson lemma}
Let $H_0$ and $H_1$ be simple, and let $X$ have a p.d.f.\ or p.m.f.\ $f_i$ under $H_i$.

\begin{definition}[Likelihood Ratio Statistic]
	The \vocab{likelihood ratio statistic} is defined by
	\begin{align*}
		\Lambda_x(H_0; H_1) = \frac{f_1(x)}{f_0(x)}
	\end{align*}
\end{definition} 

\begin{definition}[Likelihood Ratio Test]
	The \vocab{likelihood ratio test} is a test that rejects $H_0$ when $\Lambda_x$ exceeds a set value $k$, or more formally, $C = \qty{ x \colon \Lambda_x(H_0; H_1) > k }$.
\end{definition} 

\begin{lemma}[Neyman-Pearson]
	Suppose that $f_0, f_1$ are nonzero on the same set, and suppose that there exists $k > 0$ such that the likelihood ratio test with critical region $C = \qty{x \colon \Lambda_x(H_0; H_1) > k}$ has size $\alpha$.
	Then out of all tests of size upper bounded by $\alpha$, this test has the largest power.
\end{lemma}

\begin{remark}
	A likelihood ratio test with size $\alpha$ does not always exist for any given $\alpha$.
	However, in general we can find a \textit{randomised likelihood ratio test} with arbitrary size $\alpha$.
	This is a test where, for some values of $X$, we reject the null hypothesis; for some values, we fail to reject the null hypothesis; and for some values we reject the null hypothesis with a random chance of rejecting the null hypothesis.
\end{remark}

\begin{proof}
	Let $\overline C$ be the complement of $C$ in $\mathcal X$.
	Then, the likelihood ratio test has
	\begin{align*}
		\alpha &= \mathbb{P}_{H_0}(X \in C) = \int_C f_0(x) \dd{x} \\
		\beta &= \mathbb{P}_{H_1}(X \notin C) = \int_{\overline C} f_1(x) \dd{x}
	\end{align*}
	Let $C^\star$ be a critical region for a different test, with type I and II error probabilities $\alpha^\star, \beta^\star$.
	Here,
	\begin{align*}
		\alpha^\star = \int_{C^\star} f_0(x) \dd{x};\quad \beta^\star = \int_{\overline {C^\star}} f_1(x) \dd{x}
	\end{align*}
	Suppose $\alpha^\star \leq \alpha$.
	Then, we will show $\beta \leq \beta^\star$.
	\begin{align*}
		\beta - \beta^\star = \int_{\overline C} f_1(x) \dd{x} - \int_{\overline{C^\star}} f_1(x) \dd{x}
	\end{align*}
	By cancelling the integrals on the intersection, and using the definition of $C$,
	\begin{align*}
		\beta - \beta^\star & = \int_{\overline C \cap C^\star} f_1(x) \dd{x} - \int_{\overline{C^\star} \cap C} f_1(x) \dd{x}                                                                                         \\
		& = \int_{\overline C \cap C^\star} \underbrace{\frac{f_1(x)}{f_0(x)}}_{\leq k \text{ on } \overline C} f_0(x) \dd{x} - \int_{\overline{C^\star} \cap C} \underbrace{\frac{f_1(x)}{f_0(x)}}_{\geq k \text{ on } C} f_0(x) \dd{x} \\
		& \leq k \qty[ \int_{\overline C \cap C^\star} f_0(x) \dd{x} - \int_{\overline C^\star \cap C} f_0(x) \dd{x} ] \\
		& = k \qty[ \int_{\overline C \cap C^\star} f_0(x) \dd{x} + \int_{C \cap C^\star} f_0(x) \dd{x} - \int_{C \cap C^\star} f_0(x) \dd{x} - \int_{\overline C^\star \cap C} f_0(x) \dd{x} ]    \\
		& = k \qty[ \int_{C^\star} f_0(x) \dd{x} - \int_{C} f_0(x) \dd{x} ] \\
		& = k \qty[ \alpha^\star - \alpha ] \\
		& \leq 0 \qedhere
	\end{align*}
\end{proof}

\begin{example} \label{exm:5.5}
	Let $X_1, \dots, X_n \sim N(\mu, \sigma_0^2)$ be i.i.d., where $\sigma_0^2$ is known and $\mu$ is an unknown.
	We wish to find the most powerful test of fixed size $\alpha$ for the hypotheses $H_0 \colon \mu = \mu_0$ and $H_1 \colon \mu = \mu_1 > \mu_0$.
	The likelihood ratio is
	\begin{align*}
		\Lambda_x(H_0;H_1) & = \frac{(2\pi \sigma_0^2)^{-n/2} \exp{\frac{-1}{2\sigma_0^2} \sum (x_i - \mu_1)^2}}{(2\pi \sigma_0^2)^{-n/2} \exp{\frac{-1}{2\sigma_0^2} \sum (x_i - \mu_0)^2}} \\
		& = \exp{\underbrace{\frac{\mu_1 - \mu_0}{\sigma_0^2}}_{\geq 0} n \overline X + \frac{n(\mu_0^2 - \mu_1^2)}{2\sigma_0^2}}
	\end{align*}
	which depends only on $\overline X$, and is monotonically increasing with respect to the sample mean $\overline X$.
	Therefore, this is also monotonically increasing with respect to the statistic
	\begin{align*}
		Z = \sqrt{n}\frac{\overline X - \mu_0}{\sigma_0}
	\end{align*}
	Thus, $\Lambda_x > k$ if and only if $Z > k'$ for some $k'$.
	Hence, the likelihood ratio test has critical region $\qty{x\colon Z(x) > k'}$ for some $k'$.
	It thus suffices to find a critical region of $Z$ with size $\alpha$ in order to construct the most powerful test of this size.
	Under $H_0$, $Z \sim N(0,1)$.
	Hence, the critical region is given by $k' = \Phi^{-1}(1-\alpha)$.
	This is known as a \textit{$Z$-test}, since we are using the $Z$ statistic to define the critical region.
\end{example}

\subsection{\texorpdfstring{$p$}{p}-values}

\begin{definition}
	Let $C$ be a critical region of the form $\qty{ x: T(x) > k }$ for some test statistic $T$.
	Let $x^\star$ denote the observed data.
	Then, the \vocab{$p$-value} is
	\begin{align*}
		\psub{H_0}{T(X) > T(x^\star)}
	\end{align*}
\end{definition}

Typically, when reporting the results of a test, we describe the conclusion of the test as well as the $p$-value.

\begin{example}
	In \Cref{exm:5.5}, suppose $\mu_0 = 5$, $\mu_1 = 6$, $\alpha = 0.05$, and $x^\star = (5.1, 5.5, 4.9, 5.3)$.
	Here, $\overline{x^\star} = 5.2$ and $z^\star = 0.4$.
	The likelihood ratio test has critical region
	\begin{align*}
		\qty{x: Z(x) > \Phi^{-1}(0.95) \approx 1.645}
	\end{align*}
	The conclusion of the test here is to not reject $H_0$.
	The $p$-value is $1 - \Phi(z^\star) \approx 0.35$.
\end{example} 

\begin{proposition}
	Under the null hypothesis $H_0$, the $p$-value is a uniform random variable in $[0,1]$.
\end{proposition}

$p$ is a function of $x^\star$ and so if $x^\star$ has the $H_0$ distribution then $p(x^\star)$ is uniform.

\begin{proof}
	Let $F$ be the distribution of the test statistic $T$, which we will assume for this proof is continuous.
	Then $\forall \; u \in [0, 1]$,
	\begin{align*}
		\psub{H_0}{p < u} &= \psub{H_0}{1 - F(T) < u} \\
		&= \psub{H_0}{F(T) > 1-u} \\
		&= \psub{H_0}{T > F^{-1}(1-u)}\footnote{$F$ a bijection.} \\
		&= 1 - F(F^{-1}(1-u)) = u
	\end{align*}
	Thus $p \sim \operatorname{Unif}(0, 1)$.
\end{proof}

\subsection{Composite hypotheses}
Let $X \sim f_X(\wildcard \mid \theta)$ where $\theta \in \Theta$.
Let $H_0 = \theta \in \Theta_0 \subset \Theta$ and $H_1 = \theta \in \Theta_1 \subset \Theta$.
The probabilities of type I and type II error are now dependent on the precise value of $\theta$, rather than simply on which hypothesis is taken.

\begin{definition}[Power Function]
	The \vocab{power function} for a test $C$ is
	\begin{align*}
		W(\theta) = \psub{\theta}{X \in C}.
	\end{align*}
	This is the probability of rejecting $H_0$ given the true parameter is $\theta$.
\end{definition}

\begin{definition}[Size]
	The \vocab{size} of a test $C$ is the worst case Type I error probability,
	\begin{align*}
		\alpha = \sup_{\theta \in \Theta_0} W(\theta)
	\end{align*}
\end{definition} 

\begin{definition}[Uniformly Most Powerful]
	A test is \vocab{uniformly most powerful} (UMP) of size $\alpha$ if, for any test $C^\star$ with power function $W^\star$ and size upper bounded by $\alpha$, for all $\theta \in \Theta_1$ we have $W(\theta) \geq W^\star(\theta)$.
\end{definition} 

\begin{note}
	Such tests need not exist.
	In simple models, many likelihood ratio tests are uniformly most powerful.
\end{note} 

\begin{example}[One-sided Test for Normal Location]
	Let $X_1, \dots, X_n \sim N(\mu, \sigma_0^2)$ be i.i.d.\ where $\sigma_0^2$ is known and $\mu$ is unknown.
	Let $H_0 \colon \mu \leq \mu_0$ and $H_1 \colon \mu > \mu_0$ for some fixed $\mu_0$.

	We claim that the simple hypothesis test given by $H_0' \colon \mu = \mu_0$ and $H_1' \colon \mu = \mu_1 > \mu_0$ is uniformly most powerful for $H_0$ and $H_1$.
	The LRT was $C = \qty{x : z = \sqrt{n} \frac{\bar x - \mu_0}{\sigma_0} > z_\alpha}$

	The power function is
	\begin{align*}
		W(\mu) &= \psub{\mu}{\frac{\sqrt{n}(\overline X-\mu_0)}{\sigma_0} < z_\alpha} \\
		&= \psub{\mu}{\frac{\sqrt{n}(\overline X - \mu)}{\sigma_0} > z_\alpha + \frac{\sqrt{n}(\mu_0 - \mu)}{\sigma_0}} \footnote{Adding $\frac{\sqrt{n}(\mu_0 - \mu)}{\sigma_0}$ to both sides.}
		\intertext{$\frac{\sqrt{n}(\overline X - \mu)}{\sigma_0} \sim \mathcal{N}(0, 1)$ under $\mathbb{P}_\mu$.}
		&= 1 - \Phi\qty(z_\alpha + \sqrt{n} \frac{\mu_0 - \mu}{\sigma_0})
	\end{align*}
	$W(\mu)$ is monotone increasing in $\mu$, so the size is  $\sup_{\mu \in \Theta_0} W(\mu) = W(\mu_0) = \alpha$.

	It remains to show that if $C^\star$ is another test of size $\leq \alpha$ with power function $W^\star$ then $W(\mu_1) \geq W^\star(\mu_1)$ for all $\mu_1 > \mu_0$. \\
	First, observe that the critical region depends only on $\mu_0$, and not on $\mu_1$.
	In particular, for any $\mu_1 > \mu_0$, we have that the critical region $C$ is the likelihood ratio test for the simple hypothesis test $H_0' \colon \mu = \mu_0$ and $H_1' \colon \mu = \mu_1$.
	Any test $C^\star$ of $H_0$ vs $H_1$ of size $\leq \alpha$ can also be seen as a test of $H_0'$ vs $H_1'$ with size $\leq \alpha$.
	\begin{align*}
		W^\star(\mu_0) \leq \sup_{\mu < \mu_0} W^\star(\mu) \leq \alpha
	\end{align*}
	By the Neyman-Pearson lemma, $C$ has power no smaller than $C^\star$ for $H_0'$ against $H_1'$:
	\begin{align*}
		W(\mu_1) \geq W^\star(\mu_1)
	\end{align*}
	Since this is true for all $\mu_1 > \mu_0$, the result holds, and the test $C$ satisfies the property for being uniformly most powerful.
\end{example}

\subsection{Generalised likelihood ratio test}

\begin{definition}[Generalised Likelihood Ratio]
	Suppose we have hypotheses, $H_0, H_1$
	The \vocab{generalised likelihood ratio} is given by
	\begin{align*}
		\Lambda_x(H_0; H_1) = \frac{\sup_{\theta \in \Theta_1} f_X(x \mid \theta)}{\sup_{\theta \in \Theta_0} f_X(x \mid \theta)}
	\end{align*}
	Larger values of $\Lambda_x$ indicate larger departures from $H_0$.

	The \vocab{generalised likelihood ratio test} rejects the null hypothesis when $\Lambda_x$ is sufficiently large.
\end{definition}

\begin{example}[Two-sided Normal Mean Test] \label{exm:2}
	Let $X_1, \dots, X_n \sim N(\mu, \sigma_0^2)$ be i.i.d.\ where $\sigma_0^2$ is known and $\mu$ is unknown.
	Let $\Theta_0 = \{\mu_0\}$ and $\Theta_1 = \mathbb{R} \setminus \{\mu_0\}$ for some fixed $\mu_0$.
	In this model, the generalised likelihood ratio is
	\begin{align*}
		\Lambda_x(H_0; H_1) &= \frac{(2 \pi \sigma_0^2)^{-n/2} \exp{\frac{-1}{2\sigma_0^2} \Sigma_{i=1}^n (x_i - \overline X)^2}}{(2 \pi \sigma_0^2)^{-n/2} \exp{\frac{-1}{2\sigma_0^2} \Sigma_{i=1}^n (x_i - \mu_0)^2}} \\
		2\log \Lambda_x &= \frac{n}{\sigma_0^2}(\overline X - \mu_0)^2
	\end{align*}
	Under $H_0$, $\sqrt{n} \frac{\overline X - \mu_0}{\sigma_0} \sim N(0,1)$.
	Hence, $2 \log \Lambda_x \sim \chi_1^2$.
	Therefore, the critical region of this generalised likelihood ratio test is
	\begin{align*}
		C = \qty{x \colon \frac{n}{\sigma_0^2}(\overline X - \mu_0)^2 > \chi_1^2(\alpha)} = \qty{x: \qty| \sqrt{n} \frac{\bar{x} - \mu_0}{\sigma_0} | > z_{\alpha / 2} = \Phi\inv \qty(1 - \frac{\alpha}{2})}
	\end{align*}
	where $\chi_1^2(\alpha)$ is the upper $\alpha$ point of $\chi_1^2$.
	This is called a \textit{two-sided test} since there are two tails on the critical region, plotting with respect to $\sqrt{n} \frac{\overline X - \mu_0}{\sigma_0}$.
\end{example}

\begin{note}
	In general, we can approximate the distribution of $2 \log \Lambda_x$ with a $\chi^2$ distribution when $n$ is large.
\end{note} 

\subsection{Wilks' theorem}
\begin{definition}[Dimension]
	The \vocab{dimension} of a hypothesis $H_0 \colon \theta \in \Theta_0$ is the number of `free parameters' in $\Theta_0$.
\end{definition}

\begin{example}
	If $\Theta_0 = \qty{\theta \in \mathbb R^k \colon \theta_1 = \dots = \theta_p = 0}$, then the dimension of $H_0$ is $k - p$.
\end{example}

\begin{example}
	Let $A \in \mathbb R^{p \times k}$ be a $p \times k$ matrix with linearly independent rows.
	Let $b \in \mathbb R^p$ for $p < k$, then we define $\Theta_0 = \qty{\theta \in \mathbb R^k \colon A\theta = b}$.
	Then the dimension of $\theta$ is $k - p$.
\end{example} 

\begin{example}
	$\Theta_0 = \qty{\theta \in \mathbb{R}^k : \Theta_i = f_i(\phi), \phi \in \mathbb{R}^p}$.
	Here $\phi$ are the free parameters; $f_i$ need not be linear.
	Under regularity conditions, $\dim \Theta_0 = p$.
\end{example} 

\begin{definition}[Nested Hypotheses]
	\vocab{Nested hypotheses} are hypotheses of the form $H_0 \colon \theta \in \Theta_0$ and $H_1 \colon \theta \in \Theta_1$, where $\Theta_0 \subseteq \Theta_1$.
\end{definition}

\begin{theorem}[Wilks' Theorem] \label{thm:wilks}
	Suppose we have nested hypothesis and $\dim \Theta_1 - \dim \Theta_0 = p$.
	Let $X = (X_1, \dots, X_n)$ be i.i.d.\ random variables under $f_x(\wildcard\mid\theta)$.
	Then, under some regularity conditions, as $n \to \infty$ we have under $H_0$
	\begin{align*}
		2\log\Lambda_x \sim \chi_p^2
	\end{align*}
	More precisely, for any $\theta \in \Theta_0$ and any $\ell \in \mathbb R_+$,
	\begin{align*}
		\lim_{n \to \infty} \psub{\theta}{2\log \Lambda_x \leq \ell} = \prob{\Xi \leq \ell};\quad \Xi \sim \chi_p^2
	\end{align*}
\end{theorem}

\begin{proof}
	Wait for Part 2 Principles of Statistics.
\end{proof} 

\begin{remark}
	If $n$ is large, this theorem allows us to implement a generalised likelihood ratio test even if we cannot find the exact distribution of $2 \log \Lambda_x$.
	For $n$ large, the size of the test is $\approx \alpha$.
	% Frequentist guarantees obtained from such a test will be approximate.
\end{remark}

\begin{example}
	In \Cref{exm:2}, $\Theta_0 = \{\mu_0\}$, $\Theta_1 = \mathbb{R} \setminus \{\mu_0\}$ we found $2 \log \Lambda_x \sim \chi_1^2$.

	If we take $\Theta_1 = \mathbb{R}$ then we have nested hypotheses, the GLR statistic doesn't change, so $2 \log \Lambda_x \sim \chi_1^2$.
	$\dim \Theta_1 = 1$ and $\dim \Theta_0 = 0$ hence the difference in dimensions is $1$.
	Then, Wilks' theorem implies that $2 \log \Lambda_x$ is approximately distributed according to $\chi_1^2$, although the result is exact in this particular case.
\end{example}

\subsection{Goodness of fit}
Let $X_1, \dots, X_n$ be i.i.d.\ samples taking values in $\qty{1, \dots, k}$.
Let $p_i = \prob{X_1 = i}$, and let $N_i$ be the number of samples equal to $i$, so $\sum_i p_i = 1$ and $\sum_i N_i = n$.
The parameters here are $p = (p_1, \dots, p_k)$, which has $k - 1$ dimensions.
A \textit{goodness of fit test} has a null hypothesis of the form $H_0 \colon p_i = \widetilde p_i$ for all $i$, for a fixed $\widetilde p = (\widetilde p_1, \dots, \widetilde p_k)$.
The alternative hypothesis $H_1$ does not constrain $p$.

The model is $(N_1, \dots, N_k) \sim \mathrm{Multi}(n; p_1, \dots, p_k)$.
The likelihood function is
\begin{align*}
	L(p) \propto p_1^{N_1} \cdots p_k^{N_k} \implies \ell(p) = \text{constant} + \sum_i N_i \log p_i
\end{align*}
The generalised likelihood ratio is
\begin{align*}
	2 \log \Lambda_x = 2\qty(\sup_{p \in \Theta_1} \ell(p) - \sup_{p \in \Theta_0} \ell(p)) = 2\qty(\ell(\hat p) - \ell(\widetilde p))
\end{align*}
where $\hat p$ is the maximum likelihood estimator under $H_1$.
To find $\hat p$, we typically use the method of Lagrange multipliers.
\begin{align*}
	\mathcal L(p, \lambda) = \sum_i N_i \log p_i - \lambda \qty(\sum p_i - 1)
\end{align*}
We can compute that
\begin{align*}
	\hat p_i = \frac{N_i}{n}
\end{align*}
This is simply the fraction of observed samples of type $i$.

So 
\begin{align*}
	2 \log \Lambda_x &= 2\qty(\ell(\hat p) - \ell(\widetilde p)) \\
	&= 2 \sum_i N_i \log \frac{N_i}{n \widetilde{p}_i}
\end{align*}

\nameref{thm:wilks} tells us that $2 \log \Lambda_x \sim \chi_p^2$ with $p = \dim \Theta_1 - \dim \Theta_; = (k - 1) - 0 = k - 1$.

So we can reject $H_0$ with size $\approx \alpha$ when $2 \log \Lambda_x > \chi_{k - 1}^2(\alpha)$.

\begin{example} \label{exm:mendel}
	Mendel performed an experiment in which 556 different pea plants were created from a small set of ancestors.
	Each descendent was either yellow or green, and either wrinkled or smooth, giving four possibilities in total.
	The observed result was
	\begin{align*}
		N = \qty(\underbrace{315}_{SG}, \underbrace{108}_{SY}, \underbrace{102}_{WG}, \underbrace{31}_{WY})
	\end{align*}
	Mendel's theory gives a null hypothesis $H_0 \colon p = \widetilde p = \qty(\frac{9}{16}, \frac{3}{16}, \frac{3}{16}, \frac{1}{16})$.
	Here,
	\begin{align*}
		2 \log \Lambda = 0.618;\quad \sum_i \frac{(o_i - e_i)^2}{e_i} = 0.604
	\end{align*}
	These are referred to a $\chi^2_3$ distribution.
	We observe that $\chi^2_3(0.05) = 7.815$, so we fail to reject the null hypothesis with a test of size $5\%$.
	We can compute that the $p$-value is $\prob{\chi^2_3 > 0.6} \approx 0.96$, so there is a very high probability of observing a more extreme value than observed.
\end{example}

\subsection{Pearson statistic}
Let $o_i = N_i$ be the observed number of samples of type $i$, and $e_i = n \widetilde p_i$ be the expected value under the null hypothesis of the number of samples of type $i$.
Here, we can write
\begin{align*}
	2 \log \Lambda = 2 \sum_i N_i \log(\frac{N_i}{n \widetilde p_i}) = 2 \sum_i o_i \log \frac{o_i}{e_i}
\end{align*}
Let $\delta_i = o_i - e_i$.
Then
\begin{align*}
	2 \log \Lambda = 2 \sum_i (e_i + \delta_i) \log\qty(1 + \underbrace{\frac{\delta_i}{e_i}}_{\text{small when } n \text{ large}})
\end{align*}
By taking the Taylor expansion, we arrive at
\begin{align*}
	&= 2 \sum_i (e_i + \delta_i) \qty(\frac{\delta_i}{e_i} - \frac{\delta^2_i}{2e^2_i} + \dots) \\
	&= 2 \sum_i \qty(\delta_i + \frac{\delta_i^2}{e_i} - \frac{\delta_i^2}{2e_i})
\end{align*}
Note that $\sum_i \delta_i = \sum_i (o_i - e_i) = n - n = 0$, so we can simplify and find
\begin{align*}
	\sum_i \frac{\delta_i^2}{e_i} = \sum_i \frac{(o_i - e_i)^2}{e_i}
\end{align*}
This is \vocab{Pearson's $\chi^2$ statistic}.
This is also referred to a $\chi^2_{k-1}$ when performing a hypothesis test.

\begin{example}
	See \Cref{exm:mendel}.
\end{example} 

\subsection{Goodness of fit for composite null}
Suppose $H_0 \colon p_i = p_i(\theta)$ for some $\theta \in \Theta_0$, and $H_1 \colon p$ has any distribution on $\qty{1, \dots, k}$.
We can compute
\begin{align*}
	2 \log \Lambda = 2 \qty( \sup_p \ell(p) - \sup_{\theta \in \Theta} \ell(p(\theta)) )
\end{align*}
We can sometimes compute these quantities explicitly, and hence find a test which refers this test statistic to a $\chi^2_d$ distribution where $d = \dim \Theta_1 - \dim \Theta_0 = (k-1) - \dim \Theta_0$.

\begin{example}
	Consider a population of individuals who may have one of three genotypes, which occur with probabilities $(p_1, p_2, p_3) = (\theta^2, 2\theta(1-\theta), (1-\theta)^2)$ for $\theta \in [0, 1]$.
	In this case, we can find the maximum likelihood estimator under the null hypothesis to be
	\begin{align*}
		\hat \theta = \frac{2N_1 + N_2}{2n}
	\end{align*}
	Hence,
	\begin{align*}
		2 \log \Lambda = 2(\ell(\hat p) - \ell(\hat \theta))
	\end{align*}
	where $\hat{p}$ is the mle in $H_1$ whilst $\hat{\theta}$ is the mle in $H_0$.
	Last time we found $\hat p_i = \frac{N_i}{n}$.
	This can be computed explicitly and referred to a $\chi^2_1$ distribution.
	We can check that, in this model,
	\begin{align*}
		2 \log \Lambda = \sum_i o_i \log \frac{o_i}{e_i}
	\end{align*}
	where $o_i = N_i$ and $e_i = n p_i(\hat \theta)$.
	We can approximate this using the Pearson statistic, $\sum_i \frac{(o_i - e_i)^2}{e_i}$.
	Each statistic can be referred to a $\chi^2_d$ when $n$ is large by \nameref{thm:wilks}, where $d = \dim \Theta_1 - \dim \Theta_0 = 2 - 1 = 1$.
\end{example}

\subsection{Testing independence in contingency tables}
Suppose we have observations $(X_1, Y_1), \dots, (X_n, Y_n)$ which are i.i.d., where the $X_i$ take values in $1, \dots, r$ and the $Y_i$ take values in $1, \dots, c$.
We wish to test whether the $X_i$ and $Y_i$ are independent.
We will summarise this data into a sufficient statistic known as a \vocab{contingency table} $N$, given by
\begin{align*}
	N_{ij} = \abs{ \qty{\ell \colon 1 \leq \ell \leq n, (X_\ell, Y_\ell) = (i,j)} }
\end{align*}
So $N_{ij}$ is the number of samples of type $(i,j)$.

\begin{example}[Covid 19 Deaths]
	Let $X_i$ be the age group of $i$th death, $Y_i$ the week on which the $i$th death occurred.

	\begin{question}
		Are deaths decreasing faster for older age groups that had been vaccinated?
	\end{question} 

	Suppose we observe $n$ samples, and each sample has probability $p_{ij}$ of being of type $(i,j)$.
	Flattening $(N_{ij})$ into a vector, this has a multinomial distribution with parameters $(p_{ij})$ (also flattened into a vector).
	The null hypothesis is the week of death is independent of age ($X_i$ independent of $Y_i$ for each sample).
	So $H_0 \colon p_{ij} = \mathbb{P}(X_\ell = i) \mathbb{P}(Y_\ell = j) = p_{i+} p_{+j}$ where $p_{i+} = \sum_j p_{ij}$ and $p_{+j} = \sum_i p_{ij}$.
	The alternative hypothesis places no restrictions on the $p_{ij}$ apart from that it sums to 1 and has nonnegative entries.
	The generalised LRT:
	\begin{align*}
		2 \log \Lambda &= \sum_{i,j} o_{ij} \log \frac{o_{ij}}{e_{ij}} \\
		&= \sum_{i, j} N_{ij} \log \frac{N_{ij}}{n \hat{p}_{ij}}
	\end{align*}
	where $\hat p_{ij}$ is the mle under $H_0$.
	These can be found using the method of Lagrange multipliers.
	In particular,
	\begin{align*}
		\hat p_{ij} = \hat p_{i+} \hat p_{+j};\quad \hat p_{i+} = \frac{N_{i+}}{n} = \frac{1}{n} \sum_{j=1}^c N_{ij};\quad \hat p_{+j} = \frac{N_{+j}}{n} = \frac{1}{n} \sum_{i=1}^r N_{ij}
	\end{align*}
	So
	\begin{align*}
		2 \log \Lambda = 2 \sum_{i = 1}^r \sum_{j = 1}^c N_{ij} \log \frac{N_{ij}}{n \hat{p}_{i+} \hat{p}_{+j}} \approx \sum_{i,j} \frac{(o_{ij} - e_{ij})^2}{e_{ij}}
	\end{align*}
	By Wilks' theorem, these test statistics have an approximate $\chi^2_p$ distribution, where $p = \dim \Theta_1 - \dim \Theta_0 = (rc-1) - (r-1 + c-1)\footnote{We have $r - 1$ degrees of freedom in $(p_{1+}, \dots, p_{r+})$ and $c - 1$ in $(p_{+1}, \dots, p_{+r})$} = (r-1)(c-1)$.
\end{example}
The $\chi^2$ test for independence has a number of weaknesses.
\begin{enumerate}
	\item The $\chi^2$ approximation requires $n$ to be large.
	      A reasonable heuristic is to require $N_{ij} \geq 5$ for \textit{all} $i,j$.
	      If this is not possible, we can perform an \textit{exact test} (which is non-examinable).
	\item The $\chi^2$ test often has a low power.
	      Heuristically, this is because the alternative space $\Theta_1$ is too large, and there are many possible models that lie in this space.
		  One solution is to define a parametric alternative $H_1$ with fewer degrees of freedom or lump categories in the table.
\end{enumerate}

\begin{remark}
	Note that this test also applies when $n$ is a random variable with a Poisson distribution.
	This is often the case when we do not fix the number of samples.
	The proof is not provided in this course.
\end{remark} 

\subsection{Testing homogeneity in contingency tables}
Instead of just assuming $\sum_{ij} N_{ij}$ fixed, we also assume row totals are fixed.

\begin{example}
	Suppose we perform a clinical trial on 150 patients, who are randomly assigned to one of three groups of equal size.
	The first two sets take a drug with different doses, and the third set takes a placebo.

	\begin{center}
		\begin{tabular}{c | c c c | c}
			          & improved & no difference & worse &    \\\hline
			placebo   & 18       & 17            & 15    & 50 \\
			half dose & 20       & 10            & 20    & 50 \\
			full dose & 25       & 13            & 12    & 50
		\end{tabular}
	\end{center}

	In the previous section, we fixed the total number of samples.
	Here, we fix the total number of samples, and the total number of samples in each row.
	We suppose
	\begin{align*}
		N_{i1}, \dots, N_{ic} \sim \mathrm{Multinomial}(n_{i+}; p_{i1}, \dots, p_{ic})
	\end{align*}
	which are independent for each row $i$ of the table.
	The null hypothesis for homogeneity is that $p_{1j} = p_{2j} = \dots = p_{rj}$ for all $j$.
	The alternative hypothesis assumes that $p_{i1}, \dots, p_{ic}$ is any arbitrary probability vector for each row $i$.
	Under the alternative hypothesis,
	\begin{align*}
		L(p) = \prod_{i=1}^r \frac{n_{i+}!}{N_{i1}!
			\cdots N_{ic}!} p_{i1}^{N_{i1}} \cdots p_{ic}^{N_{ic}}.
	\end{align*}
	Hence,
	\begin{align*}
		\ell(p) = \text{constant} + \sum_{i,j} N_{ij} \log p_{ij}
	\end{align*}
	This is the same likelihood as the independence test above.
	To define the maximum likelihood estimator we can again use the method of Lagrange multipliers with constraints $\sum_j p_{ij} = 1$ for each $i$.
	We find
	\begin{align*}
		\hat p_{ij} = \frac{N_{ij}}{n_{i+}}
	\end{align*}
	Under the null hypothesis, we let $p_j = p_{ij}$ for any $i$.
	\begin{align*}
		\ell(p) = \text{constant} + \sum_{i,j} N_{ij} \log p_j = \sum_j N_{+j} \log p_j
	\end{align*}
	We have the constraint $\sum_j p_j = 1$.
	Using the method of Lagrange multipliers,
	\begin{align*}
		\hat p_j = \frac{N_{+j}}{n_{++}};\quad n_{++} = \sum_i n_{i+}.
	\end{align*}
	Hence,
	\begin{align*}
		2 \log \Lambda = 2 \sum_{i,j} N_{ij} \log \frac{\hat p_{ij}}{\hat p_j} = 2 \sum_{i,j} N_{ij} \log \frac{N_{ij}}{n_{i+} N_{+j} / n_{++}}
	\end{align*}
	This is precisely the same test statistic as the test for independence above.
	The only difference is that $n_{i+}$ is fixed in this model.
	Further, if $o_{ij} = N_{ij}$ and $e_{ij} = n_{i+} \hat p_j = \frac{n_{i+} N_{+j}}{n_{++}}$, we have
	\begin{align*}
		2 \log \Lambda = 2 \sum_{i,j} o_{ij} \log \frac{o_{ij}}{e_{ij}} \approx \sum_{i,j} \frac{(o_{ij} - e_{ij})^2}{e_{ij}}.
	\end{align*}
	By Wilks' theorem, this is asymptotically a $\chi^2_p$ distribution.
	Here,
	\begin{align*}
		p = \dim \Theta_1 - \dim \Theta_0 = r(c-1) - (c-1) = (r-1)(c-1)
	\end{align*}
	This is again exactly the same as in the $\chi^2$ test for independence.
	Operationally, the tests for homogeneity and independence are therefore completely identical; we reject the null hypothesis for one test if and only if we reject the null for the other.
	In the example above,
	\begin{align*}
		2 \log \Lambda = 5.129;\quad \sum_{i,j} \frac{(o_{ij} - e_{ij})^2}{e_{ij}} = 5.173
	\end{align*}
	Referring this to a $\chi^2_4$ distribution, the upper $0.05$-point is $9.488$.
	Hence, we do not reject the null hypothesis at the 5\% significance level.
\end{example}

\subsection{Tests and confidence sets}
\begin{definition}[Acceptance Region]
	The \vocab{acceptance region} $A$ of a test is the complement of the critical region.
\end{definition}

\begin{theorem}
	Let $X \sim f_X(\wildcard\mid\theta)$ for some $\theta \in \Theta$.
	\begin{enumerate}
		\item Suppose that for each $\theta_0 \in \Theta$, there exists a test of size $\alpha$ with acceptance region $A(\theta_0)$ for the  null hypothesis $\theta = \theta_0$.
		Then
		\begin{align*}
			I(X) = \qty{\theta \colon X \in A(\theta)}\footnote{$\theta$ is fixed.}
		\end{align*}
		is a $100(1-\alpha)\%$ confidence set.
		\item Now suppose there exists a set $I(X)$ which is a $100(1-\alpha)\%$ confidence set for $\theta$.
		Then
		\begin{align*}
			A(\theta_0) = \qty{x \colon \theta_0 \in I(x)}
		\end{align*}
		is the acceptance region of a test of size $\alpha$ for the hypothesis $\theta = \theta_0$.
	\end{enumerate} 
\end{theorem}

\begin{proof}
	Observe that for both parts of the theorem,
	\begin{align*}
		\theta_0 \in I(X) \iff X \in A(\theta_0) \iff \text{fail to reject } H_0 \text{ with data } X.
	\end{align*}

	For the first part, we want to show that $\mathbb{P}_{\theta_0} (\theta_0 \in I(X)) = 1 - \alpha$.
	As $\theta_0 \in I(X) \iff X \in A(\theta_0)$, $\mathbb{P}_{\theta_0} (\theta_0 \in I(X)) = \mathbb{P}_{\theta_0}(X \in A(\theta_0)) = 1 - \alpha$ as $A_{\theta_0}$ is the acceptance region of a size $\alpha$ test.

	For the second part, we want to show that $\mathbb{P}_{\theta_0} (X \notin A(\theta_0)) = \alpha$.
	As $X \in A(\theta_0) \iff \theta_0 \in I(X)$, $\mathbb{P}_{\theta_0}(X \notin A(\theta_0)) = \mathbb{P}_{\theta_0} (\theta_0 \notin I(X)) = \alpha$ as $I(X)$ is a $100(1-\alpha)\%$ confidence set.
\end{proof}

\begin{example}
	Let $X_1, \dots, X_n \sim N(\mu, \sigma_0^2)$ be i.i.d.\ with $\sigma_0^2$ known and $\mu$ unknown.
	We found that a $100(1-\alpha)\%$ confidence interval for $\mu$ is
	\begin{align*}
		I(X) = \qty(\overline X \pm \frac{Z_{\alpha/2}\sigma_0}{\sqrt{n}})
	\end{align*}
	Hence, by the second part of the theorem above, we can find a test for $H_0 \colon \mu = \mu_0$ with size $\alpha$ by
	\begin{align*}
		A(\mu_0) = \qty{ x \colon \mu_0 \in I(x) } = \qty{ x \colon \mu_0 \in \qty[\overline x \pm \frac{Z_{\alpha/2} \sigma_0}{\sqrt{n}}] }
	\end{align*}
	This is equivalent to rejecting $H_0$ when
	\begin{align*}
		\abs{\sqrt{n} \frac{\mu_0 - \overline X}{\sigma_0}} > Z_{\alpha/2}
	\end{align*}
	This is a two-sided test for normal location.
\end{example}
