\section{Cryptography}

\subsection{Cryptosystems}
We want to modify a message s.t. it becomes unintelligible to an eavesdropper Eve.
Certain (secret) information is shared between two participants Alice and Bob, called the \vocab{key}, chosen from a set of possible keys $\mathcal K$.
The unencrypted message is called the \vocab{plaintext}, which lies in a set $\mathcal M$, and the encrypted message is called the \vocab{ciphertext}, and lies in a set $\mathcal C$.
A \vocab{cryptosystem} consists of $(\mathcal K, \mathcal M, \mathcal C)$ together with the \vocab{encryption} function $e \colon \mathcal M \times \mathcal K \to \mathcal C$ and \vocab{decryption} function $d \colon \mathcal C \times \mathcal K \to \mathcal M$.
These maps have the property that $d(e(m, k), k) = m$ for all $m \in \mathcal M, k \in \mathcal K$.

\begin{example}
    Suppose $\mathcal M = \mathcal C = \qty{A, B, \dots, Z}^\star = \Sigma^\star$.
    \begin{itemize}
        \item The \vocab{simple substitution cipher} defines $\mathcal K$ to be the set of permutations of $\Sigma$.
        To encrypt a message, each letter of plaintext is replaced with its image under a chosen permutation $\pi \in \mathcal K$.
    \end{itemize}


    The \vocab{Vigen\`ere cipher} has $\mathcal K = \Sigma^d$ for some $d$.
    We identify $\Sigma$ and $\faktor{\mathbb Z}{26\mathbb Z}$.
    To encode a message we add it to the key $\operatorname{mod} 26$, repeating the key so that it is the correct length.
    For instance, encrypting the plaintext ATTACKATDAWN with the key LEMON gives ciphertext ATTACKATDAWN + LEMONLEMONLE = LXFOPVEFRNHR.
    % Note, for instance, that each occurrence of the letter A in the plaintext corresponds to a letter of the key in the ciphertext.
    If $d = 1$, this is the \vocab{Caesar cipher}.
\end{example}

\subsection{Breaking cryptosystems}
Eve may know $e$ and $d$, as well as the prob distributions of $\mathcal K, \mathcal M$, but she does not know the key itself.
She seeks to recover the plaintext from a given string of ciphertext.
There are three possible attack levels.

\begin{enumerate}
    \item (ciphertext-only) Eve only knows some piece of ciphertext.
    \item (known-plaintext) Eve knows a considerable length of plaintext and its corresponding ciphertext, but not the key.
    In other words, she knows $m$ and $e(m,k)$, but not $k$.
    \item (chosen plaintext) Eve can acquire the ciphertext for any plaintext message; she can generate $e(m,k)$ for any $m$.
\end{enumerate}

\begin{remark}
    The simple substitution cipher and Vigen\`ere cipher fail at Level 1 in English if the messages are sufficiently long, as we can perform frequency analysis.
    Even if the plaintext is suitably random, both examples can fail at Level 2.
    For modern applications, Level 3 security is desirable.
\end{remark}

Consider a cryptosystem $(\mathcal K, \mathcal M, \mathcal C)$.
We model the keys and messages as independent r.v.s $K, M$ taking values in $\mathcal K, \mathcal M$.
The ciphertext r.v. is $C = e(M, K) \in \mathcal C$.

\begin{definition}[Perfect Accuracy]
    A cryptosystem $(\mathcal M, \mathcal K, \mathcal C)$ has \vocab{perfect secrecy} if $H(M \mid C) = H(M)$, or equivalently, $M$ and $C$ are independent, or $I(M;C) = 0$.
\end{definition}

\begin{lemma}
    Perfect secrecy $\implies \abs{\mathcal K} \geq \abs{\mathcal M}$
\end{lemma}

\begin{proof}
    We need to prove that there must be at least as many possible keys as there are possible plaintext messages.
    ``Possible'' means the prob of being chosen is $> 0$.

    Fix message $m_0 \in \mathcal{M}$ and a key $k_0 \in K$, both with prob $> 0$.
    Then $c_0 = e(m_0, k_0)$ has prob $> 0$. \\
    For any possible message $m \in \mathcal{M}$,
    \begin{align*}
        \mathbb{P}(C = c_0 \mid M = m) = \mathbb{P}(C = c_0) > 0.
    \end{align*}
    So for each $m \in \mathcal{M}$ $\exists \; k \in \mathcal{K}$ s.t. $e(m, k) = c_0$.
    Thus $\abs{\mathcal{K}} \geq \abs{\mathcal{M}}$.
\end{proof}

\begin{definition}[Message Equivocation]
    The \vocab{message equivocation} is $H(M \mid C)$.
\end{definition}

\begin{definition}[Key Equivocation]
    The \vocab{key equivocation} is $H(K \mid C)$.
\end{definition}

\begin{lemma}
    $H(M \mid C) \leq H(K \mid C)$.
\end{lemma}

\begin{proof}
    Note that $M = d(C,K)$, hence $H(M \mid C, K) = 0$.
    Therefore, $H(C,K) = H(M,C,K)$.
    So
    \begin{align*}
        H(K \mid C) &= H(K,C) - H(C) \\
        &= H(M,C,K) - H(M \mid K,C) - H(C) \\
        &= H(M,K,C) - H(C) \\
        &= H(K \mid M,C) + H(M,C) - H(C) \\
        &= \underbracket{H(K \mid M, C)}_{\geq 0} + H(M \mid C)
    \end{align*}
    Hence $H(K \mid C) \geq H(M \mid C)$.
\end{proof}

Let $\mathcal M = \mathcal C = \mathcal A$, and suppose we send $n$ messages modelled as $M^{(n)} = (M_1, \dots, M_n)$ encrypted as $C^{(n)} = (C_1, \dots, C_n)$ using the same key $K$.

\begin{definition}[Unicity Distance]
    The \vocab{unicity distance} is the least $n$ s.t. $H\qty(K \mid C^{(n)}) = 0$; it is the smallest number of encrypted messages required to uniquely determine the key.
\end{definition}

Now,
\begin{align*}
    H\qty(K \mid C^{(n)}) &= H\qty(K, C^{(n)}) - H\qty(C^{(n)}) \\
    &= H\qty(K, M^{(n)}, C^{(n)}) - H\qty(C^{(n)}) \\
    &= H\qty(K, M^{(n)}) - H\qty(C^{(n)}) \\
    &= H(K) + H\qty(M^{(n)}) - H\qty(C^{(n)}) \quad \text{as $K, M^{(n)}$ are independent.}
\end{align*}
We make the following assumptions.
\begin{enumerate}
    \item All keys are equally likely, so $H(K) = \log \abs{\mathcal K}$.
    \item $H\qty(M^{(n)}) \approx nH$ for some constant $H$ and sufficiently large $n$ (true for many sources, including Bernoulli sources).
    \item All sequences of ciphertext are equally likely, so $H\qty(C^{(n)}) = n \log \abs{\mathcal A}$. Good cryptosystems satisfy this.
\end{enumerate}

Hence,
\begin{align*}
    H\qty(K \mid C^{(n)}) = \log \abs{\mathcal K} + nH - n \log \abs{\mathcal A}
\end{align*}
This is nonnegative iff
\begin{align*}
    n \leq U = \frac{\log \abs{\mathcal K}}{\log \abs{\mathcal A} - H}
\end{align*}
Equivalently, $\frac{\log \abs{\mathcal A}}{R\log \abs{\mathcal A}}$ where $R = 1 - \frac{H}{\log \abs{\mathcal A}}$ is the \vocab{redundancy} of the source.
Recall that $0 \leq H \leq \log \abs{\mathcal A}$.
To make the unicity distance large, we can make the number of keys large, or use a message source with little redundancy.

\subsection{One-time pad}
Consider streams (sequences) in $\mathbb F_2$ representing the plaintext $p_0, p_1, \dots$, the key stream $k_0, k_1, \dots$, and the ciphertext $z_0, z_1, \dots$ where $z_n = p_n + k_n$.

\begin{definition}[One-Time Pad]
    A \vocab{one-time pad} is a cryptosystem where $k$ is generated randomly; the $k_i$ are iid uniform on $0, 1$.
\end{definition}

$z = p + k$ is now a stream of iid r.v.s taking values of 0 or 1 with prob $\frac{1}{2}$.
Hence, without the key stream, deciphering is impossible, so the unicity distance is infinite.

\begin{lemma}
    A one-time pad has perfect secrecy
\end{lemma}

\begin{proof}
    $\mathbb{P}(M = m, C = c) = \mathbb{P}(M = m, K = c - m) = \mathbb{P}(M = m) \mathbb{P}(K = c - m) = \mathbb{P}(M = m) \frac{1}{2^n}$ for a message of length $n$ and $c - m$ is $\operatorname{mod} 2$.
    So $M, C$ independent.
\end{proof}

In order to effectively use a one-time pad, we need to generate a random key stream.
We then need to share the key stream to the recipient, which is exactly the initial problem.
In most applications, the one-time pad is not practical. \\
Instead, we share an initial fill $k_0, \dots, k_{d-1}$ to be used in a shared FSR of length $d$ to generate $k$.
We then apply the following result.

\begin{lemma} \label{17.2}
    Let $x_0, x_1, \dots$ be a stream in $\mathbb F_2$ produced by a FSR of length $d$.
    Then $\exists \; M, N \leq 2^d$ s.t. $x_{N+r} = x_{r} \ \forall \; r \geq M$.
\end{lemma}

\begin{proof}
    Let the register be $f \colon \mathbb F_2^d \to \mathbb F_2^d$, and let $v_i = (x_i, \dots, x_{i+d-1})$.
    Then for all $i$, we have $f(v_i) = v_{i+1}$.
    Since $\abs{\mathbb F_2^d} = 2^d$, the vectors $v_0, v_1, \dots, v_{2^d}$ cannot all be distinct.
    So $\exists \; 0 \leq a < b \leq 2^d$ s.t. $v_a = v_b$.
    Let $M = a$ and $N = b - a$, so $v_M = v_{M+N}$ so by induction\footnote{By applying $f$.} we have $v_r = v_{r+N}$ for all $r \geq M$.
\end{proof}

\begin{remark} ~
    \begin{itemize}
        \item The maximum period of a FSR of length $d$ is $2^d$.
        \item For a LFSR, the maximum period is $2^d - 1$.
        Indeed the bound in \cref{17.2} is improved by $1$ (Sheet 4).
        \item Stream ciphers using LFSR fail at level 2 due to the Berlekamp--Massey method.
        \item Why should we use this cryptosystem?
        \begin{itemize}
            \item It's cheap, fast, and easy to use.
            \item Encryption and decryption can be performed on-the-fly, without needing the entire codeword first.
            \item Error tolerant.
        \end{itemize}
    \end{itemize}
\end{remark}

% Recall that the stream produced by a LFSR is given by
% \begin{align*}
%     x_n = \sum_{i=1}^d a_{d-i} x_{n-i}
% \end{align*}
% for all $n \geq d$, and has auxiliary polynomial
% \begin{align*}
%     P(X) = X^d + a_{d-1}X^{d-1} + \dots + a_0
% \end{align*}
% with $a_d = 1$.
% The solutions to the recursion relations are linear combinations of powers of roots of $P$.
% Over $\mathbb C$, the general solution is a linear combination of $\alpha^n, n\alpha^n, \dots, n^{t-1} \alpha^n$ where $\alpha$ is a root of $P(X)$ with multiplicity $t$.

% As $n^2 = n$ in $\mathbb F_2$, we cannot use this method directly.
% First, we must work in a splitting field $K$ of $P$, a field containing $\mathbb F_2$ in which $P$ is expressible as a product of linear factors.
% In addition, we replace the $n^i \alpha^n$ term with $\binom{n}{i} \alpha^n$.
% The general solution is now a linear combination of these terms in $K$.

We can also generate new key streams from old ones.
\begin{lemma}
    Let $(x_n)$, $(y_n)$ be outputs from LFSRs of length $M, N$ respectively.
    Then,
    \begin{enumerate}
        \item the sequence $(x_n + y_n)$ is the output of a LFSR of length $M + N$;
        \item the sequence $(x_n y_n)$ is the output of a LFSR of length $MN$.
    \end{enumerate}
\end{lemma}

\begin{proof}[Proof (Not Lectured)]
    Assume for simplicity that the auxiliary polynomials $P(X), Q(X)$ each have distinct roots $\alpha_1, \alpha_M$ and $\beta_1, \dots, \beta_N$ in a field $K$ extending $\mathbb F_2$.
    Then $x_n = \sum_{i=1}^M \lambda_i \alpha_i^n$ and $y_n = \sum_{i=1}^N \mu_j \beta_j^n$ where $\lambda_i, \mu_j \in K$.
    Now, $x_n + y_n = \sum_{i=1}^M \lambda_i \alpha_i^n \sum_{i=1}^N \mu_j \beta_j^n$ is produced by a LFSR with auxiliary polynomial $P(X) Q(X)$.
    For the second part, $x_n y_n = \sum_{i=1}^M \sum_{j=1}^n \lambda_i \mu_j (\alpha_i \beta_j)^n$ is the output of a LFSR with auxiliary polynomial $\prod_{i=1}^N \prod_{j=1}^M (X - \alpha_i \beta_j)$.
\end{proof}
Adding outputs of LFSRs is no more economical than producing the same string with a single LFSR.
Muliplying streams does increase the effective length of the LFSR, but $x_n y_n = 0$ when either $x_n$ or $y_n$ are zero, so we gain little extra data.
NonLFSRs are in general hard to analyse; in particular, an eavesdropper may understand the FSR better than Alice and Bob.

\subsection{Asymmetric ciphers}
Stream ciphers are examples of symmetric cryptosystems.
In such a system, the decryption process is the same, or is easily deduced from, the encryption process.
In an asymmetric cryptosystem, the key is split into two parts: the \vocab{private key} for decryption, and the \vocab{public key} for encryption.
Knowing the encryption and decryption processes and the public key, it should still be hard to find the private key or to decrypt the messages.
This aim implies security at level 3.
In this case, there is also no key exchange problem, since the public key can be broadcast on an open channel.

We base asymmetric cryptosystems on certain mathematical problems in number theory which are believed to be `hard', such as the following.
\begin{enumerate}
    \item \underline{Factoring}.
    Let $N = pq$ for $p, q$ large prime numbers.
    Given $N$, the task is to find $p$ and $q$.
    \item \underline{Discrete logarithm problem}.
    Let $p$ be a large prime and $g$ be a primitive root mod $p$ (a generator of $\mathbb F_p^\times$).
    Given $x$, we wish to find $a$ s.t. $x \equiv g^a$ mod $p$.
\end{enumerate}

\begin{definition}[Polynomial Time]
    An algorithm runs in \vocab{polynomial time} if the number of operations needed to perform the algorithm is at most $c N^d$ where $N$ is the input size, and $c, d$ are constants.
\end{definition}

\begin{example}
    An algorithm for factoring $N$ has input size $\log_2 N$, roughly the number of bits in its binary expansion.
    Polynomial time algorithms include arithmetic operations on integers including the division algorithm, computation of greatest common divisors, and the Euclidean algorithm.
    We can also compute $x^\alpha$ mod $N$ in polynomial time using repeated squaring; this is called modular exponentiation.
    Primality testing can be performed in polynomial time.

    Polynomial time algorithms are not known for factoring and discrete logarithms.
    However, we have elementary methods for computing them that take exponential time.
    If $N = pq$, dividing $N$ by successive primes up to $\sqrt{N}$ will find $p$ and $q$ but takes $O(\sqrt{N}) = O(2^{\frac{B}{2}})$ steps where $B = \log_2 N$.

    We describe the \vocab{baby-step, giant-step} algorithm for the discrete logarithm problem.
    Set $m = \ceil*{\sqrt{p}}$, and write $a = qm + r$ for $0 \leq q, r < m$.
    Then, $x \equiv g^a = g^{qm + r}$ mod $p$, so $g^{qm} = g^{-r} x$ mod $p$.
    We list all values of $g^{qm}$ and $g^{-r}x$ mod $p$; we then sort the lists and search for a match.
    This takes $O(\sqrt{p}\log p)$ steps.

    The best known methods for solving the examples above use a factor base method, called the \vocab{modular number sieve}.
    It has running time
    \begin{align*}
        O\qty(\exp(c (\log N)^{\frac{1}{3}} (\log \log N)^{\frac{2}{3}}))
    \end{align*}
    where $c$ is a known constant.
\end{example}

\subsection{Rabin cryptosystem}
Recall that \vocab{Euler's totient function} is denoted $\varphi$, where $\varphi(n)$ is the number of integers less than $n$ which are coprime to $n$.
Equivalently, $\varphi(n) = \abs{\qty(\faktor{\mathbb Z}{n\mathbb Z})^\times}$.
By Lagrange's theorem, $a^{\varphi(N)} \equiv 1$ mod $N$ for each $a$ coprime to $N$; this result is sometimes known as the Fermat--Euler theorem.
If $N = p$ is prime, $a^p \equiv a$ mod $p$, which is Fermat's little theorem.

\begin{lemma}
    Let $p = 4k - 1$ be a prime, and let $d \in \mathbb Z$.
    If $x^2 \equiv d$ mod $p$ is solvable, one solution is $x \equiv d^k$ mod $p$.
\end{lemma}

\begin{proof}
    Suppose $x_0$ is a solution, so $x_0^2 \equiv d$ mod $p$.
    WLOG we can assume $x_0 \not\equiv 0$, or equivalently, $x_0 \nmid p$.
    Then $x_0^2 \equiv d$ so $d^{2k-1} \equiv x_0^{2(2k-1)} \equiv x_0^{p-1} \equiv 1$ by Fermat's little thm.
    Hence, $\qty(d^k)^2 \equiv d$.
\end{proof}

In the Rabin cryptosystem, the private key consists of two large distinct primes $p, q \equiv 3$ mod 4.
The public key is $N = pq$.
$\mathcal M = \mathcal C = \qty{1, \dots, N-1} = \mathbb Z_N^\times$.
We encrypt a plaintext message $m$ as $c = m^2$ mod $N$.
Usually, we restrict our messages so that $(m, N) = 1$ and $m > \sqrt{N}$.

Receiving ciphertext $c$, we can solve for $x_1, x_2$ s.t. $x_1^2 \equiv c$ mod $p$ and $x_2^2 \equiv c$ mod $q$ using the previous lemma.
Then, we use the Chinese remainder theorem, we can find $x$ s.t. $x \equiv x_1$ mod $p$ and $x \equiv x_2$ mod $q$, hence $x^2 \equiv c$ mod $N$\footnote{By squaring $x^2 \equiv x_1^2$ mod $p$ and $x^2 \equiv x_2^2$ mod $q$ so $x = pn + c = qm + c$. As $p, q$ coprime $x = pqa + c$.}.
Indeed, running the Euclidean algorithm on $p, q$ gives integers $r, s$ s.t. $rp + sq = 1$, then we can take $x = sqx_1 + rpx_2$.

\begin{lemma} ~\vspace*{-1.5\baselineskip}
    \begin{enumerate}
        \item Let $p$ be an odd prime, and let $(d, p) = 1$.
        Then $x^2 \equiv d$ mod $p$ has no or exactly two solutions.
        \item Let $N = pq$ where $p, q$ are distinct odd primes, and let $(d, N) = 1$.
        Then $x^2 \equiv d$ mod $N$ has no or exactly four solutions.
    \end{enumerate}
\end{lemma}

\begin{proof}
    \emph{Part (1).}
    If there is a solution $x$, $-x$ also works.
    If $x, y$ are solutions, then $x^2 \equiv y^2$ mod $p$ so $p \mid (x^2 - y^2) = (x-y)(x+y)$, so either $p \mid x-y$ or $p \mid x+y$, so $x = \pm y$.

    \emph{Part (2).}
    If $x_0$ is a solution, then by the Chinese remainder theorem, there exist solutions $x$ with $x \equiv \pm x_0$ mod $p$ and $x \equiv \pm x_0$ mod $q$.
    This gives four solutions as required.
    By (i), these are the only possible solutions.
\end{proof}

Hence, to decrypt the Rabin cipher, we must find all four solutions to $x^2 \equiv c$ mod $N$.
Messages should include enough redundancy to uniquely determine which of these four solutions is the intended plaintext.

\begin{theorem}
    Breaking the Rabin cryptosystem is essentially as difficult as factoring $N$.
\end{theorem}

\begin{proof}
    If we can factorise $N$ as $pq$, we have seen that we can decrypt messages.
    Conversely, suppose we can break the cryptosystem, so we have an algorithm to find square roots modulo $N$.
    Choose $x$ mod $N$ at random, and use the algorithm to find $y$ s.t. $y^2 \equiv x^2$ mod $N$.
    With prob $\frac{1}{2}$, $x \neq \pm y$ mod $N$.
    Then, $(N, x-y)$ is a nontrivial factor of $N$.
    If this fails, choose another $x$, and repeat until the prob of failure $\qty(\frac{1}{2})^r$ is acceptably low.
\end{proof}

\subsection{RSA cryptosystem}
Suppose $N = pq$ where $p, q$ are distinct odd primes.
We claim that if we know a multiple $m$ of $\varphi(N) = (p-1)(q-1)$, then factoring $N$ is `easy'.
Write $o_p(x)$ for the order of $x$ as an element of $\qty(\faktor{\mathbb Z}{p\mathbb Z})^\times$.
Write $m = 2^a b$ where $a \geq 1, b$ odd.
Let
\begin{align*}
    X = \qty{x \in \qty(\faktor{\mathbb Z}{N\mathbb Z})^\times : o_p(x^b) \neq o_q(x^b)}
\end{align*}

\begin{theorem} ~\vspace*{-1.5\baselineskip}
    \begin{enumerate}
        \item If $x \in X$, $\exists \; 0 \leq t < a$ s.t. $(x^{2^t b} - 1, N)$ is a nontrivial factor of $N$.
        \item $\abs{X} \geq \frac{1}{2} \abs{\qty(\faktor{\mathbb Z}{N\mathbb Z})^\times} = \frac{1}{2}(p-1)(q-1)$.
    \end{enumerate}
\end{theorem}

\begin{proof}
    \emph{Part (1).}
    By the Fermat--Euler theorem, $x^{\varphi(N)} \equiv 1$ mod $N$.
    Hence $x^m \equiv 1$ mod $N$.
    But $m = 2^a b$, so setting $y = x^b$ mod $N$, we obtain $y^{2^a} \equiv 1$ mod $N$.
    In particular, $o_p(y)$ and $o_q(y)$ are powers of 2.
    Since $x \in X$, $o_p(y) \neq o_q(y)$, so wlog suppose $o_p(y) < o_q(y)$.
    Let $o_p(y) = 2^t$, so $0 \leq t < a$.
    Then $y^{2^t} \equiv 1$ mod $p$, but $y^{2^t} \not\equiv 1$ mod $q$.
    So $(y^{2^t} - 1, N) = p$ as required.
\end{proof}

\begin{proof}[Proof (Non Examinable)]
    \emph{Part (2).}
    The Chinese remainder theorem provides a multiplicative group isomorphism
    \begin{align*}
        \qty(\faktor{\mathbb Z}{N\mathbb Z})^\times \to \qty(\faktor{\mathbb Z}{p\mathbb Z})^\times \times \qty(\faktor{\mathbb Z}{q\mathbb Z})^\times
    \end{align*}
    mapping $x$ to $(x \text{ mod } p, x \text{ mod } q)$.
    We claim that if we partition $\qty(\faktor{\mathbb Z}{p\mathbb Z})^\times$ according to the value of $o_p(x^b)$, then each equivalence class has size at most
    \begin{align*}
        \frac{1}{2} \abs{\qty(\faktor{\mathbb Z}{p\mathbb Z})^\times} = \frac{1}{2}(p-1)
    \end{align*}
    We show that one of these subsets has size exactly $\frac{1}{2}(p-1)$.
    Let $g$ be a primitive root mod $p$, so $\qty(\faktor{\mathbb Z}{p\mathbb Z})^\times = \genset{g}$.
    By Fermat's little theorem, $g^{p-1} \equiv 1$ mod $p$, so $g^m = g^{2^a b} \equiv 1$ mod $p$.
    Hence, $o_p(g^b)$ is a power of 2, say $2^t \leq a$.
    Let $x = g^k$ for some $0 \leq k \leq p - 2$, then $x^b = (g^b)^k$, so $o_p(x^b) = \frac{2^t}{(2^t, k)}$.
    So $o_p(x^b) = 2^t$ iff $k$ is odd, so
    \begin{align*}
        o_p(x^b) = o_p(g^{bk}) =
        \begin{cases}
            o_p(g^b) = 2^t & \text{if } k \text{ odd} \\
            < 2^t & \text{if } k \text{ even}
        \end{cases}
    \end{align*}
    Thus, $\qty{g^k \text{ mod } p \mid k \text{ odd}}$ is the set as required, proving the claim.
    To finish, for each $y \in \qty(\faktor{\mathbb Z}{q\mathbb Z})^\times$, the set
    \begin{align*}
        \qty{x \in \qty(\faktor{\mathbb Z}{p\mathbb Z})^\times \midd o_p(x^b) \neq o_q(x^b)}
    \end{align*}
    has at least $\frac{1}{2}(p-1)$ elements.
    Applying the Chinese remainder theorem,
    \begin{align*}
        \abs{X} = \abs{\qty{(x,y) \in \qty(\faktor{\mathbb Z}{p\mathbb Z})^\times \times \qty(\faktor{\mathbb Z}{q\mathbb Z})^\times \midd o_p(x^b) \neq o_q(x^b)}} \geq \frac{1}{2}(p-1)(q-1) = \frac{1}{2}\varphi(N)
    \end{align*}
\end{proof}

To factorise $N$ when $x \in X$, compute $(y^{2^t}-1, N)$ for $0 \leq t < a$.
We know it is $p$ for one of these choices, and hence obtain a factor of $N$.

\subsection{The RSA Cryptosystem}

In the RSA cryptosystem, the private key consists of large distinct primes $p, q$ chosen at random.
Let $N = pq$, and choose the \vocab{encrypting exponent} $e$ randomly s.t. $(e, \varphi(N)) = 1$, for instance taking $e$ prime larger than $p, q$.
By Euclid's algorithm, there exist $d, k$ s.t. $de - k\varphi(N) = 1$; $d$ is called the \vocab{decrypting exponent}.

The public key is $(N, e)$, and we encrypt $m \in \mathcal M$ as $c \equiv m^e$ mod $N$.
The private key is $(N, d)$, and we decrypt $c \in \mathcal C$ as $x \equiv c^d$ mod $N$.
By the Fermat--Euler theorem, $x \equiv m^{de} \equiv m^{1+k\varphi(N)} \equiv m$ mod $N$, noting that the prob that $(m, N) \neq 1$ is small enough to be ignored.
Hence, the decrypting function is inverse to the encrypting function.
%
% Suppose that Eve intercepts the ciphertext $c$, and wants to find the plaintext given the public key $(N, e)$.
% This problem is essentially as difficult as finding the factors $p$ and $q$ of $N$ so that $\varphi(N) = (p-1)(q-1)$ can be computed.
\begin{corollary}
    Finding the RSA private key, $d$, from the public key $(N, e)$ is essentially as difficult as factoring $N$.
\end{corollary}

\begin{proof}
    We have already shown that if we can factorise $N$, we can find $d$.
    Conversely, suppose there is an algorithm to find $d$ given $N$ and $e$.
    Then $de \equiv 1$ mod $\varphi(N)$.
    Taking $m = de-1$ in the proof of part (1) of the theorem above, we can factorise $N$ if we have a $x \in X$.
    A random $x \in \qty(\faktor{\mathbb{Z}}{n \mathbb{Z}})^\times$ will lie in $X$ with prob $\geq \frac{1}{2}$.
    Thus after $r$ such random choices, we find a factor of $N$ with prob $\geq 1 - \qty(\frac{1}{2})^r$.
\end{proof}

\begin{remark}
    We have shown that finding $(N, d)$ from the public key $(N, e)$ is as hard as factoring $N$.
    It is unknown whether decrypting messages sent via RSA is as hard as factoring.

    RSA avoids the issue of needing to share keys, but it is slow.
    Symmetric ciphers are often faster.
    So we are still interested in sharing keys.
\end{remark}

Shamir proposed:
\begin{example}[Shamir's padlock example]
    Let $\mathcal A = \mathbb Z_p$.
    Alice chooses $a \in \mathbb Z_{p-1}^\times$ and computes $g^a$.
    She finds $a'$ s.t. $aa' = 1$ mod $p-1$ by Euclid.
    Bob chooses $b \in \mathbb Z_{p-1}^\times$ and computes $g^b$.
    He similarly finds $b'$ s.t. $bb' = 1$ mod $p-1$.

    Let $m$ be a message in $\mathbb Z_p$.
    She encodes $m$ as $c = m^a$ mod $p$.
    She then sends this to Bob, who computes $d = c^b$ mod $p$.
    He sends this back to Alice, who computes $e = d^{a'}$ mod $p$.
    She sends this back to Bob, who computes $e^{b'}$ mod $p$.
    By Fermat's little theorem, $e^{b'} \equiv d^{a'b'} \equiv c^{ba'b'} \equiv m^{aba'b'} \equiv m$.
    % https://q.uiver.app/?q=WzAsNSxbMCwwLCJtIl0sWzEsMCwibV5hIl0sWzIsMCwiY15iIl0sWzMsMCwiZF57YSd9Il0sWzQsMCwiZV57Yid9Il0sWzAsMSwiQSJdLFsxLDIsIkIiXSxbMiwzLCJBIl0sWzMsNCwiQiJdXQ==
    \[
        \begin{tikzcd}
            m & {m^a} & {c^b} & {d^{a'}} & {e^{b'}}
            \arrow["A", from=1-1, to=1-2]
            \arrow["B", from=1-2, to=1-3]
            \arrow["A", from=1-3, to=1-4]
            \arrow["B", from=1-4, to=1-5]
        \end{tikzcd}
    \]
\end{example}
This suggests:
\begin{example}[Diffie--Hellman key exchange]
    Alice and Bob wish to agree on a secret key $k$.
    Let $p$ be a large prime, and $g$ a primitive root mod $p$.
    Alice chooses an exponent $\alpha \in \mathbb Z_{p-1}$ and sends $g^\alpha$ mod $p$ to Bob.
    Bob chooses an exponent $\beta$ and sends $g^\beta$ mod $p$ to Alice.
    Both Alice and Bob compute $k = g^{\alpha\beta}$, which can be used as their secret key.
    An eavesdropper must find $g^{\alpha\beta}$ knowing $g$, $g^{\alpha}$, and $g^{\beta}$.
    Diffie and Hellman conjectured that this problem is as difficult as solving the discrete logarithm problem.
\end{example}

\subsection{Secrecy and attacks}
Consider a message $m$ sent by Alice to Bob.
Here are some possible aims that the participants may have in communication.

\begin{enumerate}
    \item \vocab{Secrecy}: Alice and Bob can be sure that no third party can read the message.
    \item \vocab{Integrity}: Alice and Bob can be sure that no third party can alter the message.
    \item \vocab{Authenticity}: Bob can be sure that Alice sent the message.
    \item \vocab{Non-repudiation}: Bob can prove to a third party that Alice sent the message.
\end{enumerate}

\begin{example}[Authenticity using RSA]
    Suppose Alice uses a private key $(N, d)$ to encrypt $m$.
    Anyone can decrypt $m$ using the public key $(N, e)$ as $(m^d)^e = (m^e)^d = m$, but they cannot forge a message sent by Alice.
    Suppose Bob picks a random message $m$ and sends it to Alice; if Bob then receives a message back from Alice which after decryption ends in $m$, then he can be sure it comes from Alice.
\end{example}

Signature schemes preserve integrity and non-repudiation.
They also prevent tampering in the following sense.

\begin{example}[Homomorphism Attack]
    Suppose a bank sends messages of the form $(M_1, M_2)$ where $M_1$ represents the client's name and $M_2$ represents an amount of money to be transferred into their account.
    Suppose that messages are encoded using RSA as $(Z_1, Z_2) = (M_1^e, M_2^e)$, where all calculations are performed modulo $N$.
    A client $C$ transfers $\pounds 100$ to their account, and observes the encrypted message $(Z_1, Z_2)$.
    Then, sending $(Z_1, Z_2^3)$ to the bank, $C$ becomes a millionaire without breaking RSA.
    Alternatively, one could simply send $(Z_1, Z_2)$ to the bank many times, gaining more money each time; this particular attack is defeated by timestamping the messages.
\end{example}

\begin{definition}[Signed]
    A message $m$ is \vocab{signed} as $(m, s)$ where the \vocab{signature} $s = s(m,k)$ is a function of $m$ and the private key $k$.
\end{definition}

The recipient can check the signature using the public key to verify authenticity of the message.
The signature function or \vocab{trapdoor} function $s \colon \mathcal M \times \mathcal K \to \mathcal S$ is designed s.t. without knowledge of the private key, one cannot sign messages, but anyone can check whether a signature is valid.
Note that the signature is associated to each message, not to each sender.

\begin{example}[Signatures using RSA]
    Suppose Alice has a private key $(N, d)$, and broadcasts a public key $(N, e)$.
    She signs a message $m$ as $(m, s)$ where $s = m^d \text{ mod } N$.
    The signature is verified by checking $s^e = m$.

    This technique is vulnerable to the homomorphism attack.
    This is also vulnerable to the \vocab{existential forgery} attack, in which an attacker produces valid signed messages of the form $(s^e \text{ mod } N, s)$ after choosing $s$ first.
    Hopefully, such messages are not meaningful.

    To solve these problems, we could use a better signature scheme.
    In addition, rather than signing a message $m$, we instead sign the \vocab{digest} $h(m)$ where $h \colon \mathcal M \to \qty{1, \dots, N-1}$ is a \vocab{hash} function.
    A hash function is a publicly known function for which it is very difficult to find pairs of messages with matching hashes; such a pair is called a \vocab{collision}.
    Examples of hash functions include MD5 and the SHA family.
\end{example}

\subsection{Elgamal Signature Scheme}
Alice chooses a large prime $p$ and a random integer $u$ with $1 < u < p$.
Let $g$ be a primitive root mod $p$.
The public key is $p, g, y = g^u \text{ mod } p$.
The private key is $u$.
Let $h \colon \mathcal M \to \qty{1, \dots, p-1}$ be a collision-resistant hash function.

To send a message $m$ with $0 \leq m \leq p-1$, Alice randomly chooses $k$ with $1 \leq k \leq p-2$ coprime to $p-1$.
She computes $r, s$ with $1 \leq r \leq p-1$ and $1 \leq s \leq p-2$ satisfying
\begin{align*}
    r \equiv g^k \mod p;\quad h(m) \equiv ur + ks \mod (p-1)
\end{align*}
Since $k$ is coprime to $p-1$, the congruence for $s$ always has a solution.
Alice signs the message $m$ with the signature $(r, s)$.
Now,
\begin{align*}
    g^{h(m)} \equiv g^{ur + ks} \equiv (g^u)^r (g^k)^s \equiv y^r r^s \mod p
\end{align*}
Bob accepts a signature if $g^{h(m)} \equiv y^r r^s$ mod $p$.
To forge a signature, obvious attacks involve the discrete logarithm problem, finding $u$ from $y = g^u$.

\begin{lemma}
    Let $a, b, m \in \mathbb N$ and consider the congruence $ax \equiv b$ mod $m$.
    This has either no or $\gcd(a,m)$ solutions for $x$ mod $m$.
\end{lemma}

\begin{proof}
    Let $d = \gcd(a,m)$.
    If $d \nmid b$, there is no solution.
    If $d \mid b$, we can rewrite the congruence as $\frac{a}{d} x \equiv \frac{b}{d}$ mod $\frac{m}{d}$.
    Note that $\frac{a}{d}, \frac{m}{d}$ are coprime, so this congruence has a unique solution $\mod \frac{m}{d}$, hence we have $d$ solns $\mod m$.
\end{proof}

It is \underline{vital} that Alice chooses a new value of $k$ to sign each message.
Suppose she sends $m_1, m_2$ using the same value of $k$.
Denote the signatures $(r, s_1)$ and $(r, s_2)$; note that $r$ depends only on $k$ and is hence fixed.
\begin{align*}
    h(m_1) \equiv ur + ks_1 \mod (p-1);\quad h(m_2) \equiv ur + ks_2 \mod (p-1)
\end{align*}
Hence,
\begin{align*}
    h(m_1) - h(m_2) \equiv k(s_1 - s_2) \mod (p-1)
\end{align*}
By the previous lemma, there are $d = \gcd(p-1, s_1 - s_2)$ of solutions for $k$ modulo $p-1$.
Choose the solution that gives the correct value in the first congruence $r \equiv g^k$ mod $p$.
Then,
\begin{align*}
    s_1 \equiv \frac{h(m_1) - ur}{k} \mod (p-1)
\end{align*}
This gives $ur \equiv h(m_1) - ks_1$.
Hence, using the lemma again, there are $\gcd(p-1, r)$ solutions for $u$.
Choose the solution for $u$ that gives $y \equiv g^u$.
This allows us to deduce Alice's private key $u$, as well as the exponent $k$ used in both messages.

\begin{remark}
    Several existential forgeries are known, i.e. we can find solns $m, r, s$ to $g^m = y^r y^s \mod p$ but with no control over $m$.
    In practice, this is stopped by signing a hash value of the message instead of the message itself.
\end{remark}

\subsection{The digital signature algorithm}
The digital signature algorithm is a variant of the Elgamal signature scheme developed by the NSA.
The public key is $(p, q, g)$ constructed as follows.
% TODO: verify previous line
\begin{itemize}
    \item Let $p$ be a prime of exactly $N$ bits, where $N$ is a multiple of 64 s.t. $512 \leq N \leq 1024$, so $2^{N-1} < p < 2^N$.
    \item Let $q$ be a prime of 160 bits, s.t. $q \mid p-1$.
    \item Let $g \equiv h^{\frac{p-1}{q}}$ mod $p$, where $h$ is a primitive root mod $p$; in particular, $g$ is an element of order $q$ in $\mathbb Z_p^\times$.
    \item Alice chooses a private key $x$ with $1 < x < q$ and publishes $y = g^x$.
\end{itemize}
Let $m$ be a message with $0 \leq m < q$.
She chooses a random $k$ with $1 < k < q$, and computes
\begin{align*}
    s_1 \equiv (g^k \text{ mod } p) \text{ mod } q;\quad s_2 \equiv k^{-1}(m+xs_1) \text{ mod } q
\end{align*}
The signature is $(s_1, s_2)$.
To verify a signature, we perform the following procedure.
Bob computes $w \equiv s_2^{-1}$ mod $q$, $u_1 \equiv mw$ mod $q$, $u_2 \equiv s_1 w$ mod $q$, and $v = (g^{u_1} y^{u_2} \text{ mod } p) \text{ mod } q$.
He accepts the signature if $v = s_1$.
\begin{proposition}
    If a message is signed with the DSA and the message is not manipulated, the signature is accepted.
\end{proposition}
\begin{proof}
    First, note that $(m + xs_1)w = ks_2s_2^{-1}$ mod $q$.
    Now, as $g^q = 1$ mod $p$,
    \begin{align*}
        v &= (g^{u_1} y^{u_2} \text{ mod } p) \text{ mod } q \\
        &= (g^{mw} g^{xs_1 w} \text{ mod } p) \text{ mod } q \\
        &= (g^{(m + xs_1)w} \text{ mod } p) \text{ mod } q \\
        &= (g^k \text{ mod } p) \text{ mod } q \\
        &= s_1
    \end{align*}
    Hence, for a correctly signed message, the verification succeeds.
\end{proof}
Suppose that Alice sends $m_1$ to Bob and $m_2$ to Carol, and provides signatures for each message using the DSA.
One can show that if Alice uses the same value of $k$ for both transmissions, it is possible for an eavesdropper to recover the private key $x$ from the signed messages.

\subsection{Commitment schemes}
Suppose Alice wants to send a bit $m \in \qty{0,1}$ to Bob in such a way that
\begin{enumerate}
    \item Bob cannot determine the value of $m$ without Alice's help; and
    \item Alice cannot change the bit once she has sent it.
\end{enumerate}
Such a system can be used for coin tossing: suppose Alice and Bob are in different rooms, where Alice tosses a coin and Bob guesses the result.
The result of the coin and Bob's guess can be viewed as messages of this form.
As another example, consider a poll whose result cannot be viewed until everyone has voted.
We will see two examples of such a \vocab{commit-and-reveal} strategy, known as \vocab{bit commitment}.

\subsubsection{Using a Public Key Cryptosystem}
Suppose that we have a publicly known encryption function $e_A$ and a decryption function $d_A$ known only to Alice. \\
Alice makes a choice for her message $m \in \mathbb{F}_2$, and commits to Bob the ciphertext $c = e_A(m)$.
Under the assumption that the cipher is secure, Bob cannot decipher the message. \\
To reveal her choice, Alice sends her private key to Bob, who can then use it to decipher the message $d_A(c) = d_A(e_A(m)) = m$.
He can also check that $d_A, e_A$ are inverse functions and thus ensure that Alice sent the correct private key.

\subsubsection{Using Coding Theory}
Alternatively, suppose that Alice has two ways to communicate to Bob: a clear channel which transmits with no errors, and a binary symmetric channel with error prob $p$.
Suppose $0 < p < \frac{1}{2}$, and the noisy channel corrupts bits independent of any action of Alice or Bob, so neither can affect its behaviour.

\begin{itemize}
    \item Bob publishes a binary linear code $C$ of length $N$ and min dist $d$.
    \item Alice publishes a random non-trivial linear map $\theta \colon C \to \mathbb F_2$.
    \item To send a bit $m \in \mathbb F_2$, Alice chooses a random codeword $c \in C$ s.t. $\theta(c) = m$, and sends $c$ to Bob via the noisy channel. Bob receives $r = c + e \in \mathbb F_2^N$ where $e$ is the error pattern.
    \item The expected value of $d(r,c) = d(e,0)$ is $Np$. $N$ is chosen s.t. $Np \gg d$, so Bob cannot tell what the original codeword $c$ was, and hence cannot find $\theta(c) = m$.
    \item To reveal, Alice sends $c$ to Bob using the clear channel.
    Bob can check that $d(c,r) \approx Np$; if so, he accepts the message.
    \item It is possible that many more or many fewer bits of $c$ were corrupted by the noisy channel, which may make Bob reject the message even if Alice correctly committed and revealed the message.
    $N, d$ should be chosen s.t. the prob of this occurring is negligible.
    \item If it \underline{does} occur, $A$, $B$ repeat the process.
\end{itemize}

We have shown that Bob cannot read Alice's guess until she reveals it.
In addition, Alice cannot cheat by changing her guess, because she knows $c$ but not how it was corrupted by the noisy channel.
All she knows is that the received message $r$ has distance approximately $Np$ from $c$.
If she were to send $c' \neq c$, she must ensure that $d(r,c') \approx Np$, but the prob that this happens is small unless she chooses $c'$ very close to $c$.
But any two distinct codewords have distance at least $d$, so she \underline{cannot} cheat.

\subsection{Secret sharing schemes - Non Examinable?}
% This has been made nonexaminable.
Suppose that the CMS is attacked by the MIO.
% (mathematical institute, oxford)
The Faculty will retreat to a bunker known as MR2.
Entry to MR2 is controlled by a \vocab{secret}, which is a positive integer $S$.
This secret is known only to the Leader.
Each of the $n$ members of the Faculty knows a pair of numbers, called their \vocab{shadow} or \vocab{share}.
It is required that, in the absence of the Leader, any $k$ members of the Faculty can reconstruct the secret from their shadows, but any $k-1$ cannot.
\begin{definition}
    Let $k, n \in \mathbb N$ with $k < n$.
    A \vocab{$(k, n)$-threshold scheme} is a method of sharing a message $S$ among a set of $n$ participants s.t. any subset of $k$ participants can reconstruct $S$, but no subset of smaller size can reconstruct $S$.
\end{definition}
We discuss Shamir's method for implementing such a scheme.
Let $0 \leq S \leq N$ be the secret, which can be chosen at random by the Leader.
The Leader chooses and publishes a prime $p > n, N$.
They then choose independent random coefficients $a_1, \dots, a_{k-1}$ with $0 \leq a_j \leq p-1$ where we take $a_0 = S$, and distinct integers $x_1, \dots, x_n$ with $1 \leq x_j \leq p-1$.
Define
\begin{align*}
    P(r) \equiv a_0 + \sum_{j=1}^{k-1} a_j x_r^j \mod p
\end{align*}
choosing $0 \leq P(r) \leq p-1$.
The $r$th participant is given their shadow pair $(x_r, P(r))$ to be kept secret.
The Leader can then discard their computations.

Suppose $k$ members of the Faculty assemble with shadow pairs $(y_j, Q(j)) = (x_{i_j}, P(i_j))$ for $1 \leq j \leq k$.
By properties of the Vandermonde determinant,
\begin{align*}
    \det\begin{pmatrix}
    1 & y_1 & \cdots & y_1^{k-1} \\
    1 & y_2 & \cdots & y_2^{k-1} \\
    \vdots & \vdots & \ddots & \vdots \\
    1 & y_k & \cdots & y_k^{k-1}
    \end{pmatrix} = \prod_{1 \leq j < i \leq k} (y_i - y_j)
\end{align*}
The $y_i$ are distinct, so this determinant does not vanish.
Hence, we can uniquely solve the system of $k$ simultaneous equations
\begin{align*}
    z_0 + y_1 z_1 + y_1^2 z_2 + \dots + y_k^{k-1} z_{k-1} &\equiv Q(1) \\
    z_0 + y_2 z_1 + y_2^2 z_2 + \dots + y_2^{k-1} z_{k-1} &\equiv Q(2) \\
    &\vdots \\
    z_0 + y_k z_1 + y_k^2 z_2 + \dots + y_k^{k-1} z_{k-1} &\equiv Q(k)
\end{align*}
In particular, $z_0 = a_0 = S$ is the secret, as $(a_0, \dots, a_{k-1})$ is also a solution to these equations by construction.
Suppose $k-1$ people attempt to reconstruct the secret.
In this case, the Vandermonde determinant gives
\begin{align*}
    \det\begin{pmatrix}
    y_1 & y_1^2 & \cdots & y_1^{k-1} \\
    y_2 & y_2^2 & \cdots & y_2^{k-1} \\
    \vdots & \vdots & \ddots & \vdots \\
    y_{k-1} & y_{k-1}^2 & \cdots & y_{k-1}^{k-1}
    \end{pmatrix} = y_1 y_2 \dots y_{k-1} \prod_{1 \leq j < i \leq k-1} (y_i - y_j)
\end{align*}
This is nonzero modulo $p$, so the system of equations
\begin{align*}
    z_0 + y_1 z_1 + y_1^2 z_2 + \dots + y_k^{k-1} z_{k-1} &\equiv Q(1) \\
    z_0 + y_2 z_1 + y_2^2 z_2 + \dots + y_2^{k-1} z_{k-1} &\equiv Q(2) \\
    &\vdots \\
    z_0 + y_{k-1} z_1 + y_{k-1}^2 z_2 + \dots + y_{k-1}^{k-1} z_{k-1} &\equiv Q(k-1)
\end{align*}
has solutions for $z_1, \dots, z_{k-1}$ regardless of the value of $z_0$.
Thus, $k - 1$ members of the Faculty cannot reconstruct the secret $S$, or even tell which values are more likely than others.
\begin{remark}
    Note that a polynomial of degree $k - 1$ can be recovered from its values at $k$ points, but not on fewer points; this technique is known as Lagrange interpolation.
    The secret shadow pairs can be changed without altering the secret $S$; the Leader simply chooses a different random polynomial with the same constant term.
    Changing the polynomial frequently can increase security, since any eavesdropper who has gathered some shadow pairs generated from one polynomial cannot use that information to help decrypt a different polynomial.
\end{remark}
\begin{example}
    Consider a $(3, n)$-threshold scheme, where ordinary workers in a company have single shares, the vice presidents have two shares, and the Leader has three.
    In this case, the secret can be recovered by any three ordinary workers, any two workers if one of them is a vice president, or the Leader alone.
    In such \vocab{hierarchical schemes}, the `importance' of individuals determines how many of them are required to recover the secret.
\end{example}
\begin{example}
    Suppose Alice has a private key that she wishes to store securely and reliably.
    She uses a $(k, 2k-1)$-threshold scheme, where she forms $2k-1$ shadow pairs and stores them in different locations.
    As long as she does not lose more than half of the pairs, she can recover her key, hence the scheme is reliable.
    An eavesdropper needs to steal more than half of the pairs in order to recover the key, hence the scheme is secure.
\end{example}
% exercise: is S compromised if the values of the x_j are known?
