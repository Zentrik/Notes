\section{Algebraic coding theory}

\subsection{Linear codes}

\begin{definition}[Linear Code]
    A binary code $C \subseteq \mathbb F_2^n$ is \vocab{linear} if $0 \in C$, and whenever $x, y \in C$, we have $x + y \in C$.
\end{definition}

Equivalently, $C$ is a vector subspace of $\mathbb F_2^n$.

\begin{definition}[Rank]
    The \vocab{rank} of a linear code $C$, denoted $\rank C$, is its dimension as an $\mathbb F_2$-vector space.
    A linear code of length $n$ and rank $k$ is called an $(n,k)$-code.
    If it has minimum distance $d$, it is called an $(n,k,d)$-code.
\end{definition}

Let $v_1, \dots, v_k$ be a basis for $C$.
Then $C = \qty{\sum_{i=1}^k \lambda_i v_i : \lambda_i \in \mathbb F_2}$.
The size of the code is therefore $2^k$, so an $(n,k)$-code is an $[n,2^k]$-code, and an $(n,k,d)$-code is an $[n,2^k,d]$-code.
The information rate is $\frac{k}{n}$.

\begin{definition}[Weight]
    The \vocab{weight} of $x \in \mathbb F_2^n$ is $w(x) = d(x,0)$.
\end{definition}

\begin{lemma}
    The minimum distance of a linear code is the minimum weight of a nonzero codeword.
\end{lemma}

\begin{proof}
    Let $x, y \in C$.
    Then, $d(x,y) = d(x+y,0) = w(x+y)$.
    Observe that $x \neq y$ iff $x + y \neq 0$, so $d(C)$ is the minimum $w(x+y)$ for $x + y \neq 0$.
\end{proof}

\begin{definition}[Inner Product]
    Let $x, y \in \mathbb F_2^n$.
    Define $x \cdot y = \sum_{i=1}^n x_i y_i \in \mathbb F_2$.
    This is symmetric and bilinear.
\end{definition}

\begin{warning}
    There are nonzero $x$ s.t. $x \cdot x = 0$.
\end{warning}

\begin{definition}[Parity Check Code]
    Let $P \subseteq \mathbb F_2^n$.
    The \vocab{parity check code} defined by $P$ is
    \begin{align*}
        C = \qty{x \in \mathbb F_2^n : \forall p \in P,\,p \cdot x = 0}
    \end{align*}
\end{definition}

\begin{example} ~\vspace*{-1.5\baselineskip}
    \begin{enumerate}
        \item $P = \qty{11\dots 1}$ gives the simple parity check code.
        \item $P = \qty{1010101, 0110011, 0001111}$ gives Hamming's original $[7,16,3]$-code.
        \item $C^+$ and $C^-$ are linear if $C$ is linear.
    \end{enumerate}
\end{example}

\begin{lemma}
    Every parity check code is linear.
\end{lemma}

\begin{proof}
    $0 \in C$ as $p \cdot 0 = 0$.
    If $p \cdot x = 0$ and $p \cdot y = 0$ then $p \cdot (x + y) = 0$, so $x, y \in C$ implies $x + y \in C$.
\end{proof}

\begin{definition}[Dual Code]
    Let $C \subseteq \mathbb F_2^n$ be a linear code.
    The \vocab{dual code} $C^\perp$ is defined by
    \begin{align*}
        C^\perp = \qty{x \in \mathbb F_2^n : \forall y \in C,\, x \cdot y = 0}
    \end{align*}
\end{definition}

By definition, $C^\perp$ is a parity check code, and hence is linear.
Note that $C \cap C^\perp$ may contain elements other than 0.

\begin{lemma}
    $\rank C + \rank C^\perp = n$.
\end{lemma}

\begin{proof}
    One can prove this by defining $C^\perp$ as an annihilator from linear algebra.
    A proof using coding theory is shown later.
\end{proof}

\begin{corollary}
    Let $C$ be a linear code.
    Then $(C^\perp)^\perp = C$.
    In particular, all linear codes are parity check codes, defined by $C^\perp$.
\end{corollary}

\begin{proof}
    If $x \in C$, then $x \cdot y = 0$ for all $y \in C^\perp$ by definition, so $x \in (C^\perp)^\perp$.
    Then $\rank C = n - \rank C^\perp = n - (n - \rank (C^\perp)^\perp) = \rank (C^\perp)^\perp$, so $C = (C^\perp)^\perp$.
\end{proof}

\begin{definition}[Generator Matrix]
    Let $C$ be an $(n,k)$-code.
    A \vocab{generator matrix} $G$ for $C$ is a $k \times n$ matrix where the rows form a basis for $C$.
    A \vocab{parity check matrix} $H$ for $C$ is a generator matrix for the dual code $C^\perp$, so it is an $(n-k) \times n$ matrix.
\end{definition}

The codewords of a linear code can be viewed either as linear combinations of rows of $G$, or linear dependence relations between the columns of $H$, so $C = \qty{x \in \mathbb F_2^n : H x = 0}$.

\subsection{Syndrome decoding}
\begin{definition}[Syndrome]
    Let $C$ be an $(n, k)$-code.
    The \vocab{syndrome} of $x \in \mathbb F_2^n$ is $Hx$.
\end{definition}

If we receive a word $x = c + z$ where $c \in C$ and $z$ is the error pattern, $Hx = Hz$ as $Hc = 0$.
If $C$ is $e$-error correcting, we precompute $Hz$ for all $z$ for which $w(z) \leq e$.
On receiving $x$, we can compute the syndrome $Hx$ and find this entry in the table of values of $Hz$.
If successful, we decode $c = x - z$, with $d(x,c) = w(z) \leq e$.

\begin{definition}[Equivalent]
    Codes $C_1, C_2 \subseteq \mathbb F_2^n$ are \vocab{equivalent} if there exists a permutation of bits that maps codewords in $C_1$ to codewords in $C_2$.
\end{definition}

Codes are typically only considered up to equivalence.

\begin{lemma}
    Every $(n, k)$-linear code is equivalent to one with generator matrix with block form $\begin{pmatrix}
        I_k & B
    \end{pmatrix}$ for some $k \times (n - k)$ matrix $B$.
\end{lemma}

\begin{proof}
    Let $G$ be a $k \times n$ generator matrix for $C$.
    Using Gaussian elimination, we can transform $G$ into row echelon form
    \begin{align*}
        G_{ij} = \begin{cases}
            0 & j < \ell(i) \\
            1 & j = \ell(i)
        \end{cases}
    \end{align*}
    for some $\ell(1) < \ell(2) < \dots < \ell(k)$.
    Permuting the columns replaces $C$ with an equivalent code, so wlog we may assume $\ell(i) = i$.
    Hence,
    \begin{align*}
        G = \begin{pmatrix}
            1 & & \star \\
            & \ddots & & B \\
            & & 1
        \end{pmatrix}
    \end{align*}
    Further row operations eliminate $\star$ to give $G$ in the required form.
\end{proof}

A message $y \in \mathbb F_2^k$ viewed as a row vector can be encoded as $yG$.
If $G = \begin{pmatrix}
    I_k & B
\end{pmatrix}$, then $yG = (y, yB)$ where $y$ is the message and $yB$ is a string of check digits.

\begin{definition}[Systematic Code]
    A \vocab{systematic code} is any code whose codewords can be split up in this manner.
\end{definition}

We now prove the following lemma that was stated earlier.
\begin{lemma}
    $\rank C + \rank C^\perp = n$.
\end{lemma}

\begin{proof}
    Let $C$ have generator matrix $G = \begin{pmatrix}
        I_k & B
    \end{pmatrix}$.
    $G$ has $k$ linearly independent columns, so there is a linear map $\gamma \colon \mathbb F_2^n \to \mathbb F_2^k$ defined by $x \mapsto Gx$ which is surjective.
    Its kernel is $C^\perp$.
    By the rank-nullity theorem, $\dim \mathbb F_2^n = \dim \ker \gamma + \dim \Im \gamma$, so $n = \rank C + \rank C^\perp$ as required.
\end{proof}

\begin{lemma}
    An $(n, k)$-code with generator matrix $G = \begin{pmatrix}
        I_k & B
    \end{pmatrix}$ has parity check matrix $H$ of the form $\begin{pmatrix}
        -B^\transpose & I_{n-k}
    \end{pmatrix}$.
\end{lemma}

\begin{proof}
    \begin{align*}
        GH^\transpose = \begin{pmatrix}
            I_k & B
        \end{pmatrix} \begin{pmatrix}
            -B \\
            I_{n-k}
        \end{pmatrix} = B + B = 2B = 0
    \end{align*}
    So the rows of $H$ generate a subcode of $C^\perp$.
    But $\rank H = n - k$, and $\rank C^\perp = n - k$.
    So $H = C^\perp$, and $C^\perp$ has generator matrix $H$.
\end{proof}

\begin{lemma}
    Let $C$ be a linear code with parity check matrix $H$.
    Then, $d(C) = d$ iff
    \begin{enumerate}
        \item any $d - 1$ columns of $H$ are linearly independent; and
        \item a set of $d$ columns of $H$ are linearly dependent.
    \end{enumerate}
\end{lemma}

The proof is left as an exercise.
% see online for proof

\subsection{Hamming codes}

\begin{definition}[Hamming Code]
    Let $d \geq 1$, and let $n = 2^d - 1$.
    Let $H$ be the $d \times n$ matrix with columns given by the nonzero elements of $\mathbb F_2^d$.
    The \vocab{Hamming $(n, n-d)$-linear code} is the (linear) code with parity check matrix $H$.
\end{definition}

\begin{remark}
    This is only defined up to equivalence.
\end{remark}

\begin{lemma}
    The Hamming $(n, n-d)$-code $C$ has minimum distance $d(C) = 3$, and is a perfect 1-error correcting code.
\end{lemma}

\begin{proof}
    Any two columns of $H$ are linearly independent, but there are three linearly dependent columns as $n = 2^d-1$.
    Hence, $d(C) = 3$.
    Hence, $C$ is $\floor*{\frac{3-1}{2}} = 1$-error correcting.
    A perfect code is one s.t. $\abs{C} = \frac{2^n}{V(n,e)}$.
    In this case, $n = 2^d - 1$ and $e = 1$, so $\frac{2^n}{1 + 2^d - 1} = 2^{n-d} = \abs{C}$ as required.
\end{proof}

\subsection{Reed--Muller codes}
Let $X = \qty{p_1, \dots, p_n}$ be a set of size $n$.
There is a correspondence between $\mathcal P(X)$ and $\mathbb F_2^n$.
\begin{align*}
    \mathcal P(X) \xrightarrow{A \mapsto 1_A} \qty{f \colon X \to \mathbb F_2} \xrightarrow{f \mapsto (f(p_1), \dots, f(p_n))} \mathbb F_2^n
\end{align*}
The \vocab{symmetric difference} of two sets $A, B$ is $A \symmdiff B = (A\setminus B) \cup (B \setminus A)$, which corresponds to vector addition in $\mathbb F_2^n$.
Intersection $A \cap B$ corresponds to the \vocab{wedge product} $x \wedge y = (x_1 y_1, \dots, x_n y_n)$.

Let $X = \mathbb F_2^d$, so $n = 2^d = \abs{X}$.
Let $v_0 = (1, \dots, 1)$, and let $v_i = 1_{H_i}$ where $H_i = \qty{p \in X : p_i = 0}$ is the \vocab{coordinate hyperplane} ($1 \leq i \leq d$).

\begin{definition}[Reed--Muller Code]
    Let $0 \leq r \leq d$.
    The \vocab{Reed--Muller code} $RM(d,r)$ of \vocab{order} $r$ and length $2^d$ is the linear code spanned by $v_0$ and all wedge products of at most $r$ of the the $v_i$ for $1 \leq i \leq d$.
\end{definition}

By convention, the empty wedge product is $v_0$.

\begin{example}
    Let $d = 3$, and let $X = \mathbb F_2^3 = \qty{p_1, \dots, p_8}$ in binary order.
    \begin{align*}
        \begin{array}{c|cccccccc}
            X & 000 & 001 & 010 & 011 & 100 & 101 & 110 & 111 \\\hline
            v_0 & 1 & 1 & 1 & 1 & 1 & 1 & 1 & 1 \\
            v_1 & 1 & 1 & 1 & 1 & 0 & 0 & 0 & 0 \\
            v_2 & 1 & 1 & 0 & 0 & 1 & 1 & 0 & 0 \\
            v_3 & 1 & 0 & 1 & 0 & 1 & 0 & 1 & 0 \\
            v_1 \wedge v_2 & 1 & 1 & 0 & 0 & 0 & 0 & 0 & 0 \\
            v_2 \wedge v_3 & 1 & 0 & 0 & 0 & 1 & 0 & 0 & 0 \\
            v_1 \wedge v_3 & 1 & 0 & 1 & 0 & 0 & 0 & 0 & 0 \\
            v_1 \wedge v_2 \wedge v_3 & 1 & 0 & 0 & 0 & 0 & 0 & 0 & 0
        \end{array}
    \end{align*}
    A generator matrix for Hamming's original code is a $4 \times 7$ submatrix in the top-right corner.
\end{example}

$RM(3,0)$ is spanned by $v_0$, and is hence the repetition code of length 8.
$RM(3,1)$ is spanned by $v_0, v_1, v_2, v_3$, which is equivalent to a parity check extension of Hamming's original $(7,4)$-code.
$RM(3,2)$ is an $(8,7)$-code, and can be shown to be equivalent to a simple parity check code of length 8.
$RM(3,3)$ is the trivial code $\mathbb F_2^8$ of length 8.

\begin{theorem} ~\vspace*{-1.5\baselineskip}
    \begin{enumerate}
        \item The vectors $v_{i_1} \wedge \dots \wedge v_{i_s}$ for $i_1 < \dots < i_s$ and $0 \leq s \leq d$ form a basis for $\mathbb F_2^n$.
        \item The rank of $RM(d,r)$ is $\sum_{s=0}^r \binom{d}{s}$.
    \end{enumerate}
\end{theorem}

\begin{proof}
    \emph{Part (i).}
    There are $\sum_{s=0}^d \binom{d}{s} = 2^d = n$ vectors listed, so it suffices to show they are a spanning set, or equivalently $RM(d,d)$ is the trivial code.
    Let $p \in X$, and let $y_i$ be $v_i$ if $p_i = 0$ and $v_0 + v_i$ if $p_i = 1$.
    Then $1_{\qty{p}} = y_1 \wedge \dots \wedge y_d$.
    Expanding this using the distributive law, $1_{\qty{p}} \in RM(d,d)$.
    But the set of $1_{\qty{p}}$ for $p \in X$ spans $\mathbb F_2^n$, as required.

    \emph{Part (ii).}
    $RM(d,r)$ is spanned by $v_{i_1} \wedge \dots \wedge v_{i_s}$ where $i_1 < \dots < i_s$ and $0 \leq s \leq r$.
    Since these are linearly independent by (i), so a basis.
    Hence the rank of $RM(d,r)$ is the number of such vectors, which is $\sum_{s=0}^r \binom{d}{s}$.
\end{proof}

\subsection{New codes from old (again)}

\begin{definition}[Bar Product]
    Let $C_1, C_2$ be linear codes of length $n$ where $C_2 \subseteq C_1$.
    The \vocab{bar product} is $C_1 \mid C_2 = \qty{(x \mid x + y)\footnote{The concatenation of $x$ and $x + y$.} : x \in C_1, y \in C_2}$.
\end{definition}

This is a linear code of length $2n$.

\begin{lemma} ~\vspace*{-1.5\baselineskip} \label{lem:12.6}
    \begin{enumerate}
        \item $\rank (C_1 \mid C_2) = \rank C_1 + \rank C_2$.
        \item $d(C_1 \mid C_2) = \min \qty{2d(C_1), d(C_2)}$.
    \end{enumerate}
\end{lemma}

\begin{proof}
    \emph{Part (i).}
    If $C_1$ has basis $x_1, \dots, x_k$ and $C_2$ has basis $y_1, \dots, y_\ell$, then $C_1 \mid C_2$ has basis
    \begin{align*}
        \qty{(x_i \mid x_i) : 1 \leq i \leq k} \cup \qty{(0 \mid y_i) : 1 \leq i \leq \ell}
    \end{align*}

    \emph{Part (ii).}
    Let $0 \neq (x \mid x + y)\footnote{$x \in C_1, y \in C_2$ not both $0$.} \in C_1 \mid C_2$.
    If $y \neq 0$, then $w(x \mid x + y) = w(x) + w(x + y) \geq w(y) \geq d(C_2)$.
    If $y = 0$, then $w(x \mid x + y) = w(x \mid x) = 2w(x) \geq 2d(C_1)$.
    Hence, $d(C_1 \mid C_2) \geq \min\qty{2d(C_1), d(C_2)}$.

    There is a nonzero $x \in C_1$ with $w(x) = d(C_1)$, so $d(C_1 \mid C_2) \leq w(x \mid x) = 2d(C_1)$.
    There is a nonzero $y \in C_2$ with $w(y) = d(C_2)$, giving $d(C_1 \mid C_2) \leq w(0 \mid 0 + y) = d(C_2)$, giving the other inequality as required.
\end{proof}

\begin{theorem} ~\vspace*{-1.5\baselineskip}
    \begin{enumerate}
        \item $RM(d,r) = RM(d-1,r) \mid RM(d-1,r-1)$ for $0 < r < d$.
        \item $RM(d,r)$ has minimum distance $2^{d-r}$ for all $r$.
    \end{enumerate}
\end{theorem}

\begin{proof}
    \emph{Part (i).}
    $RM(d-1, r-1) \subseteq RM(d-1, r)$, so bar product defined.
    Order the elements of $X = \mathbb{F}_2^d$ s.t. $v_d = (\underbracket{0, \dots, 0}_{2^{d-1}} \mid \underbracket{1, \dots, 1}_{2^{d-1}})$ and $v_i = (v_i' \mid v_i')$ ($1 \leq i \leq d - 1$).
    If $z \in RM(d, r)$ then $z$ is sum of wedge products of $v_1, \dots, v_d$.
    Write $z = x + (y \wedge v_d)$ for $x, y$ sums of wedge products of $v_1, \dots, v_{d-1}$.
    Then $x = (x' \mid x')$\footnote{Note $x'$ is the vector containing the first $2^{d-1}$ components of $x$. Similarly for $y$.}, some $x' \in RM(d-1, r)$ and $y = (y' \mid y')$, some $y' \in RM(d-1, r-1)$.
    Then $z = x + (y \wedge v_d) = $
    \begin{align*}
        z = x + (y \wedge v_d) &= (x' \mid x') + (y' \mid y') \wedge (0, \dots, 0 \mid 1, \dots, 1) \\
        &= (x' \mid x' + y') \in RM(d-1, r) \mid RM(d-1, r-1).
    \end{align*}

    \emph{Part (ii).}
    If $r = 0$, then $RM(d,r)$ is the repetition code of length $2^d$, which has min distance $2^d$. \\
    If $r = d$, $RM(d,r)$ is the trivial code of length $2^d$, which has min distance $1 = 2^{d-d}$. \\
    We prove the remaining cases by induction on $d$.
    From part (i), $RM(d,r) = RM(d-1,r) \mid RM(d-1,r-1)$.
    By induction, the min distance of $RM(d-1,r)$ is $2^{d-1-r}$ and the min distance of $RM(d-1,r-1)$ is $2^{d-r}$.
    By part (ii) of \cref{lem:12.6}, the min distance of $RM(d,r)$ is $\min\qty{2\cdot 2^{d-1-r}, 2^{d-r}} = 2^{d-r}$.
\end{proof}

\begin{remark}
    \begin{enumerate}
        \item One could define $RM(d, 0)$ and $RM(d, d)$ and also define recursively $RM(d, r)$ as a bar product.
        \item $RM(5, 1)$ was used by NASA for the Mariner 9 mission to Mars.
        \item Decoding procedure using `successive majority verdicts' is outline in (Goldie and Pinch, pages 165-167).
    \end{enumerate}
\end{remark}

\subsection{GRM Recap}
\begin{definition}[Ring]
    A \vocab{ring} $R$ is a set with operations $+, \times$ (e.g. $\mathbb{Z}$, $\mathbb{Z}_n$).
\end{definition}

\begin{definition}[Field]
    A \vocab{field} is a commutative ring where every non-zero element has a multiplicative inverse (e.g. $\mathbb{Q}$, $\mathbb{R}$, $\mathbb{C}$, $\mathbb{F}_p$).
\end{definition}

Every field is an extension (subfield) of $\mathbb{F}_p$, in which case we say its has characteristic $p$, or of $\mathbb{Q}$, characteristic 0.

\begin{definition}[Polynomial Ring]
    A \vocab{polynomial ring} with coefficients in $R$ is $R[X] = \qty{\sum_{i=0}^{n} a_i X^i : a_i \in R, n \in \mathbb{N}_0}$.
\end{definition}

By defn, $\sum_{i=0}^n a_i X^i = 0$ iff each $a_i$ is zero.
Note that $X^2 + X \in \mathbb F_2[X]$ is nonzero, but always evaluates to zero. \\
If $F$ is a field, $F[X]$ is a Euclidean domain using the degree function as the Euclidean function, and has a Euclidean division algorithm.
If $f, g \in F[X]$, $g \neq 0$, $\exists \; q, r \in F[X]$ s.t. $f = qg + r$ with $\deg r < \deg q$.

\begin{definition}[Ideal]
    An \vocab{ideal} $I \subseteq R$ is a subgroup under $+$ s.t. $r \in R$, $x \in I \implies rx \in I$ (e.g. $2 \mathbb{Z} \triangleleft \mathbb{Z}$).
\end{definition}

\begin{definition}[Principal Ideal]
    The \vocab{principal ideal} generated by $x \in R$ is $(x) = Rx = xR = \qty{rx : r \in R}$.
\end{definition}

By division algorithm, every ideal in $\mathbb{Z}$ or $F[X]$ is principal, generated by an element of least absolute value and least degree respectively.
The generator of a prime ideal is unique up to multiplication by a unit (an element with multiplicative inverse).
$\mathbb{Z}$ has units $\qty{\pm 1}$, $F[X]$ has units $F \setminus \qty{0}$.

Every non-zero element of $\mathbb{Z}$ or $F[X]$ can be factored into irreducibles, uniquely up to order and multiplication by units.

If $I \trianglelefteq R$ ideal then set of cosets $\faktor{R}{I} = \qty{x + I : x \in R}$ is the \vocab{quotient ring} under natural choice of $+, \times$.
In practice identify $\faktor{\mathbb{Z}}{n \mathbb{Z}}$ and $\qty{0, 1, \dots, n-1}$ and agree to reduce $\mod n$ after each $+, \times$. \\
Similarly, $\faktor{F[X]}{(f(X))} \longleftrightarrow \qty{\sum_{i=0}^{n-1} a_i X^i : a_i \in F} \longleftrightarrow F^n$ where $n = \deg f$, reducing after each multiplication using division algo.

% If $F$ is a field and $f \in F[X]$, $\faktor{F[X]}{(f)}$ is in bijection with $F^n$ where $n = \deg f$, since $\faktor{F[X]}{(f)}$ is represented by the set of functions of degree less than $\deg f$.

\subsection{Cyclic Codes}

\begin{definition}[Cyclic Code]
    A linear code $C \subseteq \mathbb F_2^n$ is \vocab{cyclic} if
    \begin{align*}
        (a_0, a_1, \dots, a_{n-1}) \in C \implies (a_{n-1}, a_0, \dots, a_{n-2}) \in C
    \end{align*}
\end{definition}

We identify $\faktor{\mathbb F_2[X]}{(X^n - 1)}$ with $\mathbb F_2^n$ as above, letting $\pi(a_0, a_1, \dots, a_{n-1}) = a_0 + a_1X + \dots + a_{n-1}X^{n-1} \mod (X^n - 1)$.

\begin{lemma}
    A code $C \subseteq \mathbb F_2^n$ is cyclic iff $\mathcal C = \pi(C)$ satisfies
    \begin{enumerate}
        \item $0 \in \mathcal C$;
        \item $f, g \in \mathcal C$ implies $f + g \in \mathcal C$;
        \item $f \in \mathbb F_2[X], g \in \mathcal C$ implies $fg \in \mathcal C$.
    \end{enumerate}
\end{lemma}

Equivalently, $\mathcal C$ is an ideal of $\faktor{\mathbb F_2[X]}{(X^n - 1)}$.

\begin{proof}
    If $g(X) = a_0 + a_1X + \dots + a_{n-1}X^{n-1} \mod (X^n - 1)$, then $Xg(X) = a_{n-1} + a_0X + \dots + a_{n-2}X^{n-1} \mod (X^n - 1)$.
    So $\mathcal C$ is cyclic iff (i) and (ii) hold and if (iii)': $g(X) \in C \implies Xg(X) \in C$.
    Note (iii)' is the case $f(X) = X$ of (iii).
    In general, $f(X) = \sum a_i X^i$ so
    \begin{align*}
        f(X) g(X) &= \sum_i a_i \underbracket{X^i g(X)}_{\in \mathcal{C} \text{ by (iii)}} \in \mathcal{C} \text{ by (ii)}
    \end{align*}
\end{proof}

Henceforth, we will identify $C$ with $\mathcal C$.

\underline{Basic problem}: to find all cyclic codes of length $n$. \\
The cyclic codes of length $n$ correspond to ideals in $\faktor{\mathbb F_2[X]}{(X^n - 1)}$.
Such ideals correspond to ideals of $\mathbb F_2[X]$ that contain $X^n - 1$.
Since $\mathbb F_2[X]$ is a principal ideal domain, these ideals correspond to polynomials $g(X) \in \mathbb F_2[X]$ dividing $X^n - 1$.

\begin{theorem}
    Let $C \trianglelefteq \faktor{\mathbb F_2[X]}{(X^n - 1)}$ be a cyclic code.
    Then $\exists!$ \vocab{generating polynomial} $g(X) \in \mathbb F_2[X]$ s.t.
    \begin{enumerate}
        \item $C = \qty{f(X) g(X) \mod (X^n -1) : f(X) \in \mathbb{F}_2[X]} = (g)$;
        \item $g(X) \mid X^n - 1$.
    \end{enumerate}
    In particular, $p(X) \in \mathbb F_2[X]$ represents a codeword iff $g \mid p$.
\end{theorem}

\begin{proof}
    Let $g(X) \in \mathbb F_2[X]$ be a poly poly of least degree representing a $\neq 0$ codeword of $C$.
    Note that $\deg g < n$.
    Since $C$ is cyclic, $(g) \subseteq C$. \\
    Now let $p(X) \in \mathbb F_2[X]$ represent a codeword.
    By the division algorithm, $p = qg + r$ for $q, r \in \mathbb F_2[X]$ where $\deg r < \deg g$.
    Then, $r = p - qg \in C$ as $C$ is an ideal.
    But $\deg r < \deg g$, so $r = 0$.
    Hence, $g \mid p$.
    This shows $C \subseteq (g)$ in (i). \\
    For part (ii), let $p(X) = X^n - 1$, giving $g \mid X^n - 1$.

    Now we show uniqueness.
    Suppose $C = (g_1) = (g_2)$.
    Then $g_1 \mid g_2$ and $g_2 \mid g_1$.
    So $g_1 = cg_2$ for some unit in $c$.
    Units in $\mathbb{F}_2[X]$ are $\mathbb{F}_2 \setminus \qty{0} = \qty{1}$, so $g_1(x) = g_2(x)$.
\end{proof}

\begin{lemma}
    Let $C$ be a cyclic code of length $n$ with gen poly $g(X) = a_0 + a_1 X + \dots + a_k X^k$ with $a_k \neq 0$.
    Then $C$ has basis $\qty{g, Xg, X^2g, \dots, X^{n-k-1}g}$.
    In particular, $\rank C = n - k$.
\end{lemma}

\begin{proof}
    \underline{Linear Independence}: Suppose $f(X) g(X) = 0 \mod (X^n - 1)$ for some $f(X) \in \mathbb{F}_2[X]$ with $\deg f < n - k$.
    Then $\deg fg < n$, so $f(X)g(X) = 0$, hence $f(X) = 0$, i.e. every dependence relation is trivial.

    \underline{Spanning}: Let $p(x) \in \mathbb{F}_2[X]$ represent a codeword.
    WLOG  $\deg p < n$.
    Since $g(X)$ is the gen poly, $g(x) \mid p(X)$ i.e. $p(X) = f(X) g(X)$ for some $f(X) \in \mathbb{F}_2[X]$.
    Also $\deg f = \deg p - \deg q < n - k$, so $p(X)$ lies in the span of $g(X), \dots, X^{n - k - 1} g(X)$.
\end{proof}

\begin{corollary}
    Let $C$ be a cyclic code of length $n$ with gen poly $g(X) = a_0 + a_1 X + \dots + a_k X^k$ with $a_k \neq 0$.
    Then, a generator matrix for $C$ is given by
    \begin{align*}
        G = \begin{pmatrix}
            a_0 & a_1 & a_2 & \cdots & a_k & 0 & 0 & \cdots & 0 \\
            0 & a_0 & a_1 & \cdots & a_{k-1} & a_k & 0 & \cdots & 0 \\
            \vdots & \vdots & \vdots & \ddots & \vdots & \vdots & \vdots & \ddots & \vdots \\
            0 & 0 & 0 & \cdots & 0 & a_0 & a_1 & \cdots & a_k
        \end{pmatrix}
    \end{align*}
    This is an $(n - k) \times n$ matrix.
\end{corollary}

\begin{definition}[Parity Check polynomial]
    Let $g$ be a generator for $C$.
    The \vocab{parity check polynomial} is the polynomial $h$ s.t. $g(X) h(X) = X^n - 1$.
\end{definition}

\begin{corollary}
    Writing $h(X) = b_0 + b_1 X + \dots + b_{n-k}\footnote{$\neq 0$.} X^{n-k}$, the parity check matrix is
    \begin{align*}
        H = \begin{pmatrix}
            b_{n-k} & b_{n-k-1} & b_{n-k-2} & \cdots & b_1 & b_0 & 0 & 0 & \cdots & 0 \\
            0 & b_{n-k} & b_{n-k-1} & \cdots & b_2 & b_1 & b_0 & 0 & \cdots & 0 \\
            \vdots & \vdots & \vdots & \ddots & \vdots & \vdots & \vdots & \vdots & \ddots & \vdots \\
            0 & 0 & 0 & \cdots & 0 & b_{n-k} & b_{n-k-1} & b_{n-k-2} & \cdots & b_0
        \end{pmatrix}
    \end{align*}
    which is a $k \times n$ matrix.
\end{corollary}

\begin{proof}
    One can check that the inner product of the $i$th row of the generator matrix and the $j$th row of the parity check matrix is the coefficient of $X^{n-k-i+j}$ in $g(X) h(X) = X^n - 1$.
    Since $1 \leq i \leq n - k$ and $1 \leq j \leq k$, $0 < n - k - i + j < n$, and such coefficients are zero.
    Hence, the rows of $G$ are orthogonal to the rows of $H$.
    Note that as $b_{n-k} \neq 0$, $\rank H = k = \rank C^\perp$, so $H$ is the parity check matrix.
\end{proof}

\begin{remark}
    Given a polynomial $f(X) = \sum_{i=0}^m f_i X_i$ of degree $m$, the \vocab{reverse} polynomial is $\check{f}(X) = f_n + f_{n-1}X + \dots + f_0 X^M = X^m f\qty(\frac{1}{X})$.
    The cyclic code generated by $\check{h}$ is the dual code $C^\perp$.
\end{remark}

\begin{lemma}
    If $n$ is odd, $X^n - 1 = f_1(X) \dots f_t(X)$ where the $f_i(X)$ are distinct irreducible polys in $\mathbb F_2[X]$.
    Thus, there are $2^t$ cyclic codes of length $n$.
\end{lemma}

This is false if $n$ is even, for instance, $X^2 - 1 = (X - 1)^2$.
% The proof follows from Galois theory.

\begin{proof}
    If $X^n - 1$ has repeated factor, then $\exists$ field extension $K$ over $\mathbb{F}_2$ s.t. $X^n - 1 = (X - \lambda)^2 g(X)$ for some $\lambda \in K$, $g \in K[X]$.
    Taking formal derivatives, $nX^{n-1} = 2(X-\lambda)g(X) + (X-\lambda)^2 g'(X)$ so $n \lambda^{n-1} = 0$ so $\lambda = 0$ as $n$ odd\footnote{We are in $\mathbb{F}_2$ so $n$ even will work}.
    Also $\lambda^n = 1$ \Lightning.
\end{proof}

% \begin{exercise}
%     Show Hamming $(7,4, 3)$-code is a cyclic code and find its parity check and generating polynomial.
% \end{exercise}

\subsection{Reminders About Finite Fields}
\begin{theorem}
    Suppose $p$ prime, $\mathbb F_p = \faktor{\mathbb Z}{p\mathbb Z}$ is a field, and if $f(X) \in \mathbb F_p[X]$ is irreducible, then $K = \faktor{\mathbb F_p[X]}{(f)}$ is a field and has order $p^{\deg f}$.
    Moreover, any finite field arises in this way.
\end{theorem}

\begin{theorem}
    If $q = p^\alpha$ is a prime power where $\alpha \geq 1$, $\exists$ field $\mathbb F_q$ of order $q$ unique up to isomorphism.
\end{theorem}

\begin{warning}
    $\mathbb F_q \not\simeq \faktor{\mathbb Z}{q\mathbb Z}$ if $\alpha > 1$.
\end{warning}

\begin{theorem}
    The multiplicative group $\mathbb F_q^\times = \mathbb{F}_q \setminus \qty{0}$ is cyclic; there exists $\beta \in \mathbb F_q$ s.t. $\mathbb F_q^\times = \genset{\beta} = \qty{1, \beta, \dots, \beta^{q-2}}$.
    Such a $\beta$ is called a \vocab{primitive element}.
\end{theorem}

\subsection{BCH codes}

BCH codes are a particular type of cyclic code.

Let $n$ be an odd integer, and let $r \geq 1$ s.t. $2^r \equiv 1$ mod $n$, which always exists as $2$ is coprime to $n$.
Let $K = \mathbb F_{2^r}$, and define $\bm \mu_n(K) = \qty{x \in K : x^n = 1} \leq K^\times$, which is a cyclic group.
Since $n \mid (2^r - 1) = \abs{K^\times}$, $\bm \mu_n(K)$ is the cyclic group of order $n$.
Hence, $\bm \mu_n(K) = \qty{1, \alpha, \alpha^2, \dots, \alpha^{n-1}}$ for some \vocab{primitive $n$th root of unity} $\alpha \in K$.

\begin{definition}[Cyclic Code of Length $n$ with Defining Set]
    The \vocab{cyclic code of length $n$ with defining set} $A \subseteq \bm\mu_n(K)$ is the code
    \begin{align*}
        C = \qty{f(X) \in \faktor{\mathbb F_2[X]}{(X^n - 1)} : \forall \; a \in A,\, f(a) = 0}
    \end{align*}
\end{definition}

The gen poly $g(X)$ is the nonzero poly of least degree s.t. $g(a) = 0 \; \forall \; a \in A$.
Equivalently, $g$ is the lcm of the minimal polys of the elements of $A$.

\begin{definition}[BCH Code]
    The cyclic code of length $n$ with defining set $\qty{\alpha, \alpha^2, \dots, \alpha^{\delta - 1}}$ is a \vocab{BCH code} with \vocab{design distance} $\delta$.
\end{definition}

\begin{theorem} \label{thm:14.3}
    A BCH code $C$ with design distance $\delta$ has minimum distance $d(C) \geq \delta$.
\end{theorem}

This proof needs the following result.

\begin{lemma}
    The Vandermonde matrix satisfies
    \begin{align*}
        \det \begin{pmatrix}
            1 & 1 & 1 & \cdots & 1 \\
            x_1 & x_2 & x_3 & \cdots & x_n \\
            x_1^2 & x_2^2 & x_3^2 & \cdots & x_n^2 \\
            \vdots & \vdots & \vdots & \ddots & \vdots \\
            x_1^{n-1} & x_2^{n-1} & x_3^{n-1} & \cdots & x_n^{n-1}
        \end{pmatrix} = \prod_{1 \leq j < i \leq n} (x_i - x_j)
    \end{align*}
\end{lemma}

\begin{proof}
    Look it up.
\end{proof}

\begin{proof}[Proof of \Cref{thm:14.3}]
    Consider
    \begin{align*}
        H = \begin{pmatrix}
            1 & \alpha & \alpha^2 & \cdots & \alpha^{n-1} \\
            1 & \alpha^2 & \alpha^4 & \cdots & \alpha^{2(n-1)} \\
            \vdots & \vdots & \vdots & \ddots & \vdots \\
            1 & \alpha^{\delta - 1} & \alpha^{2(\delta - 1)} & \cdots & \alpha^{(\delta - 1)(n-1)}
        \end{pmatrix}
    \end{align*}
    This is a $(\delta - 1) \times n$ matrix.
    Any collection of $(\delta - 1)$ columns is independent as it forms a Vandermonde matrix.
    But any codeword of $C$ is a dependence relation between the columns of $H$.
    Hence every nonzero codeword has weight at least $\delta$, giving $d(C) \geq \delta$.
\end{proof}

Note that $H$ in the proof above is not a parity check matrix, as its entries do not lie in $\mathbb F_2$.
If worried about the proof, look at the addendum on moodle.

\subsubsection{Decoding BCH Codes}

Let $C$ be a cyclic code with defining set $\qty{\alpha, \alpha^2, \dots, \alpha^{\delta - 1}}$ where $\alpha \in K$ is a primitive $n$th root of unity.
By \cref{thm:14.3}, its minimum distance is at least $\delta$, so we should be able to correct $t = \floor*{\frac{\delta - 1}{2}}$ errors.
Suppose we send $c \in C$ through the channel, and receive $r = c + e$ where $e$ is the error pattern with at most $t$ nonzero errors.
Note that $r, c, e$ correspond to polynomials $r(X), c(X), e(X) = \sum_{i=0}^{n-1} e_i X^i$, and $c(\alpha^j) = 0$ for $j \in \qty{1, \dots, \delta - 1}$ as $c$ is a codeword.
Hence, $r(\alpha^j) = e(\alpha^j)$.

\begin{definition}[Error Locator Polynomial]
    The \vocab{error locator polynomial} of an error pattern $e \in \mathbb F_2^n$ is
    \begin{align*}
        \sigma(X) = \prod_{i \in \mathcal E} (1 - \alpha^i X) \in K[X]
    \end{align*}
    where $\mathcal E = \qty{i : e_i = 1}$.
\end{definition}

\underline{Aim}: Assuming that $\deg \sigma = \abs{\mathcal E} \leq t$, where $2t + 1 \leq \delta$, we want to recover $\sigma$ from $r(X)$.

\begin{theorem}
    Suppose $\deg \sigma = \abs{\mathcal E} \leq t$ where $2t + 1 \leq \delta$.
    Then $\sigma(X)$ is the \underline{unique} polynomial in $K[X]$ of least degree s.t.
    \begin{enumerate}
        \item $\sigma(0) = 1$;
        \item $\sigma(X) \sum_{j=1}^{2t} r(\alpha^j) X^j = \omega(X)$ mod $X^{2t+1}$ for some $\omega(X) \in K[X]$ of degree at most $t$.
    \end{enumerate}
\end{theorem}

\begin{proof}
    Define $\omega(X) = -X\sigma'(X)$, called the \vocab{error co-locator}.
    Hence,
    \begin{align*}
        \omega(X) = \sum_{i \in \mathcal E} \alpha^i X \prod_{\substack{j \neq i \\ j \in \mathcal{E}}} (1 - \alpha^j X)
    \end{align*}
    This polynomial has $\deg \omega = \deg \sigma$.
    Consider the ring $K\Brackets{X} = \qty{\sum_{i=0}^{\infty} p_i X^i : p_i \in K}$ of formal power series.
    In this ring,
    \begin{align*}
        \frac{\omega(X)}{\sigma(X)} = \sum_{i \in \mathcal E} \frac{\alpha^i X}{1 - \alpha^i X} = \sum_{i \in \mathcal E} \sum_{j = 1}^\infty (\alpha^i X)^j = \sum_{j=1}^\infty X^j \sum_{i \in \mathcal E} (\alpha^j)^i = \sum_{j=1}^\infty e(\alpha^j) X^j
    \end{align*}
    Hence $\sigma(X) \sum_{j=1}^\infty e(\alpha^j) X^j = \omega(X)$.
    By definition of $C$, we have $c(\alpha^j) = 0$ for all $1 \leq j \leq \delta - 1$.
    Hence $c(\alpha^j) = 0$ for $1 \leq j \leq 2t$.
    As $r = c + e$, $r(\alpha^j) = e(\alpha^j)$ for all $1 \leq j \leq 2t$, hence $\sigma(X) \sum_{j=1}^{2t} r(\alpha^j) X^j = \omega(X)$ mod $X^{2t+1}$.
    This verifies (i) and (ii) for this choice of $\omega$.
    $\omega(X) = -X\sigma'(X)$ so $\deg \omega = \deg \sigma = \abs{\mathcal E} \leq t$.

    For uniqueness, suppose there exist $\widetilde \sigma, \widetilde \omega \in K[X]$ with the properties (i), (ii).
    WLOG, we can assume $\deg \widetilde \sigma \leq \deg \sigma$.
    $\sigma(X)$ has distinct nonzero roots, so $\omega(X) = -X\sigma'(X)$ is nonzero at these roots.
    Hence $\sigma, \omega$ are coprime. \\
    By (ii), $\widetilde \sigma(X) \omega(X) = \sigma(X) \widetilde \omega(X)$ mod $X^{2t+1}$.
    But the degrees of $\sigma, \widetilde \sigma, \omega, \widetilde \omega$ are at most $t$, so this congruence is an equality.
    But $\sigma(X)$ and $\omega(X)$ are coprime, so $\sigma \mid \widetilde \sigma$, but $\deg \widetilde \sigma \leq \deg \sigma$ by assumption, so $\widetilde \sigma = \lambda \sigma$ for some $\lambda \in K$.
    By (i), $\sigma(0) = \widetilde\sigma(0)$ hence $\lambda = 1$, giving $\widetilde \sigma = \sigma$.
\end{proof}

\underline{Decoding Algorithm} \\
Suppose that we receive $r(X)$ and wish to decode it.

Recall $e(\alpha^j) = r(\alpha^j)$ for $j = 1, 2, \dots, 2t$.
\begin{itemize}
    \item Set $\sigma(X) = \sigma_0 + \sigma_1 X + \dots + \sigma_t X^t$ and $\sigma(X)(r(\alpha) X + r(\alpha^2) X^2 \dots + r(\alpha^{2t})X^{2t} + e(\alpha^{2t + 1}) X^{2t + 1} + \dots) = \sum_{i=0}^{t} \omega_i x^i$.
    \item Coeffs of $X^i$ for $t < i \leq 2i$ are $\sum_{j=0}^t \sigma_j r(\alpha^{i-j}) = 0$ which don't involve any of $e(\alpha^j)X^j$ for all $1 \leq j \leq 2t$.
    \item So we obtain a system of linear equations
    \begin{align*}
        \begin{pmatrix}
        r(\alpha^{t+1}) & r(\alpha^t) & \dots & r(\alpha) \\
        r(\alpha^{t+2}) & r(\alpha^{t+1}) & \dots & r(\alpha^2) \\
        \vdots &  &  &  \\
        r(\alpha^{2t}) & r(\alpha^{2t-1}) & \dots & r(\alpha^{t})
        \end{pmatrix}
        \begin{pmatrix}\sigma_0 \\ \sigma_1 \\ \vdots\\ \sigma_t \end{pmatrix}
        = 0
    \end{align*}
    \item So $\exists \sigma \neq 0$ in kernel.
    This determines $\sigma(X)$, hence what the errors are that we need to correct.
    % \item Then solve these polynomials over $K$, keeping solutions of least degree.
    % \item Compute $\mathcal E = \qty{i : \sigma(\alpha^{-i}) = 0}$, and check that $\abs{\mathcal E} = \deg \sigma$.
    % \item Set $e(X) = \sum_{i \in \mathcal E} X^i$, then $c(X) = r(X) + e(X)$, and check that $c$ is a codeword.
\end{itemize}

\begin{example}[Reed-Solomon]
    If $\mathbb{F} = \mathbb{F}_q$, $n = q - 1$ this is \vocab{Reed-Solomon} code.
    This is used in CDs.
    There are two RS codes over $\mathbb{F}_{2^8}$ with $\delta = 5$ with length $n = 32$, $28$ respectively.
    Error bursts caused by scratches on the CDs of about 4000 bits can be corrected.
\end{example}

\begin{example}
    % Consider $n = 7$, the roots of $X^7 - 1$ over $\mathbb{F}_2$ form cyclic group of order $7$, so every root is primitive.
    % $X^7 - 1 = (X + 1)(X^3 + X + 1)(X^3 + X^2 + 1)$ in irreducible form.
    % Let $\alpha$ be a root of $X^3 + X + 1$.
    % If $\alpha^7 = 1$, $(\alpha^2)^3 + (\alpha^2) + 1 = \alpha^6 (1 + \alpha^3 + \alpha) = 0$.
    % So $\alpha^2$ also a root of $X^3 + X + 1$.
    % Similarly $\alpha^4 = (\alpha^2)^2$ is also a root.
    % These roots are distinct.
    % \begin{align*}
    %     \therefore X^3 + X + 1
    % \end{align*}

    Consider $n = 7$, and $X^7 - 1 = (X + 1)(X^3 + X + 1)(X^3 + X^2 + 1)$ in $\mathbb F_2[X]$.
    Let $g(X) = X^3 + X + 1$, so $h(X) = (X + 1)(X^3 + X^2 + 1) = X^4 + X^2 + X + 1$.
    The parity check matrix is
    \begin{align*}
        H = \begin{pmatrix}
            1 & 0 & 1 & 1 & 1 & 0 & 0 \\
            0 & 1 & 0 & 1 & 1 & 1 & 0 \\
            0 & 0 & 1 & 0 & 1 & 1 & 1
        \end{pmatrix}
    \end{align*}
    The columns are the elements of $\mathbb F_2^3 \setminus \qty{0}$.
    This is the Hamming $(7,4)$-code.

    Let $K$ be a splitting field\footnote{This is from Galois Theory. We want to add roots till $X^7 - 1$ splits into linear factors.} for $X^7 - 1$; we can take $K = \mathbb F_8$.
    Let $\beta \in K$ be a root of $g$.
    Note that $\beta^3 = \beta + 1$, so $\beta^6 = \beta^2 + 1$, so $g(\beta^2) = 0$. %, and hence $g(\beta^4) = 0$.
    So the BCH code defined by $\qty{\beta, \beta^2}$ has generator polynomial $g(X)$, again proving that this is Hamming's $(7,4)$-code.
    This code has design distance $3$, so $d(C) \geq 3$, and we know Hamming's code has minimum distance exactly 3.
\end{example}

\subsection{Shift registers}

\begin{definition}[(General) Feedback Shift Register]
    A \vocab{(general) feedback shift register} is a map $f \colon \mathbb F_2^d \to \mathbb F_2^d$ given by
    \begin{align*}
        f(x_0, \dots, x_{d-1}) = (x_1, \dots, x_{d-1}, C(x_0, \dots, x_{d-1}))
    \end{align*}
    where $C \colon \mathbb F_2^d \to \mathbb F_2$.
    We say that the register has length $d$. \\
    The \vocab{stream} associated to an \vocab{initial fill} $(y_0, \dots, y_{d-1})$ is the sequence $y_0, y_1, \dots, y_n, \dots$ with $y_n = C(y_{n-d}, \dots, y_{n-1}) \; \forall \; n \geq d$.
\end{definition}

\begin{definition}[Linear Feedback Shift Register]
    The general feedback shift register $f \colon \mathbb F_2^d \to \mathbb F_2^d$ is a \vocab{linear feedback shift register (LFSR)} if $C$ is linear, so
    \begin{align*}
        C(x_0, \dots, x_{d-1}) = \sum_{i=0}^{d-1} a_i x_i, \quad a_i \in \mathbb{F}_2
    \end{align*}
    We usually set $a_0 = 1$.
\end{definition}

The stream produced by a LFSR is now given by the recurrence relation $y_n = \sum_{i=0}^{d-1} a_i y_{n-d+i}$.
We can define the \vocab{auxiliary polynomial} $P(X) = X^d + a_{d-1} X^{d-1} + \dots + a_1 X + a_0$.
We sometimes write $a_d = 1$, so $P(X) = \sum_{i=0}^d a_i X^i$.

\underline{Over $\mathbb{C}$}: general solution is linear combinations of $\alpha^n, n \alpha^n, \dots, n^{t-1} \alpha^n$ for $\alpha$ running over roots of $P(X)$ and $\alpha$ root of mult $t$.

\underline{Over $\mathbb{F}_2$}: $n^2 \equiv n \mod 2$ so this doesn't give enough solns.
Resolved by replacing $n^j \alpha^n$ by $\binom{n}{j} \alpha^n$.

\begin{definition}[Feedback Polynomial]
    The \vocab{feedback polynomial} is $\check{P}(X) = a_0 X^d + \dots + a_{d-1} X + 1 = \sum_{i=0}^d a_{d-i} X^i$.
    A sequence $y_0, \dots$ of elements of $\mathbb F_2$ has \vocab{generating function} $G(X) = \sum_{j=0}^\infty y_j X^j \in \mathbb F_2\Brackets{X}$.
\end{definition}

\begin{theorem}
    The stream $(y_n)_{n \geq 0}$ comes from a LFSR with auxiliary poly $P(X)$ iff its generating fcn is (formally) of the form $A(X)/\check{P}(X)$ with $A \in \mathbb F_2[X]$ s.t. $\deg A < \deg \check{P}$.
\end{theorem}

Note that $\check{P}(X) = X^{\deg P}P(X^{-1})$ is the reverse of $P$.

\begin{proof}
    Let $P(X)$ and $\check{P}(X)$ be as above.
    We require
    \begin{align*}
        G(X)\check{P}(X) = \qty(\sum_{j=0}^\infty y_j X^j) \qty(\sum_{i=0}^d a_{d-i} X^i)
    \end{align*}
    to be a poly of degree $< d$.
    This holds iff the coefficient of $X^n$ in $G(X) \check{P}(X)$ is 0 for all $n \geq d$, i.e. $\sum_{i=0}^d a_{d-i} y_{n-i} = 0$.
    This holds iff $y_n\footnote{$a_d = 1$} = \sum_{i=0}^{d-1} a_i y_{n-d + i}$ for all $n \geq d$.
    This is precisely the form of a stream that arises from a LFSR with auxiliary polynomial $P$.
\end{proof}

\begin{remark}
    The problem of recovering the LFSR from its stream and the problem of decoding BCH codes both involve writing a power series as a ratio of polynomials.
\end{remark}

\subsection{The Berlekamp--Massey method}
Let $(x_n)_{n \geq 0}$ be the output of a (binary) LFSR.
We wish to find the unknown length $d$ and values $a_0 (=1), a_1, \dots, a_{d-1}, a_d(=1)$ s.t. $a_0 x_n + \sum_{i=1}^d a_{d-i} x_{n-i} = 0$ for all $n \geq d$.
We have
\begin{align*}
    \underbrace{\begin{pmatrix}
        x_d & x_{d-1} & \cdots & x_1 & x_0 \\
        x_{d+1} & x_d & \cdots & x_2 & x_1 \\
        \vdots & \vdots & \ddots & \vdots & \vdots \\
        x_{2d-1} & x_{2d-2} & \cdots & x_d & x_{d-1} \\
        x_{2d} & x_{2d-1} & \cdots & x_{d+1} & x_d
    \end{pmatrix}}_{A_d} \begin{pmatrix}
        a_d \\
        a_{d-1} \\
        \vdots \\
        a_1 \\
        a_0
    \end{pmatrix} = 0 (\ast)
\end{align*}
We look successively at $A_0 = \begin{pmatrix}
    x_0
\end{pmatrix}, A_1 = \begin{pmatrix}
    x_1 & x_0 \\
    x_2 & x_1
\end{pmatrix}, \dots$, starting at $A_r$ if we know $d \geq r$.
For each $A_i$, we compute its determinant.
If $\abs{A_i} \neq 0$, then $d \neq i$.
If $\abs{A_i} = 0$, we solve $(\ast)$ on the assumption that $d = i$, giving a candidate for the coefficients $a_0, \dots, a_{d-1}$.
This candidate can be checked over as many terms of the stream as desired.
If the test fails, we know $d > i$.

\begin{remark}
    Usually Gaussian elimination is easier than expanding rows/ cols.
\end{remark}