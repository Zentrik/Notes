\section{Information theory}

\subsection{Sources and information rate}
\begin{definition}[Source]
    A \vocab{source} is a sequence of r.v.s $X_1, X_2, \dots$ taking values in the alphabet $\mathcal A$.
\end{definition}

\begin{example}[Bernoulli Source]
    The \vocab{Bernoulli} (or \vocab{memoryless}) source is a source where the $X_i$ are iid Bernoulli's.
\end{example}

\begin{definition}[Reliably Encodable]
    A source $X_1, X_2, \dots$ is \vocab{reliably encodable} at rate $r$ if $\exists$ subsets $A_n \subseteq \mathcal A^n$ s.t.
    \begin{enumerate}
        \item $\lim_{n \to \infty} \frac{\log \abs{A_n}}{n} = r$;
        \item $\lim_{n \to \infty} \prob{(X_1, \dots, X_n) \in A_n} = 1$.
    \end{enumerate}
\end{definition}

\begin{definition}[Information Rate]
    The \vocab{information rate} $H$ of a source is the infimum of all reliable encoding rates.
\end{definition}

\begin{example}
    $0 \leq H \leq \log\abs{\mathcal A}$, with both bounds attainable.
    The proof is left as an exercise.
\end{example}

Shannon's first coding theorem computes the information rate of certain sources, including Bernoulli sources.

Recall from IA Probability that a probability space is a tuple $(\Omega, \mathcal F, \mathbb P)$, and a discrete r.v. is a function $X \colon \Omega \to \mathcal A$.
The probability mass function is the function $p_X \colon \mathcal A \to [0,1]$ given by $p_X(x) = \prob{X = x}$.
We can consider the function $p(X) \colon \Omega \to [0,1]$ defined by the composition $p_X \circ X$, which assigns $p(X)(\omega) = \prob{X = X(\omega)}$; hence, $p(X)$ is also a r.v..

Similarly, given a source $X_1, X_2, \dots$ of r.v.s with values in $\mathcal A$, the probability mass function of any tuple $X^{(n)} = (X_1, \dots, X_n)$ is $p_{X^{(n)}}(x_1, \dots, x_n) = \prob{X_1 = x_1, \dots, X_n = x_n}$.
As $p_{X^{(n)}} \colon \mathcal A^n \to [0,1]$, and $X^{(n)} \colon \Omega \to \mathcal A^n$, we can consider $p(X^{(n)}) = p_{X^{(n)}} \circ X^{(n)}$ defined by $\omega \mapsto p_{X^{(n)}}(X^{(n)}(\omega))$.

\begin{example}
    Let $\mathcal A = \qty{A, B, C}$.
    Suppose
    \begin{align*}
        X^{(2)} = \begin{cases}
        AB & \text{with probability } 0.3 \\
        AC & \text{with probability } 0.1 \\
        BC & \text{with probability } 0.1 \\
        BA & \text{with probability } 0.2 \\
        CA & \text{with probability } 0.25 \\
        CB & \text{with probability } 0.05
    \end{cases}
    \end{align*}
    Then, $p_{X^{(2)}}(AB) = 0.3$, and so on.
    Hence,
    \begin{align*}
        p(X^{(2)}) = \begin{cases}
        0.3 & \text{with probability } 0.3 \\
        0.1 & \text{with probability } 0.2 \\
        0.2 & \text{with probability } 0.2 \\
        0.25 & \text{with probability } 0.25 \\
        0.05 & \text{with probability } 0.05
    \end{cases}
    \end{align*}
\end{example}

We say that a source $X_1,X_2, \dots$ converges in probability to a r.v. $L$ if $\forall \; \varepsilon > 0$, $\lim_{n \to \infty} \prob{\abs{X_n - L} > \varepsilon} = 0$.
We write $X_n \xrightarrow{\mathbb P} L$.
The weak law of large numbers states that if $X_1, X_2, \dots$ are iid real-valued r.v.s with finite $\expect{X_1}$, then $\frac{1}{n} \sum_{i=1}^n X_i \xrightarrow{\mathbb P} \expect{X}$.

\begin{example}[Bernoulli] \label{exm:ber}
    Let $X_1, X_2, \dots$ be a Bernoulli source.
    Then $p(X_1), p(X_2), \dots$ are iid r.v.s, and $p(X_1, \dots, X_n) = p(X_1) \dots p(X_n)$.
    Note that by the WLLN,
    \begin{align*}
        -\frac{1}{n} \log p(X_1, \dots, X_n) = -\frac{1}{n} \sum_{i=1}^n \log p(X_i) \xrightarrow{\mathbb P} \expect{-\log p(X_1)} = H(X_1)
    \end{align*}
\end{example}

\begin{lemma}
    The information rate of a Bernoulli source $X_1, X_2, \dots$ is at most the expected word length of an optimal code $c \colon \mathcal A \to \qty{0,1}^\star$ for $X_1$.
\end{lemma}

\begin{proof}
    Let $\ell_1, \ell_2, \dots$ be the codeword lengths when we encode $X_1, X_2, \dots$ using $c$.
    Let $\varepsilon > 0$.
    Let
    \begin{align*}
        A_n = \qty{x \in \mathcal A^n : c^\star(x) \text{ has length less than } n \qty(\expect{\ell_1} + \varepsilon)}
    \end{align*}
    Then,
    \begin{align*}
        \prob{(X_1, \dots, X_n) \in A_n} = \prob{\sum_{i=1}^n \ell_i \leq n \qty(\expect{\ell_1} + \varepsilon)} = \prob{\abs{\frac{1}{n} \sum_{i=1}^n \ell_i - \expect{\ell_i}} < \varepsilon} \to 1
    \end{align*}
    Now, $c$ is decipherable so $c^\star$ is injective.
    Hence, $\abs{A_n} \leq 2^{n\qty(\expect{\ell_1} + \varepsilon)}$.
    Making $A_n$ larger if necessary, we can assume $\abs{A_n} = \floor*{2^{n\qty(\expect{\ell_1} + \varepsilon)}}$.
    Taking logarithms, $\frac{\log \abs{A_n}}{n} \to \expect{\ell_1} + \varepsilon$.
    So $X_1, X_2, \dots$ is reliably encodable at rate $r = \expect{\ell_1} + \varepsilon$ for all $\varepsilon > 0$.
    Hence the information rate is at most $\expect{\ell_1}$.
\end{proof}
\begin{corollary}
    A Bernoulli source has information rate less than $H(X_1) + 1$.
\end{corollary}
\begin{proof}
    Combine the previous lemma with the noiseless coding theorem.
\end{proof}
Suppose we encode $X_1, X_2, \dots$ in blocks of size $N$.
Let $Y_1 = (X_1, \dots, X_N), Y_2 = (X_{N+1}, \dots, X_{2N})$ and so on, s.t. $Y_1, Y_2, \dots$ take values in $\mathcal A^N$.
One can show that if the source $X_1, X_2, \dots$ has information rate $H$, then $Y_1, Y_2, \dots$ has information rate $NH$.
\begin{proposition}
    The information rate $H$ of a Bernoulli source is at most $H(X_1)$.
\end{proposition}
\begin{proof}
    Apply the previous corollary to the $Y_i$ to obtain
    \begin{align*}
        NH < H(Y_1) + 1 = H(X_1, \dots, X_N) + 1 = NH(X_1) + 1 \implies H < H(X_1) + \frac{1}{N}.
    \end{align*}
    But $N > 1$ is arbitrary so can take limit.
\end{proof}

\subsection{Asymptotic equipartition property}
\begin{definition}[Asymptotic Equipartition Property (AEP)]
    A source $X_1, X_2, \dots$ satisfies the \vocab{asymptotic equipartition property} if $\exists$ a constant $H \geq 0$ s.t.
    \begin{align*}
        -\frac{1}{n} \log p(X_1, \dots, X_n) \xrightarrow{\mathbb P} H
    \end{align*}
\end{definition}

\begin{example}
    Suppose we toss a biased coin with probability $p$ of obtaining a head.
    Let $X_1, X_2, \dots$ be the results of independent coin tosses.
    If we toss the coin $N$ times, we expect $pN$ heads and $(1-p)N$ tails.
    The probability of any particular sequence of $pN$ heads and $(1-p)N$ tails is
    \begin{align*}
        p^{pN} (1-p)^{(1-p)N} = 2^{N (p \log p + (1-p) \log(1-p))} = 2^{-NH(X)}
    \end{align*}
    Not every sequence of tosses is of this form, but there is only a small probability of `atypical sequences'.
    With high probability, it is a `typical sequence' which has a probability close to $2^{-NH(X)}$.
\end{example}

\begin{lemma}
    The AEP for a source $X_1, X_2, \dots$ is equivalent to the property that $\forall \; \varepsilon > 0 \ \exists \; n_0 \in \mathbb N$ s.t. $\forall \; n \geq n_0$, $\exists \;$ `typical set' $T_n \subseteq \mathcal A^n$ s.t.
    \begin{enumerate}
        \item $\prob{(X_1, \dots, X_n) \in T_n} > 1 - \varepsilon$;
        \item $2^{-n(H+\varepsilon)} \leq p(x_1, \dots, x_n) \leq 2^{-n(H-\varepsilon)} \quad \forall \; (x_1, \dots, x_n) \in T_n$.
    \end{enumerate}
\end{lemma}

\begin{proof}[Proof sketch (Not Lectured)]
    First, we show that the AEP implies the alternative definition.
    We define
    \begin{align*}
        T_n &= \qty{(x_1, \dots, x_n) \midd \abs{-\frac{1}{n} \log p(x_1, \dots, x_n) - H} \leq \varepsilon} \\
        &= \qty{(x_1, \dots, x_n) \mid \text{condition (ii) holds}}
    \end{align*}
    For the converse,
    \begin{align*}
        \prob{\abs{\frac{1}{n}\log p(x_1, \dots, x_n) - H} < \varepsilon} \geq \prob{T_n} \to 1
    \end{align*}
\end{proof}

\subsection{Shannon's first coding theorem}
\begin{theorem}[Shannon's First Coding Theorem]
    Let $X_1, X_2, \dots$ be a source satisfying AEP with constant $H$.
    Then this source has information rate $H$.
\end{theorem}

\begin{proof}
    Let $\varepsilon > 0$, and let $T_n \subseteq \mathcal A^n$ be typical sets.
    Then, $\forall \; n \geq n_0(\varepsilon)$, $\forall \; (x_1, \dots, x_n) \in T_n$ we have $p(x_1, \dots, x_n) \geq 2^{-n(H + \varepsilon)}$.
    Therefore, $1 \geq \prob{T_n} \geq \abs{T_n} 2^{-n(H + \varepsilon)}$, giving $\frac{1}{n} \log \abs{T_n} \leq H + \varepsilon$.
    Taking $A_n = T_n$ in the defn of reliable encoding shows that the source is reliably encodable at rate $H + \varepsilon$.

    Conversely, if $H = 0$ the proof concludes, so we may assume $H > 0$.
    Let $0 < \varepsilon < \frac{H}{2}$, and suppose that the source is reliably encodable at rate $H - 2\varepsilon$ with sets $A_n \subseteq \mathcal A^n$.
    Let $T_n \subseteq \mathcal A^n$ be typical sets.
    Then, $\forall \; (x_1, \dots, x_n) \in T_n$, $p(x_1, \dots, x_n) \leq 2^{-n(H - \varepsilon)}$, so $\prob{A_n \cap T_n} \leq 2^{-n(H - \varepsilon)} \abs{A_n}$, giving
    \begin{align*}
        \frac{1}{n} \log \prob{A_n \cap T_n} \leq -(H - \varepsilon) + \frac{1}{n} \log \abs{A_n} \to -(H - \varepsilon) + (H - 2 \varepsilon) = -\varepsilon
    \end{align*}
    Then, $\log \prob{A_n \cap T_n} \to -\infty$, so $\prob{A_n \cap T_n} \to 0$.
    But $\prob{T_n} \leq \prob{A_n \cap T_n} + \prob{\mathcal A^n \setminus A_n} \to 0 + 0$, contradicting typicality.
    So we cannot reliably encode at rate $H - 2\varepsilon$, so the information rate is at least $H$.
\end{proof}

\begin{corollary}
    A Bernoulli source $X_1, X_2, \dots$ has information rate $H(X_1)$.
\end{corollary}

\begin{proof}
    In \cref{exm:ber} we showed that for a Bernoulli source, \\ $-\frac{1}{n} \log p(X_1, \dots, X_n) \xrightarrow{\mathbb P} H(X_1)$.
    So the AEP holds with $H = H(X_1)$, giving the result by Shannon's first coding theorem.
\end{proof}

\begin{remark}
    The AEP is useful for noiseless coding.
    We can encode the typical sequences using a block code, and encode the atypical sequences arbitrarily.

    Many sources, which are not necessarily Bernoulli, also satisfy AEP.
    Under suitable hypotheses, the sequence $\frac{1}{n} H(X_1, \dots, X_n)$ is decreasing and bounded below, and the AEP is satisfied with constant $H = \lim_{n \to \infty} \frac{1}{n} H(X_1, \dots, X_n)$.

    For a Bernoulli source, $H(X_1, \dots, X_n) = n H(X_1)$.
\end{remark}

\subsection{Capacity}
Consider a communication channel with input alphabet $\mathcal A$ and output alphabet $\mathcal B$.
Recall the following definitions.
A \vocab{code} of length $n$ is a subset $C \subseteq \mathcal A^n$.
The \vocab{error rate} is
\begin{align*}
        \hat e(C) = \max_{c \in C} \prob{\text{error} \mid c \text{ sent}}
    \end{align*}
The \vocab{information rate} is $\rho(C) = \frac{\log \abs{C}}{n}$.
A channel can \vocab{transmit reliably at rate $R$} if $\exists$ codes $C_1, C_2, \dots$ where $C_n$ has length $n$ s.t. $\lim_{n \to \infty} \rho(C_n) = R$ and $\lim_{n \to \infty} \hat e(C_n) = 0$.

\begin{definition}[(Operational) Capacity]
    The \vocab{(operational) capacity} of a channel is the supremum of all rates at which it can transmit reliably.
\end{definition}

Suppose we are given a source with information rate $r$ bits per second that emits symbols at a rate of $s$ symbols per second.
Suppose we also have a channel with capacity $R$ bits per transmission that transmits symbols at a rate of $S$ transmissions per second.
Usually, information theorists take $S = s = 1$.
We will show that reliable encoding and transmission is possible iff $rs \leq RS$.

We will now compute the capacity of the binary symmetric channel with error probability $p$.
\begin{proposition}
    A binary symmetric channel with error probability $p < \frac{1}{4}$ has nonzero capacity.
\end{proposition}
\begin{proof}
    Let $\delta$ be s.t. $2p < \delta < \frac{1}{2}$.
    We claim that we can reliably transmit at rate $R = 1 - H(\delta) > 0$.
    Let $C_n$ be a code of length $n$, and suppose it has minimum distance $\floor*{n\delta}$ of maximal size.
    Then, by the GSV bound,
    \begin{align*}
        \abs{C_n} = A(n, \floor*{n\delta}) \geq 2^{-n(1-H(\delta))} = 2^{nR}
    \end{align*}
    Replacing $C_n$ with a subcode if necessary, we can assume $\abs{C_n} = \floor*{2^{nR}}$, with minimum distance at least $\floor*{n\delta}$.
    Using minimum distance decoding,
    \begin{align*}
        \hat e(C_n) &\leq \prob{\text{in $n$ uses, the channel makes at least } \floor*{\frac{\floor*{n\delta}-1}{2}} \text{ errors}} \\
        &\leq \prob{\text{in $n$ uses, the channel makes at least } \floor*{\frac{n\delta - 1}{2}} \text{ errors}}
    \end{align*}
    Let $\varepsilon > 0$ be s.t. $p + \varepsilon < \frac{\delta}{2}$.
    Then, for $n$ sufficiently large, $\frac{n\delta - 1}{2} = n\qty(\frac{\delta}{2} - \frac{1}{2n}) > n(p + \varepsilon)$.
    Hence, $\hat e(C_n) \leq \prob{\text{in $n$ uses, the channel makes at least } n(p+\varepsilon) \text{ errors}}$.
    We show that this value converges to zero as $n \to \infty$ using the next lemma.
\end{proof}
\begin{lemma}
    Let $\varepsilon > 0$.
    A binary symmetric channel with error probability $p$ is used to transmit $n$ digits.
    Then,
    \begin{align*}
        \lim_{n \to \infty} \prob{\text{in $n$ uses, the channel makes at least $n(p + \varepsilon)$ errors}} = 0
    \end{align*}
\end{lemma}
\begin{proof}
    Consider r.v.s $U_i = 1\qty[\text{the $i$th digit is mistransmitted}]$.
    The $U_i$ are iid with $\prob{U_i = 1} = p$.
    In particular, $\expect{U_i} = p$.
    Therefore, the probability that the channel makes at least $n(p + \varepsilon)$ errors is
    \begin{align*}
        \prob{\sum_{i=1}^n U_i \geq n(p + \varepsilon)} \leq \prob{\abs{\frac{1}{n} \sum_{i=1}^n U_i - p} \geq \varepsilon}
    \end{align*}
    so the result holds by the weak law of large numbers.
\end{proof}

\subsection{Conditional entropy}
\begin{definition}
    Let $X, Y$ be r.v.s taking values in alphabets $\mathcal A, \mathcal B$ respectively.
    Then, the \vocab{conditional entropy} is defined by
    \begin{align*}
        H(X \mid Y = y) = - \sum_{x \in \mathcal A} \prob{X = x \mid Y = y} \log \prob{X = x \mid Y = y}
    \end{align*}
    and
    \begin{align*}
        H(X \mid Y) = \sum_{y \in \mathcal B} \prob{Y = y} H(X \mid Y = y)
    \end{align*}
    Note that $H(X \mid Y) \geq 0$.
\end{definition}
\begin{lemma}
    $H(X,Y) = H(X \mid Y) + H(Y)$.
\end{lemma}
\begin{proof}
    \begin{align*}
        H(X \mid Y) &= -\sum_{y \in \mathcal B} \sum_{x \in \mathcal A} \prob{X = x \mid Y = y} \prob{Y = y} \log \qty(\prob{X = x \mid Y = y}) \\
        &= -\sum_{y \in \mathcal B} \sum_{x \in \mathcal A} \prob{X = x \mid Y = y} \prob{Y = y} \log \qty(\frac{\prob{X = x, Y = y}}{\prob{Y = y}}) \\
        &= -\sum_{y \in \mathcal B} \sum_{x \in \mathcal A} \prob{X = x, Y = y} \qty(\log \prob{X = x, Y = y} - \log \prob{Y = y}) \\
        &= -\sum_{y \in \mathcal B} \sum_{x \in \mathcal A} \prob{X = x, Y = y} \log \prob{X = x, Y = y} \\
        &+ \sum_{y \in \mathcal B} \sum_{x \in \mathcal A} \prob{X = x, Y = y} \log \prob{Y = y} \\
        &= -\sum_{y \in \mathcal B} \sum_{x \in \mathcal A} \prob{X = x, Y = y} \log \prob{X = x, Y = y} \\
        &+ \sum_{y \in \mathcal B} \prob{Y = y} \log \prob{Y = y} \\
        &= H(X,Y) - H(Y)
    \end{align*}
\end{proof}
\begin{example}
    Let $X$ be a uniform r.v. on $\qty{1, \dots, 6}$ modelling a dice roll, and $Y$ is defined to be zero if $X$ is even, and one if $X$ is odd.
    Then, $H(X,Y) = H(X) = \log 6$ and $H(Y) = \log 2$.
    Therefore, $H(X \mid Y) = \log 3$ and $H(Y \mid X) = 0$.
\end{example}
\begin{corollary}
    $H(X\mid Y) \leq H(X)$, with equality iff $X$ and $Y$ are independent.
\end{corollary}
\begin{proof}
    Combine this result with the fact that $H(X,Y) \leq H(X) + H(Y)$ where equality holds iff $H(X), H(Y)$ are independent.
\end{proof}
Now, replace r.v.s $X$ and $Y$ with random vectors $X^{(r)} = (X_1, \dots, X_r)$ and $Y^{(s)} = (Y_1, \dots, Y_s)$.
Similarly, we can define $H(X_1, \dots, X_r \mid Y_1, \dots, Y_s) = H(X^{(r)} \mid Y^{(s)})$.
Note that $H(X,Y\mid Z)$ is the entropy of $X$ and $Y$ combined, given the value of $Z$, and is not the entropy of $X$, together with $Y$ given $Z$.
\begin{lemma}
    Let $X, Y, Z$ be r.v.s.
    Then, $H(X \mid Y) \leq H(X \mid Y, Z) + H(Z)$.
\end{lemma}
\begin{proof}
    Expand $H(X,Y,Z)$ in two ways.
    \begin{align*}
        H(Z \mid X,Y) + \underbrace{H(X\mid Y) + H(Y)}_{H(X,Y)} = H(X,Y,Z) = H(X \mid Y,Z) + \underbrace{H(Z \mid Y) + H(Y)}_{H(Y,Z)}
    \end{align*}
    Since $H(Z \mid X,Y) \geq 0$, we have
    \begin{align*}
        H(X \mid Y) \leq H(X \mid Y,Z) + H(Z \mid Y) \leq H(X \mid Y,Z) + H(Z)
    \end{align*}
\end{proof}
\begin{proposition}[Fano's inequality]
    Let $X, Y$ be r.v.s taking values in $\mathcal A$.
    Let $\abs{\mathcal A} = m$, and let $p = \prob{X \neq Y}$.
    Then $H(X \mid Y) \leq H(p) + p \log(m-1)$.
\end{proposition}
\begin{proof}
    Define $Z$ to be zero if $X = Y$ and one if $X \neq Y$.
    Then, $\prob{Z = 0} = \prob{X = Y} = 1 - p$, and $\prob{Z = 1} = \prob{X \neq Y} = p$.
    Hence, $H(Z) = H(p)$.
    Applying the previous lemma, $H(X \mid Y) \leq H(X \mid Y, Z) + H(p)$, so it suffices to show $H(X \mid Y, Z) \leq p\log(m-1)$.

    Since $Z = 0$ implies $X = Y$, $H(X \mid Y = y, Z = 0) = 0$.
    There are $m - 1$ remaining possibilities for $X$.
    Hence, $H(X \mid Y = y, Z = 1) \leq \log(m-1)$.
    \begin{align*}
        H(X \mid Y, Z) &= \sum_{y \in \mathcal A} \sum_{z \in \qty{0,1}} \prob{Y = y, Z = z} H(X \mid Y = y, Z = z) \\
        &\leq \sum_{y \in \mathcal A} \prob{Y = y, Z = 1} \log(m-1) \\
        &= \prob{Z = 1} \log (m-1) \\
        &= p\log(m-1)
    \end{align*}
    as required.
\end{proof}
Let $X$ be a r.v. describing the input to a channel and $Y$ be a r.v. describing the output of the channel.
$H(p)$ provides the information required to decide whether an error has occurred, and $p\log(m-1)$ gives the information needed to resolve that error in the worst possible case.

\subsection{Shannon's second coding theorem}
\begin{definition}
    Let $X, Y$ be r.v.s taking values in $\mathcal A$.
    The \vocab{mutual information} is $I(X;Y) = H(X) - H(X \mid Y)$.
\end{definition}
This is nonnegative, as $I(X;Y) = H(X) + H(Y) - H(X,Y) \geq 0$.
Equality holds iff $X, Y$ are independent.
Clearly, $I(X;Y) = I(Y;X)$.
\begin{definition}
    Consider a discrete memoryless channel with input alphabet $\mathcal A$ of size $m$ and output alphabet $\mathcal B$.
    Let $X$ be a r.v. taking values in $\mathcal A$, used as the input to this channel.
    Let $Y$ be the r.v. output by the channel, depending on $X$ and the channel matrix.
    The \vocab{information capacity} of the channel is $\max_{X} I(X;Y)$.
\end{definition}
The maximum is taken over all discrete r.v.s $X$ taking values in $\mathcal A$, or equivalently.
This maximum is attained since $I$ is continuous and the space
\begin{align*}
        \qty{(p_1, \dots, p_m) \in \mathbb R^m \midd p_i \geq 0, \sum_{i=1}^m p_i = 1}
    \end{align*}
is compact.
The information capacity depends only on the channel matrix.
\begin{theorem}
    For a discrete memoryless channel, the (operational) capacity is equal to the information capacity.
\end{theorem}
We prove that the operational capacity is at most the information capacity in general, and we will prove the other inequality for the special case of the binary symmetric channel.
\begin{example}
    Assuming this result holds, we compute the capacity of certain specific channels.
    \begin{enumerate}
        \item Consider the binary symmetric channel with error probability $p$, input $X$, and output $Y$.
        Let $\prob{X = 0} = \alpha, \prob{X = 1} = 1 - \alpha$, so $\prob{Y = 0} = (1-p)\alpha p(1-\alpha), \prob{Y = 1} = (1-p)(1-\alpha) + p\alpha$.
        Then, as $H(Y \mid X) = \prob{X=0} H(p) + \prob{X = 1} H(p)$,
        \begin{align*}
            C &= \max_\alpha I(X;Y) = \max_\alpha [H(Y) - H(Y\mid X)] \\
            &= \max_\alpha \qty[H(\alpha(1-p) + (1-\alpha)p) - H(p)] = 1 - H(p)
       \end{align*}
        with the maximum attained at $\alpha = \frac{1}{2}$.
        Hence, the capacity of the binary symmetric channel is $C = 1 + p \log p + (1-p) \log (1-p)$.
        If $p = 0$ or $p = 1$, $C = 1$.
        If $p = \frac{1}{2}$, $C = 0$.
        Note that $I(X;Y) = I(Y;X)$; we can choose which to calculate for convenience.
        \item Consider the binary erasure channel with erasure probability $p$, input $X$, and output $Y$.
        Let $\prob{X = 0} = \alpha, \prob{X = 1} = 1 - \alpha$, so $\prob{Y = 0} = (1-p)\alpha, \prob{Y = 1} = (1-p)(1-\alpha), \prob{Y = \star} = p$.
        We obtain
        \begin{align*}
        H(X \mid Y = 0) = 0;\quad H(X \mid Y = 1) = 0;\quad H(X \mid Y = \star) = H(\alpha)
    \end{align*}
        Therefore, $H(X \mid Y) = pH(\alpha)$, giving
        \begin{align*}
            C &= \max_\alpha I(X;Y) = \max_\alpha \qty[H(X) - H(X \mid Y)] \\
            &= \max_\alpha \qty[H(\alpha) - pH(\alpha)] = (1-p) \max_\alpha H(\alpha) = 1-p
       \end{align*}
        with maximum attained at $\alpha = \frac{1}{2}$.
    \end{enumerate}
\end{example}
We will now model using a channel $n$ times as the \vocab{$n$th extension}, replacing $\mathcal A$ with $\mathcal A^n$ and $\mathcal B$ with $\mathcal B^n$, and use the channel matrix defined by
\begin{align*}
        \prob{y_1 \dots y_n \text{ received} \mid x_1 \dots x_n \text{ sent}} = \prod_{i=1}^n \prob{y_i \mid x_i}
    \end{align*}
\begin{lemma}
    Consider a discrete memoryless channel with information capacity $C$.
    Then, its $n$th extension has information capacity $nC$.
\end{lemma}
\begin{proof}
    Let $X_1, \dots, X_n$ be the input producing an output $Y_1, \dots, Y_n$.
    Since the channel is memoryless,
    \begin{align*}
        H(Y_1, \dots, Y_n \mid X_1, \dots, X_n) = \sum_{i=1}^n H(Y_i \mid X_1, \dots, X_n) = \sum_{i=1}^n H(Y_i \mid X_i)
    \end{align*}
    Therefore,
    \begin{align*}
        I(X_1, \dots, X_n; Y_1, \dots, Y_n) &= H(Y_1, \dots, Y_n) - H(Y_1, \dots, Y_n \mid X_1, \dots, X_n) \\
        &\leq \sum_{i=1}^n H(Y_i) - \sum_{i=1}^n H(Y_i \mid X_i) \\
        &= \sum_{i=1}^n \qty[H(Y_i) - H(Y_i \mid X_i)] \\
        &= \sum_{i=1}^n I(X_i;Y_i) \leq nC
    \end{align*}
    Equality is attained by taking $X_1, \dots, X_n$ iid s.t. $I(X_i;Y_i) = C$.
    Indeed, if $X_1, \dots, X_n$ are independent, then so are $Y_1, \dots, Y_n$, so $H(Y_1, \dots, Y_n) = \sum_{i=1}^n H(Y_i)$.
    Therefore,
    \begin{align*}
        \max_{X_1, \dots, X_n} I(X_1, \dots, X_n; Y_1, \dots, Y_n) = nC
    \end{align*}
    as required.
\end{proof}
We now prove part of Shannon's second coding theorem, that the operational capacity is at most the information capacity for a discrete memoryless channel.
\begin{proof}
    Let $C$ be the information capacity.
    Suppose reliable transmission is possible at a rate $R > C$.
    Then, there is a sequence of codes $(C_n)_{n \geq 1}$ where $C_n$ has length $n$ and size $\floor*{2^{nR}}$, s.t. $\lim_{n \to \infty} \rho(C_n) = R$ and $\lim_{n \to \infty} \hat e(C_n) = 0$.

    Recall that $\hat e(C_n) = \max_{c \in C_n} \prob{\text{error} \mid c \text{ sent}}$.
    Define the \vocab{average error rate} $e(C)$ by $e(C) = \frac{1}{\abs{C_n}} \sum_{c \in C} \prob{\text{error} \mid c \text{ sent}}$.
    Note that $e(C_n) \leq \hat e(C_n)$.
    As $\hat e(C_n) \to 0$, we also have $e(C_n) \to 0$.

    Consider an input r.v. $X$ distributed uniformly over $C_n$.
    Let $Y$ be the output given by $X$ and the channel matrix.
    Then $e(C_n) = \prob{X \neq Y} = p$.
    Hence, $H(X) = \log \abs{C_n} = \log \floor*{2^{nR}} \geq nR - 1$ for sufficiently large $n$.
    Also, by Fano's inequality, $H(X \mid Y) \leq H(p) + p \log(\abs{C_n} - 1) \leq 1 + pnR$.

    Recall that $I(X;Y) = H(X) - H(X \mid Y)$.
    By the previous lemma, $nC \geq I(X;Y)$, so
    \begin{align*}
        nC \geq nR - 1 - 1 - pnR \implies pnR \geq n(R - c) - 2 \implies p \geq \frac{n(R - C) - 2}{nR}
    \end{align*}
    As $n \to \infty$, the right hand side converges to $\frac{R - C}{R} > 0$.
    This contradicts the fact that $p = e(C_n) \to 0$.
    Hence, we cannot transmit reliably at any rate which exceeds $C$, hence the capacity is at most $C$.
\end{proof}
To complete the proof of Shannon's second coding theorem for the binary symmetric channel with error probability $p$, we prove that the operational capacity is at least $1 - H(p)$.
\begin{proposition}
    Consider a binary symmetric channel with error probability $p$, and let $R < 1 - H(p)$.
    Then there exists a sequence of codes $(C_n)_{n \geq 1}$ with $C_n$ of length $n$ and size $\floor*{2^{nR}}$ s.t. $\lim_{n \to \infty} \rho(C_n) = R$ and $\lim_{n \to \infty} e(C_n) = 0$.
\end{proposition}
\begin{remark}
    This proposition deals with the average error rate, instead of the error rate $\hat e$.
\end{remark}
\begin{proof}
    We use the method of random coding.
    Without loss of generality let $p < \frac{1}{2}$.
    Let $\varepsilon > 0$ s.t. $p + \varepsilon < \frac{1}{2}$ and $R < 1 - H(p + \varepsilon)$.
    We use minimum distance decoding, and in the case of a tie, we make an arbitrary choice.
    Let $m = \floor*{2^{nR}}$, and let $C = \qty{c_1, \dots, c_m}$ be a code chosen uniformly at random from $\mathcal C = \qty{[n,m]\text{-codes}}$, a set of size $\binom{2^n}{m}$.

    Choose $1 \leq i \leq m$ uniformly at random, and send $c_i$ through the channel, and obtain an output $Y$.
    Then, $\prob{Y \text{ not decoded as } c_i}$ is the average value of $e(C)$ for $C$ ranging over $\mathcal C$, giving $\frac{1}{\abs{\mathcal C}} \sum_{C \in \mathcal C} e(C)$.
    We can choose a code $C_n \in \mathcal C$ s.t. $e(C_n) \leq \frac{1}{\abs{\mathcal C}}\sum_{C \in \mathcal C} e(C)$.
    So it suffices to show $\prob{Y \text{ not decoded as } c_i} \to 0$.

    Let $r = \floor*{n(p + \varepsilon)}$.
    Then if $B(Y,r) \cap C = \qty{c_i}$, $Y$ is correctly decoded as $c_i$.
    Therefore,
    \begin{align*}
        \prob{Y \text{ not decoded as } c_i} \leq \prob{c_i \not\in B(Y,r)} + \prob{B(Y,r) \cap C \supsetneq \qty{c_i}}
    \end{align*}
    We consider the two cases separately.

    In the first case with $d(c_i,Y) > r$, $\prob{d(c_i,Y) > r}$ is the probability that the channel makes more than $r$ errors, and hence more than $n(p + \varepsilon)$ errors.
    We have already shown that this converges to zero as $n \to \infty$.

    In the second case with $d(c_i,Y) \leq r$, if $j \neq i$,
    \begin{align*}
        \prob{c_j \in B(Y,r) \mid c_i \in B(Y,r)} = \frac{V(n,r) - 1}{2^n - 1} \leq \frac{V(n,r)}{2^n}
    \end{align*}
    Therefore,
    \begin{align*}
        \prob{B(Y,r) \cap C \supsetneq \qty{c_i}} &\leq \sum_{j \neq i} \prob{c_j \in B(Y,r), c_i \in B(Y,r)} \\
        &\leq \sum_{j \neq i} \prob{c_j \in B(Y,r) \mid c_i \in B(Y,r)} \\
        &\leq (m-1) \frac{V(n,r)}{2^n} \\
        &\leq \frac{mV(n,r)}{2^n} \\
        &\leq 2^{nR} 2^{nH(p+\varepsilon)} 2^{-n} \\
        &= 2^{n(R - (1 - H(p+\varepsilon)))} \to 0
    \end{align*}
    as required.
\end{proof}
\begin{proposition}
    We can replace $e$ with $\hat e$ in the previous result.
\end{proposition}
\begin{proof}
    Let $R'$ be s.t. $R < R' < 1 - H(p)$.
    Then, apply the previous result to $R'$ to construct a sequence of codes $(C_n')_{n \geq 1}$ of length $n$ and size $\floor*{2^{nR'}}$, where $e(C_n') \to 0$.
    Order the codewords of $C_n'$ by the probability of error given that the codeword was sent, and delete the worst half.
    This gives a code $C_n$ with $\hat e(C_n) \leq 2 e(C_n')$.
    Hence $\hat e(C_n) \to 0$ as $n \to \infty$.
    Since $C_n$ has length $n$, and size $\frac{1}{2} \floor*{2^{nR'}} = \floor*{2^{nR' - 1}}$.
    But $2^{nR' - 1} = 2^{n(R' - \frac{1}{n})} \geq 2^{nR}$ for sufficiently large $n$.
    So we can replace $C_n'$ with a code of smaller size $\floor*{2^{nR}}$ and still have $\hat e(C_n) \to 0$ and $\rho(C_n) \to R$ as $n \to \infty$.
\end{proof}
Therefore, a binary symmetric channel with error probability $p$ has operational capacity $1 - H(p)$, as we can transmit reliably at any rate $R < 1 - H(p)$, and the capacity is at most $1 - H(p)$.
The result shows that codes with certain properties exist, but does not give a way to construct them.

\subsection{The Kelly criterion}
Let $0 < p < 1$, $u > 0$, $0 \leq w < 1$.
Suppose that a coin is tossed $n$ times in succession with probability $p$ of obtaining a head.
If a stake of $k$ is paid ahead of a particular throw, the return is $ku$ if the result is a head, and the return is zero if the result is a tail.

Suppose the initial bankroll is $X_0 = 1$.
After $n$ throws, the bankroll is $X_n$.
We bet $w X_n$ on the $(n + 1)$th coin toss, retaining $(1-w)X_n$.
The bankroll after the toss is
\begin{align*}
        X_{n+1} = \begin{cases}
    X_n(wu + (1-w)) & (n + 1)\text{th toss is a head} \\
    X_n(1-w) & (n + 1)\text{th toss is a tail}
\end{cases}
    \end{align*}
Define $Y_{n+1} = \frac{X_{n+1}}{X_n}$, then the $Y_i$ are iid.
Then $\log Y_i$ is a sequence of iid r.v.s.
Note that $\log X_n = \sum_{i=1}^n \log Y_i$.
\begin{lemma}
    Let $\mu = \expect{\log Y_1}, \sigma^2 = \Var{\log Y_1}$.
    Then, if $a > 0$,
    \begin{enumerate}
        \item $\prob{\abs{\frac{1}{n} \sum_{i=1}^n \log Y_i - \mu} \geq a} \leq \frac{\sigma^2}{na^2}$ by Chebyshev's inequality;
        \item $\prob{\abs{\frac{\log X_n}{n} - \mu} \geq a} \leq \frac{\sigma^2}{na^2}$;
        \item given $\varepsilon > 0$ and $\delta > 0$, there exists $N$ s.t. $\prob{\abs{\frac{\log X_n}{n} - \mu} \geq \delta} \leq \varepsilon$ for all $n \geq N$.
    \end{enumerate}
\end{lemma}
Consider a single coin toss, with probability $p < 1$ of a head.
Suppose that a bet of $k$ on a head gives a payout of $ku$ for some payout ratio $u > 0$.
Suppose further that we have an initial bankroll of 1, and we bet $w$ on heads, retaining $1 - w$, for some $0 \leq w < 1$.
Then, if $Y$ is the expected fortune after the throw, $\expect{\log Y} = p \log(1 + (u-1)w) + (1-p) \log(1-w)$.
One can show that the value of $\expect{\log Y}$ is maximised by taking $w = 0$ if $up \leq 1$, and setting $w = \frac{up-1}{u-1}$ if $up > 1$.

Let $q = 1-p$.
If $up > 1$, at the optimum value of $w$, we find
\begin{align*}
        \expect{\log Y} = p \log p + q \log q + \log u - q \log(u-1) = -H(p) + \log u - q \log(u-1)
    \end{align*}
Kelly's criterion is that in order to maximise profit, $\expect{\log Y}$ should be optimised, given that we can bet arbitrarily many times.

One can show that if $w$ is set below the optimum, the bankroll will still increase, but does so more slowly.
If $w$ is set sufficiently high, the bankroll will tend to decrease.
