\section{Fourier Transforms}

\subsection{Definitions}
\begin{definition}[Fourier transform]
	The \vocab{Fourier transform} of a function $f(x)$ is
	\begin{align} \label{eq:8.1}
		\widetilde f(k) = \mathcal F(f)(k) = \int_{-\infty}^\infty f(x) e^{-ikx} \dd{x}
	\end{align}
	The \vocab{inverse Fourier transform} is
	\begin{align} \label{eq:8.2}
		f(x) = \mathcal F\inv\qty(\widetilde f)(x) = \frac{1}{2\pi} \int_{-\infty}^\infty \widetilde f(k) e^{ikx} \dd{k}
	\end{align}
	Different internally-consistent definitions exist, which distribute the multiplicative constants in different ways.
\end{definition}

\begin{theorem}[Fourier inversion theorem]
	For a function $f(x)$,
	\begin{align} \label{eq:8.3}
		\mathcal F\inv (\mathcal F (f))(x) = f(x)
	\end{align}
	with a sufficient condition that $f$ and $\widetilde f$ are \underline{absolutely integrable}, so
	\begin{align*}
		\int_{-\infty}^\infty \abs{f(x)} \dd{x} = M < \infty.
	\end{align*}
	In particular, $f \to 0$ as $x \to \pm \infty$.
\end{theorem}

\begin{example}
	Consider the Gaussian,
	\begin{align} \label{eq:8.4}
		f(x) = \frac{1}{\sigma \sqrt{\pi}} \exp[-\frac{x^2}{\sigma^2}]
	\end{align}
	We wish to compute its Fourier transform.
	Since $i \sin kx$ is an odd function,
	\begin{align*}
		\widetilde f(k) = \frac{1}{\sigma \sqrt{\pi}} \int_{-\infty}^\infty \exp[-\frac{x^2}{\sigma^2}] \exp[-ikx] \dd{x} = \frac{1}{\sigma \sqrt{\pi}} \int_{-\infty}^\infty \exp[-\frac{x^2}{\sigma^2}] \cos(kx) \dd{x}
	\end{align*}
	Consider, using Leibniz' rule,
	\begin{align*}
		\dv{\widetilde f}{k} = \frac{-1}{\sigma \sqrt{\pi}} \int_{-\infty}^\infty x\exp[\frac{-x^2}{\sigma^2}] \sin kx \dd{x}
	\end{align*}
	Integrating by parts,
	\begin{align*}
		\dv{\widetilde f}{k} & = \frac{1}{\sigma \sqrt{\pi}} \underbracket{\qty[\frac{\sigma^2}{2} \exp[\frac{-x^2}{\sigma^2}] \sin kx]_{-\infty}^\infty}_0 - \frac{1}{\sigma \sqrt{\pi}} \int_{-\infty}^\infty \frac{k\sigma^2}{2} \exp[\frac{-x^2}{\sigma^2}] \cos kx \dd{x} \\
        & = -\frac{k\sigma^2}{2} \widetilde f(k)
	\end{align*}
	This is a differential equation for $\widetilde f$, which gives
	\begin{align*}
		\widetilde f(k) = C \exp[-\frac{k^2\sigma^2}{4}]
	\end{align*}
	Suppose $k = 0$.
	Then, in the original expression for the Fourier transform, we can directly find $\widetilde f(0) = 1$.
	Hence $C \exp[-\frac{0^2\sigma^2}{4}] = 1 \implies C = 1$.
	Hence,
	\begin{align} \label{eq:8.5}
		\widetilde f(k) = \exp[-\frac{k^2\sigma^2}{4}]
	\end{align}
	which is another Gaussian with the width parameter inverted.
\end{example}

\begin{exercise}
    Show that $\mathcal{F}\inv(e^{-k^2 \sigma^2 / 4}) = f(x)$ (try completing the square).
\end{exercise} 

\begin{exercise}
    Show that $f(x) = e^{-a |x|}$, $a > 0$, has FT 
    \begin{align} \label{eq:8.6}
        \widetilde{f} = \frac{2a}{a^2 + k^2}
    \end{align} in two ways.
    \begin{enumerate}
        \item Integrate $2 \int_{0}^{\infty} e^{-ax} \cos kx \dd{x}$ by parts twice.
        \item Integrate $\int_{0}^{\infty} e^{-(a - ik) x} \dd{x} + \int_{-\infty}^{0} e^{(a + ik)x} \dd{x}$ directly.
    \end{enumerate} 
    Note that if $f(x) = \begin{cases}
        e^{-ax} & x > 0 \\
        0 & x \leq 0
    \end{cases}$ ($a > 0$) then 
    \addtocounter{equation}{-1}
    \begin{subequations}
        \begin{align} \label{eq:8.6a}
            \widetilde{f}(k) = \frac{1}{ik + a}
        \end{align} 
    \end{subequations}
\end{exercise} 

\subsection{Converting Fourier series into Fourier transforms}
Recall that the complex form of the Fourier series, \cref{eq:1.13}, is
\begin{align*}
	f(x) = \sum_{n=-\infty}^\infty c_n e^{ik_n x}
\end{align*}
where $k_n = \frac{n\pi}{L}$.
We can write in particular $k_n = n \Delta k$ where $\Delta k = \frac{\pi}{L}$.
Then,
\begin{align*}
	c_n = \frac{1}{2L} \int_{-L}^L f(x) e^{-ik_n x} \dd{x} = \frac{\Delta k}{2\pi} \int_{-L}^L f(x) e^{-ik_n x}\dd{x}
\end{align*}
Now, re-substituting into the Fourier series,
\begin{align*}
	f(x) = \sum_{n=-\infty}^\infty \frac{\Delta k}{2\pi} e^{i k_n x} \int_{-L}^L f(x') e^{-ik_n x'} \dd{x'}
\end{align*}
But interpreting the sum multiplied by $\Delta k$ as a Riemann integral,
\addtocounter{equation}{-1}
\begin{subequations}
	\addtocounter{equation}{1}
	\begin{align} \label{eq:8.6b}
		\sum_{n=-\infty}^{\infty} \Delta k g(k_n) \to \int_{-\infty}^{\infty} g(k) \dd{k}
	\end{align} 
\end{subequations}
So,
\begin{align*}
	f(x) \to \int_{-\infty}^\infty \frac{1}{2\pi} e^{i k_n x} \int_{-L}^L f(x') e^{-ik x'} \dd{x'} \dd{k}
\end{align*}
Taking the limit $L \to \infty$,
\begin{align*}
	f(x) = \frac{1}{2\pi} \int_{-\infty}^\infty \dd{k} e^{i k x} \int_{-\infty}^\infty \dd{x'} f(x') e^{-ik_n x'}
\end{align*}
which is the inverse Fourier transform of the Fourier transform of $f$, which gives the Fourier inversion theorem.
Note that when $f(x)$ is discontinuous at $x$, the Fourier transform gives
\begin{align} \label{eq:8.7}
	\mathcal F\inv(\mathcal F(f))(x) = \frac{1}{2}(f(x_-) + f(x_+))
\end{align}
which is analogous to the result for Fourier series.

\subsection{Properties of Fourier series}
Recall the definition of the Fourier transform.
\begin{align*}
	\widetilde f(k) = \int_{-\infty}^\infty f(x) e^{-ikx} \dd{x}
\end{align*}
\begin{proposition}[Linearity]
	The (inverse) Fourier transform is linear.
	\begin{align} \label{eq:8.8}
		h(x) = \lambda f(x) + \mu g(x) \iff \widetilde h(k) = \lambda \widetilde f(k) + \mu \widetilde g(k)
	\end{align}
\end{proposition} 

\begin{proposition}[Translation]
	Translated functions transform to multiplicative factors.
	\begin{align} \label{eq:8.9}
		h(x) = f(x - \lambda) \iff \widetilde h(k) = e^{-i\lambda k} \widetilde f(k)
	\end{align}
\end{proposition} 

\begin{proof}
	This is because
	\begin{align*}
		\widetilde h(k) = \int f(x - \lambda) e^{-ikx} \dd{x} = \int f(y) e^{-ik(y + \lambda)} \dd{y} = e^{-i\lambda k} \widetilde f(k)
	\end{align*}
\end{proof} 

\begin{proposition}[Frequency Shift]
	Frequency shifts transform to translations in frequency space.
	\begin{align} \label{eq:8.10}
		h(x) = e^{i\lambda x}f(x) \implies \widetilde h(k) = \widetilde f(k - \lambda)
	\end{align}
\end{proposition} 

\begin{proposition}[Scaling]
	A scalar multiple applied to the argument transforms into an inverse scalar multiple.
	\begin{align} \label{eq:8.11}
		h(x) = f(\lambda x) \iff \widetilde h(k) = \frac{1}{\abs{\lambda}} \widetilde f\qty(\frac{k}{\lambda})
	\end{align}
\end{proposition} 

\begin{proposition}[Multiplication by $x$]
	Multiplication by $x$ transforms into an imaginary derivative.
	\begin{align} \label{eq:8.12}
		h(x) = xf(x) \iff \widetilde h(k) = i\widetilde f'(k)
	\end{align}
\end{proposition} 

\begin{proof}
	This is because
	\begin{align*}
		\int_{-\infty}^\infty f(x) e^{-ikx} \dd{x} = \frac{-1}{i} \dv{k} \int_{-\infty}^\infty f(x) e^{-ikx} \dd{x}
	\end{align*}
\end{proof} 

\begin{proposition}[Derivatives]
	Derivatives transform into a multiplication by $ik$.
	\begin{align} \label{eq:8.13}
		h(x) = f'(x) \iff \widetilde h(k) = ik \widetilde f(k)
	\end{align}
\end{proposition} 

\begin{proof}
	This is because we can integrate by parts and find
	\begin{align*}
		\widetilde h(k) = \int_{-\infty}^\infty f'(x) e^{-ikx} \dd{x} = \underbrace{\qty[f(x) e^{-ikx}]_{-\infty}^\infty}_{=0} + ik\int_{-\infty}^\infty f(x) e^{-ikx} \dd{x}
	\end{align*}
\end{proof} 

\begin{proposition}[General duality] ~\vspace*{-1.5\baselineskip}
	\begin{align} \label{eq:8.14}
		g(x) = \widetilde{f}(x) \iff \widetilde{g}(k) = 2 \pi f(-k)
	\end{align} 
\end{proposition} 

\begin{proof}
	Consider \cref{eq:8.2} with mapping $x \mapsto -x$, we get
	\begin{align*}
		f(-x) = \frac{1}{2\pi} \int_{-\infty}^\infty \widetilde f(k) e^{-ikx} \dd{k}.
	\end{align*}
	Now swap $k$ and $x$, treating $\widetilde f$ now as a function in position space
	\begin{align*}
		f(-k) = \frac{1}{2\pi} \int_{-\infty}^\infty \widetilde f(x) e^{-ikx} \dd{x}.
	\end{align*}
	Thus
	\begin{align*}
		g(x) = \widetilde f(x) \iff \widetilde g(k) = 2\pi f(-k)
	\end{align*}
\end{proof} 

\begin{corollary} ~\vspace*{-1.5\baselineskip}
	\begin{align*}
		f(-x) = \frac{1}{2\pi} \mathcal F(\mathcal F(f))(x)
	\end{align*}
	Finally,
	\begin{align*}
		\mathcal F^4(f)(x) = 4\pi^2 f(x)
	\end{align*}
\end{corollary} 

\begin{exercise}
	Verify these properties.
\end{exercise} 

\begin{example}
	Consider a function defined by
	\begin{align*}
		f(x) = \begin{cases}
			1 & \abs{x} \leq a   \\
			0 & \text{otherwise}
		\end{cases}
	\end{align*}
	for some $a > 0$.
	By the definition of the Fourier transform,
	\begin{align} \label{eq:8.15}
		\widetilde f(k) = \int_{-\infty}^\infty f(x) e^{-ikx} \dd{x} = \int_{-a}^a e^{-ikx} \dd{x} = \int_{-a}^a \cos kx \dd{x} = \frac{2}{k} \sin ka
	\end{align}
	By the Fourier inversion theorem,
	\begin{align*}
		\frac{1}{\pi} \int_{-\infty}^\infty e^{ikx} \frac{1}{k} \sin ka \dd{k} = f(x)
	\end{align*}
	for $x \neq a$. \\
	Now, in this expression, let $x = 0$ and let $k \mapsto x$.
	We arrive at the \underline{Dirichlet discontinuous formula}.
	\begin{align} \label{eq:8.16}
		\int_0^\infty \frac{\sin ax}{x} \dd{x} = \frac{\pi}{2} \sgn a = \begin{cases}
			\frac{\pi}{2}  & a > 0 \\
			0              & a = 0 \\
			-\frac{\pi}{2} & a < 0
		\end{cases}
	\end{align}
	Here, we allow $a < 0$, so $\sin(-ax) = -\sin ax$.
\end{example}

\subsection{Convolution theorem}
We want to multiply Fourier transforms in the frequency domain (transformed space).
This is useful for filtering or processing signals.
\begin{align*}
	\widetilde h(k) = \widetilde f(k) \widetilde g(k)
\end{align*}
Consider the inverse.
\begin{align}
	h(x) &= \frac{1}{2\pi} \int_{-\infty}^\infty \widetilde f(k) \widetilde g(k) e^{ikx} \dd{k} \notag \\
	&= \frac{1}{2\pi} \int_{-\infty}^\infty \qty(\int_{-\infty}^\infty f(y) e^{-iky} \dd{y}) \widetilde g(k) e^{ikx} \dd{k} \notag \\
	&= \int_{-\infty}^\infty f(y) \qty( \frac{1}{2\pi} \int_{-\infty}^\infty e^{-iky} \widetilde g(k) e^{ikx} \dd{k} ) \dd{y} \notag \\
	&= \int_{-\infty}^\infty f(y) \qty( \frac{1}{2\pi} \int_{-\infty}^\infty \widetilde g(k) e^{ik(x-y)} \dd{k} ) \dd{y} \notag \\
	&= \int_{-\infty}^\infty f(y) g(x-y) \dd{y} \text{ by \cref{eq:8.9}} \notag \\
	&\equiv (f \ast g)(x) \label{eq:8.17}
\end{align}
where $f \ast g$ is called the \textit{convolution} of $f$ and $g$.
By duality \cref{eq:8.14}, we also have
\begin{align} \label{eq:8.18}
	h(x) = f(x) g(x) \implies \widetilde h(k) = \frac{1}{2\pi} \int_{-\infty}^\infty \widetilde f(p) \widetilde g(k-p) \dd{p} = \frac{1}{2\pi}\qty(\widetilde f \ast \widetilde g)(k)
\end{align}

\subsection{Parseval's theorem}
Consider $h(x) = g^\star(-x)$.
\begin{align*}
	\widetilde h(k) & = \int_{-\infty}^\infty g^\star(-x) e^{-ikx} \dd{x} \\
	& = \qty[\int_{-\infty}^\infty g(-x) e^{ikx} \dd{x}]^\star
	\intertext{Let $-x \mapsto y$}
	& = \qty[\int_{-\infty}^\infty g(y) e^{-iky} \dd{y}]^\star \\
	& = \widetilde g^\star(k)
\end{align*}
Substituting this into the convolution theorem \cref{eq:8.17}, with $g(x) \mapsto g^\star(-x)$, we have (RHS is the inverse Fourier transform)
\begin{align*}
	\int_{-\infty}^\infty f(y) g^\star(y-x) \dd{y} = \frac{1}{2\pi} \int_{-\infty}^\infty \widetilde f(k) \widetilde g^\star(k) e^{ikx} \dd{x}
\end{align*}
Taking $x = 0$ in this expression and mapping $y \mapsto x$, we find
\begin{align} \label{eq:8.19}
	\int_{-\infty}^\infty f(x) g^\star(x) \dd{x} = \frac{1}{2\pi} \int_{-\infty}^\infty \widetilde f(k) \widetilde g^\star(k) \dd{x}
\end{align}
Equivalently,
\begin{align} \label{eq:8.20}
	\inner{g,f} = \frac{1}{2\pi} \inner{\widetilde g, \widetilde f}
\end{align}
So the inner product is conserved under the Fourier transform (up to a factor of $2 \pi$).
Now, by setting $g^\star = f^\star$, we have
\begin{align*}
	\int_{-\infty}^\infty \abs{f(x)}^2 \dd{x} = \frac{1}{2\pi} \int_{-\infty}^\infty \abs{\widetilde f(k)}^2 \dd{k}
\end{align*}
This is Parseval's theorem.

\subsection{Fourier transforms of generalised functions}
We can apply Fourier transforms to generalised functions by considering limiting distributions.
Consider the inversion
\begin{align*}
	f(x) & = \mathcal F\inv(\mathcal F(f))(x) \\
	     & = \frac{1}{2\pi} \int_{-\infty}^\infty \qty[\int_{-\infty}^\infty f(u) e^{-iku} \dd{u}] e^{ikx} \dd{k} \\
	     & = \int_{-\infty}^\infty f(u) \underbrace{\qty[\frac{1}{2\pi} \int_{-\infty}^\infty e^{-ik(x-u)} \dd{k}]}_{\delta(x-u)} \dd{u}
\end{align*}
In order to reconstruct $f(x)$ on the right hand side for any function $f$, we must have that the bracketed term is $\delta(x-u)$.
So we identify
\begin{align*}
	\delta(x-u) = \frac{1}{2\pi} \int_{-\infty}^\infty e^{ik(x-u)} \dd{k}
\end{align*}
\begin{itemize}
	\item If $f(x) = \delta(x)$,
	\begin{align} \label{eq:8.21}
		\widetilde f(k) = \int_{-\infty}^\infty \delta(x) e^{ikx} \dd{x} = 1
	\end{align}
	This can be thought of as the Fourier transform of an infinitely thin Gaussian, which becomes an infinitely wide Gaussian (a constant).
	\item If $f(x) = 1$, then
	\begin{align} \label{eq:8.22}
		\widetilde f(k) = \int_{-\infty}^\infty e^{-ikx}\dd{x} = 2\pi \delta(k)
	\end{align}
	This can also be found by the duality formula \cref{eq:8.14}.
	\item If $f(x) = \delta(x - a)$, using \cref{eq:8.9} we have
	\begin{align} \label{eq:8.23}
		\widetilde f(k) = e^{-ika}
	\end{align}
	This is a translation of the original Fourier transform for the $\delta$ function above.
\end{itemize} 

\subsection{Trigonometric functions}
Let $f(x) = \cos \omega x = \frac{1}{2} \qty(e^{ix} + e^{-ix})$.
Then,
\begin{align} \label{eq:8.24}
	\widetilde f(k) = \pi\qty(\delta(k+\omega) + \delta(k-\omega))
\end{align}
For $f(x) = \sin \omega x$, we have
\begin{align*}
	\widetilde f(k) = i\pi\qty(\delta(k+\omega) - \delta(k-\omega))
\end{align*}
Using duality \cref{eq:8.14},
\begin{align*}
	f(x) & = \frac{1}{2}\qty(\delta(x+a) + \delta(x-a)) \implies \widetilde f(k) = \cos ka  \\
	f(x) & = \frac{1}{2i}\qty(\delta(x+a) - \delta(x-a)) \implies \widetilde f(k) = \sin ka
\end{align*}

\subsection{Heaviside functions}
Let $H(x)$ be the Heaviside function, such that $H(0) = \frac{1}{2}$.
Then, $H(x) + H(-x) = 1$ for all $x$ and is cts at $x = 0$.
We can take the Fourier transform of this and find by \cref{eq:8.22}
\begin{align*}
	\widetilde H(k) + \widetilde H(-k) = 2\pi \delta(k) \tag{$\ast$}
\end{align*}
Recall that $H'(x) = \delta(x)$, \cref{eq:6.7}.
Thus by \cref{eq:8.13,eq:8.21},
\begin{align*}
	ik \widetilde H(x) = \widetilde \delta(k) = 1 \tag{$\dagger$}
\end{align*}
Since $k \delta(k) = 0$, the two equations for $\widetilde H$ can be consistent if we take
\begin{align} \label{eq:8.25}
	\widetilde H(k) = \pi\delta(k) + \frac{1}{ik}
\end{align}

\subsection{Dirichlet discontinuous formula}
Recall the Dirichlet discontinuous formula \cref{eq:8.16}:
\begin{align*}
	\int_0^\infty \frac{\sin ax}{x} \dd{x} = \frac{\pi}{2} \sgn a = \begin{cases}
		\frac{\pi}{2}  & a > 0 \\
		0              & a = 0 \\
		-\frac{\pi}{2} & a < 0
	\end{cases}
\end{align*}
We can rewrite this as
\begin{align*}
	\frac{1}{2} \sgn x = \frac{1}{2\pi} \int_{-\infty}^\infty \frac{e^{ikx}}{ik} \dd{k}
\end{align*}
since the cosine term divided by $ik$ is odd.
Hence,
\begin{align} \label{eq:8.26}
	f(x) = \frac{1}{2} \sgn x \iff \widetilde f(k) = \frac{1}{ik}
\end{align}
This is the preferred form for a Heaviside-type function when used in Fourier transforms.

\subsection{Solving ODEs for boundary value problems}
Consider $y'' - y = f(x)$ with homogeneous boundary conditions $y \to 0$ as $x \to \pm \infty$.
Taking the Fourier transform of this expression, we find by \cref{eq:8.13}
\begin{align*}
	(-k^2 - 1) \widetilde y = \widetilde f
\end{align*}
Thus, the solution is
\begin{align*}
	\widetilde y(k) = \frac{-\widetilde f(k)}{1+k^2} \equiv \widetilde f(k) \widetilde g(k)
\end{align*}
where $\widetilde g(k) = \frac{-1}{1 + k^2}$.
Note that $\widetilde g(k)$ is the Fourier transform of $g(x) = -\frac{1}{2} e^{-\abs{x}}$, \cref{eq:8.6}.
Applying the convolution theorem \cref{eq:8.17},
\begin{align*}
	y(x) &= \int_{-\infty}^\infty f(u) g(x-u) \dd{u} \\
	     &= -\frac{1}{2} \int_{-\infty}^\infty f(u) e^{-\abs{x-u}}\dd{u} \\
	     &= -\frac{1}{2} \qty[ \int_{-\infty}^x f(u) e^{u-x}\dd{u} + \int_x^\infty f(u) e^{x-u}\dd{u} ]
\end{align*}
This is in the form of a boundary value problem Green's function \cref{eq:7.20}.
We can construct the same results by constructing the Green's function directly or by using inverse fourier transform on $\widetilde y(k)$.

\subsection{Signal processing}
Suppose we have an input signal $\mathcal I(t)$, which is acted on by some linear operator $\mathcal L_{\text{in}}$ to yield an output $\mathcal O(t)$.
The Fourier transform of the input $\widetilde{\mathcal I}(\omega)$ is called the \vocab{resolution}.
\begin{align} \label{eq:8.27}
	\widetilde{\mathcal I}(\omega) = \int_{-\infty}^\infty \mathcal I(t) e^{-i\omega t} \dd{t}
\end{align}
In the frequency domain, the action of $\mathcal L_{\text{in}}$ on $\mathcal I(t)$ means that $\widetilde{\mathcal I}(\omega)$ is multiplied by a \vocab{transfer function} $\widetilde{\mathcal R}(\omega)$ to yield outupt,
\begin{align} \label{eq:8.28}
	\mathcal O(t) = \frac{1}{2\pi} \int_{-\infty}^\infty \widetilde{\mathcal R}(\omega) \widetilde{\mathcal I}(\omega) e^{i\omega t} \dd{\omega}
\end{align}
The inverse Fourier transform of the transfer function, $\mathcal R$, is called the \vocab{response function}, which is given by
\begin{align} \label{eq:8.29}
	\mathcal R(t) = \frac{1}{2\pi} \int_{-\infty}^\infty \widetilde{\mathcal R}(\omega) e^{i \omega t}\dd{\omega}
\end{align}
By the convolution theorem,
\begin{align*}
	\mathcal O(t) = \int_{-\infty}^\infty \mathcal I(u) \mathcal R(t-u) \dd{u}
\end{align*}
Suppose there is no input ($\mathcal I(t) = 0$) for $t < 0$.
By causality, there should be zero output for the response function ($\mathcal R(t) = 0$) for $t < 0$.
Therefore, we require $0 < u < t$ and hence
\begin{align} \label{eq:8.30}
	\mathcal O(t) = \int_0^t \mathcal I(u) \mathcal R(t-u) \dd{u}
\end{align}
which resembles an initial value problem Green's function \cref{eq:7.26}.

\subsection{General transfer functions for ODEs}
Suppose an input-output relationship is given by a linear ODE (nth order).
\begin{align} \label{eq:8.31}
	\mathcal L \mathcal O(t) \equiv \qty(\sum_{i=0}^n a_i \dv[i]{x}) \mathcal O(t) \equiv \mathcal I(t)
\end{align}
Here, $\mathcal L_{\text{in}} = 1$.
We want to solve this ODE using a Fourier transform.
\begin{align*}
	(a_0 + a_1 i\omega - a_2 \omega^2 - a_3 i\omega^3 + \dots + a_n (i \omega)^n) \widetilde{\mathcal O}(\omega) = \widetilde{\mathcal I}(\omega)
\end{align*}
We can solve this algebraically in Fourier transform space.
The transfer function is
\begin{align} \label{eq:8.32}
	\widetilde{\mathcal R}(\omega) = \frac{1}{a_0 + \dots + a_n (i \omega)^n}
\end{align}
We factorise the denominator to find partial fractions.
Suppose there are $J$ distinct roots $(i \omega - c_j)^{k_j}$, where $k_j$ is the algebraic multiplicity of the $j$th root, so $\sum_{j=1}^J k_j = n$.
So we can write
\begin{align*}
	\widetilde{\mathcal R}(\omega) = \frac{1}{(i \omega - c_1)^{k_1} \dots (i \omega - c_J)^{k_J}}
\end{align*}
Expressing this as partial fractions,
\begin{align} \label{eq:8.33}
	\widetilde{\mathcal R}(\omega) = \sum_{j=1}^J \sum_{m=1}^{k_i} \frac{\Gamma_{jm}}{(i\omega - c_j)^m}
\end{align}
The $\Gamma_{jm}$ terms are constant.
To solve this, we must find the inverse Fourier transform of $(i\omega - a)^{-m}$.
Recall that \cref{eq:8.6a}
\begin{align*}
	\mathcal F\inv\qty(\frac{1}{i\omega - a}) = \begin{cases}
		e^{at} & t > 0 \\
		0      & t < 0
	\end{cases}
\end{align*}
for $\Re a < 0$.
So we will require $\Re c_j < 0$ for all $j$ to eliminate exponentially growing solutions.
Note that for $m = 2$,
\begin{align*}
	i \dv{\omega} \qty(\frac{1}{i \omega - a}) = \frac{1}{(i \omega - a)^2}
\end{align*}
and recall \cref{eq:8.12}
\begin{align*}
	\mathcal F (t f(t)) = i \mathcal F'(\omega)
\end{align*}
Hence,
\begin{align*}
	\mathcal F\inv\qty(\frac{1}{(i \omega - a)^2}) = \begin{cases}
		t e^{at} & t > 0 \\
		0        & t < 0
	\end{cases}
\end{align*}
Inductively, we arrive at
\begin{align} \label{eq:8.34}
	\mathcal F\inv\qty(\frac{1}{(i \omega - a)^m}) = \begin{cases}
		\frac{t^{m-1}}{(m-1)!} e^{at} & t > 0 \\
		0                             & t < 0
	\end{cases}
\end{align}
We can therefore invert any transfer function to obtain the response function.
Thus the response function takes the form
\begin{align} \label{eq:8.35}
	\mathcal R(t) = \sum_{j=1}^J \sum_{m=1}^{k_i} \Gamma_{jm} \frac{t^{m-1}}{(m-1)!} e^{c_j t},\quad t > 0
\end{align}
and zero for $t < 0$.
We can now solve such differential equations, \cref{eq:8.31}, in Green's function form \cref{eq:8.30}, or directly invert $\widetilde{\mathcal R}(\omega) \widetilde{\mathcal I}(\omega)$ for a polynomial $\widetilde{\mathcal I}(\omega)$.

\subsection{Damped oscillator}
We can use the Fourier transform method to solve the differential equation
\begin{align*}
	\mathcal L y \equiv y'' + 2py' + (p^2 + q^2)y = f(t)
\end{align*}
where $p > 0$.
Consider homogeneous boundary conditions $y(0) = y'(0) = 0$.
The Fourier transform is
\begin{align*}
	(i\omega)^2 \widetilde y + 2 i p \omega \widetilde y + (p^2 + q^2) \widetilde y = \widetilde f
\end{align*}
Hence,
\begin{align*}
	\widetilde y = \frac{\widetilde f}{-\omega^2 + 2ip\omega + p^2 + q^2} \equiv \widetilde R \widetilde f
\end{align*}
We can invert this using the convolution theorem by inverting $\widetilde R$.
\begin{align*}
	y(t) = \int_0^t \mathcal R(t-\tau) f(\tau) \dd{\tau}
\end{align*}
where the response function is
\begin{align*}
	\mathcal R(t - \tau) = \frac{1}{2\pi} \int_{-\infty}^\infty \frac{e^{i\omega(t-\tau)}}{p^2 + q^2 + 2ip\omega - \omega^2} \dd{\omega}
\end{align*}
We can show that $\mathcal L \mathcal R(t-\tau) = \delta(t-\tau)$ using \cref{eq:8.23}; in other words, $\mathcal R$ is the Green's function (Sheet 3, Q4).

\subsection{Discrete sampling and the Nyquist frequency}
Suppose a signal $h(t)$ is sampled at equal times $t_n = n\Delta$ with a time step $\Delta$ and values
\begin{align} \label{eq:8.36}
	h_n = h(t_n) = h(n\Delta),\quad n \in \mathbb{Z}
\end{align}
The sampling frequency is therefore $\Delta\inv$, so the sampling angular velocity is $\omega_s = 2\pi f_s = \frac{2\pi}{\Delta}$.

\begin{definition}[Nyquist Frequency]
	The \vocab{Nyquist frequency} is the highest frequency actually sampled at $\Delta$,
	\begin{align} \label{eq:8.37}
		f_c = \frac{1}{2 \Delta}
	\end{align} 
\end{definition} 

Suppose we have a signal $g_f$ with a given frequency $f$.
We will write
\begin{align} \label{eq:8.38}
	g_f(t) = A \cos(2\pi f t + \phi) = \Re \qty(A e^{2 \pi i f t + \phi}) = \frac{1}{2} \qty(A e^{2 \pi i f t + \phi}) + \frac{1}{2} \qty(A e^{-2 \pi i f t + \phi})
\end{align}
where $A \in \mathbb R$.
Note that this signal has two `frequencies'; a positive and a negative frequency.
The combination of these frequencies gives the full wave.

Suppose we sample $g_f(t)$ at the Nyquist frequency, so $f = f_c$.
Then,
\begin{align}
	g_{f_c}(t_n) & = A \cos(2 \pi \frac{1}{2\Delta} n \Delta + \phi) \notag \\
	& = A \cos(\pi n + \phi) \notag \\
	& = A \cos \pi n \cos \phi + A \sin \pi n \sin \phi \notag \\
	& = A' \cos(2\pi f_c t_n) \label{eq:8.39}
\end{align}
where $A' = A \cos \phi$.
This has removed half of the information about the wave; the amplitude and the phase have become degenerate.
We have lost phase/amplitude information, there is no longer any distinction between them.
We can identify $f_c$ with $-f_c$ when considering the remaining information; we say that the two frequencies are \textit{aliased} together.

Now, suppose we sample at greater than the Nyquist frequency, in particular $f = f_c + \delta f > f_c$, where for simplicity we let $\delta f < f_c$.
As an exercise, show that
\begin{align}
	g_f(t_n) &= A \cos(2\pi (f_c + \delta f)t_n + \phi) \notag \\
	&= A \cos(2\pi (f_c - \delta f)t_n - \phi) \label{eq:8.40}
\end{align}
So frequencies above the Nyquist frequency are reinterpreted after the sampling as a frequency lower than the Nyquist frequency.
This aliases $f_c + \delta f$ with $f_c - \delta f$.

\subsection{Nyquist-Shannon sampling theorem}
\begin{definition}[Bandwith-Limited]
	A signal $g(t)$ is \vocab{bandwidth-limited} if it contains no frequencies above $\omega_{\max} = 2\pi f_{\max}$.
	In other words, $\widetilde g(\omega) = 0$ for all $\abs{\omega} > \omega_{\max}$.
	In this case,
	\begin{align} \label{eq:8.41}
		g(t) = \frac{1}{2\pi} \int_{-\infty}^\infty \widetilde g(\omega) e^{i\omega t} \dd{\omega} = \frac{1}{2\pi} \int_{-\omega_{\max}}^{\omega_{\max}} \widetilde g(\omega) e^{i\omega t} \dd{\omega}
	\end{align}
\end{definition}

Suppose we set the sampling rate to the Nyquist frequency, so $\Delta = \frac{1}{2f_{\max}}$.
Then,
\begin{align*}
	g_n \equiv g(t_n) = \frac{1}{2\pi} \int_{-\omega_{\max}}^{\omega_{\max}} \widetilde g(\omega) e^{i\pi n \omega / \omega_{\max}} \dd{\omega}
\end{align*}
This is a complex Fourier series coefficient \cref{eq:1.13} $c_n$, multiplied by $\frac{\omega_{\max}}{\pi}$.
The Fourier series is periodic in $\omega$ with period $2 \omega_{\max}$, not in space or time.
\begin{align} \label{eq:8.42}
	\widetilde g_\mathrm{per}(\omega) = \frac{\pi}{\omega_{\max}} \sum_{n=-\infty}^\infty g_n e^{-i \pi n \omega / \omega_{\max}}
\end{align}
The actual Fourier transform $\widetilde g$ is found by multiplying by a top hat window function
\begin{align*}
	\widetilde h(\omega) = \begin{cases}
		1 & \abs{\omega} \leq \omega_{\max} \\
		0 & \text{otherwise}
	\end{cases}
\end{align*}
Hence,
\begin{align} \label{eq:8.4f}
	\widetilde g(\omega) = \widetilde g_\mathrm{per}(\omega) \widetilde h(\omega)
\end{align}
Note that this relation is exact.
Inverting this expression,
\begin{align*}
	g(t) & = \frac{1}{2\pi} \int_{-\infty}^\infty \widetilde g_\mathrm{per}(\omega) \widetilde h(\omega) e^{i \omega t} \dd{\omega} \\
	& = \frac{1}{2\omega_{\max}} \sum_{n=-\infty}^\infty g_n \int_{-\omega_{\max}}^{\omega_{\max}} \exp(i \omega\qty(t - \frac{n \pi}{\omega_{\max}})) \dd{\omega}
\end{align*}
Only the cosine term is even, hence
\begin{align} \label{eq:8.44}
	g(t) = \frac{1}{2\omega_{\max}} \sum_{n=-\infty}^\infty g_n \frac{\sin(\omega_{\max} t - \pi n)}{\omega_{\max} t - \pi n}
\end{align}
Hence, $g(t)$ can be written \textit{exactly} as a combination of countably many discrete sample points.

\subsection{Discrete Fourier transform}
Suppose we have a finite number of samples 
\begin{align} \label{eq:8.45}
	h_m = h(t_m) \text{ for } t_m = m \Delta, \text{ where } m = 0,\dots, N-1
\end{align}
We will approximate the Fourier transform for $N$ frequencies within the Nyquist frequency $f_c = \frac{1}{2\Delta}$, using equally-spaced frequencies, given by $\Delta_f = \frac{1}{N\Delta}$ in the range $-f_c \leq f \leq f_c$.
We could take the convention $f_n = n \Delta_f = \frac{n}{N\Delta}$ for $n = -\frac{N}{2}, \dots, \frac{N}{2}$.
However, this overcounts the Nyquist frequency (which is aliased, \cref{eq:8.39}), giving $N + 1$ frequencies instead of the desired $N$.
Since frequencies above the Nyquist frequency are aliased to below it, \cref{eq:8.40}:
\begin{align*}
	\qty(\frac{N}{2} + m) \Delta f = f_c + \delta f \mapsto \qty(\frac{N}{2} - m)\Delta f = -(f_c - \delta f)
\end{align*}
we can instead use the convention $f_n = n \Delta_f = \frac{n}{N\Delta}$ for
\begin{align} \label{eq:8.46}
	n = 0, \dots, N - 1
\end{align} 
This counts the Nyquist frequency only once.

The \vocab{discrete FT} at a frequency $f_n$ becomes
\begin{align}
	\widetilde h(f_n) & = \int_{-\infty}^\infty h(t) e^{-2\pi if_n t} \dd{t} \notag \\
	& \approx \Delta \sum_{m=0}^{N-1} h_m e^{-2\pi i f_n t_m} \notag \\
	& = \Delta \sum_{m=0}^{N-1} h_m e^{-2\pi i m n / N} \notag \\
	& = \Delta \widetilde h_d(f_n) \label{eq:8.47}
\end{align}
where the function $\widetilde h_d(f_n)$ is the discrete Fourier transform.

The matrix 
\begin{align} \label{eq:8.48}
	[\mathrm{DFT}]_{mn} = e^{-2\pi i m n / N},\ m, n = 0, 1, \dots, N-1
\end{align} defines the discrete Fourier transform for the vector $h = \qty{h_m}$.
The discrete Fourier transform is then
\begin{align*}
	\widetilde h_d = [\mathrm{DFT}] h
\end{align*}
By inverting the discrete Fourier transform matrix, we find
\begin{align*}
	h = [\mathrm{DFT}]\inv \widetilde h_d = \frac{1}{N} [\mathrm{DFT}]^\dagger \widetilde h_d
\end{align*}
since the inverse of the discrete Fourier transform matrix is its adjoint.
The matrix is built from roots of unity $\omega = e^{-2\pi i/N}$.
So, for instance, $n = 4$ gives $\omega = e^{-2\pi i/4} = -i$ giving
\begin{align*}
	[\mathrm{DFT}] = \begin{pmatrix}
		1 & 1  & 1  & 1  \\
		1 & -i & -1 & i  \\
		1 & -1 & 1  & -1 \\
		1 & i  & -1 & -i
	\end{pmatrix}
\end{align*}
The inverse discrete Fourier transform is
\begin{align}
	h_m &= h(t_m) \notag \\
	    &= \frac{1}{2\pi} \int_{-\infty}^\infty \widetilde h(\omega) e^{i \omega t_m} \dd{\omega} \notag \\
	    &= \int_{-\infty}^\infty \widetilde h(f) e^{2\pi i f t_m} \dd{f} \notag \\
	    &\approx \frac{1}{\Delta N} \sum_{n=0}^{N-1} \Delta \widetilde h_d(f_n) e^{2\pi i m n / N} \notag \\
	    &= \frac{1}{N} \sum_{n=0}^{N-1} \widetilde h_n e^{2\pi i m n / N} \label{eq:8.48}
\end{align}
Hence, we can interpolate the initial function from its samples.
\begin{align*}
	h(t) = \frac{1}{N} \sum_{n=0}^{N-1} \widetilde h_n e^{2\pi i n t / N}
\end{align*}
Parseval's theorem becomes,
\begin{align} \label{eq:8.49}
	\sum_{m=0}^{N-1} \abs{h_m}^2 = \frac{1}{N} \sum_{n=0}^{N-1} \abs{\widetilde h_n}^2
\end{align}
\begin{exercise}
	Prove this.
\end{exercise} 
The convolution theorem for $g_m, h_m$ is
\begin{align} \label{eq:8.50}
	c_k = \sum_{m=0}^{N-1} g_m h_{k-m} \iff \widetilde c_k = \widetilde g_k \widetilde h_k
\end{align}

\subsection{Fast Fourier transform (non-examinable)}
While the discrete Fourier transform is an order $O(N^2)$ operation, we can reduce this into an order $O(n \log N)$ operation.
Such a simplification is called the \vocab{fast Fourier transform}.
We can split the discrete Fourier transform into even and odd parts, noting that $\omega_N = e^{-2\pi i / N}$ implies $\omega_N^2 = e^{-2 \pi i / (N/2)} = \omega_{N/2}$
\begin{align*}
	\widetilde h_k & = \sum_{n=0}^{N-1} h_n \omega_N^{nk}                                                                           \\
    & = \sum_{m=0}^{N/2-1} h_{2m} \omega_N^{2mk} + \sum_{m=0}^{N/2-1} h_{2m + 1} \omega_N^{(2m+1)k}                  \\
    & = \sum_{m=0}^{N/2-1} h_{2m} (\omega_N^2)^{mk} + \omega_N^k \sum_{m=0}^{N/2-1} h_{2m + 1} (\omega_N^2)^{mk}     \\
    & = \sum_{m=0}^{N/2-1} h_{2m} (\omega_{N/2})^{mk} + \omega_N^k \sum_{m=0}^{N/2-1} h_{2m + 1} (\omega_{N/2})^{mk} \\
\end{align*}
This algorithm iteratively reduces the Fourier transform's complexity by a factor of two, until the trivial case of finding the discrete Fourier transform of two data points.