\part{Groups}

\section{Review of IA Groups}

\textit{This section contains material covered by IA Groups.}

\subsection{Definitions}
A \textit{group} is a pair $(G, \cdot)$ where $G$ is a set and $\cdot \colon G \times G \to G$ is a binary operation on $G$, satisfying
\begin{itemize}
	\item $a \cdot (b \cdot c) = (a \cdot b) \cdot c$;
	\item there exists $e \in G$ such that for all $g \in G$, we have $g \cdot e = e \cdot g = g$; and
	\item for all $g \in G$, there exists an inverse $h \in G$ such that $g \cdot h = h \cdot g = e$.
\end{itemize}
\begin{remark}
	\begin{enumerate}
		\item Sometimes, such as in IA Groups, a closure axiom is also specified.
		      However, this is implicit in the type definition of $\cdot$.
		      In practice, this must normally be checked explicitly.
		\item Additive and multiplicative notation will be used interchangeably.
		      For additive notation, the inverse of $g$ is denoted $-g$, and for multiplicative notation, the inverse is instead denoted $g\inv$.
		      The identity element is sometimes denoted $0$ in additive notation and $1$ in multiplicative notation.
	\end{enumerate}
\end{remark}
A subset $H \subseteq G$ is a \textit{subgroup} of $G$, written $H \leq G$, if $h \cdot h' \in H$ for all $h, h' \in H$, and $(H, \cdot)$ is a group.
The closure axiom must be checked, since we are restricting the definition of $\cdot$ to a smaller set.
\begin{remark}
	A non-empty subset $H \subseteq G$ is a subgroup of $G$ if and only if
	\begin{align*}
		a, b \in H \implies a \cdot b\inv \in H
	\end{align*}
\end{remark}
An \textit{abelian} group is a group such that $a \cdot b = b \cdot a$ for all $a, b$ in the group.
The \textit{direct product} of two groups $G, H$, written $G \times H$, is the group over the Cartesian product $G \times H$ with operation $\cdot$ defined such that $(g_1, h_1) \cdot (g_2, h_2) = (g_1 \cdot_G g_2, h_1 \cdot_H h_2)$.

\subsection{Cosets}
Let $H \leq G$.
Then, the \textit{left cosets} of $H$ in $G$ are the sets $gH$ for all $g \in G$.
The set of left cosets partitions $G$.
Each coset has the same cardinality as $H$.
Lagrange's theorem states that if $G$ is a finite group and $H \leq G$, we have $\abs{G} = \abs{H} \cdot [G \colon H]$, where $[G \colon H]$ is the number of left cosets of $H$ in $G$.
$[G \colon H]$ is known as the \textit{index} of $H$ in $G$.
We can construct Lagrange's theorem analogously using right cosets.
Hence, the index of a subgroup is independent of the choice of whether to use left or right cosets; the number of left cosets is equal to the number of right cosets.

\subsection{Order}
Let $g \in G$.
If there exists $n \geq 1$ such that $g^n = 1$, then the least such $n$ is the \textit{order} of $g$.
If no such $n$ exists, we say that $g$ has infinite order.
If $g$ has order $d$, then:
\begin{enumerate}
	\item $g^n = 1 \implies d \mid n$;
	\item $\genset{g} = \qty{1, g, \dots, g^{d-1}} \leq G$, and by Lagrange's theorem (if $G$ is finite) $d \mid \abs{G}$.
\end{enumerate}

\subsection{Normality and quotients}
A subgroup $H \leq G$ is \textit{normal}, written $H \trianglelefteq G$, if $g\inv H g = H$ for all $g \in G$.
In other words, $H$ is preserved under conjugation over $G$.
If $H \trianglelefteq G$, then the set $\faktor{G}{H}$ of left cosets of $H$ in $G$ forms the \textit{quotient group}.
The group action is defined by $g_1 H \cdot g_2 H = (g_1 \cdot g_2) H$.
This can be shown to be well-defined.

\subsection{Homomorphisms}
Let $G, H$ be groups.
A function $\phi \colon G \to H$ is a \textit{group homomorphism} if $\phi(g_1 \cdot_G g_2) = \phi(g_1) \cdot_H \phi(g_2)$ for all $g_1, g_2 \in G$.
The \textit{kernel} of $\phi$ is defined to be $\ker \phi = \qty{g \in G \colon \phi(g) = 1}$, and the \textit{image} of $\phi$ is $\Im \phi = \qty{\phi(g) \colon g \in G}$.
The kernel is a normal subgroup of $G$, and the image is a subgroup of $H$.

\subsection{Isomorphisms}
An \textit{isomorphism} is a homomorphism that is bijective.
This yields an inverse function, which is of course also an isomorphism.
If $\varphi \colon G \to H$ is an isomorphism, we say that $G$ and $H$ are isomorphic, written $G \cong H$.
Isomorphism is an equivalence relation.
The isomorphism theorems are
\begin{enumerate}
	\item if $\varphi \colon G \to H$, then $\faktor{G}{\ker \varphi} \cong \Im \varphi$;
	\item if $H \leq G$ and $N \trianglelefteq G$, then $H \cap N \trianglelefteq H$ and $\faktor{H}{H \cap N} \cong \faktor{HN}{N}$;
	\item if $N \leq M \leq G$ such that $N \trianglelefteq G$ and $M \trianglelefteq G$, then $\faktor{M}{N} \trianglelefteq \faktor{G}{N}$, and $\faktor{G/N}{M/N} = \faktor{G}{M}$.
\end{enumerate}
