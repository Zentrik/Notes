\section{Finite abelian groups}

\subsection{Products of cyclic groups}
\begin{theorem} \label{thm:6.1}
	Every finite abelian group is isomorphic to a product of cyclic groups.
\end{theorem}

The proof for this theorem will be provided later in the course.
Note that the isomorphism provided for by the theorem is not unique.
An example of such behaviour is the following lemma.

\begin{lemma} \label{lem:6.2}
	Let $m, n \in \mathbb{N}$ be coprime integers.
	Then $C_m \times C_n \cong C_{mn}$.
\end{lemma}

\begin{proof}
	Let $g, h$ be generators of $C_m$ and $C_n$.
	Then consider the element $(g, h)^k = (g^k, h^k)$, which has order $mn$.
	Thus $\genset{(g,h)}$ has order $mn$.
	So every element in $C_m \times C_n$ is expressible in this way, giving $\genset{(g,h)} = C_m \times C_n$.
\end{proof}

\begin{corollary} \label{cor:6.3}
	Let $G$ be a finite abelian group.
	Then $G \cong C_{n_1} \times \dots \times C_{n_k}$ where each $n_i$ is a power of a prime.
\end{corollary}

\begin{proof}
	If $n_i = p_1 a^1 \cdots p^r a^r$ where the $p_i$ are distinct primes, then applying \Cref{lem:6.2} inductively gives $C_{n_i}$ as a product of cyclic groups which have orders that are powers of primes.

	We can apply this to the theorem that every finite abelian group is isomorphic to a product of cyclic groups to find the result.
\end{proof}

Later, we will prove the following refinement of \Cref{thm:6.1}
\begin{theorem} \label{thm:6.4}
	Let $G$ be a finite abelian group.
	Then $G \cong C_{d_1} \times \dots \times C_{d_t}$ where $d_i \mid d_{i+1}$ for all $i$.
\end{theorem}

\begin{remark}
	The integers $n_1, \dots, n_k$ in \Cref{cor:6.3} are unique up to ordering.
	The integers $d_1, \dots, d_t$ in \Cref{thm:6.4} are also unique, assuming that $d_1 > 1$.
	The proofs will be omitted - but works by counting the number of elements of $G$ of each prime power order.
\end{remark}

\begin{example}
	The abelian groups of order 8 are exactly $C_8$, $C_2 \times C_4$, and $C_2 \times C_2 \times C_2$.
\end{example}

\begin{example}
	The abelian groups of order 12 are, using the corollary \Cref{cor:6.3}, $C_2 \times C_2 \times C_3$, $C_4 \times C_3$, and using \Cref{thm:6.4}, $C_2 \times C_6$ and $C_{12}$.
	However, $C_2 \times C_3 \cong C_6$ and $C_3 \times C_4 \cong C_{12}$, so the groups derived are isomorphic.
\end{example} 

\begin{definition}[Exponent]
	The \vocab{exponent} of a group $G$ is the least integer $n \geq 1$ such that $g^n = 1$ for all $g \in G$.
	Equivalently, the exponent is the lowest common multiple of the orders of elements in $G$.
\end{definition}

\begin{example}
	The exponent of $A_4$ is $\mathrm{lcm}\qty{2, 3} = 6$.
\end{example}

\begin{corollary}[Structure Theorem]
	Let $G$ be a finite abelian group.
	Then $G$ contains an element which has order equal to the exponent of $G$.
\end{corollary}

\begin{proof}
	If $G \cong C_{d_1} \times \dots \times C_{d_t}$ for $d_i \mid d_{i+1}$, every $g \in G$ has order dividing $d_t$.
	Hence the exponent is $d_t$, and we can choose a generator of $C_{d_t}$ to obtain an element in $G$ of the same order\footnote{Say $o(h) = d_t$ with $h \in C_{d_t}$ then $(e, e, \dots, e, h) \in G$ and has order $d_t$}.
\end{proof}
