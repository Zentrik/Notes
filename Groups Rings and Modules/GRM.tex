%&../preamble

\def\npart {IB}
\def\nterm {Lent}
\def\nyear {2022}
\def\nlecturer {Dr R Zhou}
\def\ncourse {GRM}

\def\encodingdefault{TU}\normalfont
\ifnum 0\ifxetex 1\fi\ifluatex 1\fi=0 % if pdftex
  \usepackage[T1]{fontenc}
  \usepackage[utf8]{inputenc}
  \usepackage{textcomp} % provide euro and other symbols
\else % if luatex or xetex
  % \usepackage{unicode-math}
  % \defaultfontfeatures{Scale=MatchLowercase}
  % \defaultfontfeatures[\rmfamily]{Ligatures=TeX,Scale=1}
  % \DeclareMathAlphabet{\mathcal}{OMS}{cmsy}{m}{n}
  % \let\mathbb\relax % remove the definition by unicode-math
  % \DeclareMathAlphabet{\mathbb}{U}{msb}{m}{n}
\fi


\usetikzlibrary{external}
\tikzset{external/system call={xelatex -fmt=../preamble.fmt \tikzexternalcheckshellescape -halt-on-error -interaction=batchmode -jobname "\image" "\texsource"}} % path is relative to file that includes preamble
\tikzexternalize

\hypersetup{
  pdftitle={Part \npart\ - \ncourse},
  pdfauthor={\nauthor},
  pdfsubject={Cambridge Maths Notes: Part \npart\ - \ncourse},
  pdfkeywords={Cambridge Mathematics Maths Math \npart\ \nterm\ \nyear\ \ncourse}
}

\author{Based on lectures by \nlecturer \\\small Notes taken by \nauthor}
\date{\nterm\ \nyear}
\title{Part \npart\ --- \ncourse}

\tikzsetexternalprefix{figtemp/}
\author{Based on lectures by \nlecturer \ and notes by thirdsgames.co.uk}

\usepackage{faktor}
\newcommand{\genset}[1]{\left\langle{} #1 \right\rangle}
\newcommand{\ngenset}[1]{\left\langle\!\left\langle{} #1 \right\rangle\!\right\rangle}
\newcommand{\acts}{\curvearrowright}
\DeclarePairedDelimiter\Brackets{[\![}{]\!]}

\DeclareMathOperator{\Sym}{Sym}
\DeclareMathOperator{\Orb}{Orb}
\DeclareMathOperator{\Stab}{Stab}
\DeclareMathOperator{\ccl}{ccl}
\DeclareMathOperator{\adj}{adj}
\DeclareMathOperator{\sgn}{sgn}
\DeclareMathOperator{\id}{id}

\begin{document}
    \maketitle
    \tableofcontents
    
    \setcounter{section}{-1}

    \part{Groups}

\section{Review of IA Groups}

\textit{This section contains material covered by IA Groups.}

\subsection{Definitions}
A \textit{group} is a pair $(G, \cdot)$ where $G$ is a set and $\cdot : G \times G \to G$ is a binary operation on $G$, satisfying
\begin{itemize}
	\item $a \cdot (b \cdot c) = (a \cdot b) \cdot c$;
	\item there exists $e \in G$ such that for all $g \in G$, we have $g \cdot e = e \cdot g = g$; and
	\item for all $g \in G$, there exists an inverse $h \in G$ such that $g \cdot h = h \cdot g = e$.
\end{itemize}
\begin{remark}
	\begin{enumerate}
		\item Sometimes, such as in IA Groups, a closure axiom is also specified.
		      However, this is implicit in the type definition of $\cdot$.
		      In practice, this must normally be checked explicitly.
		\item Additive and multiplicative notation will be used interchangeably.
		      For additive notation, the inverse of $g$ is denoted $-g$, and for multiplicative notation, the inverse is instead denoted $g\inv$.
		      The identity element is sometimes denoted $0$ in additive notation and $1$ in multiplicative notation.
	\end{enumerate}
\end{remark}
A subset $H \subseteq G$ is a \textit{subgroup} of $G$, written $H \leq G$, if $h \cdot h' \in H$ for all $h, h' \in H$, and $(H, \cdot)$ is a group.
The closure axiom must be checked, since we are restricting the definition of $\cdot$ to a smaller set.
\begin{remark}
	A non-empty subset $H \subseteq G$ is a subgroup of $G$ if and only if
	\begin{align*}
		a, b \in H \implies a \cdot b\inv \in H
	\end{align*}
\end{remark}
An \textit{abelian} group is a group such that $a \cdot b = b \cdot a$ for all $a, b$ in the group.
The \textit{direct product} of two groups $G, H$, written $G \times H$, is the group over the Cartesian product $G \times H$ with operation $\cdot$ defined such that $(g_1, h_1) \cdot (g_2, h_2) = (g_1 \cdot_G g_2, h_1 \cdot_H h_2)$.

\subsection{Cosets}
Let $H \leq G$.
Then, the \textit{left cosets} of $H$ in $G$ are the sets $gH$ for all $g \in G$.
The set of left cosets partitions $G$.
Each coset has the same cardinality as $H$.
Lagrange's theorem states that if $G$ is a finite group and $H \leq G$, we have $\abs{G} = \abs{H} \cdot [G : H]$, where $[G : H]$ is the number of left cosets of $H$ in $G$.
$[G : H]$ is known as the \textit{index} of $H$ in $G$.
We can construct Lagrange's theorem analogously using right cosets.
Hence, the index of a subgroup is independent of the choice of whether to use left or right cosets; the number of left cosets is equal to the number of right cosets.

\subsection{Order}
Let $g \in G$.
If there exists $n \geq 1$ such that $g^n = 1$, then the least such $n$ is the \textit{order} of $g$.
If no such $n$ exists, we say that $g$ has infinite order.
If $g$ has order $d$, then:
\begin{enumerate}
	\item $g^n = 1 \implies d \mid n$;
	\item $\genset{g} = \qty{1, g, \dots, g^{d-1}} \leq G$, and by Lagrange's theorem (if $G$ is finite) $d \mid \abs{G}$.
\end{enumerate}

\subsection{Normality and quotients}
A subgroup $H \leq G$ is \textit{normal}, written $H \trianglelefteq G$, if $g\inv H g = H$ for all $g \in G$.
In other words, $H$ is preserved under conjugation over $G$.
If $H \trianglelefteq G$, then the set $\faktor{G}{H}$ of left cosets of $H$ in $G$ forms the \textit{quotient group}.
The group action is defined by $g_1 H \cdot g_2 H = (g_1 \cdot g_2) H$.
This can be shown to be well-defined.

\subsection{Homomorphisms}
Let $G, H$ be groups.
A function $\phi : G \to H$ is a \textit{group homomorphism} if $\phi(g_1 \cdot_G g_2) = \phi(g_1) \cdot_H \phi(g_2)$ for all $g_1, g_2 \in G$.
The \textit{kernel} of $\phi$ is defined to be $\ker \phi = \qty{g \in G : \phi(g) = 1}$, and the \textit{image} of $\phi$ is $\Im \phi = \qty{\phi(g) : g \in G}$.
The kernel is a normal subgroup of $G$, and the image is a subgroup of $H$.

\subsection{Isomorphisms}
An \textit{isomorphism} is a homomorphism that is bijective.
This yields an inverse function, which is of course also an isomorphism.
If $\varphi : G \to H$ is an isomorphism, we say that $G$ and $H$ are isomorphic, written $G \cong H$.
Isomorphism is an equivalence relation.

\begin{theorem}[First Isomorphism Theorem]
	If $\varphi : G \to H$, then $\faktor{G}{\ker \varphi} \cong \Im \varphi$;
\end{theorem}

\begin{theorem}[Second Isomorphism Theorem]
	If $H \leq G$ and $K \trianglelefteq G$, then $H \cap K \trianglelefteq H$ and $\faktor{H}{H \cap N} \cong \faktor{HN}{N}$
\end{theorem}

\begin{proof}
	Let $h_1 k_1, h_2 k_2 \in HK$ (so $h_1, h_2 \in H$, $k_1, k_2 \in K$).
	$h_1 k_1 (h_2 k_2)\inv = \underbracket{h_1 h_2\inv}_{\in H} \underbracket{h_2 k_1 k_2\inv h_2\inv}_{\in K} \in HK$.
	Thus $HK \subset G$ (by a previous Remark)

	Let $\phi : H \to \faktor{G}{K}, h \mapsto hK$.
	This is the composite of $H \to G$ and the quotient map $G \to \faktor{G}{K}$, hence $\phi$ a group homomorphism.
	$\ker \phi = \{h \in H : hK = K\} = H \cap K \trianglelefteq H$ and $\Im \phi = \{hK : h \in H\} = \faktor{HK}{K}$.

	First isomorphism theorem implies $\faktor{H}{H \cap K} \cong \faktor{HK}{K}$.
\end{proof} 

\begin{remark} \label{rem:1.2}
	Suppose $K \trianglelefteq G$.
	There is a bijection between subgroups of $\faktor{G}{K}$ and subgroups of $G$ containing $K$.
	This also restricts to a bijection between normal subgroups of $\faktor{G}{K}$ and normal subgroups of $G$ containing $K$.
\end{remark} 

\begin{theorem}[Third Isomorphism Theorem]
	Let $K \leq H \leq G$ be normal subgroups of $G$.
	Then $\faktor{G/K}{H/K} = \faktor{G}{H}$
\end{theorem}

\begin{proof}
	Let $\phi : \faktor{G}{K} \to \faktor{G}{H}, gK \mapsto gH$.
	If $g_1 K = g_2 K$, then $g_2\inv g_1 \in K \subset H \implies g_1 H = g_2 H$.
	Thus $\phi$ well-defined.

	$\phi$ is a surjective group homomorphism with kernel $\faktor{H}{K}$.
\end{proof} 
    \include{01-simple.tex}
    \section{Group actions}

\subsection{Definitions}
\begin{definition}[Symmetric Group]
	Let $X$ be a set.
	Then $\Sym(X)$ is the group of permutations of $X$; that is, the group of all bijections of $X$ to itself under composition.
	The identity can be written $\id$ or $\id_X$.
\end{definition}

\begin{definition}[Permuation Group]
	A group $G$ is a permutation group of degree $n$ if $G \leq \Sym(X)$ where $\abs{X} = n$.
\end{definition}

\begin{example}
	The symmetric group $S_n$ is exactly equal to $\Sym(\qty{1, \dots, n})$, so is a permutation group of order $n$.
	$A_n$ is also a permutation group of order $n$, as it is a subgroup of $S_n$.
	$D_{2n}$ is a permutation group of order $n$.
\end{example}

\begin{definition}[Group Actions]
	A \vocab{group action} of a group $G$ on a set $X$ is a function $\alpha : G \times X \to X$ satisfying
	\begin{align*}
		\alpha(e, x) = x;\quad \alpha(g_1 \cdot g_2, x) = \alpha(g_1, \alpha(g_2, x))
	\end{align*}
	for all $g_1, g_2 \in G, x \in X$.
	The group action may be written $\ast$, defined by $g \ast x \equiv \alpha(g,x)$.
\end{definition}

\begin{proposition}
	An action of a group $G$ on a set $X$ is uniquely characterised by a group homomorphism $\varphi : G \to \Sym(X)$.
\end{proposition}

\begin{proof}
	For all $g \in G$, we can define $\varphi_g : X \to X$ by $x \mapsto g \ast x$.
	Then, for all $x \in X$,
	\begin{align*}
		\varphi_{g_1 g_2} (x) = (g_1 g_2) \ast x = g_1 \ast (g_2 \ast x) = \varphi_{g_1}(\varphi_{g_2}(x))
	\end{align*}
	Thus $\varphi_{g_1 g_2} = \varphi_{g_1} \circ \varphi_{g_2}$.
	In particular, $\varphi_g \circ \varphi_{g\inv} = \varphi_e$.
	We now define
	\begin{align*}
		\varphi : G \to \Sym(X);\quad \varphi(g) = \varphi_g \implies \varphi(g)(x) = g \ast x
	\end{align*}
	This is a homomorphism.

	Conversely, any group homomorphism $\varphi : G \to \Sym(X)$ induces a group action $\ast$ by $g \ast x = \varphi(g)$.
	This yields $e \ast x = \varphi(e)(x) = \id x = x$ and $(g_1 g_2) \ast x = \varphi(g_1 g_2) x = \varphi(g_1) \varphi(g_2) x = g_1 \ast (g_2 \ast x)$ as required.
\end{proof}

\begin{definition}[Permutation Representation]
	The homomorphism $\varphi : G \to \Sym(X)$ defined in the above proof is called a \vocab{permutation representation} of $G$.
\end{definition}

\begin{definition}[Orbit, Stabiliser]
	Let $G$ act on $X$.
	Then,
	\begin{enumerate}
		\item the \vocab{orbit} of $x \in X$ is $\Orb_G(x) = \qty{g \ast x : g \in G} \subseteq X$;
		\item the \vocab{stabiliser} of $x \in X$ is $G_x = \qty{g \in G : g \ast x = x} \leq G$.
	\end{enumerate}
\end{definition}

\begin{definition}[Transitive Group Action]
	If there is only orbit, i.e. $\Orb_G(x) = X \quad \forall \; x$ then the group action is \vocab{transitive}.
\end{definition} 

\begin{definition}[Kernel]
	The \vocab{kernel} of a permutation representation is $\bigcap_{x \in X} G_x$.
\end{definition} 

\begin{remark}
	The kernel of the permutation representation $\varphi$ is also referred to as the kernel of the group action itself.
\end{remark} 

\begin{definition}[Faithful Group Action]
	If the kernel is trivial the action is said to be \vocab{faithful}.
\end{definition} 

\begin{theorem}[Orbit-stabiliser theorem]
	The orbit $\Orb_G(x)$ bijects with the set $\faktor{G}{G_x}$ of left cosets of $G_x$ in $G$ (which may not be a quotient group).
	In particular, if $G$ is finite, we have
	\begin{align*}
		\abs{G} = \abs{\Orb(x)} \cdot \abs{G_x}
	\end{align*}
\end{theorem}

\begin{example}
	If $G$ is the group of symmetries of a cube and we let $X$ be the set of vertices in the cube, $G$ acts on $X$.
	Here, for all $x \in X$, $\abs{\Orb(x)} = 8$ and $\abs{G_x} = 6$ (including reflections), hence $\abs{G} = 48$.
\end{example}

\begin{remark}
	The orbits partition $X$.

	Note that $G_{g \ast x} = g G_x g\inv$.
	Hence, if $x, y$ lie in the same orbit, their stabilisers are conjugate.
\end{remark}

\subsection{Examples}
\begin{example}
	$G$ acts on itself by left multiplication.
	This is known as the \vocab{left regular action}.
	The kernel is trivial, hence the action is faithful.
	The action is transitive, since for all $g_1, g_2 \in G$, the element $g_2 g_1\inv$ maps $g_1$ to $g_2$.
\end{example}

\begin{theorem}[Cayley's theorem]
	Any finite group $G$ is a permutation group of order $\abs{G}$; it is isomorphic to a subgroup of $S_\abs{G}$.
\end{theorem}

\begin{example}
	Let $H \leq G$.
	Then $G$ acts on $\faktor{G}{H}$ by left multiplication, where $\faktor{G}{H}$ is the set of left cosets of $H$ in $G$.
	This is known as the \vocab{left coset action}.
	This action is transitive using the construction above for the left regular action.
	We have $\ker\varphi = \bigcap_{x \in G} xHx\inv$, which is the largest normal subgroup of $G$ contained within $H$.
\end{example}

\begin{theorem}
	Let $G$ be a non-abelian simple group, and $H \leq G$ with index $n > 1$.
	Then $n \geq 5$ and $G$ is isomorphic to a subgroup of $A_n$.
\end{theorem}

\begin{proof}
	Let $G$ act on $X = \faktor{G}{H}$ by left multiplication.
	Let $\varphi : G \to \Sym(X)$ be the permutation representation associated to this group action.

	Since $G$ is simple, $\ker \varphi = 1$ or $\ker \varphi = G$.
	If $\ker \varphi = G$, then $\Im\varphi = 1_{S_n}$, which is a contradiction since $G$ acts transitively on $X$ and $|X| > 1$.
	Thus $\ker \varphi = 1$, and $G \cong \Im\varphi \leq S_n$.

	Since $G \leq S_n$ and $A_n \triangleleft S_n$, the second isomorphism theorem shows that $G \cap A_n \triangleleft G$, and
	\begin{align*}
		\faktor{G}{G \cap A_n} \cong \faktor{GA_n}{A_n} \leq \faktor{S_n}{A_n} \cong C_2
	\end{align*}
	Since $G$ is simple, $G \cap A_n = 1$ or $G$.
	If $G \cap A_n = 1$, then $G$ is isomorphic to a subgroup of $C_2$, but this is false, since $G$ is non-abelian.
	Hence $G \cap A_n = G$ so $G \leq A_n$.
	Finally, if $n \leq 4$ we can check manually that $A_n$ is not simple; $A_n$ has no non-abelian simple subgroups.
\end{proof}

\subsection{Conjugation actions}
\begin{example}
	Let $G$ act on $G$ by conjugation, so $g \ast x = g x g\inv$.
	This is known as the \vocab{conjugation action}.
\end{example}

\begin{definition}[Conjugacy Class, Centraliser, Centre]
	The orbit of the conjugation action is called the \vocab{conjugacy class} of a given element $x \in G$, written $\ccl_G(x)$.
	The stabiliser of the conjugation action is the set $C_x$ of elements which commute with a given element $x$, called the \vocab{centraliser} of $x$ in $G$.
	The kernel of $\varphi$ is the set $Z(G)$ of elements which commute with all elements in $x$, which is the \vocab{centre} of $G$.
	This is always a normal subgroup.
\end{definition}

\begin{remark}
	$\varphi : G \to G$ satisfies
	\begin{align*}
		\varphi(g)(h_1 h_2) = g h_1 h_2 g\inv = h h_1 g\inv g h_2 g\inv = \varphi(g)(h_1) \varphi(g)(h_2)
	\end{align*}
	Hence $\varphi(g)$ is a group homomorphism for all $g$.
	It is also a bijection, hence $\varphi(g)$ is an isomorphism from $G \to G$.
\end{remark}

\begin{definition}[Automorphism]
	An isomorphism from a group to itself is known as an \vocab{automorphism}.
	We define $\mathrm{Aut}(G)$ to be the set of all group automorphisms of a given group.
	This set is a group.
	Note, $\mathrm{Aut}(G) \leq \Sym(G)$, and the $\varphi : G \to \Sym(G)$ above has image in $\mathrm{Aut}(G)$.
\end{definition}

\begin{example}
	Let $X$ be the set of subgroups of $G$.
	Then $G$ acts on $X$ by conjugation: $g \ast H = g H g\inv$.
	The stabiliser of a subgroup $H$ is $\qty{ g \in G : gHg\inv = H } = N_G(H)$, called the \vocab{normaliser} of $H$ in $G$.
	The normaliser of $H$ is the largest subgroup of $G$ that contains $H$ as a normal subgroup.
	In particular, $H \triangleleft G$ if and only if $N_G(H) = G$.
\end{example}

    \include{03-alternating.tex}
    \section{$p$-groups}

\subsection{$p$-groups}
\begin{definition}[$p$-group]
	Let $p$ be a prime.
	A finite group $G$ is a \vocab{$p$-group} if $\abs{G} = p^n$ for $n \geq 1$.
\end{definition}

\begin{theorem}
	If $G$ is a $p$-group, the centre $Z(G)$ is non-trivial.
\end{theorem}

\begin{proof}
	For $g \in G$, due to the orbit-stabiliser theorem, $\abs{\ccl(g)} \abs{C(g)} = p^n$.
	In particular, $\abs{\ccl(g)}$ divides $p^n$, and they partition $G$.
	Since $G$ is a disjoint union of conjugacy classes, modulo $p$ we have
	\begin{align*}
		\abs{G} \equiv \text{number of conjugacy classes of size } 1 \equiv 0 \implies \abs{Z(G)} \equiv 0
	\end{align*}
	Hence $Z(G)$ has order zero modulo $p$ so it cannot be trivial.
	We can check this by noting that $g \in Z(G) \iff x\inv g x = g$ for all $x$, which is true if and only if $\ccl_G(g) = \qty{g}$.
\end{proof}

\begin{corollary}
	The only simple $p$-groups are the cyclic groups of order $p$.
\end{corollary}

\begin{proof}
	Let $G$ be a simple $p$-group.
	Since $Z(G)$ is a normal subgroup of $G$, we have $Z(G) = 1$ or $Z(G) = G$.
	But $Z(G)$ may not be trivial, so $Z(G) = G$.
	This implies $G$ is abelian.
	The only abelian simple groups are cyclic of prime order by \cref{lem:1.3}, hence $G \cong C_p$.
\end{proof}

\begin{corollary}
	Let $G$ be a $p$-group of order $p^n$.
	Then $G$ has a subgroup of order $p^r$ for all $r \in \qty{0, \dots, n}$.
\end{corollary}

\begin{proof}
	Recall from \cref{lem:1.4} that any group $G$ has a composition series $1 = G_1 \triangleleft \dots \triangleleft G_N = G$ where each quotient $\faktor{G_{i+1}}{G_i}$ is simple.

	Since $G$ is a $p$-group, $\faktor{G_{i+1}}{G_i}$ is also a $p$-group.
	Each successive quotient is an order $p$ group by the previous corollary, so we have a composition series of nested subgroups of order $p^r$ for all $r \in \qty{0, \dots, n}$.
\end{proof}

\begin{lemma}
	Let $G$ be a group.
	If $\faktor{G}{Z(G)}$ is cyclic, then $G$ is abelian.
	This then implies that $Z(G) = G$, so in particular $\faktor{G}{Z(G)} = 1$.
\end{lemma}

\begin{proof}
	Let $g Z(G)$ be a generator for $\faktor{G}{Z(G)}$.
	Then, each coset of $Z(G)$ in $G$ is of the form $g^r Z(G)$ for some $r \in \mathbb Z$.
	Thus, $G = \qty{g^r z \colon r \in \mathbb Z, z \in Z(G)}$.
	Now, we multiply two elements of this group and find
	\begin{align*}
		g^{r_1} z_1 g^{r_2} z_2 = g^{r_1 + r_2} z_1 z_2 = g^{r_1 + r_2} z_2 z_1 = z_2 z_1 g^{r_1 + r_2} = g^{r_2} z_2 g^{r_1} z_1
	\end{align*}
	So any two elements in $G$ commute.
\end{proof}

\begin{corollary}
	Any group of order $p^2$ is abelian.
\end{corollary}

\begin{proof}
	Let $G$ be a group of order $p^2$.
	Then $\abs{Z(G)} \in \qty{1, p, p^2}$.
	The centre cannot be trivial as proven above, since $G$ is a $p$-group.
	If $\abs{Z(G)} = p$, we have that $\faktor{G}{Z(G)}$ is cyclic as it has order $p$.
	Applying the previous lemma, $G$ is abelian.
	However, this is a contradiction since the centre of an abelian group is the group itself.
	If $\abs{Z(G)} = p^2$ then $Z(G) = G$ and then $G$ is clearly abelian.
\end{proof}

\subsection{Sylow theorems}
\begin{theorem}[Sylow Theorems]
	Let $G$ be a finite group of order $p^a m$ where $p$ is a prime and $p$ does not divide $m$.
	Then:
	\begin{enumerate}
		\item The set $\mathrm{Syl}_p(G) = \qty{P \leq G \colon \abs{P} = p^a}$ of Sylow $p$-subgroups is non-empty.
		\item All Sylow $p$-subgroups are conjugate.
		\item The amount of Sylow $p$-subgroups $n_p = \abs{\mathrm{Syl}_p(G)}$ satisfies
		      \begin{align*}
			      n_p \equiv 1 \mod p;\quad n_p \mid \abs{G} \implies n_p \mid m
		      \end{align*}
	\end{enumerate}
\end{theorem}

\begin{proof}
	\begin{enumerate}
		\item Let $\Omega$ be the set of all \underline{subsets} of $G$ of order $p^a$.
		      We can directly find
		      \begin{align*}
			      \abs{\Omega} = \binom{p^a m}{p^a} = \frac{p^a m}{p^a} \cdot \frac{p^a m - 1}{p^a - 1} \cdots \frac{p^a m - p^a + 1}{1}
		      \end{align*}
		      Note that for $0 \leq k < p^a$, the numbers $p^a m - k$ and $p^a - k$ are divisible by the same power of $p$.
		      In particular, $\abs{\Omega}$ is coprime to $p$.

		      Let $G$ act on $\Omega$ by left-multiplication, so $g \ast X = \qty{gx \colon x \in X}$.
		      For any $X \in \Omega$, the orbit-stabiliser theorem can be applied to show that
		      \begin{align*}
			      \abs{G_X} \abs{\mathrm{orb}_G(X)} = \abs{G} = p^a m
		      \end{align*}
		      Since $|\Omega|$ is coprime to $p$, there must exist an orbit with size coprime to $p$, since orbits partition $\Omega$.
		      For such an $X$, $p^a \mid \abs{G_X}$.

		      Conversely, note that if $g \in G$ and $x \in X$, then $g \in (gx\inv) \ast X$.
		      Hence, we can consider
		      \begin{align*}
			      G = \bigcup_{g \in G} g \ast X = \bigcup_{Y \in \mathrm{orb}_G(X)} Y
		      \end{align*}
		      Thus $\abs{G} \leq \abs{\mathrm{orb}_G(X)} \cdot \abs{X}$, giving $\abs{G_X} = \frac{\abs{G}}{\abs{\mathrm{orb}_G(X)}} \leq \abs{X} = p^a$.

		      As $p^a \mid \abs{G_X}$ we must have $\abs{G_X} = p^a$.
		      In other words, the stabiliser $G_X$ is a Sylow $p$-subgroup of $G$.
		\item We will prove a stronger result for this part of the proof.
		      \begin{lemma} \label{lem:4.7}
				If $P$ is a Sylow $p$-subgroup and $Q \leq G$ is a $p$-subgroup, then $Q \leq g P g\inv$ for some $g \in G$.
			  \end{lemma} 

		      Indeed, let $Q$ act on the set of left cosets of $P$ in $G$ by left multiplication.
		      By the orbit-stabiliser theorem, each orbit has size which divides $\abs{Q} = p^k$ for some $k$.
		      Hence each orbit has size $p^r$ for some $r$.

		      Since $\faktor{G}{P}$ has size $m$, which is coprime to $p$, there must exist an orbit of size 1\footnote{Sum of the orbit sizes is $m$, $m$ coprime to $p$.}.
		      Therefore there exists $g \in G$ such that $q \ast gP = gP$ for all $q \in Q$.
		      Equivalently, $g\inv q g \in P$ for all $q \in Q$.
		      This implies that $Q \leq gPg\inv$ as required.
		      This then weakens to the second part of the Sylow theorems.
		\item Let $G$ act on $\mathrm{Syl}_p(G)$ by conjugation.
		      Part (ii) of the Sylow theorems implies that this action is transitive.
		      By the orbit-stabiliser theorem, $n_p = \abs{\mathrm{Syl}_p(G)} \mid \abs{G}$.

		      Let $P \in \mathrm{Syl}_p(G)$.
		      Then let $P$ act on $\mathrm{Syl}_p(G)$ by conjugation.
		      Since $P$ is a Sylow $p$-subgroup, the orbits of this action have size dividing $\abs{P} = p^a$, so the size is some power of $p$. \\
		      To show $n_p \equiv 1 \mod p$, it suffices to show that $\qty{P}$ is the unique orbit of size 1, as the orbits of other sizes are multiples of $p$. \\
		      Suppose $\qty{Q}$ is another orbit of size 1, so $Q$ is a Sylow $p$-subgroup which is preserved under conjugation by $P$.
		      Thus $P$ normalises $Q$, so $P \leq N_G(Q)$.
		      Notice that $P$ and $Q$ are both Sylow $p$-subgroups of $N_G(Q)$.
		      By (ii), $P$ and $Q$ are conjugate inside $N_G(Q)$.
		      Hence $P = Q$ since $Q \trianglelefteq N_G(Q)$.
		      Thus, $\abs{P}$ is the unique orbit of size 1, so $n_p \equiv 1 \mod p$ as required.
	\end{enumerate}
\end{proof}

\begin{corollary}
	If $n_p = 1$, then there is only one Sylow $p$-subgroup, and it is normal.
\end{corollary}

\begin{proof}
	Let $g \in G$ and $P \in \mathrm{Syl}_p(G)$.
	Then $g P g\inv$ is a Sylow $p$-subgroup, hence $g P g\inv = P$.
	$P$ is normal in $G$.
\end{proof}

\begin{remark}
	When $G$ acts on $\mathrm{Syl}_p(G)$ by conjugation, the orbit is $\mathrm{Syl}_p(G)$ and the stabiliser is the normaliser.  
\end{remark}

\begin{example}
	Let $G$ be a group with $\abs{G} = 1000 = 2^3 \cdot 5^3$.
	Here, $n_5 \equiv 1 \mod 5$, and $n_5 \mid 8$, hence $n_5 = 1$.
	Thus the unique Sylow 5-subgroup is normal.
	Hence no group of order 1000 is simple.
\end{example}

\begin{example}
	Let $G$ be a group with $\abs{G} = 132 = 2^2 \cdot 3 \cdot 11$.
	$n_{11}$ satisfies $n_{11} \equiv 1 \mod 11$ and $n_{11} \mid 12$, thus $n_{11} \in \qty{1, 12}$.

	Suppose $G$ is simple.

	Then $n_{11} = 12$\footnote{If $n_{11} = 1$ then we have a normal subgroup by the previous corollary.}.
	The amount of Sylow 3-subgroups satisfies $n_3 \equiv 1 \mod 3$ and $n_3 \mid 44$ so $n_3 \in \qty{1, 4, 22}$.
	Since $G$ is simple, $n_3 \in \qty{4, 22}$.

	Suppose $n_3 = 4$.
	Then $G$ acts on $\mathrm{Syl}_3(G)$ by conjugation, and this generates a group homomorphism $\varphi \colon G \to S_4$.
	But the kernel of this homomorphism is a normal subgroup of $G$, so $\ker \varphi$ is trivial or $G$ itself as $G$ simple.
	If $\ker \varphi = G$, then $\Im \varphi$ is trivial, contradicting Sylow's second theorem.
	If $\ker \varphi = 1$, then $\Im \varphi$ has order $132 > |S_4|$ \Lightning.

	Thus $n_3 = 22$ and recall $n_{11} = 12$.
	This means that $G$ has $22 \cdot (3-1) = 44$ elements of order 3\footnote{Each group in $\mathrm{Syl}_3(G)$ intersect trivially, as if they didn't any non trivial element in the intersection would generate both groups as they're all $C_3$.}, and further $G$ has $12 \cdot (11 - 1) = 120$ elements of order 11.
	However, the sum of these two totals is more than the total of 132 elements, so this is a contradiction.
	Hence $G$ is not simple.
\end{example}

    \section{Matrix groups}

\subsection{Definitions}
Let $F$ be a field, such as $\mathbb C$ or $\faktor{\mathbb Z}{p \mathbb Z}$.

\begin{definition}[Gneeral Linear Group]
	Let $GL_n(F)$ be set of $n \times n$ invertible matrices over $F$, which is called the \vocab{general linear group}.
\end{definition} 

\begin{definition}[Special Linear Group]
	Let $SL_n(F)$ be set of $n \times n$ matrices with determinant one over $F$, which is called the \vocab{special linear group}.
\end{definition} 

\begin{remark}
	$SL_n(F)$ is the kernel of the determinant homomorphism on $GL_n(F)$, so $SL_n(F) \triangleleft GL_n(F)$.
\end{remark} 	

\begin{definition}[Scalar Matrices]
	Let $Z \triangleleft GL_n(F)$ denote the subgroup of \vocab{scalar matrices}, the group of nonzero multiples of the identity.
\end{definition} 

\begin{remark}
	$Z$ is the centre of $GL_n(F)$.
\end{remark} 

\begin{definition}[Projective General Linear Group]
	The group $PGL_n(F) = \faktor{GL_n(F)}{Z}$ is called the \vocab{projective general linear group}. 
\end{definition} 

\begin{definition}[Projective Special Linear Group]
	The \vocab{projective special linear group} is $PSL_n(F) = \faktor{SL_n(F)}{Z \cap SL_n(F)}$.
\end{definition} 

\begin{remark}
	By the second isomorphism theorem, $PSL_n(F)$ is isomorphic to $\faktor{Z \cdot SL_n(F)}{Z}$, which is a subgroup of $PGL_n(F)$.
\end{remark} 

\begin{example} \label{exm:5.1}
	Consider the finite group $G = GL_n\qty(\faktor{\mathbb Z}{p\mathbb Z})$.
	A list of $n$ vectors in $\faktor{\mathbb Z}{p\mathbb Z}$ are the columns of a matrix $A \in G$ iff the vectors are linearly independent.
	Hence, by considering dimensionality of subspaces generated by each column,
	\begin{align*}
		\abs{G} & = (p^n - 1)(p^n - p)(p^n - p^2) \cdots (p^n - p^{n-1})      \\
		        & = p^{1+2+\dots+(n-1)} (p^n - 1)(p^{n-1} - 1) \cdots (p - 1) \\
		        & = p^{\binom{n}{2}} \prod_{i=1}^n (p^i - 1)
	\end{align*}
	Hence the Sylow $p$-subgroups have size $p^{\binom{n}{2}}$.
	Let $U$ be the set of upper triangular matrices with ones on the diagonal.
	This forms a Sylow $p$-subgroup of $G$, since there are $\binom{n}{2}$ entries in a given upper triangular matrix, and there are $p$ choices for such an entry.
\end{example}

\subsection{M\"obius maps in modular arithmetic}
Recall that $PGL_2(\mathbb C)$ acts on $\mathbb C \cup \qty{\infty}$ by M\"obius transformations.
Likewise, $PGL_2\qty(\faktor{\mathbb Z}{p\mathbb Z})$ acts on $\faktor{\mathbb Z}{p\mathbb Z} \cup \qty{\infty}$ by M\"obius transformations.
For a matrix
\begin{align*}
	A = \begin{pmatrix}
		a & b \\
		c & d
	\end{pmatrix} \in GL_2\qty(\faktor{\mathbb Z}{p\mathbb Z});\quad A \colon z \mapsto \frac{az+b}{cz+d}
\end{align*}
Since the scalar matrices act trivially, we obtain an action on the projective general linear group instead of the general linear group by quotienting out the scalar matrices.

We can represent $\infty$ as an integer, say, $p$, for the purposes of constructing a permutation representation.

\begin{lemma} \label{lem:5.2}
	The permutation representation $PGL_2\qty(\faktor{\mathbb Z}{p\mathbb Z}) \to S_{p+1}$ is injective (and is an isomorphism if $p = 2$ or $p = 3$).
\end{lemma}

\begin{proof}
	Suppose that $\frac{az+b}{cz+d} = z$ for all $z \in \faktor{\mathbb Z}{p\mathbb Z} \cup \qty{\infty}$. \\
	Since $z = 0$, we have $b = 0$. \\
	Since $z = \infty$, we find $c = 0$. \\
	Thus the matrix is diagonal. \\
	Finally, since $z = 1$, $\frac{a}{d} = 1$ hence $a = d$. \\
	Thus the matrix is scalar.
	So the permutation representation from $PGL_2\qty(\faktor{\mathbb Z}{p \mathbb Z})$ has trivial kernel, giving injectivity as required.

	If $p = 2$ or $p = 3$ we can compute the orders of relevant groups manually and show that the permutation representation is an isomorphism.
\end{proof}

\begin{lemma}
	Let $p$ be an odd prime.
	Then
	\begin{align*}
		\abs{PSL_2\qty(\faktor{\mathbb Z}{p\mathbb Z})} = \frac{(p-1)p(p+1)}{2}
	\end{align*}
\end{lemma}

\begin{proof}
	By \cref{exm:5.1},
	\begin{align*}
		\abs{GL_2\qty(\faktor{\mathbb Z}{p\mathbb Z})} = p(p^2 - 1)(p - 1)
	\end{align*}
	The homomorphism $GL_2\qty(\faktor{\mathbb Z}{p\mathbb Z}) \to \qty(\faktor{\mathbb Z}{p\mathbb Z})^\times$ given by the determinant is surjective.
	Since $SL_2\qty(\faktor{\mathbb Z}{p\mathbb Z})$ is the kernel of this homomorphism, we have
	\begin{align*}
		\abs{SL_2\qty(\faktor{\mathbb Z}{p\mathbb Z})} = \frac{GL_2\qty(\faktor{\mathbb Z}{p\mathbb Z})}{p - 1} = p(p-1)(p+1)
	\end{align*}
	Now, if $
	\begin{pmatrix}
		\lambda & 0 \\ 0 & \lambda
	\end{pmatrix}
	$ is an element of the special linear group, then $\lambda^2 \equiv 1 \mod p$.
	Then, $p \mid (\lambda - 1)(\lambda + 1)$ hence $\lambda \equiv \pm 1 \mod p$.
	Thus,
	\begin{align*}
		Z \cap SL_2\qty(\faktor{\mathbb Z}{p\mathbb Z}) = \qty{\pm I}
	\end{align*}
	and $\pm I$ are distinct since $p > 2$.

	Hence the order of the projective special linear group is half the order of the special linear group as required.
\end{proof}

\begin{example}
	Let $G = PSL_2\qty(\faktor{\mathbb Z}{5\mathbb Z})$.
	Then by the previous lemma, $\abs{G} = 60$.
	Let $G$ act on $\faktor{\mathbb Z}{5\mathbb Z} \cup \qty{\infty}$ by M\"obius transformations.
	The permutation representation $\varphi \colon G \to \Sym(\qty{0,1,2,3,4,\infty}) \cong S_6$ is injective by \Cref{lem:5.2}.

	\begin{claim}
		$\Im \varphi \subseteq A_6$, i.e. $\psi : G \overset{\phi}\to S_6 \overset{\sgn}\to \{\pm 1\}$ is trivial.
	\end{claim} 

	\begin{proof}
		Let $h\in G$, and suppose $h$ has order $2^n m$ for odd $m$ and so $o(h^m) = 2^n$.
		If $\psi(h^m) = 1$, then since $\psi$ is a group homomorphism we have $\psi(h)^m = 1$ giving $\psi(h) \neq -1 \implies \psi(h) = 1$.

		So to show $\psi$ is trivial, it suffices to show $\psi(g) = 1$ for all $g \in G$ with order a power of 2.

		By \Cref{lem:4.7}, if $g$ has order a power of 2, it is contained in a Sylow 2-subgroup.
		Then it suffices to show that $\psi(H) = 1$ for all Sylow 2-subgroups $H$.
		But since $\ker \psi \triangleleft G$ and all Sylow 2-subgroups are conjugate, it suffices to show $\psi(H) = 1$ for a single Sylow 2-subgroup $H$.

		The Sylow 2-subgroup must have order 4.
		Hence consider
		\begin{align*}
			H = \genset{ \begin{pmatrix}
					2 & 0 \\
					0 & 3
				\end{pmatrix} \qty{\pm I}, \begin{pmatrix}
					0  & 1 \\
					-1 & 0
				\end{pmatrix} \qty{\pm I} }
		\end{align*}
		Both of these elements square to the identity element inside the projective special linear group.
		This generates a group of order 4 which is necessarily a Sylow 2-subgroup.
		We can explicitly compute the action of $H$ on $\qty{0,1,2,3,4,\infty}$.
		\begin{align*}
			\varphi\qty(\begin{pmatrix}
					2 & 0 \\
					0 & 3
				\end{pmatrix}) = (1\ 4)(2\ 3);\quad \varphi\qty(\begin{pmatrix}
					0  & 1 \\
					-1 & 0
				\end{pmatrix}) = (0\ \infty)(1\ 4)
		\end{align*}
		These are products of two transpositions, hence even permutations.
		Thus $\psi(H) = 1$, proving the claim that $G \leq A_6$.
	\end{proof} 

	We can prove that for any $G \leq A_6$ of order 60, we have $G \cong A_5$; this is a question from the example sheets.
\end{example}

\subsection{Properties}
The following properties will not be proven in this course.
\begin{itemize}
	\item $PSL_n\qty(\faktor{\mathbb Z}{p\mathbb Z})$ is simple for all $n \geq 2$ and $p$ prime, except where $n = 2$ and $p = 2, 3$.
	      Such groups are called finite groups of \textit{Lie type}.
	\item The smallest non-abelian simple groups are $A_5 \cong PSL_2\qty(\faktor{\mathbb Z}{5\mathbb Z})$, then $PSL_2\qty(\faktor{\mathbb Z}{7\mathbb Z}) \cong GL_3\qty(\faktor{\mathbb Z}{2\mathbb Z})$ which has order 168.
\end{itemize}

    \section{Finite abelian groups}

\subsection{Products of cyclic groups}
\begin{theorem} \label{thm:6.1}
	Every finite abelian group is isomorphic to a product of cyclic groups.
\end{theorem}

The proof for this theorem will be provided later in the course.
Note that the isomorphism provided for by the theorem is not unique.
An example of such behaviour is the following lemma.

\begin{lemma} \label{lem:6.2}
	Let $m, n \in \mathbb{N}$ be coprime integers.
	Then $C_m \times C_n \cong C_{mn}$.
\end{lemma}

\begin{proof}
	Let $g, h$ be generators of $C_m$ and $C_n$.
	Then consider the element $(g, h)^k = (g^k, h^k)$, which has order $mn$.
	Thus $\genset{(g,h)}$ has order $mn$.
	So every element in $C_m \times C_n$ is expressible in this way, giving $\genset{(g,h)} = C_m \times C_n$.
\end{proof}

\begin{corollary} \label{cor:6.3}
	Let $G$ be a finite abelian group.
	Then $G \cong C_{n_1} \times \dots \times C_{n_k}$ where each $n_i$ is a power of a prime.
\end{corollary}

\begin{proof}
	If $n_i = p_1 a^1 \cdots p^r a^r$ where the $p_i$ are distinct primes, then applying \Cref{lem:6.2} inductively gives $C_{n_i}$ as a product of cyclic groups which have orders that are powers of primes.

	We can apply this to the theorem that every finite abelian group is isomorphic to a product of cyclic groups to find the result.
\end{proof}

Later, we will prove the following refinement of \Cref{thm:6.1}
\begin{theorem} \label{thm:6.4}
	Let $G$ be a finite abelian group.
	Then $G \cong C_{d_1} \times \dots \times C_{d_t}$ where $d_i \mid d_{i+1}$ for all $i$.
\end{theorem}

\begin{remark}
	The integers $n_1, \dots, n_k$ in \Cref{cor:6.3} are unique up to ordering.
	The integers $d_1, \dots, d_t$ in \Cref{thm:6.4} are also unique, assuming that $d_1 > 1$.
	The proofs will be omitted - but works by counting the number of elements of $G$ of each prime power order.
\end{remark}

\begin{example}
	The abelian groups of order 8 are exactly $C_8$, $C_2 \times C_4$, and $C_2 \times C_2 \times C_2$.
\end{example}

\begin{example}
	The abelian groups of order 12 are, using the corollary \Cref{cor:6.3}, $C_2 \times C_2 \times C_3$, $C_4 \times C_3$, and using \Cref{thm:6.4}, $C_2 \times C_6$ and $C_{12}$.
	However, $C_2 \times C_3 \cong C_6$ and $C_3 \times C_4 \cong C_{12}$, so the groups derived are isomorphic.
\end{example} 

\begin{definition}[Exponent]
	The \vocab{exponent} of a group $G$ is the least integer $n \geq 1$ such that $g^n = 1$ for all $g \in G$.
	Equivalently, the exponent is the lowest common multiple of the orders of elements in $G$.
\end{definition}

\begin{example}
	The exponent of $A_4$ is $\mathrm{lcm}\qty{2, 3} = 6$.
\end{example}

\begin{corollary}[Structure Theorem]
	Let $G$ be a finite abelian group.
	Then $G$ contains an element which has order equal to the exponent of $G$.
\end{corollary}

\begin{proof}
	If $G \cong C_{d_1} \times \dots \times C_{d_t}$ for $d_i \mid d_{i+1}$, every $g \in G$ has order dividing $d_t$.
	Hence the exponent is $d_t$, and we can choose a generator of $C_{d_t}$ to obtain an element in $G$ of the same order\footnote{Say $o(h) = d_t$ with $h \in C_{d_t}$ then $(e, e, \dots, e, h) \in G$ and has order $d_t$}.
\end{proof}

    \include{07-rings.tex}
    % \section{Factorisation in integral domains}
    % \input{09_factorisation.tex}
    % \section{Algebraic integers}
    % \input{10_algebraic_integers.tex}
    % \section{Noetherian rings}
    % \input{11_noetherian_rings.tex}
    % \section{Modules}
    % \input{12_modules.tex}
\end{document}
