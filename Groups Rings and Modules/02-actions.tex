\section{Group actions}

\subsection{Definitions}
\begin{definition}[Symmetric Group]
	Let $X$ be a set.
	Then $\Sym(X)$ is the group of permutations of $X$; that is, the group of all bijections of $X$ to itself under composition.
	The identity can be written $\id$ or $\id_X$.
\end{definition}

\begin{definition}[Permuation Group]
	A group $G$ is a permutation group of degree $n$ if $G \leq \Sym(X)$ where $\abs{X} = n$.
\end{definition}

\begin{example}
	The symmetric group $S_n$ is exactly equal to $\Sym(\qty{1, \dots, n})$, so is a permutation group of order $n$.
	$A_n$ is also a permutation group of order $n$, as it is a subgroup of $S_n$.
	$D_{2n}$ is a permutation group of order $n$.
\end{example}

\begin{definition}[Group Actions]
	A \vocab{group action} of a group $G$ on a set $X$ is a function $\alpha : G \times X \to X$ satisfying
	\begin{align*}
		\alpha(e, x) = x;\quad \alpha(g_1 \cdot g_2, x) = \alpha(g_1, \alpha(g_2, x))
	\end{align*}
	for all $g_1, g_2 \in G, x \in X$.
	The group action may be written $\ast$, defined by $g \ast x \equiv \alpha(g,x)$.
\end{definition}

\begin{proposition}
	An action of a group $G$ on a set $X$ is uniquely characterised by a group homomorphism $\varphi : G \to \Sym(X)$.
\end{proposition}

\begin{proof}
	For all $g \in G$, we can define $\varphi_g : X \to X$ by $x \mapsto g \ast x$.
	Then, for all $x \in X$,
	\begin{align*}
		\varphi_{g_1 g_2} (x) = (g_1 g_2) \ast x = g_1 \ast (g_2 \ast x) = \varphi_{g_1}(\varphi_{g_2}(x))
	\end{align*}
	Thus $\varphi_{g_1 g_2} = \varphi_{g_1} \circ \varphi_{g_2}$.
	In particular, $\varphi_g \circ \varphi_{g\inv} = \varphi_e$.
	We now define
	\begin{align*}
		\varphi : G \to \Sym(X);\quad \varphi(g) = \varphi_g \implies \varphi(g)(x) = g \ast x
	\end{align*}
	This is a homomorphism.

	Conversely, any group homomorphism $\varphi : G \to \Sym(X)$ induces a group action $\ast$ by $g \ast x = \varphi(g)$.
	This yields $e \ast x = \varphi(e)(x) = \id x = x$ and $(g_1 g_2) \ast x = \varphi(g_1 g_2) x = \varphi(g_1) \varphi(g_2) x = g_1 \ast (g_2 \ast x)$ as required.
\end{proof}

\begin{definition}[Permutation Representation]
	The homomorphism $\varphi : G \to \Sym(X)$ defined in the above proof is called a \vocab{permutation representation} of $G$.
\end{definition}

\begin{definition}[Orbit, Stabiliser]
	Let $G$ act on $X$.
	Then,
	\begin{enumerate}
		\item the \vocab{orbit} of $x \in X$ is $\Orb_G(x) = \qty{g \ast x : g \in G} \subseteq X$;
		\item the \vocab{stabiliser} of $x \in X$ is $G_x = \qty{g \in G : g \ast x = x} \leq G$.
	\end{enumerate}
\end{definition}

\begin{definition}[Transitive Group Action]
	If there is only orbit, i.e. $\Orb_G(x) = X \quad \forall \; x$ then the group action is \vocab{transitive}.
\end{definition} 

\begin{definition}[Kernel]
	The \vocab{kernel} of a permutation representation is $\bigcap_{x \in X} G_x$.
\end{definition} 

\begin{remark}
	The kernel of the permutation representation $\varphi$ is also referred to as the kernel of the group action itself.
\end{remark} 

\begin{definition}[Faithful Group Action]
	If the kernel is trivial the action is said to be \vocab{faithful}.
\end{definition} 

\begin{theorem}[Orbit-stabiliser theorem]
	The orbit $\Orb_G(x)$ bijects with the set $\faktor{G}{G_x}$ of left cosets of $G_x$ in $G$ (which may not be a quotient group).
	In particular, if $G$ is finite, we have
	\begin{align*}
		\abs{G} = \abs{\Orb(x)} \cdot \abs{G_x}
	\end{align*}
\end{theorem}

\begin{example}
	If $G$ is the group of symmetries of a cube and we let $X$ be the set of vertices in the cube, $G$ acts on $X$.
	Here, for all $x \in X$, $\abs{\Orb(x)} = 8$ and $\abs{G_x} = 6$ (including reflections), hence $\abs{G} = 48$.
\end{example}

\begin{remark}
	The orbits partition $X$.

	Note that $G_{g \ast x} = g G_x g\inv$.
	Hence, if $x, y$ lie in the same orbit, their stabilisers are conjugate.
\end{remark}

\subsection{Examples}
\begin{example}
	$G$ acts on itself by left multiplication.
	This is known as the \vocab{left regular action}.
	The kernel is trivial, hence the action is faithful.
	The action is transitive, since for all $g_1, g_2 \in G$, the element $g_2 g_1\inv$ maps $g_1$ to $g_2$.
\end{example}

\begin{theorem}[Cayley's theorem]
	Any finite group $G$ is a permutation group of order $\abs{G}$; it is isomorphic to a subgroup of $S_\abs{G}$.
\end{theorem}

\begin{example}
	Let $H \leq G$.
	Then $G$ acts on $\faktor{G}{H}$ by left multiplication, where $\faktor{G}{H}$ is the set of left cosets of $H$ in $G$.
	This is known as the \vocab{left coset action}.
	This action is transitive using the construction above for the left regular action.
	We have $\ker\varphi = \bigcap_{x \in G} xHx\inv$, which is the largest normal subgroup of $G$ contained within $H$.
\end{example}

\begin{theorem}
	Let $G$ be a non-abelian simple group, and $H \leq G$ with index $n > 1$.
	Then $n \geq 5$ and $G$ is isomorphic to a subgroup of $A_n$.
\end{theorem}

\begin{proof}
	Let $G$ act on $X = \faktor{G}{H}$ by left multiplication.
	Let $\varphi : G \to \Sym(X)$ be the permutation representation associated to this group action.

	Since $G$ is simple, $\ker \varphi = 1$ or $\ker \varphi = G$.
	If $\ker \varphi = G$, then $\Im\varphi = 1_{S_n}$, which is a contradiction since $G$ acts transitively on $X$ and $|X| > 1$.
	Thus $\ker \varphi = 1$, and $G \cong \Im\varphi \leq S_n$.

	Since $G \leq S_n$ and $A_n \triangleleft S_n$, the second isomorphism theorem shows that $G \cap A_n \triangleleft G$, and
	\begin{align*}
		\faktor{G}{G \cap A_n} \cong \faktor{GA_n}{A_n} \leq \faktor{S_n}{A_n} \cong C_2
	\end{align*}
	Since $G$ is simple, $G \cap A_n = 1$ or $G$.
	If $G \cap A_n = 1$, then $G$ is isomorphic to a subgroup of $C_2$, but this is false, since $G$ is non-abelian.
	Hence $G \cap A_n = G$ so $G \leq A_n$.
	Finally, if $n \leq 4$ we can check manually that $A_n$ is not simple; $A_n$ has no non-abelian simple subgroups.
\end{proof}

\subsection{Conjugation actions}
\begin{example}
	Let $G$ act on $G$ by conjugation, so $g \ast x = g x g\inv$.
	This is known as the \vocab{conjugation action}.
\end{example}

\begin{definition}[Conjugacy Class, Centraliser, Centre]
	The orbit of the conjugation action is called the \vocab{conjugacy class} of a given element $x \in G$, written $\ccl_G(x)$.
	The stabiliser of the conjugation action is the set $C_x$ of elements which commute with a given element $x$, called the \vocab{centraliser} of $x$ in $G$.
	The kernel of $\varphi$ is the set $Z(G)$ of elements which commute with all elements in $x$, which is the \vocab{centre} of $G$.
	This is always a normal subgroup.
\end{definition}

\begin{remark}
	$\varphi : G \to G$ satisfies
	\begin{align*}
		\varphi(g)(h_1 h_2) = g h_1 h_2 g\inv = h h_1 g\inv g h_2 g\inv = \varphi(g)(h_1) \varphi(g)(h_2)
	\end{align*}
	Hence $\varphi(g)$ is a group homomorphism for all $g$.
	It is also a bijection, hence $\varphi(g)$ is an isomorphism from $G \to G$.
\end{remark}

\begin{definition}[Automorphism]
	An isomorphism from a group to itself is known as an \vocab{automorphism}.
	We define $\mathrm{Aut}(G)$ to be the set of all group automorphisms of a given group.
	This set is a group.
	Note, $\mathrm{Aut}(G) \leq \Sym(G)$, and the $\varphi : G \to \Sym(G)$ above has image in $\mathrm{Aut}(G)$.
\end{definition}

\begin{example}
	Let $X$ be the set of subgroups of $G$.
	Then $G$ acts on $X$ by conjugation: $g \ast H = g H g\inv$.
	The stabiliser of a subgroup $H$ is $\qty{ g \in G : gHg\inv = H } = N_G(H)$, called the \vocab{normaliser} of $H$ in $G$.
	The normaliser of $H$ is the largest subgroup of $G$ that contains $H$ as a normal subgroup.
	In particular, $H \triangleleft G$ if and only if $N_G(H) = G$.
\end{example}
