\section{Introduction}

\begin{notation}
	$A_n \uparrow A$ means that the sequence $A_n$ is increasing ($A_1 \subseteq A_2 \subseteq \dots$) and $\bigcup_n A_n = A$.
\end{notation}

\subsection{Definitions}

\begin{definition}[$\sigma$-algebra]
	Let $E$ be a (nonempty) set. A collection $\mathcal E$ of subsets of $E$ is called a \vocab{$\sigma$-algebra} if the following properties hold:
	\begin{itemize}
		\item $\emptyset \in \mathcal E$;
		\item $A \in \mathcal E \implies A^c = E \setminus A \in \mathcal E$;
		\item if $(A_n)_{n \in \mathbb N}$ is a countable collection of sets in $\mathcal E$, $\bigcup_{n \in \mathbb N} A_n \in \mathcal E$.
	\end{itemize}
\end{definition}

\begin{example}
	Let $\mathcal E = \qty{\emptyset, E}$.
	This is a $\sigma$-algebra.
	Also, $\mathcal P(E) = \qty{A \subseteq E}$ is a $\sigma$-algebra.
\end{example}

\begin{remark}
	Since $\bigcap_n A_n = \qty(\bigcup_n A_n^c)^c$, any $\sigma$-algebra $\mathcal E$ is closed under countable intersections as well as under countable unions.
	Note that $B \setminus A = B \cap A^c \in \mathcal E$, so $\sigma$-algebras are closed under set difference.
\end{remark}

\begin{definition}[Measurable Space and Set]
	A set $E$ with a $\sigma$-algebra $\mathcal E$ is called a \vocab{measurable space}.
	The elements of $\mathcal E$ are called \vocab{measurable sets}.
\end{definition}

\begin{definition}[Measure]
	A \vocab{measure} $\mu$ is a set function $\mu : \mathcal E \to [0,\infty]$, such that $\mu(\emptyset) = 0$, and for a sequence $(A_n)_{n \in \mathbb N}$ such that the $A_n$ are disjoint, we have
	\[ \mu\qty(\bigcup_{n \in \mathbb N} A_n) = \sum_{n \in \mathbb N} \mu(A_n) \]
	This is the \vocab{countable additivity} property of the measure.
\end{definition}

\begin{remark}
	$(E, \mathcal{E}, \mu)$ is a measure space.
\end{remark}

\begin{remark}
	If $E$ is countable, then for any $A \in \mathcal P(E)$ and measure $\mu$, we have
	\[ \mu(A) = \mu\qty(\bigcup_{x\in A} \qty{x}) = \sum_{x \in A} \mu(\qty{x}) \]
	Hence, measures are uniquely defined by the measure of each singleton.

	Define $m : E \to [0, \infty]$ s.t. $m(x) = \mu(\{x\})$, such an $m$ is called a ``mass function'', and measures $\mu$ are in $1$-$1$ correspondence with the mass function $m$.
	This corresponds to the notion of a probability mass function.

	Here $\mathcal{E} = \mathcal{P}(E)$ and this is the theory in elementary discrete prob. (when $\mu(\{x\}) = 1 \; \forall x \in E$, $\mu$ is called the counting measure. Here $\mu(A) = |A| \; \forall A \subset E$).

	For uncountable $E$  however, the story is not so simple and $\mathcal{E} = \mathcal{P}(E)$ is generally not feasible. Indeed measures are defined on $\sigma$-algebra ``generated'' by a smaller class $\mathcal{A}$ of simple subsets of $E$.
\end{remark}

\begin{definition}[Generated $\sigma-$algebra]
	For a collection $\mathcal A$ of subsets of $E$, we define the $\sigma$-algebra \vocab{$\sigma(A)$ generated by $\mathcal A$} by
	\[ \sigma(\mathcal A) = \qty{A \subseteq E \colon A \in \mathcal E \text{ for all $\sigma$-algebras } \mathcal E \supseteq \mathcal A} \]
	So it is the smallest $\sigma$-algebra containing $\mathcal A$.
	Equivalently,
	\[ \sigma(\mathcal A) = \bigcap_{\mathcal E \supseteq \mathcal A,\ \mathcal E \text{ a $\sigma$-algebra}} \mathcal E \]
\end{definition}

\begin{question}
	Why is $\sigma(A)$ a $\sigma$-algebra? See Sheet 1, Q1.
\end{question}

\subsection{Rings and algebras}
The class $\mathcal{A}$ will usually satisfy some properties too, let $E$ be a set and $\mathcal{A}$ a collection of subsets of $E$.
To construct good generators, we define the following.

\begin{definition}[Ring]
	$\mathcal A \subseteq \mathcal P(E)$ is called a \vocab{ring} over $E$ if $\emptyset \in \mathcal A$ and $A, B \in \mathcal A$ implies $B \setminus A \in \mathcal A$ and $A \cup B \in \mathcal A$.
\end{definition}

Rings are easier to manage than $\sigma$-algebras because there are only finitary operators.

\begin{definition}[Algebra]
	$\mathcal A$ is called an \vocab{algebra} over $E$ if $\emptyset \in \mathcal A$ and $A, B \in \mathcal A$ implies $A^c \in \mathcal A$ and $A \cup B \in \mathcal A$.
\end{definition}

\begin{remark}
	Rings are closed under symmetric difference $A \symmdiff B = (B \setminus A) \cup (A \setminus B)$, and are closed under intersections $A \cap B = A \cup B \setminus A \symmdiff B$.
	Algebras are rings, because $B \setminus A = B \cap A^c = (B^c \cup A)^c$.
	Not all rings are algebras, because rings do not need to include the entire space.
\end{remark}

The idea:
\begin{itemize}
	\item Define a set function on a suitable collection $\mathcal{A}$.
	\item Extend the set function to a measure on $\sigma(\mathcal{A})$. (Carath\'eodory's Extension theorem)
	\item Such an extension is unique. (Dynkin's Lemma)
\end{itemize}

Goal: Start with a ``measure'' on $\mathcal{A}$ that has some nice properties and then extend it to $\sigma(A)$.

\begin{definition}[Set Function]
	A \vocab{set function} on a collection $\mathcal A$ of subsets of $E$, where $\emptyset \in \mathcal A$, is a map $\mu \colon \mathcal A \to [0,\infty]$ such that $\mu(\emptyset) = 0$.
	\begin{itemize}
		\item We say $\mu$ is \vocab{increasing} if $\mu(A) \leq \mu(B)$ for all $A \subseteq B$ in $\mathcal A$.
		\item We say $\mu$ is \vocab{additive} if $\mu(A \cup B) = \mu(A) + \mu(B)$ for disjoint $A, B \in \mathcal A$ and $A \cup B \in \mathcal A$.
		\item We say $\mu$ is \vocab{countably additive} if $\mu\qty(\bigcup_n A_n) = \sum_n \mu(A_n)$ for disjoint sequences $A_n$ where $\bigcup_n A_n$ and each $A_n$ lie in $\mathcal A$.
		\item We say $\mu$ is \vocab{countably subadditive} if $\mu\qty(\bigcup_n A_n) \leq \sum_n \mu(A_n)$ for arbitrary sequences $A_n$ under the above conditions.
	\end{itemize}
\end{definition}

\begin{remark}
	If $\mu$ is countably additive set function on $\mathcal{A}$ and $\mathcal{A}$ is a ring then $\mu$ satisfies all the previous listed properties.
\end{remark}

\begin{proposition}[Disjointification of countable unions]
	Consider $\bigcup_n A_n$ for $A_n \in \mathcal E$, where $\mathcal E$ is a $\sigma$-algebra (or a ring, if the union is finite).
	Then there exist $B_n \in \mathcal E$ that are disjoint such that $\bigcup_n A_n = \bigcup_n B_n$.
\end{proposition}

\begin{proof}
	Define $\widetilde A_n = \bigcup_{j \leq n} A_j$, then $B_n = \widetilde A_n \setminus \widetilde A_{n-1}$.
\end{proof}

\begin{remark} \label{rem:1}
	A measure satisfies all four of the above conditions. Countable additivity implies the other conditions. Proof on Sheet 1.
\end{remark}

\begin{theorem}[Carath\'eodory's theorem] \label{thm:car}
	Let $\mu$ be a countably additive set function on a ring $\mathcal A$ of subsets of $E$.
	Then there exists a measure $\mu^\star$ on $\sigma(\mathcal A)$ such that $\eval{\mu^\star}_{\mathcal A} = \mu$.
\end{theorem}

We will later prove that this extended measure is unique.

\begin{proof}[Proof (Non Examinable)]
	For $B \subseteq E$, we define the \vocab{outer measure} $\mu^\star$ as
	\[ \mu^\star(B) = \inf \qty{\sum_{n \in \mathbb N} \mu(A_n) : A_n \in \mathcal A, B \subseteq \bigcup_{n \in \mathbb N} A_n} \]
	If there is no sequence $A_n$ such that $B \subseteq \bigcup_{n \in \mathbb N} A_n$, we declare the outer measure $\mu^\star(B)$ to be $\infty$.
	Clearly, $\mu^\star(\emptyset)$ and $\mu^\star$ is increasing, so $\mu^\star$ is an increasing set fcn on $\mathcal{P}(E)$.

	\begin{definition}[$\mu^\star$ measurable]
		A set $A \subseteq E$ \vocab{$\mu^\star$ measurable} if $\forall B \subseteq E \ \mu^\star(B) = \mu^\star(B \cap A) + \mu^\star(B \cap A^c)$.
	\end{definition}

	We define the class
	\[ \mathcal M = \qty{A \subseteq E : A \text{ is $\mu^\star$ measurable}} \]
	We shall show that $M$ is a $\sigma$-algebra that contains $\mathcal{A}$, $\mu^\star \mid_M$ is a measure on $M$ that extends $\mu$ (i.e. $\eval{\mu^\star}_\mathcal{A} = \mu$).

	\emph{Step 1.} $\mu^\star$ is countably sub-additive on $\mathcal P(E)$:
	It suffices to prove that for $B \subseteq E$ and $B_n \subseteq E$ such that $B \subseteq \bigcup_n B_n$ we have
	\begin{equation}
		\mu^\star(B) \leq \sum_n \mu^\star(B_n)
		\tag{\(\dagger\)}
	\end{equation}
	We can assume without loss of generality that $\mu^\star(B_n) < \infty$ for all $n$, otherwise there is nothing to prove.
	For all $\epsilon > 0$ there exists a collection $A_{n,m} \in \mathcal{A}$ such that $B_n \subseteq \bigcup_m A_{n,m}$ and
	\[ \mu^\star(B_n) + \frac{\epsilon}{2^n} \geq \sum_m \mu(A_{n,m}) \]
	as we took an infimum.
	Now, since $\mu^\star$ is increasing, and $B \subseteq \bigcup_n B_n \subseteq \bigcup_n \bigcup_m A_{n,m}$, we have
	\[ \mu^\star(B) \leq \mu^\star\qty(\bigcup_{n,m} A_{n,m}) \leq \sum_{n,m} \mu(A_{n,m}) \leq \sum_n \mu^\star(B_n) + \sum_n \frac{\epsilon}{2^n} = \sum_n \mu^\star(B_n) + \epsilon \]
	Since $\epsilon$ was arbitrary in the construction, $(\dagger)$ follows by construction.

	\emph{Step 2.} $\mu^\star$ extends $\mu$:
	Let $A \in \mathcal A$, and we want to show $\mu^\star(A) = \mu(A)$.

	We can write $A = A \cup \emptyset \cup \dots$, hence $\mu^\star(A) \leq \mu(A) + 0 + \dots = \mu(A)$ by definition of $\mu^\star$.

	If $\mu^\star$ is infinite, there is nothing to prove.

	We need to prove the converse, that $\mu(A) \leq \mu^\star(A)$.
	For the finite case, suppose there is a sequence $A_n$ where $\mu(A_n) < \infty$ and $A \subseteq \bigcup_n A_n$.
	Then, $A = \bigcup_n (A \cap A_n)$, which is a union of elements of the ring $\mathcal A$.
	As $\mu$ is countably additive on $\mathcal{A}$ and $\mathcal{A}$ is a ring, $\mu$ is countably subadditive on $\mathcal{A}$ and increasing by \cref{rem:1}.
	Hence $\mu(A) \leq \sum_n \mu(A \cap A_n) \leq \sum_n \mu(A_n)$.
	Since the $A_n$ were arbitrary taking the infimum over $A_n$, we have $\mu(A) \leq \mu^\star(A)$ as required.

	\emph{Step 3.} $\mathcal M \supseteq \mathcal A$:
	Let $A \in \mathcal A$.
	We must show that for all $B \subseteq E$, $\mu^\star(B) = \mu^\star(B \cap A) + \mu^\star(B \cap A^c)$.

	We have $B \subseteq (B \cap A) \cup (B \cap A^c) \cup \emptyset \cup \dots$, hence by countable subadditivity $(\dagger)$, $\mu^\star(B) \leq \mu^\star(B \cap A) + \mu^\star(B \cap A^c)$.

	It now suffices to prove the converse, that $\mu^\star(B) \geq \mu^\star(B \cap A) + \mu^\star(B \cap A^c)$. \\
	We can assume $\mu^\star(B)$ is finite, and so $\forall \epsilon > 0 \; \exists A_n \in \mathcal A$ s.t. $B \subseteq \bigcup_n A_n$ and $\mu^\star(B) + \epsilon \geq \sum_n \mu(A_n)$.
	Now, $B \cap A \subseteq \bigcup_n (A_n \cap A)$, and $B \cap A^c \subseteq \bigcup_n (A_n \cap A^c)$.
	All of the members of these two unions are elements of $\mathcal A$, since $A_n \cap A^c = A_n \setminus A$.
	Therefore,
	\begin{align*}
		\mu^\star(B \cap A) + \mu^\star(B \cap A^c) &\leq \sum_n \mu(A_n \cap A) + \sum_n \mu(A_n \cap A^c) \\
		&\leq \sum_n \qty[ \mu(A_n \cap A) + \mu(A_n \cap A^c) ] \\
		&\leq \sum_n \mu(A_n) \leq \mu^\star(B) + \epsilon
	\end{align*}
	Since $\epsilon$ was arbitrary, $\mu^\star(B) = \mu^\star(B \cap A) + \mu^\star(B \cap A^c)$ as required.

	\emph{Step 4.} $\mathcal M$ is an algebra:
	Clearly $\emptyset$ lies in $\mathcal M$, and by the symmetry in the definition of $\mathcal M$, complements lie in $\mathcal M$.
	We need to check $\mathcal M$ is stable under finite intersections.
	Let $A_1, A_2 \in \mathcal M$ and let $B \subseteq E$.
	We have
	\begin{align*}
		\mu^\star(B) &= \mu^\star(B \cap A_1) + \mu^\star(B \cap A_1^c) \text{ as $A_1 \in M$} \\
		&= \mu^\star(B \cap A_1 \cap A_2) + \mu^\star(B \cap A_1 \cap A_2^c) + \mu^\star(B \cap A_1^c) \text{ taking $\tilde{B} = B \cap A_1$}
	\end{align*}
	We can write $A_1 \cap A_2^c = (A_1 \cap A_2^c)^c \cap A_1$, and $A_1^c = (A_1 \cap A_2)^c \cap A_1^c$.
	Hence
	\begin{align*}
		\mu^\star(B) &= \mu^\star(B \cap A_1 \cap A_2) + \underbracket{\mu^\star(B \cap (A_1 \cap A_2)^c \cap A_1) + \mu^\star(B \cap (A_1 \cap A_2)^c \cap A_1^c)}_{\mu^\star(B \cap (A_1 \cap A_2)^c) \text{ as } A_1 \in M} \\
		&= \mu^\star(B \cap A_1 \cap A_2) + \mu^\star(B \cap (A_1 \cap A_2)^c)
	\end{align*}
	which is the requirement for $A_1 \cap A_2$ to lie in $\mathcal M$.

	\emph{Step 5.} $\mathcal M$ is a $\sigma$-algebra and $\mu^\star$ is a measure on $\mathcal M$: \\
	It suffices now to show that $\mathcal M$ has countable unions and the measure respects these countable unions.
	Let $A = \bigcup_n A_n$ for $A_n \in \mathcal M$.
	Without loss of generality, let the $A_n$ be disjoint.
	We want to show $A \in \mathcal M$, and that $\mu^\star(A) = \sum_n \mu^\star(A_n)$.

	By $(\dagger)$, we have for any $B \subseteq E$ $\mu^\star(B) \leq \mu^\star(B \cap A) + \mu^\star(B \cap A^c) + 0 + \dots$ so we need to check only the converse of this inequality.
	Also, $\mu^\star(A) \leq \sum_n \mu^\star(A_n)$, so we need only check the converse of this inequality as well.
	Similarly to before,
	\begin{align*}
		\mu^\star(B) &= \mu^\star(B \cap A_1) + \mu^\star(B \cap A_1^c) \\
		&= \mu^\star(B \cap A_1) + \mu^\star(B \cap \underbracket{A_1^c \cap A_2}_{A_2 \text{ as $A_1, A_2$ disjoint}}) + \mu^\star(B \cap A_1^c \cap A_2^c) \\
		&= \mu^\star(B \cap A_1) + \mu^\star(B \cap A_2) + \mu^\star(B \cap A_1^c \cap A_2^c) \\
		&= \mu^\star(B \cap A_1) + \mu^\star(B \cap A_2) + \mu^\star(B \cap A_1^c \cap A_2^c \cap A_3) + \mu^\star(B \cap A_1^c \cap A_2^c \cap A_3^c) \\
		&= \mu^\star(B \cap A_1) + \mu^\star(B \cap A_2) + \mu^\star(B \cap A_3) + \mu^\star(B \cap A_1^c \cap A_2^c \cap A_3^c) \\
		&= \cdots \\
		&= \sum_{n \leq N} \mu^\star(B \cap A_n) + \mu^\star(B \cap A_1^c \cap \dots \cap A_N^c)
	\end{align*}
	Since $\bigcup_{n \leq N} A_n \subseteq A$, we have $\bigcap_{n \leq N} A_n^c \supseteq A^c$.
	$\mu^\star$ is increasing, hence, taking limits,
	\[ \mu^\star(B) \geq \sum_{n=1}^\infty \mu^\star(B \cap A_n) + \mu^\star(B \cap A^c) \]
	By $(\dagger)$,
	\[ \mu^\star(B) \geq \mu^\star(B \cap A) + \mu^\star(B \cap A^c) \]
	as required.
	Hence $\mathcal M$ is a $\sigma$-algebra.
	For the other inequality, we take the above result for $B = A$.
	\[ \mu^\star(A) \geq \sum_{n=1}^\infty \mu^\star(A \cap A_n) + \mu^\star(A \cap A^c) = \sum_{n=1}^\infty \mu^\star(A_n) \]
	So $\mu^\star$ is countably additive on $\mathcal M$ and is hence a measure on $\mathcal M$.
\end{proof}

\subsection{Uniqueness of extension}
To address uniqueness of extension, we introduce further subclasses of $\mathcal{P}(E)$. Let $\mathcal{A}$ be a collection of subsets of $E$.

\begin{definition}[$\pi$-system]
	A collection $\mathcal A$ of subsets of $E$ is called a \vocab{$\pi$-system} if $\emptyset \in \mathcal A$ and $A, B \in \mathcal A \implies A \cap B \in \mathcal A$.
\end{definition}

\begin{definition}[$d$-system] \label{def:d}
	A collection $\mathcal A$ of subsets of $E$ is called a \vocab{$d$-system} if
	\begin{itemize}
		\item $E \in \mathcal{A}$;
		\item $A, B \in \mathcal{A}$ and $A \subseteq B$ then $B \setminus A \in \mathcal{A}$;
		\item $A_n \in \mathcal{A}$ is an increasing sequence of sets then $\bigcup_n A_n \in \mathcal{A}$.
	\end{itemize}
\end{definition}

\begin{remark}
	Equivalently, $\mathcal{A}$ is a $d$-system if
	\begin{itemize}
		\item $\emptyset \in \mathcal{A}$;
		\item $A \in \mathcal{A} \implies A^c \in \mathcal{A}$
		\item $A_n \in \mathcal{A}$ is a sequence of disjoint sets then $\bigcup_n A_n \in \mathcal{A}$.
	\end{itemize}
	The difference between this and a $\sigma$-algebra is the requirement for disjoint sets.

	Proof on Sheet 1.
\end{remark}

\begin{proposition}
	A $d$-system which is also a $\pi$-system is a $\sigma$-algebra.
\end{proposition}

\begin{proof}
	Sheet 1.
\end{proof}

\begin{lemma}[Dynkin's Lemma/$\pi$-$\lambda$/$\pi$-$d$ theorem] \label{lem:dyn}
	Let $\mathcal A$ be a $\pi$-system.
	Then any $d$-system that contains $\mathcal A$ also contains $\sigma(\mathcal A)$.
\end{lemma}

\begin{proof}
	We define
	\[ \mathcal D = \bigcap_{\mathcal D' \text{ is a } d \text{-system};\; \mathcal D' \supseteq \mathcal A} \mathcal D' \]
	We can show this is a $d$-system (proof same as in $\sigma(\mathcal{A})$ on Sheet 1).
	It suffices to prove that $\mathcal D$ is a $\pi$-system, because then it is a $\sigma$-algebra\footnote{As $\mathcal{D} \supseteq \mathcal{A}$ and $\sigma(\mathcal{A})$ the intersection of all $\sigma$-algebras containing $\mathcal{A}$, $\mathcal{D} \supseteq \sigma(\mathcal{A})$.}.

	We now define
	\[ \mathcal D' = \qty{B \in \mathcal D : \forall A \in \mathcal A, B \cap A \in \mathcal D} \]
	We can see that $\mathcal A \subseteq \mathcal{D}'$, as $\mathcal A$ is a $\pi$-system.

	We now show that $\mathcal D'$ is a $d$-system, fix $A \in \mathcal{A}$.
	\begin{itemize}
		\item Clearly $E \cap A = A \in \mathcal A \subseteq \mathcal D'$ hence $E \in \mathcal D'$.
		\item Let $B_1, B_2 \in \mathcal D'$ such that $B_1 \subseteq B_2$.
		Then $(B_2 \setminus B_1) \cap A = (B_2 \cap A) \setminus (B_1 \cap A)$, and since $B_i \cap A \in \mathcal D$ this difference also lies in $\mathcal D$, so $B_2 \setminus B_1 \in \mathcal D'$.
		\item Now, suppose $B_n$ is an increasing sequence converging to $B$, and $B_n \in \mathcal D'$.
		Then $B_n \cap A \in \mathcal D$, and $\mathcal D$ is a $d$-system, we have $B \cap A \in \mathcal D$, so $B \in \mathcal D'$.
	\end{itemize}

	Hence $\mathcal D'$ is a $d$-system.
	Also, $\mathcal D' \subseteq \mathcal D$ by construction of $\mathcal D'$.
	But also $\mathcal{A} \subseteq \mathcal{D}'$ and $\mathcal{D}'$ is a $d$-system so $\mathcal{D} \subset \mathcal{D}'$ as $\mathcal{D}$ is the smallest $d$-system containing $\mathcal{A}$.
	Thus $\mathcal D = \mathcal D'$, i.e $\forall B \in \mathcal{D}$ and $A \in \mathcal{A}, B \cap A \in \mathcal{D} \ (\ast)$.

	We then define
	\[ \mathcal D'' = \qty{B \in \mathcal D : \forall A \in \mathcal D, B \cap A \in \mathcal D} \]
	Note that $\mathcal A \subseteq \mathcal D''$ by $(\ast)$.
	Running the same argument as before, we can show that $\mathcal D''$ is a $d$-system. So $\mathcal{D}'' = \mathbb{D}$.
	But then (by the definition of $\mathcal{D}''$), $\forall B \in \mathcal{D}, A \in \mathcal{D} \implies B \cap A \in \mathcal{D}$, i.e. $\mathcal{D}$ is a $\pi$-system (check that $\emptyset \in \mathcal{D}$).

	So $\mathcal{D}$ is a $\sigma$-algebra containing $\mathcal{A}$, hence $\mathcal{D} \supseteq \sigma(\mathcal{A})$.
\end{proof}

\begin{theorem}[Uniqueness of Extension] \label{thm:uni}
	Let $\mu_1, \mu_2$ be measures on a measurable space $(E, \mathcal E)$, such that $\mu_1(E) = \mu_2(E) < \infty$.
	Suppose that $\mu_1$ and $\mu_2$ coincide on a $\pi$-system $\mathcal A$, such that $\mathcal E \subseteq \sigma(\mathcal A)$.
	Then $\mu_1 = \mu_2$ on $\sigma(\mathcal A)$, and hence on $\mathcal E$.
\end{theorem}

\begin{proof}
	We define
	\[ \mathcal D = \qty{A \in \mathcal E : \mu_1(A) = \mu_2(A)} \]
	This collection contains $\mathcal A$ by assumption.
	By Dynkin's lemma, it suffices to prove $\mathcal D$ is a $d$-system, because then $\mathcal D \supseteq \sigma(\mathcal A) \supseteq \mathcal E$ giving $\mathcal D = \mathcal E$ as $\mathcal{D} \subseteq \mathcal{E}$.

	\begin{itemize}
		\item $\emptyset \in \mathcal{D}$, since $\mu_1(\emptyset) = \mu_2(\emptyset) = 0$;
		\item $A \in \mathcal{D} \implies \mu_1(A) = \mu_2(A)$, thus $\mu_1(A^c) = \mu_1(E) - \mu_1(A) = \mu_2(E) - \mu_2(A) = \mu_2(A^c)$, so $A^c \in \mathcal{D}$ ($\mu_1, \mu_2$ finite so this works);
		\item Let $A_n \in \mathcal{D}$ be a disjoint sequence then, $\mu_1(\bigcup_n A_n) = \sum \mu_1(A_n) = \sum \mu_2(A_n) = \mu_2(\bigcup_n A_n)$ by countable additivity. So $\bigcup_n A_n \in \mathcal{D}$.
	\end{itemize}
	So $\mathcal{D}$ is a $d$-system.
\end{proof}

\begin{remark}
	If $A_n \in \mathcal{A}$ an increasing sequence, then $\mu(\mathcal{A}) = \lim_{n \to \infty} \mu(A_n)$.
	Use this to show that $\mathcal{D}$ is a $d$-system satisfying conditions in \nameref{def:d}.

	The above theorem applies to finite measures ($\mu$ such that $\mu(E) < \infty$) only.
	However, the theorem can be extended to measures that are $\sigma$-finite, for which $E = \bigcup_{n \in \mathbb N} E_n$ where $\mu(E_n) < \infty$.
\end{remark}

\begin{question}
	How to show all sets of a $\sigma$-algebra $\mathcal{E}$ generated by $\mathcal{A}$ has a certain property $\mathcal{P}$?
\end{question}

\begin{answer} \label{ans:1}
	Consider set $\mathcal{G} = \{A \subseteq E : A \text{ has the property } \mathcal{P}\}$ and have that all elements of $\mathcal{A}$ have the property $\mathcal{P}$.

	Method 1: Show that $\mathcal{G}$ is a $\sigma$-algebra, as it then must contain $\sigma(\mathcal{A}) = \mathcal{E}$.

	Method 2: Show that $\mathcal{G}$ is a $d$-system and pick $\mathcal{A}$ s.t. it is a $\pi$-system and use \nameref{lem:dyn}.

	Method 3: Monotone Convergence Theorem, we will see it shortly.
\end{answer}

\subsection{Borel measures}

\begin{definition}[Borel Sets]
	Let $(E, \tau)$ be a Hausdorff topological space.
	The $\sigma$-algebra generated by the open sets of $E$, i.e. $\sigma(\mathcal{A})$ where $\mathcal{A} = \{A \subseteq E : A \text{ open}\}$, is called the \vocab{Borel $\sigma$-algebra} on $E$, denoted $\mathcal B(E)$. \\
	A measure $\mu$ on $(E, \mathcal B(E))$ is called a \vocab{Borel measure on $E$}.

	Members of $\mathcal B(E)$ are called \vocab{Borel sets}.
\end{definition}

\begin{notation}
	We write $\mathcal B = \mathcal B(\mathbb R)$.
\end{notation}

\begin{definition}[Radon Measure]
	A \vocab{Radon measure} is a Borel measure $\mu$ on $E$ such that $\mu(K) < \infty$ for all $K \subseteq E$ compact.
\end{definition}
Note that in a Hausdorff space, compact sets are closed and hence measurable.

\begin{definition}[Probability Measure]
	If $\mu(E) = 1$, $\mu$ is called a \vocab{probability measure} on $E$, and $(E, \mathcal{E}, \mu)$ is called a probability space, typically denoted instead by $(\Omega, \mathcal{F}, \mathcal{P})$.
\end{definition}

\begin{definition}[Finite Measure]
	If $\mu(E) < \infty$, $\mu$ is a \vocab{finite measure} on $E$.
\end{definition}

\begin{definition}[$\sigma$-finite Measure]
	If $\exists$ sequence $E_n \in \mathcal{E}$ s.t. $\mu(E_n) < \infty \; \forall n$ and $E = \bigcup_n E_n$, then $\mu$ is called a \vocab{$\sigma$-finite measure}.
\end{definition}

\begin{remark}
	Arguments that hold for finite measures can usually be extended to $\sigma$-finite measures.
\end{remark}

\subsection{Lebesgue measure}
One of the main goals for this course is to define a notion of volume for arbitrary sets, we can do this by constructing a Borel measure $\mu$ on $\mathcal{B}(\mathbb{R}^d)$ s.t $\mu \left( \prod_{i=1}^d (a_i, b_i) \right) = \prod_{i=1}^d (b_i - a_i)$ where $a_i < b_i$ corresponding to the usual notion of volume of rectangles.

% We will construct a unique Borel measure $\mu$ on $\mathbb R^d$ such that
% \[ \mu\qty(\prod_{i=1}^d [a_i, b_i]) = \prod_{i=1}^d \abs{b_i - a_i} \]
Initially, we will perform this construction for $d = 1$, and later we will consider product measures to extend this to higher dimensions.

\begin{theorem}[Construction of the Lebesgue measure]
	There exists a unique Borel measure $\mu$ on $\mathbb R$ such that
	\begin{align*}
		a < b \implies \mu \left( (a,b] \right) = b - a. \tag{$\dagger$}
	\end{align*}
	$\mu$ is called the Lebesgue measure on $\mathbb{R}$.
\end{theorem}

\begin{proof}
	First we shall prove the existence of the measure and then uniqueness.

	Consider the ring $\mathcal{A}$ of finite unions of disjoint intervals\footnote{We take semi intervals as for $\mathcal{A}$ to be a ring, we require the set difference to be in $\mathcal{A}$.} of the form
	\[ \mathcal{A} = (a_1,b_1] \cup \dots \cup (a_n,b_n] \]
	where $a_1 \leq b_1 \leq a_2 \leq \dots \leq a_n \leq b_n$.
	Note that $\sigma(\mathcal{A}) = \mathcal{B}$ (see Example Sheets\footnote{as all open intervals are in $\sigma(\mathcal{A})$ and open intervals generate open sets}).

	Define for each $A \in \mathcal{A}$
	\begin{align*}
		\mu(A) = \sum_{i=1}^{n}  (b_i - a_i).
	\end{align*}
	This agrees with $(\dagger)$ for $(a, b]$.
	This is additive and well-defined (check).

	% $\mu$ is additive, and well-defined since if $A = \bigcup_j C_j = \bigcup_k D_k$ for distinct disjoint unions, we can write $C_j = \bigcup_k (C_j \cap D_k)$ and $D_k = \bigcup_j (D_k \cap C_j)$, giving
	% \[ \mu(A) = \mu\qty(\bigcup_j C_j) = \sum_j \mu(C_j) = \sum_j \mu\qty(\bigcup_j (C_j \cap D_k)) = \sum_j \sum_k \mu(C_j \cap D_k) = \mu\qty(\bigcup_k D_k) \]

	So, the existence of $\mu$ on $\sigma(\mathcal{A}) = \mathcal{B}$ follows from \nameref{thm:car} if we can show that $\mu$ is \emph{countable additive} on $\mathcal{A}$.

	\begin{remark}
		Suppose $\mu$ a finitely additive set function on a ring $\mathcal{A}$.
		Then $\mu$ is countable additive iff
		\begin{itemize}
			\item $A_n \uparrow\footnote{increasing sequence tending to $A$} A; A_n, A \in \mathcal{A} \implies \mu(A_n) \uparrow \mu(A)$ .
			\item In addition, if $\mu$ is finite and $A_n \downarrow A$ s.t. $A_n, A \in \mathcal{\mathcal{A}}$ then $\mu(A_n) \downarrow \mu(A)$\footnote{E.g. let $A_n = [n, \infty)$ with the Lebesgue measure then $A_n \downarrow \emptyset$. But $\mu(A_n) = \infty$ whilst $\mu(\emptyset) = 0$}.
		\end{itemize}
		See Example Sheet for proof.
	\end{remark}

	So showing $\mu$ is countably additive on $\mathcal{A}$ is equivalent to showing the following \\
	If $A_n \in \mathcal{A}, A_n \downarrow \emptyset$ then $\mu(A_n) \downarrow 0$.
	We require that $\mu$ is finite, as $A_n$ decreasing we require $A_1$ to have finite measure. ?????

	We shall prove this by contradiction.

	Suppose this is not the case, so there exist $\epsilon > 0$ and $B_n \in \mathcal A$ such that $B_n \downarrow \emptyset$ but $\mu(B_n) \geq 2\epsilon$ for infinitely many $n$ (and so wlog for all $n$). \\
	We can approximate $B_n$ from within by a sequence $\overline C_n\footnote{$\overline C_n$ means the closure of $C_n$, i.e. make it a closed set by including the left endpoint} \in \mathcal{A}$ s.t. $C_n \subseteq B_n$ and $\mu(B_n \setminus C_n) \leq \epsilon / 2^n$.
	Suppose $B_n = \bigcup_{i=1}^{N_n} (a_{ni},b_{ni}]$, then define $C_n = \bigcup_{i=1}^{N_n} (a_{ni}+\frac{2^{-n}\epsilon}{N_n}, b_{ni}]$.
	Note that the $C_n$ lie in $\mathcal A$, and $\mu(B_n \setminus C_n) \leq 2^{-n}\epsilon$.
	Since $B_n$ is decreasing, we have $B_N = \bigcap_{n \leq N} B_n$, and
	\begin{align*}
		B_N \setminus (C_1 \cap \dots \cap C_N) = B_n \cap \qty(\bigcup_{n \leq N} C_n^c) = \bigcup_{n \leq N} B_N \setminus C_n \subseteq \bigcup_{n \leq N} B_n \setminus C_n
	\end{align*}
	Since $\mu$ is increasing and finitely additive and thus subadditive on $\mathcal{A}$,
	\begin{align*}
		\mu(B_N \setminus (C_1 \cap \dots \cap C_N)) \leq \mu\qty(\bigcup_{n \leq N} B_n \setminus C_n) \leq \sum_{n \leq N} \mu(B_n \setminus C_n) \leq \sum_{n \leq N} 2^{-N}\epsilon \leq \epsilon
	\end{align*}

	Since $\mu(B_N) \geq 2\epsilon$, additivity implies that $\mu(C_1 \cap \dots \cap C_N) \geq \epsilon$.
	This means that $C_1 \cap \dots \cap C_N$ cannot be empty.
	We can add the left endpoints of the intervals, giving $K_N = \overline C_1 \cap \dots \cap \overline C_N \neq \emptyset$.
	By Analysis I, $K_N$ is a nested sequence of bounded nonempty closed intervals and therefore there is a point $x \in \mathbb R$ such that $x \in K_N$ for all $N$\footnote{As completeness of $\mathbb{R}$ implies $\bigcap_n K_n$ is closed and non empty.}.
	But $K_N \subseteq \overline C_N \subseteq B_N$, so $x \in \bigcap_N B_n$, which is a contradiction since $\bigcap_N B_N$ is empty.
	Therefore, a measure $\mu$ on $\mathcal B$ exists.

	Now we prove uniqueness.
	Suppose $\mu, \lambda$ are measures such that the measure of an interval $(a,b]$ is $b - a$.
	We define truncated measures for $A \in \mathcal{B}$
	\begin{align*}
		\mu_n(A) &= \mu \left( A \cap (n,n+1) \right) \\
		\lambda_n(A) &= \lambda \left( A \cap (n,n+1] \right)
	\end{align*}
	Then $\mu_n, \lambda_n$ are \emph{probability measures} on $\mathcal{B}$ and $\mu_n = \lambda_n$ on the $\pi$-system of intervals of the form $(a, b]$ with $a < b$\footnote{As $(a, b] \cap (c, d] = \emptyset$ or $(e, f]$.}.
	This $\pi$-system generates $\mathcal{B}$, so by the uniqueness theorem for finite measures (\cref{thm:uni}) $\mu_n = \lambda_n$ on $\mathcal B$.
	Hence $\forall A \in \mathcal{B}$
	\begin{align*}
		\mu(A) &= \mu\left(\bigcup_n A \cap (n,n+1]\right) \\
		&= \sum_{n \in \mathbb Z} \mu(A \cap (n,n+1]) \\
		&= \sum_{n \in \mathbb Z} \mu_n(A) \\
		&= \sum_{n \in \mathbb Z} \lambda_n(A) = \dots = \lambda(A)
	\end{align*}
\end{proof}

\begin{definition}[Lebesgue null set]
	A Borel set $B \in \mathcal B$ is called a \vocab{Lebesgue null set} if $\lambda(B) = 0$ where $\lambda$ is the Lebesgue measure.
\end{definition}

\begin{remark}
	A singleton $\qty{x}$ can be written as $\bigcap_n \left(x-\frac 1n, x\right]$, hence $\lambda({x}) = \lim_n \frac 1n = 0$.
	Hence singletons are null sets.
	In particular, $\lambda((a,b)) = \lambda((a,b]) = \lambda([a,b)) = \lambda([a,b])$.
	Any countable set $Q = \bigcup_q \qty{q}$ is a null set.
	Not all null sets are countable; the Cantor set is an example.

	The Lebesgue measure is \emph{translation-invariant}.
	Let $x \in \mathbb R$, then the set $B + x = \qty{b + x : b \in B}$ lies in $\mathcal B$ iff $B \in \mathcal B$, and in this case, it satisfies $\lambda(B + x) = \lambda(B)$.
	We can define the translated Lebesgue measure $\lambda_x(B) = \lambda(B + x)$ for all $B \in \mathcal B$, then $\lambda_x((a,b]) = \lambda((a, b] + x) = \lambda((a + x, b+x]) = b - a = \lambda((a, b])$.
	So $\lambda_x = \lambda$ on the $\pi$-system of intervals and so $\lambda_x = \lambda$ on the sigma algebra $\mathcal{B}$ (i.e. $\forall B \in \mathcal{B}, \lambda(B+x) = \lambda(B)$).

	\begin{question}
		Is the Lebesgue measure the only such translation invariant measure on $\mathcal{B}$?
	\end{question}

	Carath\'eodory's theorem extends $\lambda$ from $\mathcal{A}$ to not just $\sigma(\mathcal{A}) = \mathcal{B}$, but actually to $\mathcal{M}$, the set of outer-measurable sets $M \supseteq \mathcal{B}$, but how large is $\mathcal{M}?$

	The class of outer measurable sets $\mathcal M$ used in Carath\'eodory's extension theorem is here called the class of Lebesgue measurable sets.
	This class, the Lebesgue $\sigma$-algebra, can be shown to be
	\[ \mathcal M = \qty{A \cup N : A \in \mathcal B, N \subseteq B, B \in \mathcal B, \lambda(B) = 0 } \supsetneq \mathcal B \]
\end{remark}

\subsection{Existence of non-measurable sets}

We now show that $\mathcal{B} \subsetneq \mathcal{P}(\mathbb{R})$ (in fact $\mathcal{M}_{leb} \subsetneq \mathcal{P}(\mathbb{R})$).

% Assuming the axiom of choice, there exists a non-measurable set of reals.
Consider $E = [0,1)$ with addition defined modulo one.
By the same argument as before, the Lebesgue measure is translation-invariant modulo one.
Consider the subgroup $Q = E \cap \mathbb Q$ of $(E, +)$.
We define $x \sim y$ for $x, y \in E$ if $x - y \in Q$.
% Then, this gives equivalence classes $[x] = \qty{y \in E \colon x \sim y}$ for all $x \in E$.
Assuming the axiom of choice (uncountable version), we can select a representative from each equivalence class, and denote by $S$ the set of such representatives.
We shall show that $S \notin \mathcal{B}$.

We can partition $E$ into the union of its cosets, so $E = \bigcup_{q \in Q} (S + q)$ is a disjoint\footnote{Suppose $s_1 + q_1 = s_2 + q_2$ then $s_1 - s_2 = q_1 - q_2 \in \mathbb{Q}$ but then $s_1, s_2 \in S$ by definition \Lightning.} union.

Suppose $S$ is a Borel set.
Then $S + q$ is also a Borel set\footnote{Consider $\mathcal{G} = \qty{B \in \mathcal{B} : B + x \in \mathcal{B}}$ we can show this is a $\sigma$-algebra, see \cpageref{ans:1}.}.
Therefore by translation invariance of $\lambda$ and by countably additivity,
\begin{align*}
	\lambda([0, 1)) = 1 = \lambda\qty(\bigcup_{q \in Q}(S+q)) = \sum_{q \in Q} \lambda(S+q) = \sum_{q \in Q} \lambda(S)
\end{align*}
But no value for $\lambda(S) \in [0,\infty]$ can be assigned to make this equation hold.
Therefore $S$ is not a Borel set.

\begin{remark}
	We can extend this proof to show that $S \notin \mathcal{M}_{leb}$.
\end{remark}

One can further show that $\lambda$ cannot be extended to all subsets $\mathcal P(E)$.
\begin{theorem}[Banach - Kuratowski]
	Assuming the continuum hypothesis, there exists no measure $\mu$ on the set $\mathcal P([0,1))$ such that $\mu([0,1)) = 1$ and $\mu(\qty{x}) = 0$ for $x \in [0,1)$.
\end{theorem}

Henceforth, whenever we are on a metric space $E$, we will work with $\mathcal{B}(E)$, which will be perfectly satisfactory.

\subsection{Probability spaces}

\begin{definition}
	If a measure space $(E, \mathcal E, \mu)$ has $\mu(E) = 1$, we call it a \vocab{probability space}, and instead write $(\Omega, \mathcal F, \mathbb P)$.
	We call $\Omega$ the outcome space or sample space, $\mathcal F$ the set of events, and $\mathbb P$ the probability measure.
\end{definition}

The axioms of probability theory (Kolmogorov, 1933), are

\begin{enumerate}
	\item $\prob{\Omega} = 1, \mathbb{P}(\emptyset) = 0$;
	\item $0 \leq \prob{E} \leq 1$ for all $E \in \mathcal F$;
	\item if $A_n$ are a disjoint sequence of events in $\mathcal F$, then $\prob{\bigcup_n A_n} = \sum_n \prob{A_n}$.
\end{enumerate}

This is exactly what is required by our definition: $\mathbb P$ is a measure on a $\sigma$-algebra.

\begin{remark} \
	\begin{itemize}
		\item $\prob{\bigcup_n A_n} \leq \sum_n \prob{A_n}$ for all sequences $A_n \in \mathcal{F}$;
		\item $A_n \uparrow A \implies \mathbb{P}(A_n) \uparrow \mathbb{P}(A)$;
		\item $A_n \downarrow A \implies \mathbb{P}(A_n) \downarrow \mathbb{P}(A)$ as $\mathbb{P}$ a finite measure.
	\end{itemize}
\end{remark}

This definition is what separates probability from analysis.
\begin{definition}[Independent]
	Events $(A_i, i \in I), A_i \in \mathcal{F}$ are \vocab{independent} if for all finite $J \subseteq I$, we have
	\begin{align*}
		\prob{\bigcap_{j \in J} A_j} = \prod_{j \in J} \prob{A_j}.
	\end{align*}
	$\sigma$-algebras $(\mathcal{A}_i, i \in I), A_i \subseteq \mathcal{F}$ are \vocab{independent} if for any $A_j \in \mathcal A_j$, where $J \subseteq I$ is finite, the $A_j$ are independent.
\end{definition}

Kolmogorov showed that these definitions are sufficient to derive the law of large numbers.

\begin{proposition}
	Let $\mathcal A_1, \mathcal A_2$ be $\pi$-systems of sets in $\mathcal F$.
	Suppose $\prob{A_1 \cap A_2} = \prob{A_1} \prob{A_2}$ for all $A_1 \in \mathcal A_1, A_2 \in \mathcal A_2$.
	Then the $\sigma$-algebras $\sigma(\mathcal A_1), \sigma(\mathcal A_2)$ are independent.
\end{proposition}

\begin{proof}
	Fix $A_1 \in \mathcal{A}_1$, and define for all $A \in \sigma(\mathcal{A}_2)$.
	\begin{align*}
		\mu(A) = \mathbb{P}(A_1 \cap A),\ \nu(A) = \mathbb{P}(A_1) \mathbb{P}(A).
	\end{align*}
	Then $\mu, \nu$ are finite measures and they agree on the $\pi$-system $\mathcal{A}_2$. Hence by \nameref{thm:uni}, $\mu(A) = \nu(A) \; \forall A \in \sigma(\mathcal{A}_2)$, i.e. $\mathbb{P}(A_1 \cap A) = \mathbb{P}(A_1)\mathbb{P}(A) \; \forall A_1 \in \mathcal{A}_1, A_2 \in \sigma(\mathcal{A}_2)$.

	Now repeat same argument, but now by fixing $A_2 \in \mathcolor{red}{\sigma(\mathcal{A}_2)}$ define for all $A \in \sigma(\mathcal{A}_1)$
	\begin{align*}
		\mu'(A) = \mathbb{P}(A \cap A_2),\ \nu'(A) = \mathbb{P}(A) \mathbb{P}(A_2).
	\end{align*}
	Then $\mu', \nu'$ are finite measures and they agree on the $\pi$-system $\mathcal{A}_1$. Hence by \nameref{thm:uni}, $\mu'(A) = \nu'(A) \; \forall A \in \sigma(\mathcal{A}_1)$, i.e. $\mathbb{P}(A_1 \cap A) = \mathbb{P}(A_1)\mathbb{P}(A) \; \forall A_1 \in \sigma(\mathcal{A}_1), A_2 \in \sigma(\mathcal{A}_2)$.

\end{proof}

This follows by uniqueness.

\subsection{Borel--Cantelli lemmas}

\begin{definition}
	Let $A_n \in \mathcal F$ be a sequence of events.
	Then the \vocab{limit superior} of $A_n$ is
	\begin{align*}
		\limsup_n A_n = \bigcap_n \bigcup_{m \geq n} A_m = \qty{A_n \text{ infinitely often}}\footnote{Consider $\omega$, if $\omega \in \limsup_n A_n$ then $\forall n, \omega \in \bigcup_{m \geq n} A_m$ thus $\omega$ must be in an infinite number of $A_n$s.}
	\end{align*}
	The \vocab{limit inferior} of $A_n$ is
	\[ \liminf_n A_n = \bigcup_n \bigcap_{m \geq n} A_m = \qty{A_n \text{ eventually}}\footnote{$\omega$ is in all but finitely many $A_n$.} \]
\end{definition}

\begin{lemma}[First Borel--Cantelli lemma]
	Let $A_n \in \mathcal F$ be a sequence of events such that $\sum_n \prob{A_n} < \infty$.
	Then $\prob{A_n \text{ infinitely often}} = 0$.
\end{lemma}

\begin{proof}
	For all $n$, we have
	\[ \prob{\limsup_n A_n} = \prob{\bigcap_n \bigcup_{m \geq n} A_m} \leq \prob{\bigcup_{m \geq n} A_m} \leq\footnote{By countable subadditivity} \sum_{m \geq n} \prob{A_m} \to 0 \]
\end{proof}

This proof did not require that $\mathbb P$ be a probability measure, just that it is a measure.
Therefore, we can use this for arbitrary measures.

\begin{lemma}[Second Borel--Cantelli lemma]
	Let $A_n \in \mathcal F$ be a sequence of independent events with $\sum_n \prob{A_n} = \infty$.
	Then $\prob{A_n \text{ infinitely often}} = 1$.
\end{lemma}

\begin{proof}
	By independence, for all $N \geq n \in \mathbb N$ and using $1 - a \leq e^{-a}$, we find
	\[ \prob{\bigcap_{m=n}^N A_m^c} = \prod_{m=n}^N \qty(1 - \prob{A_m}) \leq \prod_{m=n}^N e^{-\prob{A_m}} = e^{-\sum_{m=n}^N \prob{A_m}} \]
	As $N \to \infty$, this approaches zero. \\
	Since $\bigcap_{m=n}^N A_m^c$ decreases to $\bigcap_{m=n}^\infty A_m^c$, $\prob{\bigcap_{m=n}^\infty A_m^c} = 0$ as $\prob{\bigcap_{m=n}^\infty A_m^c} \leq \prob{\bigcap_{m=n}^N A_m^c} \leq e^{-\sum_{m=n}^N \prob{A_m}} \to 0$.
	So by taking complements $\mathbb{P}(\bigcup_{m=n}^\infty A_n) = 1 \;\forall n (\dagger)$.

	Let $B_n = \bigcup_{m=n}^\infty A_m$, $B_n$ decreasing and so $B_n \downarrow \bigcap_n B_n = \bigcap_n \bigcup_{m \geq n} A_m = \{A_n \text{ i.o}\}\footnote{$A_n$ occurs infinitely often}$.
	As $\mathbb{P}(B_n) = 1$ by $(\dagger)$, $\mathbb{P}{\{A_n \text{ i.o}\}} = \lim_{n \to \infty} \mathbb{P}(B_n) = 1$ as probabilities are a finite measure.
\end{proof}

\begin{remark}
	If $A_n$ independent, then $\{A_n \text{ i.o}\}$ has either probability $0$ or $1$ and is called a ``tail event''.
	Kolmogorov 0-1 law shows this is true for all ``tail events''.
\end{remark}