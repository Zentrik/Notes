%&../preamble

\def\npart {II}
\def\nterm {Michaelmas}
\def\nyear {2023}
\def\nlecturer {Dr Sarkar}
\def\ncourse {Probability and Measure}

\def\encodingdefault{TU}\normalfont
\ifnum 0\ifxetex 1\fi\ifluatex 1\fi=0 % if pdftex
  \usepackage[T1]{fontenc}
  \usepackage[utf8]{inputenc}
  \usepackage{textcomp} % provide euro and other symbols
\else % if luatex or xetex
  % \usepackage{unicode-math}
  % \defaultfontfeatures{Scale=MatchLowercase}
  % \defaultfontfeatures[\rmfamily]{Ligatures=TeX,Scale=1}
  % \DeclareMathAlphabet{\mathcal}{OMS}{cmsy}{m}{n}
  % \let\mathbb\relax % remove the definition by unicode-math
  % \DeclareMathAlphabet{\mathbb}{U}{msb}{m}{n}
\fi

\usetikzlibrary{external}
\tikzset{external/system call={xelatex -fmt=../preamble.fmt \tikzexternalcheckshellescape -halt-on-error -interaction=batchmode -jobname "\image" "\texsource"}} % path is relative to file that includes preamble
\tikzexternalize

\providetoggle{DontSetTitleAuthorDate}

\nottoggle{DontSetTitleAuthorDate}{
  \hypersetup{
    pdftitle={Part \npart\ - \ncourse},
    pdfsubject={Cambridge Maths Notes: Part \npart\ - \ncourse},
    pdfkeywords={Cambridge Mathematics Maths Math \npart\ \nterm\ \nyear\ \ncourse}
  }

  \author{Based on lectures by \nlecturer}
  \date{\nterm\ \nyear}
  \title{Part \npart\ --- \ncourse}
}{}

\tikzsetexternalprefix{figtemp/}
\usepackage{relsize}

\newcommand{\symmdiff}{\mathrel{\raisebox{1pt}{$\mathsmaller\triangle$}}}
\newcommand{\prob}[1]{\mathbb{P}\left({#1}\right)}
\let \emptyset \varnothing
\newcommand{\expect}[1]{\mathbb{E}\left[{#1}\right]}
% \DeclarePairedDelimiter\ceil{\lceil}{\rceil}
\DeclarePairedDelimiter\floor{\lfloor}{\rfloor}
% \renewcommand{\norm}[1]{\left \lVert #1 \right \rVert}
\newcommand{\wildcard}{{}\cdot{}}
\DeclareMathOperator*{\esssup}{ess\ sup}
% \DeclarePairedDelimiter\Brackets{[\![}{]\!]}
\newcommand{\hhat}[1]{\hat{\hat{#1}}}
\newcommand{\midd}{\,\middle|\,}

% \includeonly{07_ergodic_theory.tex}

\setcounter{section}{-1}

\begin{document}
    \maketitle
    \tableofcontents

    \section{Holes in classical theory}

    Analysis

    \begin{enumerate}
        \item What is the ``volume'' of a subset of $\mathbb{R}^d$.
        \item Integration (Riemann Integration has holes)
        \begin{itemize}
            \item $\{f_n\}$ a sequence of continuous functions on $[0, 1]$ s.t.
            \begin{itemize}
                \item $0 \leq f_n(x) \leq 1 \; \forall \; x \in [0, 1]$.
                \item $f_n(x)$ is monotonically decreasing on $n \to \infty$, i.e. $f_n(x) \geq f_{n+1}(x) \; \forall \; x$/
            \end{itemize}
            So, $\lim_{n \to \infty} f_n(x)$ exists. But $f$ is not Riemann integrable . We want a theory of integration s.t. $f$ is integrable and $\lim_{n \to \infty} \int_{0}^{1} f_n(x) \dd{x} = \int_{0}^{1} f(x) \dd{x}$.
        \end{itemize}
        \item $L^1 = ()$
        If $f \in L^1$ is $f$ Riemann integrable? Will have to change the definition of integral. $L^2$ a hilbert space
    \end{enumerate}

    Probability

    \begin{enumerate}
        \item Discrete probability has its limitations,
        \begin{itemize}
            \item Toss a unbiased coin 5 times. What is the probability if getting 3 heads?
            \item Take an infinite sequence of coin tosses ($E = \{0, 1\}^\mathbb{N}$ which is uncountable) and an event A that depends on that infinite sequence. How do you define $\mathbb{P}(A)$?
            E.g. $X_i \sim \operatorname{Ber}\left( \frac{1}{2} \right)$ and $A = \frac{\sum_{i=1}^{n} X_i}{n}$, the average number of heads.
            By strong law of large numbers $\mathbb{P}\left( \frac{\sum_{i=1}^{n} X_i}{n} \to \frac{1}{2} \right) = 1$.
            \item How to draw a point uniformly at random from $[0, 1]$? $U \sim U[0, 1]$.
            Probability needs axioms to be made rigorous.
        \end{itemize}
        \item Define Expectation for a r.v.. Also would want the following if $0 \leq X_n \leq 1$ and $X_n \downarrow X$ then $\mathbb{E} X_n \to \mathbb{E} X$.
    \end{enumerate}

    \section{Introduction}

\begin{notation}
	$A_n \uparrow A$ means that the sequence $A_n$ is increasing ($A_1 \subseteq A_2 \subseteq \dots$) and $\bigcup_n A_n = A$.
\end{notation}

\subsection{Definitions}

\begin{definition}[$\sigma$-algebra]
	Let $E$ be a (nonempty) set. A collection $\mathcal E$ of subsets of $E$ is called a \vocab{$\sigma$-algebra} if the following properties hold:
	\begin{itemize}
		\item $\emptyset \in \mathcal E$;
		\item $A \in \mathcal E \implies A^c = E \setminus A \in \mathcal E$;
		\item if $(A_n)_{n \in \mathbb N}$ is a countable collection of sets in $\mathcal E$, $\bigcup_{n \in \mathbb N} A_n \in \mathcal E$.
	\end{itemize}
\end{definition}

\begin{example}
	Let $\mathcal E = \qty{\emptyset, E}$.
	This is a $\sigma$-algebra.
	Also, $\mathcal P(E) = \qty{A \subseteq E}$ is a $\sigma$-algebra.
\end{example}

\begin{remark}
	Since $\bigcap_n A_n = \qty(\bigcup_n A_n^c)^c$, any $\sigma$-algebra $\mathcal E$ is closed under countable intersections as well as under countable unions.
	Note that $B \setminus A = B \cap A^c \in \mathcal E$, so $\sigma$-algebras are closed under set difference.
\end{remark}

\begin{definition}[Measurable Space and Set]
	A set $E$ with a $\sigma$-algebra $\mathcal E$ is called a \vocab{measurable space}.
	The elements of $\mathcal E$ are called \vocab{measurable sets}.
\end{definition}

\begin{definition}[Measure]
	A \vocab{measure} $\mu$ is a set function $\mu : \mathcal E \to [0,\infty]$, such that $\mu(\emptyset) = 0$, and for a sequence $(A_n)_{n \in \mathbb N}$ such that the $A_n$ are disjoint, we have
	\[ \mu\qty(\bigcup_{n \in \mathbb N} A_n) = \sum_{n \in \mathbb N} \mu(A_n) \]
	This is the \vocab{countable additivity} property of the measure.
\end{definition}

\begin{remark}
	$(E, \mathcal{E}, \mu)$ is a measure space.
\end{remark}

\begin{remark}
	If $E$ is countable, then for any $A \in \mathcal P(E)$ and measure $\mu$, we have
	\[ \mu(A) = \mu\qty(\bigcup_{x\in A} \qty{x}) = \sum_{x \in A} \mu(\qty{x}) \]
	Hence, measures are uniquely defined by the measure of each singleton.

	Define $m : E \to [0, \infty]$ s.t. $m(x) = \mu(\{x\})$, such an $m$ is called a ``mass function'', and measures $\mu$ are in $1$-$1$ correspondence with the mass function $m$.
	This corresponds to the notion of a probability mass function.

	Here $\mathcal{E} = \mathcal{P}(E)$ and this is the theory in elementary discrete prob. (when $\mu(\{x\}) = 1 \; \forall x \in E$, $\mu$ is called the counting measure. Here $\mu(A) = |A| \; \forall A \subset E$).

	For uncountable $E$  however, the story is not so simple and $\mathcal{E} = \mathcal{P}(E)$ is generally not feasible. Indeed measures are defined on $\sigma$-algebra ``generated'' by a smaller class $\mathcal{A}$ of simple subsets of $E$.
\end{remark}

\begin{definition}[Generated $\sigma-$algebra]
	For a collection $\mathcal A$ of subsets of $E$, we define the $\sigma$-algebra \vocab{$\sigma(A)$ generated by $\mathcal A$} by
	\[ \sigma(\mathcal A) = \qty{A \subseteq E \colon A \in \mathcal E \text{ for all $\sigma$-algebras } \mathcal E \supseteq \mathcal A} \]
	So it is the smallest $\sigma$-algebra containing $\mathcal A$.
	Equivalently,
	\[ \sigma(\mathcal A) = \bigcap_{\mathcal E \supseteq \mathcal A,\ \mathcal E \text{ a $\sigma$-algebra}} \mathcal E \]
\end{definition}

\begin{question}
	Why is $\sigma(A)$ a $\sigma$-algebra? See Sheet 1, Q1.
\end{question}

\subsection{Rings and algebras}
The class $\mathcal{A}$ will usually satisfy some properties too, let $E$ be a set and $\mathcal{A}$ a collection of subsets of $E$.
To construct good generators, we define the following.

\begin{definition}[Ring]
	$\mathcal A \subseteq \mathcal P(E)$ is called a \vocab{ring} over $E$ if $\emptyset \in \mathcal A$ and $A, B \in \mathcal A$ implies $B \setminus A \in \mathcal A$ and $A \cup B \in \mathcal A$.
\end{definition}

Rings are easier to manage than $\sigma$-algebras because there are only finitary operators.

\begin{definition}[Algebra]
	$\mathcal A$ is called an \vocab{algebra} over $E$ if $\emptyset \in \mathcal A$ and $A, B \in \mathcal A$ implies $A^c \in \mathcal A$ and $A \cup B \in \mathcal A$.
\end{definition}

\begin{remark}
	Rings are closed under symmetric difference $A \symmdiff B = (B \setminus A) \cup (A \setminus B)$, and are closed under intersections $A \cap B = A \cup B \setminus A \symmdiff B$.
	Algebras are rings, because $B \setminus A = B \cap A^c = (B^c \cup A)^c$.
	Not all rings are algebras, because rings do not need to include the entire space.
\end{remark}

The idea:
\begin{itemize}
	\item Define a set function on a suitable collection $\mathcal{A}$.
	\item Extend the set function to a measure on $\sigma(\mathcal{A})$. (Carath\'eodory's Extension theorem)
	\item Such an extension is unique. (Dynkin's Lemma)
\end{itemize}

Goal: Start with a ``measure'' on $\mathcal{A}$ that has some nice properties and then extend it to $\sigma(A)$.

\begin{definition}[Set Function]
	A \vocab{set function} on a collection $\mathcal A$ of subsets of $E$, where $\emptyset \in \mathcal A$, is a map $\mu \colon \mathcal A \to [0,\infty]$ such that $\mu(\emptyset) = 0$.
	\begin{itemize}
		\item We say $\mu$ is \vocab{increasing} if $\mu(A) \leq \mu(B)$ for all $A \subseteq B$ in $\mathcal A$.
		\item We say $\mu$ is \vocab{additive} if $\mu(A \cup B) = \mu(A) + \mu(B)$ for disjoint $A, B \in \mathcal A$ and $A \cup B \in \mathcal A$.
		\item We say $\mu$ is \vocab{countably additive} if $\mu\qty(\bigcup_n A_n) = \sum_n \mu(A_n)$ for disjoint sequences $A_n$ where $\bigcup_n A_n$ and each $A_n$ lie in $\mathcal A$.
		\item We say $\mu$ is \vocab{countably subadditive} if $\mu\qty(\bigcup_n A_n) \leq \sum_n \mu(A_n)$ for arbitrary sequences $A_n$ under the above conditions.
	\end{itemize}
\end{definition}

\begin{remark}
	If $\mu$ is countably additive set function on $\mathcal{A}$ and $\mathcal{A}$ is a ring then $\mu$ satisfies all the previous listed properties.
\end{remark}

\begin{proposition}[Disjointification of countable unions]
	Consider $\bigcup_n A_n$ for $A_n \in \mathcal E$, where $\mathcal E$ is a $\sigma$-algebra (or a ring, if the union is finite).
	Then there exist $B_n \in \mathcal E$ that are disjoint such that $\bigcup_n A_n = \bigcup_n B_n$.
\end{proposition}

\begin{proof}
	Define $\widetilde A_n = \bigcup_{j \leq n} A_j$, then $B_n = \widetilde A_n \setminus \widetilde A_{n-1}$.
\end{proof}

\begin{remark} \label{rem:1}
	A measure satisfies all four of the above conditions. Countable additivity implies the other conditions. Proof on Sheet 1.
\end{remark}

\begin{theorem}[Carath\'eodory's theorem] \label{thm:car}
	Let $\mu$ be a countably additive set function on a ring $\mathcal A$ of subsets of $E$.
	Then there exists a measure $\mu^\star$ on $\sigma(\mathcal A)$ such that $\eval{\mu^\star}_{\mathcal A} = \mu$.
\end{theorem}

We will later prove that this extended measure is unique.

\begin{proof}[Proof (Non Examinable)]
	For $B \subseteq E$, we define the \vocab{outer measure} $\mu^\star$ as
	\[ \mu^\star(B) = \inf \qty{\sum_{n \in \mathbb N} \mu(A_n) : A_n \in \mathcal A, B \subseteq \bigcup_{n \in \mathbb N} A_n} \]
	If there is no sequence $A_n$ such that $B \subseteq \bigcup_{n \in \mathbb N} A_n$, we declare the outer measure $\mu^\star(B)$ to be $\infty$.
	Clearly, $\mu^\star(\emptyset)$ and $\mu^\star$ is increasing, so $\mu^\star$ is an increasing set fcn on $\mathcal{P}(E)$.

	\begin{definition}[$\mu^\star$ measurable]
		A set $A \subseteq E$ \vocab{$\mu^\star$ measurable} if $\forall B \subseteq E \ \mu^\star(B) = \mu^\star(B \cap A) + \mu^\star(B \cap A^c)$.
	\end{definition}

	We define the class
	\[ \mathcal M = \qty{A \subseteq E : A \text{ is $\mu^\star$ measurable}} \]
	We shall show that $M$ is a $\sigma$-algebra that contains $\mathcal{A}$, $\mu^\star \mid_M$ is a measure on $M$ that extends $\mu$ (i.e. $\eval{\mu^\star}_\mathcal{A} = \mu$).

	\emph{Step 1.} $\mu^\star$ is countably sub-additive on $\mathcal P(E)$:
	It suffices to prove that for $B \subseteq E$ and $B_n \subseteq E$ such that $B \subseteq \bigcup_n B_n$ we have
	\begin{equation}
		\mu^\star(B) \leq \sum_n \mu^\star(B_n)
		\tag{\(\dagger\)}
	\end{equation}
	We can assume without loss of generality that $\mu^\star(B_n) < \infty$ for all $n$, otherwise there is nothing to prove.
	For all $\epsilon > 0$ there exists a collection $A_{n,m} \in \mathcal{A}$ such that $B_n \subseteq \bigcup_m A_{n,m}$ and
	\[ \mu^\star(B_n) + \frac{\epsilon}{2^n} \geq \sum_m \mu(A_{n,m}) \]
	as we took an infimum.
	Now, since $\mu^\star$ is increasing, and $B \subseteq \bigcup_n B_n \subseteq \bigcup_n \bigcup_m A_{n,m}$, we have
	\[ \mu^\star(B) \leq \mu^\star\qty(\bigcup_{n,m} A_{n,m}) \leq \sum_{n,m} \mu(A_{n,m}) \leq \sum_n \mu^\star(B_n) + \sum_n \frac{\epsilon}{2^n} = \sum_n \mu^\star(B_n) + \epsilon \]
	Since $\epsilon$ was arbitrary in the construction, $(\dagger)$ follows by construction.

	\emph{Step 2.} $\mu^\star$ extends $\mu$:
	Let $A \in \mathcal A$, and we want to show $\mu^\star(A) = \mu(A)$.

	We can write $A = A \cup \emptyset \cup \dots$, hence $\mu^\star(A) \leq \mu(A) + 0 + \dots = \mu(A)$ by definition of $\mu^\star$.

	If $\mu^\star$ is infinite, there is nothing to prove.

	We need to prove the converse, that $\mu(A) \leq \mu^\star(A)$.
	For the finite case, suppose there is a sequence $A_n$ where $\mu(A_n) < \infty$ and $A \subseteq \bigcup_n A_n$.
	Then, $A = \bigcup_n (A \cap A_n)$, which is a union of elements of the ring $\mathcal A$.
	As $\mu$ is countably additive on $\mathcal{A}$ and $\mathcal{A}$ is a ring, $\mu$ is countably subadditive on $\mathcal{A}$ and increasing by \cref{rem:1}.
	Hence $\mu(A) \leq \sum_n \mu(A \cap A_n) \leq \sum_n \mu(A_n)$.
	Since the $A_n$ were arbitrary taking the infimum over $A_n$, we have $\mu(A) \leq \mu^\star(A)$ as required.

	\emph{Step 3.} $\mathcal M \supseteq \mathcal A$:
	Let $A \in \mathcal A$.
	We must show that for all $B \subseteq E$, $\mu^\star(B) = \mu^\star(B \cap A) + \mu^\star(B \cap A^c)$.

	We have $B \subseteq (B \cap A) \cup (B \cap A^c) \cup \emptyset \cup \dots$, hence by countable subadditivity $(\dagger)$, $\mu^\star(B) \leq \mu^\star(B \cap A) + \mu^\star(B \cap A^c)$.

	It now suffices to prove the converse, that $\mu^\star(B) \geq \mu^\star(B \cap A) + \mu^\star(B \cap A^c)$. \\
	We can assume $\mu^\star(B)$ is finite, and so $\forall \epsilon > 0 \; \exists A_n \in \mathcal A$ s.t. $B \subseteq \bigcup_n A_n$ and $\mu^\star(B) + \epsilon \geq \sum_n \mu(A_n)$.
	Now, $B \cap A \subseteq \bigcup_n (A_n \cap A)$, and $B \cap A^c \subseteq \bigcup_n (A_n \cap A^c)$.
	All of the members of these two unions are elements of $\mathcal A$, since $A_n \cap A^c = A_n \setminus A$.
	Therefore,
	\begin{align*}
		\mu^\star(B \cap A) + \mu^\star(B \cap A^c) &\leq \sum_n \mu(A_n \cap A) + \sum_n \mu(A_n \cap A^c) \\
		&\leq \sum_n \qty[ \mu(A_n \cap A) + \mu(A_n \cap A^c) ] \\
		&\leq \sum_n \mu(A_n) \leq \mu^\star(B) + \epsilon
	\end{align*}
	Since $\epsilon$ was arbitrary, $\mu^\star(B) = \mu^\star(B \cap A) + \mu^\star(B \cap A^c)$ as required.

	\emph{Step 4.} $\mathcal M$ is an algebra:
	Clearly $\emptyset$ lies in $\mathcal M$, and by the symmetry in the definition of $\mathcal M$, complements lie in $\mathcal M$.
	We need to check $\mathcal M$ is stable under finite intersections.
	Let $A_1, A_2 \in \mathcal M$ and let $B \subseteq E$.
	We have
	\begin{align*}
		\mu^\star(B) &= \mu^\star(B \cap A_1) + \mu^\star(B \cap A_1^c) \text{ as $A_1 \in M$} \\
		&= \mu^\star(B \cap A_1 \cap A_2) + \mu^\star(B \cap A_1 \cap A_2^c) + \mu^\star(B \cap A_1^c) \text{ taking $\tilde{B} = B \cap A_1$}
	\end{align*}
	We can write $A_1 \cap A_2^c = (A_1 \cap A_2^c)^c \cap A_1$, and $A_1^c = (A_1 \cap A_2)^c \cap A_1^c$.
	Hence
	\begin{align*}
		\mu^\star(B) &= \mu^\star(B \cap A_1 \cap A_2) + \underbracket{\mu^\star(B \cap (A_1 \cap A_2)^c \cap A_1) + \mu^\star(B \cap (A_1 \cap A_2)^c \cap A_1^c)}_{\mu^\star(B \cap (A_1 \cap A_2)^c) \text{ as } A_1 \in M} \\
		&= \mu^\star(B \cap A_1 \cap A_2) + \mu^\star(B \cap (A_1 \cap A_2)^c)
	\end{align*}
	which is the requirement for $A_1 \cap A_2$ to lie in $\mathcal M$.

	\emph{Step 5.} $\mathcal M$ is a $\sigma$-algebra and $\mu^\star$ is a measure on $\mathcal M$: \\
	It suffices now to show that $\mathcal M$ has countable unions and the measure respects these countable unions.
	Let $A = \bigcup_n A_n$ for $A_n \in \mathcal M$.
	Without loss of generality, let the $A_n$ be disjoint.
	We want to show $A \in \mathcal M$, and that $\mu^\star(A) = \sum_n \mu^\star(A_n)$.

	By $(\dagger)$, we have for any $B \subseteq E$ $\mu^\star(B) \leq \mu^\star(B \cap A) + \mu^\star(B \cap A^c) + 0 + \dots$ so we need to check only the converse of this inequality.
	Also, $\mu^\star(A) \leq \sum_n \mu^\star(A_n)$, so we need only check the converse of this inequality as well.
	Similarly to before,
	\begin{align*}
		\mu^\star(B) &= \mu^\star(B \cap A_1) + \mu^\star(B \cap A_1^c) \\
		&= \mu^\star(B \cap A_1) + \mu^\star(B \cap \underbracket{A_1^c \cap A_2}_{A_2 \text{ as $A_1, A_2$ disjoint}}) + \mu^\star(B \cap A_1^c \cap A_2^c) \\
		&= \mu^\star(B \cap A_1) + \mu^\star(B \cap A_2) + \mu^\star(B \cap A_1^c \cap A_2^c) \\
		&= \mu^\star(B \cap A_1) + \mu^\star(B \cap A_2) + \mu^\star(B \cap A_1^c \cap A_2^c \cap A_3) + \mu^\star(B \cap A_1^c \cap A_2^c \cap A_3^c) \\
		&= \mu^\star(B \cap A_1) + \mu^\star(B \cap A_2) + \mu^\star(B \cap A_3) + \mu^\star(B \cap A_1^c \cap A_2^c \cap A_3^c) \\
		&= \cdots \\
		&= \sum_{n \leq N} \mu^\star(B \cap A_n) + \mu^\star(B \cap A_1^c \cap \dots \cap A_N^c)
	\end{align*}
	Since $\bigcup_{n \leq N} A_n \subseteq A$, we have $\bigcap_{n \leq N} A_n^c \supseteq A^c$.
	$\mu^\star$ is increasing, hence, taking limits,
	\[ \mu^\star(B) \geq \sum_{n=1}^\infty \mu^\star(B \cap A_n) + \mu^\star(B \cap A^c) \]
	By $(\dagger)$,
	\[ \mu^\star(B) \geq \mu^\star(B \cap A) + \mu^\star(B \cap A^c) \]
	as required.
	Hence $\mathcal M$ is a $\sigma$-algebra.
	For the other inequality, we take the above result for $B = A$.
	\[ \mu^\star(A) \geq \sum_{n=1}^\infty \mu^\star(A \cap A_n) + \mu^\star(A \cap A^c) = \sum_{n=1}^\infty \mu^\star(A_n) \]
	So $\mu^\star$ is countably additive on $\mathcal M$ and is hence a measure on $\mathcal M$.
\end{proof}

\subsection{Uniqueness of extension}
To address uniqueness of extension, we introduce further subclasses of $\mathcal{P}(E)$. Let $\mathcal{A}$ be a collection of subsets of $E$.

\begin{definition}[$\pi$-system]
	A collection $\mathcal A$ of subsets of $E$ is called a \vocab{$\pi$-system} if $\emptyset \in \mathcal A$ and $A, B \in \mathcal A \implies A \cap B \in \mathcal A$.
\end{definition}

\begin{definition}[$d$-system] \label{def:d}
	A collection $\mathcal A$ of subsets of $E$ is called a \vocab{$d$-system} if
	\begin{itemize}
		\item $E \in \mathcal{A}$;
		\item $A, B \in \mathcal{A}$ and $A \subseteq B$ then $B \setminus A \in \mathcal{A}$;
		\item $A_n \in \mathcal{A}$ is an increasing sequence of sets then $\bigcup_n A_n \in \mathcal{A}$.
	\end{itemize}
\end{definition}

\begin{remark}
	Equivalently, $\mathcal{A}$ is a $d$-system if
	\begin{itemize}
		\item $\emptyset \in \mathcal{A}$;
		\item $A \in \mathcal{A} \implies A^c \in \mathcal{A}$
		\item $A_n \in \mathcal{A}$ is a sequence of disjoint sets then $\bigcup_n A_n \in \mathcal{A}$.
	\end{itemize}
	The difference between this and a $\sigma$-algebra is the requirement for disjoint sets.

	Proof on Sheet 1.
\end{remark}

\begin{proposition}
	A $d$-system which is also a $\pi$-system is a $\sigma$-algebra.
\end{proposition}

\begin{proof}
	Sheet 1.
\end{proof}

\begin{lemma}[Dynkin's Lemma/$\pi$-$\lambda$/$\pi$-$d$ theorem] \label{lem:dyn}
	Let $\mathcal A$ be a $\pi$-system.
	Then any $d$-system that contains $\mathcal A$ also contains $\sigma(\mathcal A)$.
\end{lemma}

\begin{proof}
	We define
	\[ \mathcal D = \bigcap_{\mathcal D' \text{ is a } d \text{-system};\; \mathcal D' \supseteq \mathcal A} \mathcal D' \]
	We can show this is a $d$-system (proof same as in $\sigma(\mathcal{A})$ on Sheet 1).
	It suffices to prove that $\mathcal D$ is a $\pi$-system, because then it is a $\sigma$-algebra\footnote{As $\mathcal{D} \supseteq \mathcal{A}$ and $\sigma(\mathcal{A})$ the intersection of all $\sigma$-algebras containing $\mathcal{A}$, $\mathcal{D} \supseteq \sigma(\mathcal{A})$.}.

	We now define
	\[ \mathcal D' = \qty{B \in \mathcal D : \forall A \in \mathcal A, B \cap A \in \mathcal D} \]
	We can see that $\mathcal A \subseteq \mathcal{D}'$, as $\mathcal A$ is a $\pi$-system.

	We now show that $\mathcal D'$ is a $d$-system, fix $A \in \mathcal{A}$.
	\begin{itemize}
		\item Clearly $E \cap A = A \in \mathcal A \subseteq \mathcal D'$ hence $E \in \mathcal D'$.
		\item Let $B_1, B_2 \in \mathcal D'$ such that $B_1 \subseteq B_2$.
		Then $(B_2 \setminus B_1) \cap A = (B_2 \cap A) \setminus (B_1 \cap A)$, and since $B_i \cap A \in \mathcal D$ this difference also lies in $\mathcal D$, so $B_2 \setminus B_1 \in \mathcal D'$.
		\item Now, suppose $B_n$ is an increasing sequence converging to $B$, and $B_n \in \mathcal D'$.
		Then $B_n \cap A \in \mathcal D$, and $\mathcal D$ is a $d$-system, we have $B \cap A \in \mathcal D$, so $B \in \mathcal D'$.
	\end{itemize}

	Hence $\mathcal D'$ is a $d$-system.
	Also, $\mathcal D' \subseteq \mathcal D$ by construction of $\mathcal D'$.
	But also $\mathcal{A} \subseteq \mathcal{D}'$ and $\mathcal{D}'$ is a $d$-system so $\mathcal{D} \subset \mathcal{D}'$ as $\mathcal{D}$ is the smallest $d$-system containing $\mathcal{A}$.
	Thus $\mathcal D = \mathcal D'$, i.e $\forall B \in \mathcal{D}$ and $A \in \mathcal{A}, B \cap A \in \mathcal{D} \ (\ast)$.

	We then define
	\[ \mathcal D'' = \qty{B \in \mathcal D : \forall A \in \mathcal D, B \cap A \in \mathcal D} \]
	Note that $\mathcal A \subseteq \mathcal D''$ by $(\ast)$.
	Running the same argument as before, we can show that $\mathcal D''$ is a $d$-system. So $\mathcal{D}'' = \mathcal{D}$.
	But then (by the definition of $\mathcal{D}''$), $\forall B \in \mathcal{D}, A \in \mathcal{D} \implies B \cap A \in \mathcal{D}$, i.e. $\mathcal{D}$ is a $\pi$-system (check that $\emptyset \in \mathcal{D}$).

	So $\mathcal{D}$ is a $\sigma$-algebra containing $\mathcal{A}$, hence $\mathcal{D} \supseteq \sigma(\mathcal{A})$.
\end{proof}

\begin{theorem}[Uniqueness of Extension] \label{thm:uni}
	Let $\mu_1, \mu_2$ be measures on a measurable space $(E, \mathcal E)$, such that $\mu_1(E) = \mu_2(E) < \infty$.
	Suppose that $\mu_1$ and $\mu_2$ coincide on a $\pi$-system $\mathcal A$, such that $\mathcal E \subseteq \sigma(\mathcal A)$.
	Then $\mu_1 = \mu_2$ on $\sigma(\mathcal A)$, and hence on $\mathcal E$.
\end{theorem}

\begin{proof}
	We define
	\[ \mathcal D = \qty{A \in \mathcal E : \mu_1(A) = \mu_2(A)} \]
	This collection contains $\mathcal A$ by assumption.
	By Dynkin's lemma, it suffices to prove $\mathcal D$ is a $d$-system, because then $\mathcal D \supseteq \sigma(\mathcal A) \supseteq \mathcal E$ giving $\mathcal D = \mathcal E$ as $\mathcal{D} \subseteq \mathcal{E}$.

	\begin{itemize}
		\item $\emptyset \in \mathcal{D}$, since $\mu_1(\emptyset) = \mu_2(\emptyset) = 0$;
		\item $A \in \mathcal{D} \implies \mu_1(A) = \mu_2(A)$, thus $\mu_1(A^c) = \mu_1(E) - \mu_1(A) = \mu_2(E) - \mu_2(A) = \mu_2(A^c)$, so $A^c \in \mathcal{D}$ ($\mu_1, \mu_2$ finite so this works);
		\item Let $A_n \in \mathcal{D}$ be a disjoint sequence then, $\mu_1(\bigcup_n A_n) = \sum \mu_1(A_n) = \sum \mu_2(A_n) = \mu_2(\bigcup_n A_n)$ by countable additivity. So $\bigcup_n A_n \in \mathcal{D}$.
	\end{itemize}
	So $\mathcal{D}$ is a $d$-system.
\end{proof}

\begin{remark}
	If $A_n \in \mathcal{A}$ an increasing sequence s.t. $A_n \uparrow A$, then $\mu(A) = \lim_{n \to \infty} \mu(A_n)$.
	Use this to show that $\mathcal{D}$ is a $d$-system satisfying conditions in \nameref{def:d}.

	The above theorem applies to finite measures ($\mu$ such that $\mu(E) < \infty$) only.
	However, the theorem can be extended to measures that are $\sigma$-finite, for which $E = \bigcup_{n \in \mathbb N} E_n$ where $\mu(E_n) < \infty$.
\end{remark}

\begin{question}
	How to show all sets of a $\sigma$-algebra $\mathcal{E}$ generated by $\mathcal{A}$ has a certain property $\mathcal{P}$?
\end{question}

\begin{answer} \label{ans:1}
	Consider set $\mathcal{G} = \{A \subseteq E : A \text{ has the property } \mathcal{P}\}$ and have that all elements of $\mathcal{A}$ have the property $\mathcal{P}$.

	Method 1: Show that $\mathcal{G}$ is a $\sigma$-algebra, as it then must contain $\sigma(\mathcal{A}) = \mathcal{E}$.

	Method 2: Show that $\mathcal{G}$ is a $d$-system and pick $\mathcal{A}$ s.t. it is a $\pi$-system and use \nameref{lem:dyn}.

	Method 3: Monotone Convergence Theorem, we will see it shortly.
\end{answer}

\subsection{Borel measures}

\begin{definition}[Borel Sets]
	Let $(E, \tau)$ be a Hausdorff topological space.
	The $\sigma$-algebra generated by the open sets of $E$, i.e. $\sigma(\mathcal{A})$ where $\mathcal{A} = \{A \subseteq E : A \text{ open}\}$, is called the \vocab{Borel $\sigma$-algebra} on $E$, denoted $\mathcal B(E)$. \\
	A measure $\mu$ on $(E, \mathcal B(E))$ is called a \vocab{Borel measure on $E$}.

	Members of $\mathcal B(E)$ are called \vocab{Borel sets}.
\end{definition}

\begin{notation}
	We write $\mathcal B = \mathcal B(\mathbb R)$.
\end{notation}

\begin{definition}[Radon Measure]
	A \vocab{Radon measure} is a Borel measure $\mu$ on $E$ such that $\mu(K) < \infty$ for all $K \subseteq E$ compact.
\end{definition}
Note that in a Hausdorff space, compact sets are closed and hence measurable.

\begin{definition}[Probability Measure]
	If $\mu(E) = 1$, $\mu$ is called a \vocab{probability measure} on $E$, and $(E, \mathcal{E}, \mu)$ is called a probability space, typically denoted instead by $(\Omega, \mathcal{F}, \mathcal{P})$.
\end{definition}

\begin{definition}[Finite Measure]
	If $\mu(E) < \infty$, $\mu$ is a \vocab{finite measure} on $E$.
\end{definition}

\begin{definition}[$\sigma$-Finite Measure]
	If $\exists$ countable sequence $E_n \in \mathcal{E}$ s.t. $\mu(E_n) < \infty \; \forall n$ and $E = \bigcup_n E_n$, then $\mu$ is called a \vocab{$\sigma$-finite measure}.
\end{definition}

\begin{remark}
	Arguments that hold for finite measures can usually be extended to $\sigma$-finite measures.
\end{remark}

\subsection{Lebesgue measure}
One of the main goals for this course is to define a notion of volume for arbitrary sets, we can do this by constructing a Borel measure $\mu$ on $\mathcal{B}(\mathbb{R}^d)$ s.t $\mu \left( \prod_{i=1}^d (a_i, b_i) \right) = \prod_{i=1}^d (b_i - a_i)$ where $a_i < b_i$ corresponding to the usual notion of volume of rectangles.

% We will construct a unique Borel measure $\mu$ on $\mathbb R^d$ such that
% \[ \mu\qty(\prod_{i=1}^d [a_i, b_i]) = \prod_{i=1}^d \abs{b_i - a_i} \]
Initially, we will perform this construction for $d = 1$, and later we will consider product measures to extend this to higher dimensions.

\begin{theorem}[Construction of the Lebesgue Measure] \label{thm:leb}
	There exists a unique Borel measure $\mu$ on $\mathbb R$ such that
	\begin{align*}
		a < b \implies \mu \left( (a,b] \right) = b - a. \tag{$\dagger$}
	\end{align*}
	$\mu$ is called the Lebesgue measure on $\mathbb{R}$.
\end{theorem}

\begin{proof}
	First we shall prove the existence of the measure and then uniqueness.

	Consider the ring $\mathcal{A}$ of finite unions of disjoint intervals\footnote{We take semi intervals as for $\mathcal{A}$ to be a ring, we require the set difference to be in $\mathcal{A}$.} of the form
	\[ \mathcal{A} = (a_1,b_1] \cup \dots \cup (a_n,b_n] \]
	where $a_1 \leq b_1 \leq a_2 \leq \dots \leq a_n \leq b_n$.
	Note that $\sigma(\mathcal{A}) = \mathcal{B}$ (see Example Sheets\footnote{as all open intervals are in $\sigma(\mathcal{A})$ and open intervals generate open sets}).

	Define for each $A \in \mathcal{A}$
	\begin{align*}
		\mu(A) = \sum_{i=1}^{n}  (b_i - a_i).
	\end{align*}
	This agrees with $(\dagger)$ for $(a, b]$.
	This is additive and well-defined (check).

	% $\mu$ is additive, and well-defined since if $A = \bigcup_j C_j = \bigcup_k D_k$ for distinct disjoint unions, we can write $C_j = \bigcup_k (C_j \cap D_k)$ and $D_k = \bigcup_j (D_k \cap C_j)$, giving
	% \[ \mu(A) = \mu\qty(\bigcup_j C_j) = \sum_j \mu(C_j) = \sum_j \mu\qty(\bigcup_j (C_j \cap D_k)) = \sum_j \sum_k \mu(C_j \cap D_k) = \mu\qty(\bigcup_k D_k) \]

	So, the existence of $\mu$ on $\sigma(\mathcal{A}) = \mathcal{B}$ follows from \nameref{thm:car} if we can show that $\mu$ is \emph{countably additive} on $\mathcal{A}$.

	\begin{remark}
		Suppose $\mu$ a finitely additive set function on a ring $\mathcal{A}$.
		Then $\mu$ is countably additive iff
		\begin{itemize}
			\item $A_n \uparrow\footnote{increasing sequence tending to $A$} A; A_n, A \in \mathcal{A} \implies \mu(A_n) \uparrow \mu(A)$ .
			\item In addition, if $\mu$ is finite and $A_n \downarrow A$ s.t. $A_n, A \in \mathcal{\mathcal{A}}$ then $\mu(A_n) \downarrow \mu(A)$\footnote{E.g. let $A_n = [n, \infty)$ with the Lebesgue measure then $A_n \downarrow \emptyset$. But $\mu(A_n) = \infty$ whilst $\mu(\emptyset) = 0$}.
		\end{itemize}
		Note, these conditions are both iff separately. See Sheet 1 for proof.
	\end{remark}

	So showing $\mu$ is countably additive on $\mathcal{A}$ is equivalent to showing the following \\
	If $A_n \in \mathcal{A}, A_n \downarrow \emptyset$ then $\mu(A_n) \downarrow 0$.
	As $A_1 \supseteq A_2 \supseteq \dots$ we can consider $\mu$ restricted to $A_1$ which is finite, as $A_1$ a finite union of finite disjoint intervals.\footnote{We are actually using, if $\mu$ finitely additive on a ring $\mathcal{A}$. Then $\mu$ is countably additive iff $A_n \downarrow \emptyset, \mu(A_1) < \infty \implies \mu(A_n) \downarrow 0$.}

	We shall prove this by contradiction.

	Suppose this is not the case, so there exist $\epsilon > 0$ and $B_n \in \mathcal A$ such that $B_n \downarrow \emptyset$ but $\mu(B_n) \geq 2\epsilon$ for infinitely many $n$ (and so wlog for all $n$). \\
	We can approximate $B_n$ from within by a sequence $\overline C_n\footnote{$\overline C_n$ means the closure of $C_n$, i.e. make it a closed set by including the left endpoint.} \in \mathcal{A}$ s.t. $C_n \subseteq B_n$ and $\mu(B_n \setminus C_n) \leq \epsilon / 2^n$.
	Suppose $B_n = \bigcup_{i=1}^{N_n} (a_{n_i},b_{n_i}]$, then define $C_n = \bigcup_{i=1}^{N_n} (a_{n_i}+\frac{2^{-n}\epsilon}{N_n}, b_{n_i}]$.
	Note that the $C_n$ lie in $\mathcal A$, and $\mu(B_n \setminus C_n) \leq 2^{-n}\epsilon$.
	Since $B_n$ is decreasing, we have $B_N = \bigcap_{n \leq N} B_n$, and
	\begin{align*}
		B_N \setminus (C_1 \cap \dots \cap C_N) = B_n \cap \qty(\bigcup_{n \leq N} C_n^c) = \bigcup_{n \leq N} B_N \setminus C_n \subseteq \bigcup_{n \leq N} B_n \setminus C_n
	\end{align*}
	Since $\mu$ is increasing and finitely additive and thus subadditive on $\mathcal{A}$,
	\begin{align*}
		\mu(B_N \setminus (C_1 \cap \dots \cap C_N)) \leq \mu\qty(\bigcup_{n \leq N} B_n \setminus C_n) \leq \sum_{n \leq N} \mu(B_n \setminus C_n) \leq \sum_{n \leq N} 2^{-N}\epsilon \leq \epsilon
	\end{align*}

	Since $\mu(B_N) \geq 2\epsilon$, additivity implies that $\mu(C_1 \cap \dots \cap C_N) \geq \epsilon$.
	This means that $C_1 \cap \dots \cap C_N$ cannot be empty.
	We can add the left endpoints of the intervals, giving $K_N = \overline C_1 \cap \dots \cap \overline C_N \neq \emptyset$.
	By Analysis I, $K_N$ is a nested sequence of bounded nonempty closed intervals and therefore there is a point $x \in \mathbb R$ such that $x \in K_N$ for all $N$\footnote{As completeness of $\mathbb{R}$ implies $\bigcap_n K_n$ is closed and non empty.}.
	But $K_N \subseteq \overline C_N \subseteq B_N$, so $x \in \bigcap_N B_n$, which is a contradiction since $\bigcap_N B_N$ is empty.
	Therefore, a measure $\mu$ on $\mathcal B$ exists.

	Now we prove uniqueness.
	Suppose $\mu, \lambda$ are measures such that the measure of an interval $(a,b]$ is $b - a$.
	We define truncated measures for $A \in \mathcal{B}$
	\begin{align*}
		\mu_n(A) &= \mu \left( A \cap (n,n+1) \right) \\
		\lambda_n(A) &= \lambda \left( A \cap (n,n+1] \right)
	\end{align*}
	Then $\mu_n, \lambda_n$ are \emph{probability measures} on $\mathcal{B}$ and $\mu_n = \lambda_n$ on the $\pi$-system of intervals of the form $(a, b]$ with $a < b$\footnote{As $(a, b] \cap (c, d] = \emptyset$ or $(e, f]$.}.
	This $\pi$-system generates $\mathcal{B}$, so by the uniqueness theorem for finite measures (\nameref{thm:uni}) $\mu_n = \lambda_n$ on $\mathcal B$.
	Hence $\forall A \in \mathcal{B}$
	\begin{align*}
		\mu(A) &= \mu\left(\bigcup_n A \cap (n,n+1]\right) \\
		&= \sum_{n \in \mathbb Z} \mu(A \cap (n,n+1]) \\
		&= \sum_{n \in \mathbb Z} \mu_n(A) \\
		&= \sum_{n \in \mathbb Z} \lambda_n(A) = \dots = \lambda(A)
	\end{align*}
\end{proof}

\begin{definition}[Lebesgue Null Set]
	A Borel set $B \in \mathcal B$ is called a \vocab{Lebesgue null set} if $\lambda(B) = 0$ where $\lambda$ is the Lebesgue measure.
\end{definition}

\begin{remark}
	A singleton $\qty{x}$ can be written as $\bigcap_n \left(x-\frac 1n, x\right]$, hence $\lambda({x}) = \lim_n \frac 1n = 0$.
	Hence singletons are null sets.
	In particular, $\lambda((a,b)) = \lambda((a,b]) = \lambda([a,b)) = \lambda([a,b])$.
	Any countable set $Q = \bigcup_q \qty{q}$ is a null set.
	Not all null sets are countable; the Cantor set is an example.

	The Lebesgue measure is \emph{translation-invariant}.
	Let $x \in \mathbb R$, then the set $B + x = \qty{b + x : b \in B}$ lies in $\mathcal B$ iff $B \in \mathcal B$, and in this case, it satisfies $\lambda(B + x) = \lambda(B)$.
	We can define the translated Lebesgue measure $\lambda_x(B) = \lambda(B + x)$ for all $B \in \mathcal B$, then $\lambda_x((a,b]) = \lambda((a, b] + x) = \lambda((a + x, b+x]) = b - a = \lambda((a, b])$.
	So $\lambda_x = \lambda$ on the $\pi$-system of intervals and so $\lambda_x = \lambda$ on the sigma algebra $\mathcal{B}$ (i.e. $\forall B \in \mathcal{B}, \lambda(B+x) = \lambda(B)$).

	\begin{question}
		Is the Lebesgue measure the only such translation invariant measure on $\mathcal{B}$?
	\end{question}

	Carath\'eodory's theorem extends $\lambda$ from $\mathcal{A}$ to not just $\sigma(\mathcal{A}) = \mathcal{B}$, but actually to $\mathcal{M}$, the set of outer-measurable sets $M \supseteq \mathcal{B}$, but how large is $\mathcal{M}?$

	The class of outer measurable sets $\mathcal M$ used in Carath\'eodory's extension theorem is here called the class of \vocab{Lebesgue measurable sets}.
	This class, the Lebesgue $\sigma$-algebra, can be shown to be
	\[ \mathcal M = \qty{A \cup N : A \in \mathcal B, N \subseteq B, B \in \mathcal B, \lambda(B) = 0 } \supsetneq \mathcal B \]
\end{remark}

\subsection{Existence of non-measurable sets}

We now show that $\mathcal{B} \subsetneq \mathcal{P}(\mathbb{R})$ (in fact $\mathcal{M}_{leb} \subsetneq \mathcal{P}(\mathbb{R})$).

% Assuming the axiom of choice, there exists a non-measurable set of reals.
Consider $E = [0,1)$ with addition defined modulo one.
By the same argument as before, the Lebesgue measure is translation-invariant modulo one.
Consider the subgroup $Q = E \cap \mathbb Q$ of $(E, +)$.
We define $x \sim y$ for $x, y \in E$ if $x - y \in Q$.
% Then, this gives equivalence classes $[x] = \qty{y \in E \colon x \sim y}$ for all $x \in E$.
Assuming the axiom of choice (uncountable version), we can select a representative from each equivalence class, and denote by $S$ the set of such representatives.
We shall show that $S \notin \mathcal{B}$.

We can partition $E$ into the union of its cosets, so $E = \bigcup_{q \in Q} (S + q)$ is a disjoint\footnote{Suppose $s_1 + q_1 = s_2 + q_2$ then $s_1 - s_2 = q_1 - q_2 \in \mathbb{Q}$ but then $s_1, s_2 \in S$ by definition \Lightning.} union.

Suppose $S$ is a Borel set.
Then $S + q$ is also a Borel set\footnote{Consider $\mathcal{G} = \qty{B \in \mathcal{B} : B + x \in \mathcal{B}}$ we can show this is a $\sigma$-algebra, see \cpageref{ans:1}.}.
Therefore by translation invariance of $\lambda$ and by countable additivity,
\begin{align*}
	\lambda([0, 1)) = 1 = \lambda\qty(\bigcup_{q \in Q}(S+q)) = \sum_{q \in Q} \lambda(S+q) = \sum_{q \in Q} \lambda(S)
\end{align*}
But no value for $\lambda(S) \in [0,\infty]$ can be assigned to make this equation hold.
Therefore $S$ is not a Borel set.

\begin{remark}
	We can extend this proof to show that $S \notin \mathcal{M}_{leb}$.
\end{remark}

One can further show that $\lambda$ cannot be extended to all subsets $\mathcal P(E)$.
\begin{theorem}[Banach - Kuratowski]
	Assuming the continuum hypothesis, there exists no measure $\mu$ on the set $\mathcal P([0,1))$ such that $\mu([0,1)) = 1$ and $\mu(\qty{x}) = 0$ for $x \in [0,1)$.
\end{theorem}

Henceforth, whenever we are on a metric space $E$, we will work with $\mathcal{B}(E)$, which will be perfectly satisfactory.

\subsection{Probability spaces}

\begin{definition}
	If a measure space $(E, \mathcal E, \mu)$ has $\mu(E) = 1$, we call it a \vocab{probability space}, and instead write $(\Omega, \mathcal F, \mathbb P)$.
	We call $\Omega$ the outcome space or sample space, $\mathcal F$ the set of events, and $\mathbb P$ the probability measure.
\end{definition}

The axioms of probability theory (Kolmogorov, 1933), are

\begin{enumerate}
	\item $\prob{\Omega} = 1, \mathbb{P}(\emptyset) = 0$;
	\item $0 \leq \prob{E} \leq 1$ for all $E \in \mathcal F$;
	\item if $A_n$ are a disjoint sequence of events in $\mathcal F$, then $\prob{\bigcup_n A_n} = \sum_n \prob{A_n}$.
\end{enumerate}

This is exactly what is required by our definition: $\mathbb P$ is a measure on a $\sigma$-algebra.

\begin{remark} \
	\begin{itemize}
		\item $\prob{\bigcup_n A_n} \leq \sum_n \prob{A_n}$ for all sequences $A_n \in \mathcal{F}$;
		\item $A_n \uparrow A \implies \mathbb{P}(A_n) \uparrow \mathbb{P}(A)$;
		\item $A_n \downarrow A \implies \mathbb{P}(A_n) \downarrow \mathbb{P}(A)$ as $\mathbb{P}$ a finite measure.
	\end{itemize}
\end{remark}

This definition is what separates probability from analysis.
\begin{definition}[Independent]
	Events $(A_i, i \in I), A_i \in \mathcal{F}$ are \vocab{independent} if for all finite $J \subseteq I$, we have
	\begin{align*}
		\prob{\bigcap_{j \in J} A_j} = \prod_{j \in J} \prob{A_j}.
	\end{align*}
	$\sigma$-algebras $(\mathcal{A}_i, i \in I), \mathcal{A}_i \subseteq \mathcal{F}$ are \vocab{independent} if for any $A_j \in \mathcal A_j$, where $J \subseteq I$ is finite, the $A_j$ are independent.
\end{definition}

Kolmogorov showed that these definitions are sufficient to derive the law of large numbers.

\begin{proposition}
	Let $\mathcal A_1, \mathcal A_2$ be $\pi$-systems of sets in $\mathcal F$.
	Suppose $\prob{A_1 \cap A_2} = \prob{A_1} \prob{A_2}$ for all $A_1 \in \mathcal A_1, A_2 \in \mathcal A_2$.
	Then the $\sigma$-algebras $\sigma(\mathcal A_1), \sigma(\mathcal A_2)$ are independent.
\end{proposition}

\begin{proof}
	Fix $A_1 \in \mathcal{A}_1$, and define for all $A \in \sigma(\mathcal{A}_2)$.
	\begin{align*}
		\mu(A) = \mathbb{P}(A_1 \cap A),\ \nu(A) = \mathbb{P}(A_1) \mathbb{P}(A).
	\end{align*}
	Then $\mu, \nu$ are finite measures and they agree on the $\pi$-system $\mathcal{A}_2$. Hence by \nameref{thm:uni}, $\mu(A) = \nu(A) \; \forall A \in \sigma(\mathcal{A}_2)$, i.e. $\mathbb{P}(A_1 \cap A) = \mathbb{P}(A_1)\mathbb{P}(A) \; \forall A_1 \in \mathcal{A}_1, A_2 \in \sigma(\mathcal{A}_2)$.

	Now repeat same argument, but now by fixing $A_2 \in \mathcolor{red}{\sigma(\mathcal{A}_2)}$ define for all $A \in \sigma(\mathcal{A}_1)$
	\begin{align*}
		\mu'(A) = \mathbb{P}(A \cap A_2),\ \nu'(A) = \mathbb{P}(A) \mathbb{P}(A_2).
	\end{align*}
	Then $\mu', \nu'$ are finite measures and they agree on the $\pi$-system $\mathcal{A}_1$. Hence by \nameref{thm:uni}, $\mu'(A) = \nu'(A) \; \forall A \in \sigma(\mathcal{A}_1)$, i.e. $\mathbb{P}(A_1 \cap A) = \mathbb{P}(A_1)\mathbb{P}(A) \; \forall A_1 \in \sigma(\mathcal{A}_1), A_2 \in \sigma(\mathcal{A}_2)$.

\end{proof}

This follows by uniqueness.

\subsection{Borel--Cantelli lemmas}

\begin{definition}
	Let $A_n \in \mathcal F$ be a sequence of events.
	Then the \vocab{limit superior} of $A_n$ is
	\begin{align*}
		\limsup_n A_n = \bigcap_n \bigcup_{m \geq n} A_m = \qty{A_n \text{ infinitely often}}\footnote{Consider $\omega$, if $\omega \in \limsup_n A_n$ then $\forall n, \omega \in \bigcup_{m \geq n} A_m$ thus $\omega$ must be in an infinite number of $A_n$s.}
	\end{align*}
	The \vocab{limit inferior} of $A_n$ is
	\[ \liminf_n A_n = \bigcup_n \bigcap_{m \geq n} A_m = \qty{A_n \text{ eventually}}\footnote{$\omega$ is in all but finitely many $A_n$.} \]
\end{definition}

\begin{lemma}[First Borel--Cantelli lemma]
	Let $A_n \in \mathcal F$ be a sequence of events such that $\sum_n \prob{A_n} < \infty$.
	Then $\prob{A_n \text{ infinitely often}} = 0$.
\end{lemma}

\begin{proof}
	For all $n$, we have
	\[ \prob{\limsup_n A_n} = \prob{\bigcap_n \bigcup_{m \geq n} A_m} \leq \prob{\bigcup_{m \geq n} A_m} \leq\footnote{By countable subadditivity} \sum_{m \geq n} \prob{A_m} \to 0 \]
\end{proof}

This proof did not require that $\mathbb P$ be a probability measure, just that it is a measure.
Therefore, we can use this for arbitrary measures.

\begin{lemma}[Second Borel--Cantelli lemma]
	Let $A_n \in \mathcal F$ be a sequence of independent events with $\sum_n \prob{A_n} = \infty$.
	Then $\prob{A_n \text{ infinitely often}} = 1$.
\end{lemma}

\begin{proof}
	By independence, for all $N \geq n \in \mathbb N$ and using $1 - a \leq e^{-a}$, we find
	\[ \prob{\bigcap_{m=n}^N A_m^c} = \prod_{m=n}^N \qty(1 - \prob{A_m}) \leq \prod_{m=n}^N e^{-\prob{A_m}} = e^{-\sum_{m=n}^N \prob{A_m}} \]
	As $N \to \infty$, this approaches zero. \\
	Since $\bigcap_{m=n}^N A_m^c$ decreases to $\bigcap_{m=n}^\infty A_m^c$, $\prob{\bigcap_{m=n}^\infty A_m^c} = 0$ as $\prob{\bigcap_{m=n}^\infty A_m^c} \leq \prob{\bigcap_{m=n}^N A_m^c} \leq e^{-\sum_{m=n}^N \prob{A_m}} \to 0$.
	So by taking complements $\mathbb{P}(\bigcup_{m=n}^\infty A_n) = 1 \; \forall n \ (\dagger)$.

	Let $B_n = \bigcup_{m=n}^\infty A_m$, $B_n$ decreasing and so $B_n \downarrow \bigcap_n B_n = \bigcap_n \bigcup_{m \geq n} A_m = \{A_n \text{ i.o}\}\footnote{$A_n$ occurs infinitely often}$.
	As $\mathbb{P}(B_n) = 1$ by $(\dagger)$, $\mathbb{P}(\{A_n \text{ i.o}\}) = \lim_{n \to \infty} \mathbb{P}(B_n) = 1$ as probabilities are a finite measure\footnote{Recall the equivalent condition to countable additivity given in the proof of \nameref{thm:leb}.}.
\end{proof}

\begin{remark}
	If $A_n$ independent, then $\{A_n \text{ i.o}\}$ has either probability $0$ or $1$ and is called a ``tail event''.
	Kolmogorov 0-1 law shows this is true for all ``tail events''.
\end{remark}
    \section{Measurable Functions}
\subsection{Definition}
\begin{definition}[Measurable]
	Let $(E, \mathcal E), (G, \mathcal G)$ be measurable spaces.
	A function $f \colon E \to G$ is called \vocab{measurable} if $f^{-1}(A) \in \mathcal E \ \forall \; A \in \mathcal{G}$, where $f\inv(A)$ is the preimage of $A$ under $f$ i.e. $f\inv(A) = \qty{x \in E : f(x) \in A}$.
\end{definition}
% Informally, the preimage of a measurable set under a measurable function is measurable.

If $G = \mathbb R$ and $\mathcal G = \mathcal B$, we can just say that $f \colon (E, \mathcal E) \to G$ is measurable.
Moreover, if $E$ is a topological space and $\mathcal E = \mathcal B(E)$, we say $f$ is Borel measurable.

Note that preimages $f^{-1}$ commute with many set operations such as intersection, union, and complement.
This implies that $\qty{f^{-1}(A) : A \in \mathcal G}$ is a $\sigma$-algebra over $E$, and likewise, $\qty{A : f^{-1}(A) \in \mathcal E}$ is a $\sigma$-algebra over $G$.
Hence, if $\mathcal A$ is a collection of subsets s.t. $G \supset \sigma(\mathcal{A})$ then if $f^{-1}(A) \in \mathcal E$ for all $A \in \mathcal A$, the class $\qty{A : f^{-1} \in \mathcal E}$ is a $\sigma$-algebra that contains $\mathcal A$ and so $\sigma(\mathcal{A})$.
So $f$ is measurable.

If $f \colon (E, \mathcal E) \to \mathbb R$, the collection $\mathcal A = \qty{(-\infty,y] \colon y \in \mathbb R}$ generates $\mathcal B$ (Sheet 1).
Hence $f$ is Borel measurable iff $f^{-1}((-\infty,y]) = \qty{x \in E : f(x) \leq y} \in \mathcal E$ for all $y \in \mathbb R$.

If $E$ is a topological space and $\mathcal E = \mathcal B(E)$, then if $f \colon E \to \mathbb R$ is continuous, the preimages of open sets $B$ are open, and hence Borel sets.
The open sets in $\mathbb R$ generate the $\sigma$-algebra $\mathcal B$.
Hence, continuous functions to the real line are measurable.

\begin{example}
	Consider the indicator function $1_A$ of a set $A \subset E$. $1_A\inv(1) = A$ and $1_A\inv(0) = A^c$ hence measurable iff $A \in \mathcal E$.
\end{example}

\begin{example}
	The composition of measurable functions is measurable.
	Note that given a collection of maps $\qty{f_i \colon E \to (G,\mathcal G) : i \in I}$, we can make them all measurable by taking $\mathcal E$ to be a large enough $\sigma$-algebra, for instance $\sigma\qty(\qty{f_i^{-1}(A) : A \in \mathcal G, i \in I})$ called the $\sigma$-algebra generated by $\{f_i\}_{i \in I}$.
\end{example}

\begin{proposition}
	If $f_1, f_2, \dots$ are measurable $\mathbb{R}$-valued. Then $f_1 + f_2$, $f_1 f_2$, $\inf_n f_n$, $\sup_n f_n$, $\liminf f_n$, $\limsup f_n$ are all measurable.
\end{proposition}

\begin{proof}
	See Sheet 1.
\end{proof}

\subsection{Monotone Class Theorem}
\begin{theorem}[Monotone Class Theorem] \label{thm:monclass}
	Let $(E, \mathcal{E})$ be a measurable space and $\mathcal A$ be a $\pi$-system that generates the $\sigma$-algebra $\mathcal E$.
	Let $\mathcal V$ be a vector space of bounded maps from $E$ to $\mathbb R$ s.t.
	\begin{enumerate}
		\item $1_E \in \mathcal V$;
		\item $1_A \in \mathcal V$ for all $A \in \mathcal A$;
		\item if $f$ is bounded and $f_n \in \mathcal V$ are nonnegative functions that form an increasing sequence that converge pointwise to $f$ on $E$, then $f \in \mathcal V$.
	\end{enumerate}
	Then $\mathcal V$ contains all bounded measurable functions $f \colon E \to \mathbb R$.
\end{theorem}

\begin{proof}
	Define $\mathcal D = \qty{A \in \mathcal E : 1_A \in \mathcal V}$.
	Then $\mathcal{D}$ is a $d$-system as $1_E \in \mathcal{V}$ and for $A \subseteq B$, $1_{B \setminus A} = 1_B - 1_A \in \mathcal{V}$ as $\mathcal{V}$ a vector space so $B \setminus A \in \mathcal{D}$. \\
	If $A_n \in \mathcal D$ increases to $A$, we have $1_{A_n}$ increases pointwise to $1_A$, which lies in $\mathcal V$ by the (3.) so $A \in \mathcal{D}$.


	$\mathcal{D}$ contains $\mathcal A$ by (2.), as well as $E$ itself.
	So by Dynkin's lemma $\mathcal{D}$ contains $\sigma(\mathcal{A}) = \mathcal{E}$ so $\mathcal E = \mathcal D$ i.e. $1_A \in V \ \forall \; A \in \mathcal{E}$.

	Since $V$ a vector space it contains all finite linear combinations of indicators of measurable sets.
	Let $f \colon E \to \mathbb R$ be a bounded measurable function, which we will assume at first is nonnegative.
	We define
	\begin{align*}
		f_n(x) &= 2^{-n} \lfloor 2^n f(x) \rfloor \\
		&= 2^{-n} \sum_{j=0}^\infty 1_{A_{n, j}}(x) \\
		A_{n, j} &= \qty{2^n f(x) \in [j, j+1)} \\
		&= f\inv\left(\left[\frac{j}{2^n}, \frac{j+1}{2^n}\right)\right) \in \mathcal{E}.
	\end{align*}
	As $f$ is bounded we do not need an infinite sum but only a finite one.
	Then $f_n \leq f \leq f_n + 2^{-n}$.
	Hence $\abs{f_n - f} \leq 2^{-n} \to 0$ and $f_n \uparrow f$.

	So $0 \leq f_n \uparrow f, f_n \in \mathcal{V}$ and $f$ is bounded non-negative so $f \in \mathcal{V}$ by (3.).

	Finally, for any $f$ bounded and measurable, $f = f^+\footnote{$\max(f, 0)$} - f^-\footnote{$\max(-f, 0)$}$. $f^+, f^-$ are bounded, nonnegative and measurable, so in $\mathcal{V}$ and $\mathcal{V}$ a vector space thus $f \in \mathcal{V}$.
\end{proof}

\subsection{Image measures}

\begin{definition}[Image Measure]
	Let $f \colon (E,\mathcal E) \to (G,\mathcal G)$ be a measurable function and $\mu$ a measure on $(E, \mathcal E)$.
	Then the \vocab{image measure} $\nu = \mu \circ f^{-1}$ is obtained from assigning $\nu(A) = \mu(f^{-1}(A))$ for all $A \in \mathcal G$.
\end{definition}

\begin{remark}
	This is well defined as $f\inv(A) \in \mathcal{E}$ as $f$ measurable. $\nu$ is countably additive because the preimage satisfies set operations and $\mu$ countably additive (See Sheet 1).
\end{remark}

Starting from the Lebesgue measure, we can get all probability measures (in fact we can get all Radon measures) in this way.

% TODO: Define right-continuous
\begin{definition}[Right-Continuous]
	A function $f$ is \vocab{right-continuous} if $x_n \downarrow x \implies f(x_n) \to f(x)$.
\end{definition}

\begin{lemma}
	Let $g \colon \mathbb R \to \mathbb R$ be a non-constant, increasing, right-continuous function, and set $g(\pm\infty) = \lim_{z \to \pm \infty} g(z)$.
	On $I = (g(-\infty), g(+\infty))$ we define the \vocab{generalised inverse} $f : I \to \mathbb{R}$ by
	\[ f(x) = \inf \qty{y \in \mathbb R : g(y) \geq x}. \]
	Then $f$ is increasing, left-continuous, and $f(x) \leq y$ iff $x \leq g(y)$ for all $x \in I, y \in \mathbb R$.
\end{lemma}

\begin{remark}
	$f$ and $g$ form a Galois connection.
\end{remark}

\begin{proof}
	Fix $x \in I$. \\
	Let $J_x = \qty{y \in \mathbb R : g(y) \geq x}$.
	Since $x > g(-\infty)$, $J_x$ is nonempty and bounded below.
	Hence $f(x)$ is a well-defined real number. \\
	If $y \in J_x$, then $y' \geq y$ implies $y' \in J_x$ since $g$ is increasing.
	Since $g$ is right-continuous, if $y_n \downarrow y$, and all $y_n \in J_x$, then $g(y) = \lim_n g(y_n) \geq x$ so $y \in J_x$. \\
	So $J_x = [f(x), \infty)$.
	Hence $f(x) \leq y \iff x \leq g(y)$ as required.

	If $x \leq x'$, we have $J_x \supseteq J_{x'}$ (as $y \in J_x \Longleftarrow y \in J_x'$), i.e. $[f(x), \infty) \supseteq [f(x'), \infty)$ so $f(x) \leq f(x')$. \\
	Similarly, if $x_n \uparrow x$, we have $J_x = \bigcap_n J_{x_n}$\footnote{As $y \in \bigcap_n J_{x_n} \iff g(y) \geq x_n \ \forall \; n \iff g(y) \geq x \iff y \in J_x$.} so $[f(x), \infty) = \bigcap_n [f(x_n), \infty)$ so $f(x_n) \to f(x)$ as $x_n \to x$.
\end{proof}

\begin{theorem}
	Let $g \colon \mathbb R \to \mathbb R$ as in the previous lemma.
	Then $\exists$ a unique Radon measure $\mu_g$ on $\mathbb R$ such that $\mu_g((a,b]) = g(b) - g(a)$ for all $a < b$.
	Further, all Radon measures on $\mathbb{R}$ can be obtained in this way.
\end{theorem}

\begin{proof}
	Define $I, f$ as in the previous lemma and $\lambda$ the Lebesgue measure on $I$.

	$f$ is Borel measurable since $f^{-1}((-\infty,z]) = \qty{x \in I \colon f(x) \leq z} = \qty{x \in I \colon x \leq g(z)} = (-g(\infty),g(z)] \in \mathcal{B}$. As $\qty{(-\infty,z] : z \in \mathbb{R}}$ generate $\mathcal{B}$, $f$ measurable.

	Therefore, the image measure $\mu_g = \lambda \circ f^{-1}$ exists on $\mathcal{B}$.
	Then for any $-\infty < a < b < \infty$, we have
	\begin{align*}
		\mu_g((a,b]) &= \lambda \left( f^{-1}\left( (a,b] \right) \right) \\
		&= \lambda \left( \qty{x \colon a < f(x) \leq f(b)} \right) \\
		&= \lambda \left( \qty{x \colon g(a) < x \leq g(b)} \right) \\
		&= g(b) - g(a)
	\end{align*}
	By the \nameref{thm:uni} for $\sigma$-finite measures, $\mu_g$ is uniquely defined.
	% Since $g$ maps into $\mathbb R$, $g(b) - g(a) \in \mathbb R$ so any compact set has finite measure as it is a subset of a closed bounded interval.

	Conversely, let $\nu$ be a Radon measure on $\mathbb R$.
	Define $g : \mathbb{R} \to \mathbb{R}$ as
	\[ g(y) = \begin{cases}
		\nu((0,y]) & \text{if } y \geq 0 \\
		-\nu((y,0]) & \text{if } y < 0
	\end{cases} \]
	$\nu$ Radon tells us that $g$ is finite.
	Easy to check $g$ is right-continuous\footnote{For $y_n \downarrow y$ where $y \geq 0$, $(0, y_n] \downarrow (0, y]$ and then $\nu((0, y_n]) \downarrow \nu((0, y])$ by countably additivity. Similarly for $y < 0$.}.
	This is an increasing function in $y$, since $\nu$ is a measure.
	Finally, $\nu((a,b]) = g(b) - g(a)$ which can be seen by case analysis and additivity of the measure $\nu$.
	By uniqueness as before, this characterises $\nu$ in its entirety.
\end{proof}

\begin{remark}
	Such image measures $\mu_g$ are called \vocab{Lebesgue--Stieltjes measures} associated with $g$, where $g$ is the \vocab{Stieltjes distribution}.
\end{remark}

\begin{example}
	Fix $x \in \mathbb{R}$ and take $g = 1_{[x,\infty)}$.
	Then $\mu_g = \delta_x$ the \emph{dirac measure at $x$} defined for all $A \in \mathcal{B}$ by
	\[ \delta_x(A) = \begin{cases}
		1 & \text{if } x \in A \\
		0 & \text{otherwise}
	\end{cases} \]
\end{example}

\subsection{Random variables}

\begin{definition}[Random Variable]
	Let $(\Omega, \mathcal F, \mathbb P)$ be a probability space, and $(E, \mathcal E)$ be a measurable space.
	If $X : \Omega \to E$ a measurable function then $X$ is a \vocab{random variable} in $E$.
\end{definition}
When $E = \mathbb R$ or $\mathbb R^d$ with the Borel $\sigma$-algebra, we simply call $X$ a random variable or random vector.

\begin{example}
	$X$ models a ``random'' outcome of an experiment, e.g. when tossing a coin $\Omega = \{H, T\}, X = \text{\# heads} : \Omega \to \{0, 1\}$.
\end{example}

\begin{definition}[Distribution]
	The \vocab{law} or \vocab{distribution} $\mu_X$ of a random variable $X$ is given by the image measure $\mu_X = \mathbb P \circ X^{-1}$.
	It is a measure on $(E, \mathcal{E})$.

	When $(E, \mathcal{E}) = (\mathbb{R}, \mathcal{B})$, $\mu_X$ is uniquely determined by its values on any $\pi$-system, we shall take $\qty{(-\infty, x] : x \in \mathbb{R}}$ and
	\begin{align*}
		F_X(z) = \mu_X((-\infty, z]) = \mathbb P(X^{-1}(-\infty,z]) = \prob{\qty{\omega \in \Omega : X(\omega) \leq z}} = \prob{X \leq z}
	\end{align*}
	The function $F_x$ is called the \vocab{distribution function} of $X$, because it uniquely determines the distribution of $X$.
\end{definition}

Using the properties of measures, we can show that any distribution function satisfies:

\begin{enumerate}
	\item $F_X$ is increasing;
	\item $F_X$ is right-continuous\footnote{$x_n \downarrow x \implies (-\infty, x_n] \downarrow (-\infty, x]$ hence by countable additivity of $\mathbb{P} \circ X\inv$.};
	\item $F_X(-\infty) = \lim_{z \to -\infty} F_X(z) = \mu_X(\varnothing) = 0$;
	\item $F_X(\infty) = \lim_{z \to \infty} F_X(z) = \mu_X(\mathbb R) = \prob{\Omega} = 1$.
\end{enumerate}

\begin{proposition}
	% Given any function $F_X : \mathbb{R} \to [0, 1]$ satisfying each property, we can obtain a random variable $X$ on $(\Omega, \mathcal F, \mathbb P) = ((0,1), \mathcal B((0,1)), \mu)$ by $X(\omega) = \inf\qty{x : \omega \leq f(x)}$, and then $F_X$ is the distribution function of $X$.
	Given any function $F$ satisfying the previous properties, $\exists \;$ a random variable $X$ s.t. $F = F_X$.
\end{proposition}

\begin{proof}
	Let $\Omega = (0, 1)$, $\mathcal{F} = \mathcal{B}(0, 1)$, $\mathbb{P}$ the Lebesgue measure $\eval{\lambda}_{(0, 1)}$. \\
	Let $F$ be any function satisfying the properties, then $F$ is increasing and right continuous so we can define the generalised inverse
	\begin{align*}
		X(\omega) = \inf\qty{x : \omega \leq F(x)} : (0, 1) \to \mathbb{R}
	\end{align*}
	Hence $X$ is a measurable function and thus a random variable.
	\begin{align*}
		F_X(x) &= \mathbb{P}(X \leq x) = \mathbb{P}(\qty{\omega \in \Omega : X(\omega) \leq x}) = \mathbb{P}(\qty{\omega \in \Omega : \omega \leq F(x)}) \\
		&= \mathbb{P}(\qty{\omega \in (0, 1) : \omega \leq F(x)}) \\
		&= \mathbb{P}((0, F(x)]) \\
		&= F(x) - 0
	\end{align*}
\end{proof}

\begin{remark}
	This is similar to what we saw in IB Probability, if we have $F$ then r.v. $F\inv(U)$ where $U \sim U(0, 1)$ has the distribution function $F$, where $F\inv$ is the generalised inverse.
\end{remark}

\begin{definition}[Independent]
	Consider a countable collection $(X_i \colon (\Omega, \mathcal F, \mathbb P) \to (E, \mathcal E))$ for $i \in I$.
	This collection of random variables is called \vocab{independent} if the $\sigma$-algebras $\sigma\qty(X_i)$ are independent, recall $\sigma(X_i) $ is generated by $\qty{X_i\inv(A) \colon A \in \mathcal E}$, the smallest $\sigma$-algebra s.t. $X_i$ measurable.
\end{definition}

For $(E, \mathcal E) = (\mathbb R, \mathcal B)$ we show on an Sheet 1 that this is equivalent to the condition
\[ \prob{X_1 \leq x_1, \dots, X_n \leq x_n} = \prob{X_1 \leq x_1} \dots \prob{X_n \leq x_n} \]
for all finite subsets $\qty{X_1, \dots, X_n}$ of the $X_i$.

\subsection{Constructing independent random variables}

\begin{question}
	Given a distribution function $F$, we know $\exists$ a r.v. $X$ corresponding to it.
	But given an infinite sequence of distribution functions $F_1, F_2, \dots$ does $\exists$ independent r.v. $(X_1, X_2, \dots)$ corresponding to them?
\end{question}

Let $(\Omega, \mathcal F, \mathbb P) = ((0,1), \mathcal B(0, 1), \eval{\lambda}_{(0,1)})$.
We start with Bernoulli random variables.

Any $\omega \in (0,1)$ has a binary representation given by $(\omega_i) \in \qty{0,1}^{\mathbb N}$ where $\omega = \sum_{i=1}^{\infty} 2^{-i} \omega_i$, which is unique if we exclude infinitely long tails of zeroes from the binary representation (same reasoning as $1.00000\ldots = 0.99999\dots$).

\begin{definition}[$n$th Rademacher function]
	The \vocab{$n$th Rademacher function} $R_n : \Omega \to {0, 1}$ is given by $R_n(\omega) = \omega_n$, it extracts the $n$th bit from the binary expansion.
\end{definition}

Observe that $R_1 = 1_{(1/2, 1]}$, $R_2 = 1_{(1/4, 1/2]} + 1_{(3/4, 1]}$ and so on.
Since each $R_n$ can be given as the sum of finite ($2^{n-1}$) indicator functions on measurable sets, they are measurable functions and are hence random variables.

\begin{claim}
	$R_i$ are iid $\operatorname{Ber}(\frac{1}{2})$.
\end{claim}

\begin{proof}
	$\prob{R_n = 1} = \frac{1}{2} = \prob{R_n = 0}$ can be checked by induction.

	We now show they are independent.
	For a finite set $(x_i)_{i=1}^n$, by considering the size of the intervals that $\omega$ can lie in,
	\begin{align*}
		\prob{R_1 = x_1, \dots, R_n = x_n} = 2^{-n} = \prob{R_1 = x_1} \dots \prob{R_n = x_n}
	\end{align*}
\end{proof}

Therefore, the $R_n$ are all independent, so countable sequences of independent random variables indeed exist. \\
The next step is to construct a sequence of iid r.v.s on $\operatorname{U}(0, 1)$.

Now, take a bijection $m \colon \mathbb N^2 \to \mathbb N$ and define $Y_{k,n} = R_{m(k, n)}$, the Rademacher functions.
We now define $Y_n = \sum_{k=1}^\infty 2^{-k} Y_{k,n}$\footnote{This converges for all $\omega \in \Omega$ since $\abs{Y_{k,n}} \leq 1$.}.

\begin{claim}
	$Y_n$ are iid $\operatorname{U}(0, 1)$, i.e. $\mu_{Y_n} = \eval{\lambda}_{(0, 1)}$ and $Y_n$ independent.
\end{claim}

\begin{lemma}
	Any measurable functions of independent random variables are independent.
\end{lemma}

\begin{proof}
	They are independent because the $Y_i$ are measurable functions of independent random variables, e.g. $Y_1$ is a measurable function of $Y_{1,1}, Y_{2, 1}, \dots$; $Y_2$ of $Y_{1, 2}, Y_{2, 2}, \dots$

	% The distribution of $Y_n$ is identified on the $\pi$-system of intervals $(\frac{i}{2^m}, \frac{i+1}{2^m}], i = 0, 1, \dots, 2^{m-1}$ for $m \in \mathbb{N}$.

	The $\pi$-system of intervals $\left( \frac{i}{2^m}, \frac{i+1}{2^m} \right]$ for $i = 0, \dots, 2^m - 1$ for $m \in \mathbb{N}$ generates $\mathcal B(0, 1)$ as $\mathbb{Q}$ dense in $\mathbb{R}$.
	So by \cref{thm:uni} the distribution of $Y_n$ is identified on the intervals.
	\begin{align*}
		\prob{Y_n \in \left( \frac{i}{2^m}, \frac{i+1}{2^m} \right]} &= \prob{\frac{i}{2^m} < \sum_{k=1}^\infty 2^{-k} Y_{k,n} \leq \frac{i+1}{2^n}}\footnote{This specifies the first $m$ digits in the binary expansion of $Y_n$.} \\
		&= \mathbb{P}(Y_{1,n} = y_1, \dots, Y_{m,n} = y_m) \text{ where } \frac{i}{2^m} = 0.y_1 y_2 \dots y_m \\
		&= \prod_{i=1}^m \mathbb{P}(Y_{m, n} = y_m) \text{ by independence.} \\
		&= 2^{-m} = \lambda\left( \frac{i}{2^m}, \frac{i+1}{2^m} \right]
	\end{align*}
	Hence $\mu_{Y_n} = \eval{\lambda}_{(0,1)}$ on the $\pi$-system and so on $\mathcal{B}(0, 1)$.
\end{proof}

As before, set $G_n(x) = F_n\inv(x)$ which is the generalised inverse.
Then $G_n$ are Borel functions, set $X_n = G_n(Y_n)$ for $n \in \mathbb{N}$, then as before $F_{X_n} = F_n$ and $X_n$ are independent as $Y_n$ are.

\subsection{Convergence of measurable functions}
Let $(E, \mathcal{E}, \mu)$ be a measure space. Let $A \in \mathcal{E}$ be defined by some property.

\begin{definition}[Almost everywhere]
	We say that a property defining a set $A \in \mathcal E$ holds \vocab{$\mu$-almost everywhere} if $\mu(A^c) = 0$.
\end{definition}

\begin{definition}[Almost surely]
	If $\mu$ is a $\mathbb P$- measure, we say a property holds \vocab{$\mathbb P$-almost surely} or \vocab{with probability one}, if $\mathbb{P}(A^c) = 0$, i.e. if $\mathbb P(A) = 1$.
\end{definition}

\begin{definition}[Convergence almost everywhere]
	If $f_n$ and $f$ are measurable functions on $(E,\mathcal E,\mu) \to (\mathbb{R}, \mathcal{B})$, we say \vocab{$f_n$ converges to $f$ $\mu$-almost everywhere} if $\mu(\qty{x \in E : f_n(x) \nrightarrow f(x)}) = 0$.

	For r.v.s, we say $X_n \to X$ \vocab{$\mathbb P$-almost surely} if $\mathbb{P}(\qty{\omega \in \Omega : X_n(\omega) \to X(\omega)}) = 1$.
\end{definition}

\begin{definition}[Convergence in Measure]
	We say \vocab{$f_n$ converges to $f$ in $\mu$-measure} if for all $\epsilon > 0$
	\begin{align*}
		\mu(\qty{x\in E : \abs{f_n(x) - f(x)} > \epsilon}) \to 0,
	\end{align*} as $n \to \infty$.

	We say $X_n \to X$ in $\mathbb{P}$-probability if $\forall \; \epsilon > 0$
	\begin{align*}
		\mathbb{P}(\abs{X_n - X} > \epsilon) \to 0
	\end{align*} as $n \to \infty$.
\end{definition}

\begin{theorem}
	Let $f_n \colon (E,\mathcal E,\mu) \to \mathbb R$ be measurable functions.
	\begin{enumerate}
		\item If $\mu(E) < \infty$, then $f_n \to 0$ a.e. $\implies f_n \to 0$ in measure;
		\item If $f_n \to 0$ in measure, $\exists$ subsequence $n_k$ s.t. $f_{n_k} \to 0$ a.e.
	\end{enumerate}
	% If $\mu(E) < \infty$, then $f_n \to 0$ a.e. $\implies$ $f_n \to 0$ in measure;
\end{theorem}

\begin{example}
	Let $f_n = 1_{(n, \infty)}$ and the Lebesgue measure, then $f_n \to 0$ a.e. but $\mu(|f_n| > \epsilon) = \infty \; \forall \; n$.
\end{example}

\begin{proof}
	Fix $\epsilon > 0$.
	Suppose $f_n \to 0$ a.e., then for every $n$,
	\begin{align*}
		\mu(E) \geq \mu(\abs{f_n} \leq \epsilon) \geq \mu\qty(\bigcap_{m \geq n} \qty{\abs{f_m} \leq \epsilon})
	\end{align*}
	Let $A_n = \bigcap_{m \geq n} \qty{\abs{f_m} \leq \epsilon}$ which is increasing to $\bigcup_n \bigcap_{m \geq n} \qty{\abs{f_m} \leq \epsilon}$.
	So by the countable additivity of $\mu$,
	\begin{align*}
		\mu\qty(\bigcap_{m \geq n} \qty{\abs{f_m} \leq \epsilon}) &\to \mu\qty(\bigcup_n \bigcap_{m \geq n} \qty{\abs{f_m} \leq \epsilon}) \\
		&= \mu\qty(\abs{f_n} \leq \epsilon \text{ eventually}) \\
		&\geq \mu(\abs{f_n} \to 0) \\
		&= \mu(E) \text{ as $f_n \to 0$ a.e. and $\mu$ finite.}
	\end{align*}
	Hence,
	\begin{align*}
		\liminf_{n \to \infty} \mu(\abs{f_n} \leq \epsilon) = \mu(E) \implies \limsup_{n \to \infty} \mu(\abs{f_n} > \epsilon) \leq 0 \implies \mu(\abs{f_n} > \epsilon) \to 0
	\end{align*}
\end{proof}

\begin{proof}
	Suppose $f_n \to 0$ in measure, choosing $\epsilon = \frac{1}{k}$ we have
	\begin{align*}
		\mu\qty(\abs{f_n} > \frac{1}{k}) \to 0.
	\end{align*}
	So we can choose $n_k$ s.t. $\mu\qty(\abs{f_n} > \frac{1}{k}) \leq \frac{1}{k^2}$.
	We can choose $n_{k+1}$ in the same way s.t. $n_{k+1} > n_k$.
	So we get a subsequence $n_k$ s.t. $\mu\qty(\abs{f_{n_k}} > \frac{1}{k}) < \frac{1}{k^2}$.
	Also $\sum_k \frac{1}{k^2} < \infty$, so $\sum_k \mu\qty(\abs{f_{n_k}} > \frac{1}{k}) < \infty$.
	So by the first Borel--Cantelli lemma, we have
	\begin{align*}
		\mu\qty(\underbracket{\abs{f_{n_k}} > \frac{1}{k} \text{ infinitely often}}_{f_{n_k} \not\to 0}) = 0
	\end{align*}
	so $f_{n_k} \to 0$ a.e.
\end{proof}

\begin{remark}
	The first statement is false if $\mu(E)$ is infinite: consider $f_n = 1_{(n,\infty)}$ on $(\mathbb R,\mathcal B,\mu)$, since $f_n \to 0$ almost everywhere but $\mu(f_n) = \infty$.

	The second statement is false if we do not restrict to subsequences: consider independent events $A_n$ such that $\prob{A_n} = \frac{1}{n}$, then $1_{A_n} \to 0$ in probability since $\prob{1_{A_n} > \epsilon} = \prob{A_n} = \frac{1}{n} \to 0$, but $\sum_n \prob{A_n} = \infty$, and by the second Borel--Cantelli lemma, $\prob{1_{A_n} > \epsilon \text{ infinitely often}} = 1$, so $1_{A_n} \nrightarrow 0$ almost surely.
\end{remark}

\begin{definition}[Convergence in Distribution]
	For $X$ and $X_n$ a sequence of r.v.s, we say $X_n \overset{d}{\to} X$\footnote{$X_n$ converges to $X$ in distribution} if $F_{X_n}(t) \to F_X(t)$ as $n\to\infty$ for all $t \in \mathbb{R}$ which are continuity points of $F_X$.
\end{definition}

\begin{remark}
	This definition does not require $X_n$ to be defined on the same probability space.
\end{remark}

\begin{remark}
	If $X_n \to X$ in probability, then $X_n \overset{d}{\to} X$, see Sheet 2 for proof.
\end{remark}

\begin{example}
	Let $(X_n)_{n \in \mathbb N}$ be iid $\operatorname{Exp}(1)$, i.e. $\prob{X_n > x} = e^{-x}$ for $x \geq 0$.

	\begin{question}
		Find a deterministic fcn $g : \mathbb{N} \to \mathbb{R}$ s.t. a.s. $\limsup \frac{X_n}{g(n)} = 1$.
	\end{question}

	Define $A_n = \qty{X_n \geq \alpha \log n}$ where $\alpha > 0$, so $\prob{A_n} = n^{-\alpha}$, and in particular, $\sum_n \prob{A_n} < \infty$ if and only if $\alpha > 1$.
	By the Borel--Cantelli lemmas, we have for all $\epsilon > 0$,
	\[ \prob{\frac{X_n}{\log n} \geq 1 \text{ infinitely often}} = 1;\quad \prob{\frac{X_n}{\log n} \geq 1 + \epsilon \text{ infinitely often}} = 0 \]
	In other words, $\mathbb{P}(\limsup_n \frac{X_n}{\log n} = 1) = 1$.
\end{example}

\subsection{Kolmogorov's zero-one law}
Let $(X_n)_{n \in \mathbb N}$ be a sequence of r.v.s.
We can define $\mathcal T_n = \sigma(X_{n+1}, X_{n+2}, \dots)\footnote{The smallest $\sigma$-algebra s.t. $X_{n+1}, \dots$ are measurable.}$.
Let $\mathcal T = \bigcap_{n \in \mathbb N} \mathcal T_n$ be the \vocab{tail $\sigma$-algebra}, which contains all events in $\mathcal F$ that depend only on the `limiting behaviour' of $(X_n)$.

\begin{theorem}[Kolmogorov 0-1 Law]
	Let $(X_n)_{n \in \mathbb N}$ be a sequence of independent r.v.s.
	Let $A \in \mathcal T$ be an event in the tail $\sigma$-algebra.
	Then $\prob{A} = 1$ or $\prob{A} = 0$. \\
	If $Y \colon (\Omega,\mathcal T) \to (\mathbb R,\mathcal B)$ is measurable, it is constant almost surely.
\end{theorem}

\begin{proof}
	Let $\mathcal F_n = \sigma(X_1, \dots, X_n)$.
	Then $\mathcal{F}_n$ is generated by the $\pi$-system of sets $A = \qty(X_1 \leq x_1, \dots, X_n \leq x_n)$ for any $x_i \in \mathbb R$. \\
	Note that the $\pi$-system of sets $B = \qty(X_{n+1} \leq x_{n+1}, \dots, X_{n+k} \leq x_{n+k})$, for arbitrary $k \in \mathbb N$ and $x_i \in \mathbb R$, generates $\mathcal T_n$. \\
	By independence of the sequence, we see that $\prob{A \cap B} = \prob{A} \prob{B}$ for all such sets $A, B$, and so the $\sigma$-algebras $\mathcal T_n, \mathcal F_n$ generated by these $\pi$-systems are independent.
	As $\mathcal{T} \subseteq \mathcal{T}_n$, $\mathcal{F}_n$ and $\mathcal{T}$ are independent $\forall \; n$.

	Let $\mathcal F_\infty = \sigma(X_1, X_2, \dots)$.
	Then, $\bigcup_n \mathcal F_n$ is a $\pi$-system that generates $\mathcal F_\infty$.
	As $\mathcal{F}_n$ and $\mathcal{T}$ are independent $\forall \; n$, $\bigcup_n \mathcal F_n$ independent of $\mathcal{T}$.
	So $\mathcal{F}_\infty$, $\mathcal{T}$ are independent.

	Since $\mathcal T \subseteq \mathcal F_\infty$, if $A \in \mathcal T$, $A$ is independent from $A \in \mathcal{F}_\infty$.
	So $\prob{A} = \prob{A \cap A} = \prob{A}\prob{A}$, so $\prob{A}^2 - \prob{A} = 0$ as required.

	Finally, if $Y \colon (\Omega,\mathcal T) \to (\mathbb R,\mathcal B)$ measurable, the preimages of $\qty{Y \leq y}$ lie in $\mathcal T$, which give probability one or zero.
	Let $c = \inf\qty{y : F_Y(y) = 1}$, so $Y = c$ almost surely.
\end{proof}


\begin{remark}
	This tells us that for $X_i$ iid with finite expectation, $\liminf_{n \to \infty} \frac{1}{n} \sum_{i=1}^{n} X_i$, $\limsup_{n \to \infty} \frac{1}{n} \sum_{i=1}^{n} X_i$ are constants a.s.
\end{remark}
    \section{Integration}
\subsection{Notation}
Let $f \colon (E, \mathcal E, \mu) \to \mathbb R$ be measurable and $f \geq 0$\footnote{$f$ is measurable when mapped to $\mathbb{R}$ and $f \geq 0$, this is different from saying $f$ non-negative, measurable.}.

\begin{notation}
	We will then define the integral with respect to $\mu$, either written $\mu(f)$ or $\int_E f \dd{\mu} = \int_E f(x) \dd{\mu(x)}$.

	When $(E, \mathcal{E}, \mu) = (\mathbb{R}, \mathcal{B}, \lambda)$, we write it as $\int f(x) dx$.
\end{notation}
\begin{notation}
	If $X$ is a random variable, we will define its expectation $\expect{X} = \int_\Omega X \dd{\mathbb P} = \int_\Omega X(\omega) \dd{\mathbb P(\omega)}$.
\end{notation}

\subsection{Definition}
\begin{definition}[Simple]
	We say that a function $f \colon (E,\mathcal E,\mu) \to \mathbb R$ is \vocab{simple} if it is of the form
	\begin{align*}
		f = \sum_{k=1}^m a_k 1_{A_k};\quad a_k \geq 0;\quad A_k \in \mathcal E;\quad m \in \mathbb N
	\end{align*}
\end{definition}

\begin{definition}[$\mu$-integral]
	The \vocab{$\mu$-integral} of a simple function $f$ defined as above is
	\[ \mu(f) = \sum_{k=1}^m a_k \mu(A_k)\footnote{Note we take $0 \cdot \infty = 0.$} \]
	which is independent of the choice of representation of the simple function, i.e. well-defined.
\end{definition}

\begin{remark} \
	\begin{itemize}
		\item We have $\mu(\alpha f + \beta g) = \alpha \mu(f) + \beta \mu(g)$ for all nonnegative coefficients $\alpha, \beta$ and simple functions $f, g$.
		\item If $g \leq f$, $\mu(g) \leq \mu(f)$, so $\mu$ is increasing.
		\item $f = 0$ a.e. $\iff \mu(f) = 0$.
	\end{itemize}
\end{remark}

\begin{definition}[$\mu$-integral]
	For a general non-negative function $f \colon (E,\mathcal E,\mu) \to \mathbb R$, we define its \vocab{$\mu$-integral} to be
	\begin{align*}
		\mu(f) = \sup\qty{\mu(g) : g \leq f, g \text{ simple}}
	\end{align*}
	which agrees with the above definition for simple functions.
\end{definition}

Clearly if $0 \leq f_1 \leq f_2$ then $\mu(f_1) \leq \mu(f_2)$.

Now, for $f \colon (E,\mathcal E,\mu) \to \mathbb R$ measurable but not necessarily non-negative, we define $f^+ = \max(f,0)$ and $f^- = \max(-f,0)$, so that $f = f^+ - f^-$ and $\abs{f} = f^+ + f^-$.
\begin{definition}[$\mu$-integrable]
	A measurable function $f \colon (E,\mathcal E,\mu) \to \mathbb R$ is \vocab{$\mu$-integrable} if $\mu(\abs{f}) < \infty$.
	In this case, we define its integral to be
	\[ \mu(f) = \mu(f^+) - \mu(f^-) \]
	which is a well-defined real number.
\end{definition}
Later we shall prove that $\mu(|f|) = \mu(f^+) + \mu(f^-)$ hence $|\mu(f)| \leq \mu(|f|)$.

If one of $\mu(f^+)$ or $\mu(f^-)$ is $\infty$ and the other finite, we define $\mu(f)$ to be $\infty$ or $-\infty$ respectively (though $f$ is not integrable).

\subsection{Monotone Convergence Theorem}
\begin{notation} \
	\begin{itemize}
		\item 	We say $x_n \uparrow x$ to mean $x_n \leq x_{n+1} \; \forall n$ and $x_n \to x$.
		\item 	We say $f_n \uparrow f$ to mean $f_n(x) \leq f_{n+1}(x) \; \forall n$ and $f_n(x) \to f$.
	\end{itemize}
\end{notation}

\begin{theorem}[Monotone Convergence Theorem] \label{thm:mct}
	Let $f_n, f \colon (E,\mathcal E,\mu) \to \mathbb R$ be measurable and non-negative s.t. $f_n \uparrow f$.
	Then, $\mu(f_n) \uparrow \mu(f)$.
\end{theorem}

\begin{remark}
	This is a theorem that allows us to interchange a pair of limits, $\mu(f) = \mu\qty(\lim_n f_n) = \lim_n \mu(f_n)$, i.e. $\lim_n \int f_n \dd{\mu} = \int \lim_n f_n \dd{\mu}$ for $f_n \geq 0$ and $f_n \uparrow f$. \\
	If $g_n \geq 0$, letting $f_n = \sum_{k=1}^{n} g_k$ and $f_n \uparrow f = \sum_{k=1}^{\infty} g_k$ we get $\lim_n \int \sum_{k=1}^{n} g_k \dd{\mu} = \int \sum_{k=1}^\infty g_k \dd{\mu} \implies \sum_{k=1}^{\infty} \int g_k \dd{\mu} = \int \sum_k g_k \dd{\mu}$ or equivalently $\mu\qty(\sum_k g_k) = \sum_k \mu(g_k)$.
	This generalises the countable additivity of $\mu$ to integrals of non-negative functions.

	If we consider the approximating sequence $\widetilde f_n = 2^{-n} \floor*{2^n f}$, as defined in the monotone class theorem, then this is a non-negative sequence converging to $f$.
	So in particular, $\mu(f)$ is equal to the limit of the integrals of these simple functions.

	It suffices to require convergence of $f_n \to f$ a.e., the general argument does not need to change.
	The non-negativity constraint is not required if the first term in the sequence $f_0$ is integrable, by subtracting $f_0$ from every term.
\end{remark}

\begin{proof}
	Recall that $\mu(f) = \sup\qty{\mu(g) : g \leq f, g \text{ simple}}$.
	Let $M = \sup_n \mu(f_n)$, then $\mu(f_n) \uparrow M$.

	We now show $M = \mu(f)$.

	Since $f_n \leq f$, $\mu(f_n) \leq \mu(f)$, so taking suprema, $M \leq \mu(f)$.

	Now, we need to show $\mu(f) \leq M$, or equivalently, $\mu(g) \leq M$ for all simple $g$ s.t. $g \leq f$, so by taking suprema, $\mu(f) = \sup_g \mu(g) \leq M$. \\
	Now let $g = \sum_{k=1}^m a_k 1_{A_k}$ where $a_k \geq 0$ and wlog the $A_k \in \mathcal E$ are disjoint.
	We define $g_n = \min (\overline f_n, g)$, where $\overline f_n$ is the $n$th approximation of $f_n$ by simple functions as in the \nameref{thm:monclass}.
	So $g_n$ is simple, $g_n \leq \overline{f}_n \leq f_n \uparrow f$, so $g_n \uparrow \min(f, g) = g$.
	I.e. $g_n \uparrow g$ and $g_n$ simple with $g_n \leq f_n$.

	Fix $\epsilon \in (0, 1)$, and define sets $A_k(n) = \qty{x \in A_k : g_n(x) \geq (1-\varepsilon) a_k}$.
	Since $g = a_k$ on $A_k$, and since $g_n \uparrow g$, $A_k(n) \uparrow A_k$ for all $k$.
	Since $\mu$ is a measure, $\mu(A_k(n)) \uparrow \mu(A_k)$ by countable additivity.

	Also, we have $g_n 1_{A_k} \geq g_n 1_{A_k(n)} \geq (1-\epsilon)a_k 1_{A_k(n)}$ as $A_k(n) \subseteq A_k$.
	So as $\mu(\cdot)$ is increasing, we have $\mu(g_n 1_{A_k}) \geq \mu\qty((1-\epsilon)a_k 1_{A_k(n)})$ and so $\mu(g_n 1_{A_k}) \geq (1-\epsilon)a_k \mu(1_{A_k(n)})$ as they are simple functions.

	Finally, $g_n = \sum_{k=1}^n g_n 1_{A_k}$ as $g_n \leq g$ and $g$ supported on $\bigcup_{k=1}^n A_k$ and $A_k$ disjoint.
	So
	So as $g_n 1_{A_k}$ is simple,
	\begin{align*}
		\mu(g_n) &= \mu\qty(\sum_{k=1}^{n} g_n 1_{A_k}) \\
		&= \sum_{k=1}^{n} \mu(g_n 1_{A_k}) \\
		&\geq \sum_{k=1}^{n} (1-\epsilon)a_k \mu(A_k(n)) \\
		&\uparrow \sum_{k=1}^{n} (1-\epsilon) a_k \mu(A_k) \\
		&= (1 - \epsilon) \mu(g).
	\end{align*}
	Then,
	\begin{align*}
		(1-\epsilon)\mu(g) \leq \lim_n \mu(g_n) \leq\footnote{As $g_n \leq f_n$} \lim_n \mu(f_n) \leq M
	\end{align*} so $\mu(g) \leq \frac{M}{1 - \epsilon} \; \forall \epsilon \in (0, 1)$ hence $\mu(g) \leq M$.
\end{proof}

\subsection{Linearity of Integral}
\begin{theorem}[Linearity of Integral]
	Let $f, g \colon (E, \mathcal E, \mu) \to \mathbb R$ be nonnegative measurable functions.
	Then $\forall \alpha, \beta \geq 0$,
	\begin{itemize}
		\item $\mu(\alpha f + \beta g) = \alpha \mu(f) + \beta \mu(g)$;
		\item $f \leq g \implies \mu(f) \leq \mu(g)$;
		\item $f = 0$ a.e. $\iff \mu(f) = 0$.
	\end{itemize}
\end{theorem}

\begin{proof}
	If $\widetilde f_n, \widetilde g_n$ are the approximations of $f$ and $g$ by simple functions from the \nameref{thm:monclass} let $f_n = \min(\widetilde f_n, n)$\footnote{This ensures that $f_n$ is not an infinite sum of indicators, as discussed in proof of \nameref{thm:monclass} (we assumed $f$ bounded).} and $g_n = \min(\widetilde g_n, n)$.
	Then $f_n, g_n$ are simple and $f_n \uparrow f$ and $g_n \uparrow g$.
	Then $\alpha f_n + \beta g_n \uparrow \alpha f + \beta g$, so by MCT\footnote{\nameref{thm:mct}}, $\mu(f_n) \uparrow \mu(f)$, $\mu(g_n) \uparrow \mu(g)$ and $\mu(\alpha f_n + \beta g_n) \uparrow \mu(\alpha f + \beta g)$.
	As $f_n$, $g_n$ simple $\mu(\alpha f_n + \beta g_n) = \alpha \mu(f_n) + \beta \mu(g_n) \uparrow \alpha \mu(f) + \beta \mu(g)$.
	So $\alpha \mu(f) + \beta \mu(g) = \mu(\alpha f + \beta g)$.

	The second part is obvious from definition.

	If $f = 0$ a.e, then $0 \leq f_n \leq f$, so $f_n = 0$ a.e. but $f_n$ simple $\implies \mu(f_n) = 0$.
	As $\mu(f_n) \uparrow \mu(f)$ so $\mu(f) = 0$. \\
	Conversely, if $\mu(f) = 0$, then $0 \leq \mu(f_n) \uparrow \mu(f)$ so $\mu(f_n) = 0 \; \forall n \implies f_n = 0$ a.e..
	But $f_n \uparrow f \implies f = 0$ a.e.
\end{proof}

\begin{remark}
	Functions such as $1_{\mathbb Q}$ are integrable and have integral zero.
	They are `identified' with the zero element in the theory of integration.
\end{remark}

\begin{theorem}[Linearity of Integral]
	Let $f, g \colon (E, \mathcal E, \mu) \to \mathbb R$ be integrable.
	Then $\forall \alpha, \beta \in \mathbb{R}$,
	\begin{itemize}
		\item $\mu(\alpha f + \beta g) = \alpha \mu(f) + \beta \mu(g)$;
		\item $f \leq g \implies \mu(f) \leq \mu(g)$;
		\item $f = 0$ a.e. $\implies \mu(f) = 0$.
	\end{itemize}
\end{theorem}

\begin{proof}
	Left as an exercise, just use $f = f^+ - f^-$ and use definitions and $\mu(f) = \mu(f^+) - \mu(f^-)$ etc.
\end{proof}

\subsection{Fatou's lemma}
\begin{example}
	Let $f_n = 1_{(n, n+1)}$, $f_n \geq 0$ with $f_n \to 0$ as $n \to \infty$.
	$\lambda(f_n) = 1$ but $\lambda(0) = 0$.
\end{example}

\begin{lemma}[Fatou's lemma]
	Let $f_n \colon (E, \mathcal E, \mu) \to \mathbb R$ be measurable, non-negative functions.
	Then $\mu (\liminf_n f_n) \leq \liminf_n \mu(f_n)$.
\end{lemma}

\begin{remark}
	Recall that $\liminf_n x_n = \sup_n \inf_{m \geq n} x_m$ and $\limsup_n x_n = \inf_n \sup_{m \geq n} x_m$.
	In particular, $\limsup_n x_n = \liminf_n x_n$ implies that $\lim_n x_n$ exists and is equal to $\limsup_n x_n$ and $\liminf_n x_n$.
	Hence, if the $f_n$ converge to some measurable function $f$, we must have $\mu(f) \leq \liminf_n \mu(f_n)$.
\end{remark}

\begin{proof}
	We have $\inf_{m \geq n} f_m \leq f_k$ for all $k \geq n$, so by taking integrals, $\mu\qty(\inf_{m \geq n} f_m) \leq \mu(f_k)$.
	Thus,
	\begin{align*}
		\mu\qty(\inf_{m \geq n} f_m) \leq \inf_{k \geq n} \mu(f_k) \leq \sup_n \inf_{k \geq n} \mu(f_k) = \liminf \mu(f_k) \tag{$\dagger$}
	\end{align*}
	Note that $\inf_{m \geq n} f_m$ increases to $\sup_n \inf_{m \geq n} f_m = \liminf_n f_n$.

	Let $g_n = \inf_{m \geq n} f_n$, then $g_n \geq 0$ and $g_n \uparrow \sup_n g_n = \sup_n \inf_{m \geq n} f_m = \liminf_n f_n$.
	By MCT $\mu(g_n) \uparrow \mu(\liminf_n f_n)$ so by taking limits in $(\dagger)$, $\mu(\liminf_n f_n) \leq \liminf \mu(f_n)$.
\end{proof}

\subsection{Dominated Convergence Theorem}
\begin{theorem}[Dominated Convergence Theorem] \label{thm:dct}
	Let $f_n, f \colon (E, \mathcal E, \mu)$ be measurable functions s.t. $\abs{f_n} \leq g$ a.e., for some integrable fcn $g$, so $\mu(g) < \infty$\footnote{Note $g \geq \abs{f_n} \geq 0$.}, and $f_n \to f$ pointwise (or a.e.) on $E$. \\
	Then $f_n$ and $f$ are also integrable, and $\mu(f_n) \to \mu(f)$.
\end{theorem}

\begin{proof}
	Clearly $\mu(\abs{f_n}) \leq \mu(g) < \infty$, so the $f_n$ are integrable.
	Taking limits in $\abs{f_n} \leq g$, we have $\abs{f} \leq g$, so $f$ is also integrable by the same argument and as the limit of measurable fcns is measurable.

	Now, $g \pm f_n \geq 0$, and converges pointwise to $g \pm f$.
	Since limits are equal to the limit inferior when they exist, by Fatou's lemma, we have
	\[ \mu(g) + \mu(f) = \mu(g + f) = \mu\qty(\liminf_n (g + f_n)) \leq \liminf_n \mu(g + f_n) = \mu(g) + \liminf_n \mu(f_n) \]
	Hence $\mu(f) \leq \liminf_n \mu(f_n)$ as $\mu(g)$ finite.
	Likewise, $\mu(g) - \mu(f) \leq \mu(g) - \limsup_n \mu(f_n)$, so $\mu(f) \geq \limsup_n \mu(f_n)$, so
	\[ \limsup_n \mu(f_n) \leq \mu(f) \leq \liminf_n \mu(f_n) \]
	But since $\liminf_n \mu(f_n) \leq \limsup_n \mu(f_n)$, the result follows.
\end{proof}

\begin{remark}
	In fact, $\mu(\abs{f_n - f}) \to 0$ as $|f_n - f| \leq |f_n| + |f| \leq g + g = 2g$ and $2g$ is integrable so by DCT (\nameref{thm:dct}) proved.

	If $X_n \to X$ $\mathbb{P}$ a.s., and $|X_n| \leq Y$ and $\mathbb{E}[Y] < \infty$ then $\mathbb{E}[X_n] \to \mathbb{E}[X]$ and $\mathbb{E}[\abs{X_n - X}] \to 0$. \\
	In particular, if $\abs{X_n} \leq M \; \forall n$, for some $M > 0, M \in \mathbb{R}$ then $\mathbb{E}[\abs{X_n - X}] \to 0$ (Bounded Convergence Theorem)\footnote{This works as with finite measure then $\mathbb{E}[M]$ is finite}.
\end{remark}

\begin{remark}
	DCT also holds for convergence in $\mathbb{P}$-prob, where if $X_n \to X$ in $\mathbb{P}$ probability then we get $\mathbb{E}[X_n] \to \mathbb{E}[X]$ and $\mathbb{E}[\abs{X_n - X}] \to 0$.
\end{remark}

\begin{proof}
	Suppose $\mathbb{E}[\abs{X_n - X}] \not\to 0$.
	Then $\exists$ a subsequence $n_k$ s.t. $\mathbb{E}\qty[\abs{X_{n_k} - X}] > \epsilon \; \forall k$ for some $\epsilon > 0$.
	Now $X_n \to X$ in $\mathbb{P}$ prob then $X_{n_k} \to X$ in $\mathbb{P}$ prob by definition.
	By \cref{thm:inprobinmeasure}, $\exists n_{k_{l}}$ s.t. $X_{n_{k_l}} \to X$ a.s.
	But then by DCT, $\mathbb{E}\qty[\abs{X_{n_{k_l}} - X}] \to 0$ \Lightning.
\end{proof}

\begin{theorem}[Bounded Convergence Theorem]
	If $X_n \to X$ in $\mathbb{P}$ prob and $\abs{X_n} \leq M$ for some constant $M > 0$, $\forall n \geq 0$.
	Then $\mathbb{E}[\abs{X_n - X}] \to 0$.
\end{theorem}

This is quite useful in probability.

\begin{example}
	Let $E = [0,1]$ with the Lebesgue measure.
	Let $f_n \to f$ pointwise and the $f_n$ are uniformly bounded, so $\sup_n \norm{f_n}_\infty \leq g$ for some $g \in \mathbb R$.
	Then since $\mu(g) = g < \infty$, the DCT implies that $f_n, f$ are integrable and $\mu(f_n) \to \mu(f)$ as $n \to \infty$.

	In particular, no notion of uniform convergence of the $f_n$ is required as in Riemann Integrals.
\end{example}

\begin{remark}
	FTC states that
	\begin{enumerate}
		\item Let $f : [a, b] \to \mathbb{R}$ be continuous and set $F(t) = \int_{a}^{t} f(x) \dd{x}$\footnote{This is a Lebesgue integral}.
		Then $F$ is differentiable on $[a, b]$ with $F' = f$.
		\item Let $F : [a, b] \to \mathbb{R}$ be differentiable and $F'$ is continuous, then $\int_{a}^{b} F'(x) \dd{x} = F(b) - F(a)$.
	\end{enumerate}
	The proof of the fundamental theorem of calculus requires only the fact that
	\[ \int_x^{x + h} \dd{t} = h \]
	This is a fact which is obviously true of the Riemann integral and also of the Lebesgue integral.

	Therefore, for any continuous function $f \colon [0,1] \to \mathbb R$, we have
	\[ \underbrace{\int_0^x f(t) \dd{t}}_{\text{Riemann integral}} = F(x) = \underbrace{\int_0^x f(t) \dd{\mu(t)}}_{\text{Lebesgue integral}} \]
	So these integrals coincide for continuous functions.
\end{remark}

\begin{remark}
	We can generalise the FTC for Lebesgue integrals: \\
	If $f : [a, b] \to \mathbb{R}$ is Lebesgue integrable and $F(t) = \int_{a}^{t} f(x) \dd{x}$.
	Then,
	\begin{align*}
		\lim_{h \to 0} \frac{F(t+h) - F(t)}{h} = \lim_{h \to 0} \frac{\int_{t}^{t + h} f(x) \dd{x}}{h} = f(t) \text{ a.e.}
	\end{align*}
	This is the Lebesgue differentiation theorem, studied in Analysis of Functions.
\end{remark}

\begin{remark}
	We can show that all Riemann integrable functions are $\mu^\star$-measurable, where $\mu^\star$ is the outer measure of the Lebesgue measure, as defined in the proof of Carath\'eodory's theorem.
	However, there exist certain Riemann integrable functions that are not Borel measurable.
	We can modify such an $f$ on a Lebesgue measure $0$ set to make it Borel measurable, i.e. $\exists \widetilde f$ s.t. $\widetilde{f} = f$ on $A$ and $\lambda(A^c) = 0$ and $\int \widetilde{f} \dd{x} = \int f \dd{x}$.

	A (bounded) Riemann integrable fcn $f : [a, b] \to \mathbb{R}$ is Lebesgue integrable in the following sense.
	If $f$ is bounded on $[a, b]$, $f$ is $R$-integrable iff
	\begin{align*}
		\lambda\qty(\mu\qty(\qty{x \in [0,1] : f \text{ is discontinuous at } x})) = 0,
	\end{align*} i.e. $f$ is continuous a.e.

	The standard techniques of Riemann integration, such as substitution and integration by parts, extend to all bounded measurable functions by the monotone class theorem.
\end{remark}

\begin{example}
	$1_\mathbb{Q}$ on $[0, 1]$ is a bounded function on a bounded interval.
	The set of discontinuity points is $[0, 1]$ which is not measure $0$, thus not Riemann Integrable.
	But this is Lebesgue integrable and $1_\mathbb{Q} = 0$ $\lambda$ a.s., so $\lambda(1_\mathbb{Q}) = 0$.
\end{example}

\begin{theorem}[Substitution Formula]
	Let $\phi : [a, b] \to \mathbb{R}$, $\phi$ strictly increasing and continuously differentiable.
	Then $\forall g$ Borel fcns, $g \geq 0$ on $[\phi(a), \phi(b)]$, $\int_{\phi(a)}^{\phi(b)} g(y) \dd{y} = \int_a^b g(\phi(x)) \phi'(x) \dd{x} (\star)$.
\end{theorem}

\begin{proof}
	Let $\mathcal{V}$ be the set of all measurable fcns $g$ for which $(\star)$ holds.
	Then by linearity of integral, $\mathcal{V}$ is a vector space.
	\begin{itemize}
		\item $1 \in \mathcal{V}$ by FTC (2), $1_{(c, d]} \in \mathcal{V}$ by FTC (2).
		\item If $f_n \in \mathcal{V}$, $f_n \uparrow \mathcal{V}$, $f_n \geq 0$ then by \nameref{thm:mct}.
	\end{itemize}
	Hence by \nameref{thm:monclass}, $(\star)$ holds $\forall g \geq 0$ measurable.
\end{proof}

\begin{theorem}[Differentiation Under The Integral Sign]
	Let $U \subseteq \mathbb R$ be an open set and $(E, \mathcal E, \mu)$ be a measure space.
	Let $f \colon U \times E \to \mathbb R$ be s.t.
	\begin{enumerate}
		\item $x \mapsto f(t, x)$ is integrable $\forall t$;
		\item $t \mapsto f(t,x)$ is differentiable $\forall x \in E$;
		\item $\exists g : E \to \mathbb{R}$ integrable s.t. $\abs{\pdv{f}{t}(t, x)} < g(x) \; \forall t \in U, x \in E$;
	\end{enumerate}
	Then $x \mapsto \pdv{f}{t}(t, x)$ is integrable $\forall t$ and,
	\[ F(t) = \int_E f(t,x) \dd{\mu(x)} \implies F'(t) = \int_E \pdv{f}{t}\qty(t,x) \dd{\mu(x)} \]
\end{theorem}

\begin{proof}
	Fix $t$.
	By the mean value theorem,
	\[ g_h(x) = \frac{f(t + h, x) - f(t, x)}{h} - \pdv{f}{t}\qty(t,x) \implies \abs{g_h(x)} = \abs{\pdv{f}{t}\qty(\widetilde t, x) - \pdv{f}{t}\qty(t, x)} \leq 2g(x) \]
	Note that $g$ is integrable.
	By differentiability of $f$, we have $g_h \to 0$ as $h \to 0$, so by DCT, $\mu(g_h) \to \mu(0) = 0$.
	By linearity of the integral,
	\[ \mu(g_h) = \frac{\int_E f(t + h, x) - f(t, x) \dd{\mu(x)}}{h} - \int_E \pdv{f}{t}\qty(t,x) \dd{\mu(x)} \]
	Hence, $\frac{F(t+h) - F(t)}{h} - F'(t) \to 0$.
\end{proof}

\begin{example}[Integrals and Image Measures.]
	For a measurable function $f \colon (E, \mathcal E, \mu) \to (G, \mathcal G)$ with image measure $\nu = \mu \circ f\inv$ on $(G, \mathcal{G})$.
	If $g \colon G \to \mathbb R$ is a measurable, non-negative function then,
	\begin{align*}
		\mu \circ f\inv(g) = \nu(g) = \int_G g \dd{\nu} = \int_G g \dd{\mu\circ f^{-1}} = \int_E g(f(x)) \dd{\mu(x)} = \mu(g \circ f)
	\end{align*}
	Proof on Sheet 2, use monotone class theorem and first prove for $g$ indicator fcns and then simple functions.

	In particular, for $X : (\Omega, \mathcal{F}, \mathbb{P}) \to \mathbb{R}$ and $X \geq 0$ measurable, we have,
	\[ \expect{g(X)} = \int_\Omega g(X(\omega)) \dd{\mathbb P(\omega)} = \int g \dd{\mu_X}, \]
	where $\mu_X = \mathbb{P} \circ X\inv$ is the distribution of $X$.
\end{example}

\begin{example}[Densities of Measures]
	If $f \colon (E, \mathcal E, \mu) \to \mathbb R$ is a measurable non-negative function, we can define $\nu(A) = \mu(f 1_A)$ for any measurable set $A$, which is again a measure on $(E, \mathcal E)$ by the \nameref{thm:mct}.
	For $A_n$ disjoint,
	\begin{align*}
		\nu\qty(\bigcup_{i=1}^\infty A_i) &= \mu(f 1_{\cup A_i}) \\
		&= \mu(f \sum_{i=1}^{\infty} 1_{A_i}) \\
		&= \mu(\sum_{i=1}^{\infty} f 1_{A_i}) \\
		&= \sum_{i=1}^{\infty} \mu(f 1_{A_i}) \text{ by MCT} \\
		&= \sum_{i=1}^{\infty} \nu(A_i).
	\end{align*}

	In particular, if $g \colon (E, \mathcal E) \to \mathbb R$ is measurable, $\nu(g) = \mu(fg) = \int_E g(x) f(x) \dd{\mu(x)} = \int_E g \dd{\nu(f)}$.
	This follows by definition for $g$ indicator functions, by additivity extends to simple functions and by \nameref{thm:mct} to all measurable non-negative functions.

	We call $f$ the \vocab{density} of $\nu$ with respect to $\mu$.
	This is unique as $\mu(f 1_A) = \mu(g 1_A) \ \forall A \in \mathcal{E} \implies f = g$ $\mu$ a.e. (proved on Sheet 2).

	In particular, for $\mu = \lambda$, $\forall f$ Borel $\exists$ a Borel measure $\nu$ on $\mathbb{R}$ given by $\nu(A) = \int_A f(x) \dd{x}$ and then $\forall g$ Borel, $g \geq 0$ $\nu(g) = \int f(x) g(x) \dd{x}$.
	We say $\nu$ has \vocab{density} $f$.
	This $\nu$ is a prob measure on $(\mathbb{R}, \mathcal{B})$ iff $\int f(x) \dd{x} = 1$.

	For $\lambda : (\Omega, \mathcal{F}, \mathbb{P}) \to \mathbb{R}$, if the law $\mu_X = \mathbb{P} \circ X\inv$ has the density $f_X$ (wrt $\lambda$), we call $f_X$ the \vocab{probability density function} of $X$.
	Then $\mathbb{P}(X \in A) = \mathbb{P} \circ X\inv (A) = \mu_X(A) = \int_A f_X(x) \dd{x} \ \forall A \in \mathcal{B}$ and $\forall g$ Borel, $g \geq 0$.
	Taking $A = (-\infty, x]$, we get $\mathbb{P}(X \leq x) = F_X(x) = \int_{-\infty}^{x} f_X(x) \dd{x}$. \\
	$\mathbb{E}[g(x)] = \int g(x) \dd{\mu_X(x)}$ from previous example and $\int g(x) \dd{\mu_X(x)} = \nu(g) = \int g(x) f_X(x) \dd{x}$.
\end{example}

    \section{Product Measures}
\subsection{Integration in product spaces}
Let $(E_1, \mathcal E_1, \mu_1), (E_2, \mathcal E_2, \mu_2)$ be finite measure spaces.
On $E = E_1 \times E_2$, we can consider the $\pi$-system of `rectangles' $\mathcal A = \qty{A_1 \times A_2 : A_1 \in \mathcal E_1, A_2 \in \mathcal E_2}$.
Then we define the $\sigma$-algebra $\mathcal{E} = \mathcal E_1 \otimes \mathcal E_2 = \sigma(\mathcal A)$ on the product space.

If the $E_i$ are topological spaces with a countable basis, then $\mathcal B(E_1 \times E_2) = \mathcal B(E_1) \otimes \mathcal B(E_2)$ where we take the product topology.

\begin{lemma}
	Let $f \colon (E, \mathcal E) \to \mathbb R$ be measurable.
	Then $\forall \; x_1 \in E_1$, the fcn $(x_2 \mapsto f(x_1, x_2)) \colon (E_2, \mathcal E_2) \to \mathbb R$ is $\mathcal E_2$-measurable.
\end{lemma}

\begin{proof}
	Let
	\[ \mathcal V = \qty{f \colon (E,\mathcal E) \to \mathbb R : f \text{ bounded, measurable, conclusion of the lemma holds}} \]
	This is a $\mathbb R$-vector space, and $1_E, 1_A \in \mathcal{V} \ \forall \; A = A_1 \times A_2 \in \mathcal A$, since $1_A(x_1, x_2) = 1_{A_1}(x_1) 1_{A_2}(x_2)$ thus fixing $x_1$ gives $0$ or $1_{A_2}$.

	Now, let $0 \leq f_n$ increase to $f$, $f_n \in \mathcal V$.
	Then $(x_2 \mapsto f(x_1, x_2)) = \lim_n (x_2 \mapsto f_n(x_1, x_2))$, so it is $\mathcal E_2$-measurable as it's a limit of a sequence of measurable functions.
	Then by the \nameref{thm:monclass}, $\mathcal V$ contains all bounded measurable functions.
	This extends to all measurable functions by truncating the absolute value of $f$ to $n \in \mathbb N$, then the sequence of such bounded truncations converges pointwise to $f$.
\end{proof}

\begin{lemma} \label{lem:4-2}
	Let $f \colon (E, \mathcal E) \to \mathbb R$ be measurable s.t.
	\begin{enumerate}
		\item $f$ is bounded; or
		\item $f$ is nonnegative.
	\end{enumerate}
	Then the map $x_1 \mapsto \int_{E_2} f(x_1,x_2) \dd{\mu_2(x_2)}$ is $\mu_1$-measurable and is bounded\footnote{As $\mu_2$ is a finite measure.} or nonnegative respectively.
\end{lemma}

\begin{remark}
	In case (ii), the map on $x_1$ may evaluate to infinity, but the set of values
	\[ \qty{x_1 \in E_1 : \int_{E_2} f(x_1,x_2) \dd{\mu_2(x_2)} = \infty} \]
	lies in $\mathcal E_1$.

	Generally, a fcn $f$ taking values in $[0, \infty]$ is measurable means $f\inv(\{\infty\}) \in \mathcal{E}_1$ and $f\inv(A) \in \mathcal{E}_1 \ \forall \; A \in \mathcal{B}$.
\end{remark}

\begin{proof}
	Let
	\[ \mathcal V = \qty{f \colon (E,\mathcal E) \to \mathbb R : f \text{ bounded, measurable, conclusion of the lemma holds}} \]
	This is a vector space by linearity of the integral.
	$1_E \in \mathcal V$, since $\int_{E_2} 1_E(x_1,x_2) \dd{\mu_2(x_2)} = 1_{E_1} \mu_2(E_2)$ is non-negative and bounded.
	$1_A \in \mathcal V \ \ \forall \; A \in \mathcal A$, because $1_{A_1}(x_1) \mu_2(A_2)$ is $\mathcal E_1$-measurable, non-negative, and bounded since it is at most $\mu_2(E_2) < \infty$.

	Now let $f_n$ be a sequence of non-negative functions that increase to $f$, where $f_n \in \mathcal V$.
	Then by the \nameref{thm:mct},
	\[ \int_{E_2} \lim_{n \to \infty} f_n(x_1, x_2) \dd{\mu_2(x_2)} = \lim_{n \to \infty} \int_{E_2} f_n(x_1, x_2) \dd{\mu_2(x_2)} \]
	is an increasing limit of $\mathcal E_1$-measurable functions, so is $\mathcal E_1$-measurable.
	It is bounded by $\mu_2(E_2) \norm{f}_\infty$, or non-negative as required.
	So $f \in \mathcal V$.
	By the \nameref{thm:monclass}, the result for bounded functions holds.

	% In case (ii), we can take a bounded approximation in $\mathcal V$ of an arbitrary measurable function $f$ to conclude the proof.
\end{proof}

\begin{theorem}[Product Measure]
	There $\exists$ a unique measure $\mu = \mu_1 \otimes \mu_2$ on $(E, \mathcal E)$ such that $\mu(A_1 \times A_2) = \mu_1(A_1) \mu_2(A_2)$ for all $A_1 \in \mathcal E_1$, $A_2 \in \mathcal E_2$.
\end{theorem}

\begin{proof}
	$\mathcal A$ is a $\pi$-system generating $\mathcal E$ and $\mu$ a finite measure, so by the \nameref{thm:uni}, $\mu$ unique.

	We define for $A \in \mathcal{E}$,
	\[ \mu(A) = \int_{E_1} \qty( \int_{E_2} 1_A(x_1,x_2) \dd{\mu_2(x_2)} ) \dd{\mu_1(x_1)}. \]
	This is well-defined by the two previous lemmas.

	We have
	\begin{align*}
		\mu(A_1 \times A_2) &= \int_{E_1} \qty( \int_{E_2} 1_{A_1}(x_1) 1_{A_2}(x_2) \dd{\mu_2(x_2)} ) \dd{\mu_1(x_1)} \\
		&= \int_{E_1} 1_{A_1}(x_1) \mu_2(A_2) \dd{\mu_1(x_1)} \\
		&= \mu_1(A_1) \mu_2(A_2)
	\end{align*}
	Clearly $\mu(\varnothing) = 0$, so it suffices to show countable additivity.
	Let $A_n$ be disjoint sets in $\mathcal E$.
	Then
	\[ 1_{\qty(\bigcup_n A_n)} = \sum_n 1_{A_n} = \lim_{n \to \infty} \sum_{i=1}^n 1_{A_n} \]
	Then by the \nameref{thm:mct} and the previous lemmas,
	\begin{align*}
		\mu\qty(\bigcup_n A_n) &= \int_{E_1} \qty( \int_{E_2} \lim_{n \to \infty} \sum_{i=1}^n 1_{A_i} \dd{\mu_2(x_2)} ) \dd{\mu_1(x_1)} \\
		&= \int_{E_1} \qty( \lim_{n \to \infty} \int_{E_2} \sum_{i=1}^n 1_{A_i} \dd{\mu_2(x_2)} ) \dd{\mu_1(x_1)} \\
		&= \lim_{n \to \infty} \int_{E_1} \qty( \int_{E_2} \sum_{i=1}^n 1_{A_i} \dd{\mu_2(x_2)} ) \dd{\mu_1(x_1)} \\
		&= \lim_{n \to \infty} \sum_{i=1}^n \int_{E_1} \qty( \int_{E_2} 1_{A_i} \dd{\mu_2(x_2)} ) \dd{\mu_1(x_1)} \\
		&= \lim_{n \to \infty} \sum_{i=1}^n \mu(A_i) \\
		&= \sum_{n=1}^\infty \mu(A_n)
	\end{align*}
\end{proof}

\begin{remark}
	Note $\mu(A) = \int_{E_2} \qty(\int_{E_1} 1_A(x_1, x_2) \dd{\mu_1(x_1)}) \dd{\mu_2(x_2)}$ by just swapping the order of integration in the previous lemmas and proofs and then by \nameref{lem:dyn}.
\end{remark}

\subsection{Fubini's theorem}
\begin{theorem}[Fubini-Tonelli]
	Let $(E, \mathcal E, \mu) = (E_1 \times E_2, \mathcal E_1 \otimes \mathcal E_2, \mu_1 \otimes \mu_2)$ be a finite measure space.
	\begin{enumerate}
		\item Let $f \colon E \to \mathbb R$ be measurable, non-negative.
		Then
		\begin{align*}
			\mu(f) &= \int_E f \dd{\mu} \\
			&= \int_{E_1} \qty( \int_{E_2} f(x_1,x_2) \dd{\mu_2(x_2)} ) \dd{\mu_1(x_1)} \\
			&= \int_{E_2} \qty( \int_{E_1} f(x_1,x_2) \dd{\mu_1(x_1)} ) \dd{\mu_2(x_2)}
		\end{align*}
		\item Let $f \colon E \to \mathbb R$ be a $\mu$-integrable function (on the product measure).
		Let
		\[ A_1 = \qty{x_1 \in E_1 : \int_{E_2} \abs{f(x_1,x_2)} \dd{\mu_2(x_2)} < \infty}. \]
		Define $f_1 : E_1 \to \mathbb{R}$ by $f_1(x_1) = \int_{E_2} f(x_1,x_2) \dd{\mu_2(x_2)}$ on $A_1$ and 0 elsewhere. \\
		Then $\mu_1(A_1^c) = 0$, $f_1$ is $\mu_1$-integrable and $\mu(f) = \mu_1(f_1) = \mu_1(f_1 1_{A_1})$, and defining $A_2$ symmetrically, $\mu(f) = \mu_2(f_2) = \mu_2(f_2 1_{A_2})$.
	\end{enumerate}
\end{theorem}

\begin{remark}
	If $f$ is bounded, $A_1 = E_1$.
	Note, for $f(x_1,x_2) = \frac{x_1^2-x_2^2}{(x_1^2+x_2^2)^2}$ on $(0,1)^2$, we have $\mu_1(f_1) \neq \mu_2(f_2)$, but $f$ is not Lebesgue integrable on $(0,1)^2$.
\end{remark}

\begin{proof}
	By the definition of the product measure, first statement is true for $f = 1_A$ for $A \in \mathcal{E}$.
	Then, by linearity of the integral, this extends to simple functions.
	For general fcn $f \geq 0$ by \nameref{thm:mct} and the standard approximation by simple fcns $f_n = \min(2^{-n} \floor{2^n f}, n)$, the first statement follows.

	Now let $f$ be $\mu$-integrable.
	Define $h : E_1 \to [0, \infty]$ as $h(x_1) = \int_{E_2} \abs{f(x_1,x_2)} \dd{\mu_2(x_2)}$.
	By \Cref{lem:4-2}, $h$ is measurable (as $\abs{f} \geq 0$), is non-negative, so $A_1 \in \mathcal{E}_1$\footnote{$h$ measurable $\implies h\inv(\{\infty\}) \in \mathcal{E}_1$. $A_1 = h\inv(\{\infty\})^c$ thus in $\mathcal{E}_1$.}. \\
	Then by the first part, $\mu_1(h) \leq \mu(\abs{f}) < \infty$.
	So $f_1$ is $\mu_1$-integrable.
	We have $\mu_1(A_1^c) = 0$, otherwise $\mu_1(h) \geq \mu_1(h 1_{A_1^c}) = \infty$ \Lightning.

	Setting, $f_1^\pm = \int_{E_2} f^\pm(x_1,x_2) \dd{\mu_2(x_2)}$ we see than $f_1 = (f_1^+ - f_1^-) 1_{A_1}$.
	Also by the first part, $\mu_1(f_1^+) = \mu(f^+) < \infty$ and $\mu_1(f_1^-) = \mu(f^-) < \infty$.
	Hence, $\mu(f) =\footnote{As $f$ integrable} \mu(f^+) - \mu(f^-) = \mu_1(f_1^+) - \mu_1(f_1^-) =\footnote{As $f_1$ integrable due to $\mu_1(A_1^c) = 0$.} \mu_1(f_1)$ as required.
\end{proof}

\begin{remark}
	The proofs above extend to $\sigma$-finite measures $\mu$.

	Let $(E_i, \mathcal E_i, \mu_i)$ be measure spaces with $\sigma$-finite measures.
	Note that $(\mathcal E_1 \otimes \mathcal E_2) \otimes \mathcal E_3 = \mathcal E_1 \otimes (\mathcal E_2 \otimes \mathcal E_3)$, by a $\pi$-system argument using Dynkin's lemma.
	So we can iterate the construction of the product measure to obtain a measure $\mu_1 \otimes \dots \otimes \mu_n$\footnote{This is associative.}, which is a unique measure on $\qty(\prod_{i=1}^n E_i, \bigotimes_{i=1}^n \mathcal E_i)$ with the property that the measure of a hypercube $\mu(A_1 \times A_n)$ is the product of the measures of its sides $\mu_i(A_i)$.

	In particular, we have constructed the Lebesgue measure $\mu^n = \bigotimes_{i=1}^n \mu$ on $\mathbb R^n$.
	Applying Fubini's theorem, for functions $f$ that are either non-negative and measurable or $\mu^n$-integrable, we have
	\[ \int_{\mathbb R^n} f \dd{\mu^n} = \idotsint_{\mathbb R \dots \mathbb R} f(x_1, \dots, x_n) \dd{\mu(x_1)} \dots \dd{\mu(x_n)} \]
\end{remark}

\subsection{Product probability spaces and independence}
\begin{proposition}
	Let $X_1, \dots, X_n$ be r.v.s, $X_i : (\Omega, \mathcal F, \mathbb P) \to (E_i, \mathcal{E}_i)$.
	Set $(E, \mathcal E) = \qty(\prod_{i=1}^n E_i, \bigotimes_{i=1}^n \mathcal E_i)$.
	Consider $X \colon (\Omega, \mathcal F, \mathbb{P}) \to (E, \mathcal E)$ given by $X(\omega) = (X_1(\omega), X_2(\omega), \dots, X_n(\omega))$.
	Then $X$ is $\mathcal{E}$-measurable and the following are equivalent.
	\begin{enumerate}
		\item $X_1, \dots, X_n$ are independent random variables;
		\item $\mu_X = \bigotimes_{i=1}^n \mu_{X_i}$;
		\item for all bounded and measurable $f_i \colon E_i \to \mathbb R$, $\expect{\prod_{i=1}^n f_i(X_i)} = \prod_{i=1}^n \expect{f_i(X_i)}$.
	\end{enumerate}
\end{proposition}
\begin{proof}
	To show $X$ measurable suffices to check $X\inv(A_1 \times \dots \times A_n) \in \mathcal{F}$, where $A_i \in \mathcal{E}_i \ \forall \; i$ as this is a $\pi$-system generating $\mathcal{E}$.
	\begin{align*}
		X\inv(A_1 \times \dots \times A_n) &= \qty{\omega : X_1(\omega) \in A_1, \dots, X_n(\omega) \in A_n} \\
		&= \bigcap_{i = 1}^n X_i\inv(A_i).
	\end{align*}
	$X_i$ measurable so $X_i\inv(A_i) \in \mathcal{F}$ and so the intersection is in $\mathcal{F}$.

	(1) $\implies$ (2):
	Consider the $\pi$-system $\mathcal A$ of rectangles $A = \prod_{i=1}^n A_i$ for $A_i \in \mathcal E_i$, as this generates $\mathcal{E}$ suffices to check equality on it.

	Since $\mu_X$ is an image measure, then
	\begin{align*}
		\mu_X(A_1 \times \dots \times A_n) = \prob{X_1 \in A_1, \dots, X_n \in A_n} = \prob{X_1} \dots \prob{A_n} &= \prod_{i=1}^n \mu_{X_i}(A_i) \\
		&= \qty(\bigotimes_{i=1}^n \mu_{X_i})(A).
	\end{align*}

	(2) $\implies$ (3):
	By Fubini's theorem,
	\begin{align*}
		\expect{\prod_{i=1}^n f_i(X_i)} &= \mu_X\qty(\prod_{i=1}^n f_i(x_i)) \\
		&= \int_E f(x) \dd{\mu_X(x)} \\
		&= \idotsint_{E_i} \qty(\prod_{i=1}^n f_i(x_i)) \dd{\mu_{X_1}(x_1)} \dots \dd{\mu_{X_2}(x_2)} \\
		&= \prod_{i=1}^n \int_{E_i} f_i(x_i) \dd{\mu_{X_i}(x_i)} \\
		&= \prod_{i=1}^n \expect{f_i(X_i)}
	\end{align*}

	(3) $\implies$ (1):
	Let $f_i = 1_{A_i}$ for any $A_i \in \mathcal E_i$.
	These are bounded and measurable functions.
	Then
	\[ \prob{X_1 \in A_1, \dots, X_n \in A_n} = \expect{\prod_{i=1}^n 1_{A_i}(X_i)} = \prod_{i=1}^n \expect{1_{A_i}(X_i)} = \prod_{i=1}^n \prob{X_i \in A_i} \]
	So the $\sigma$-algebras generated by the $X_i$ are independent as required.
\end{proof}

    \section{$L^p$ Spaces, Norms and Inequalities}
\subsection{Norms}
\begin{definition}[Norm]
	A \vocab{norm} on a real vector space is a map $\norm{\wildcard}_V \colon V \to \mathbb [0, \infty)$ s.t.
	\begin{enumerate}
		\item $\norm{\lambda v} = \abs{\lambda} \cdot \norm{v}$;
		\item $\norm{u + v} \leq \norm{u} + \norm{v}$;
		\item $\norm{v} = 0 \iff v = 0$.
	\end{enumerate}
\end{definition}

\begin{definition}
	Let $(E, \mathcal E, \mu)$ be a measure space.
	We define $L^p(E,\mathcal E,\mu) = L^p(\mu) = L^p$ for the space of measurable functions $f \colon E \to \mathbb R$ s.t. $\norm{f}_p$ is finite, where
	\[ \norm{f}_p = \begin{cases}
		\qty(\int_E \abs{f(x)}^p \dd{\mu(x)})^{\frac{1}{p}} & 1 \leq p < \infty \\
		\esssup \abs{f} = \inf \qty{\lambda \geq 0 : \abs{f} \leq \lambda \text{ $\mu$-a.e.}} & p = \infty
	\end{cases} \]
\end{definition}

We must check that $\norm{\wildcard}_p$ as defined is a norm. \\
Clearly (1) holds for all $1 \leq p < \infty$ by linearity of integral and for $p = \infty$ its obvious. \\
Property (2) holds for $p = 1$ and $p = \infty$, and we will prove later that this holds for other values of $p$ by Minkowski inequality. \\
The last property does not hold: $f = 0$ implies $\norm{f}_p = 0$, but $\norm{f}_p = 0$ implies only that $\abs{f}^p = 0$ a.e., so $f$ is zero a.e. on $E$.

Therefore, to rigorously define the norm, we must construct the quotient space $\mathcal L^p$ of functions that coincide a.e..
We write $[f]$ for the equivalence class of functions that are equal a.e.
The functional $\norm{\wildcard}_p$ is then a norm on $\mathcal L^p = \qty{[f] : f \in L^p}$.

\begin{proposition}[Chebyshev's inequality, Markov's inequality]
	Let $f \colon E \to \mathbb R$ be non-negative and measurable.
	Then $\forall \lambda > 0$,
	\[ \mu(\qty{x \in E : f(x) \geq \lambda}) = \mu(f \geq \lambda) \leq \frac{\mu(f)}{\lambda} \]
\end{proposition}

\begin{proof}
	Integrate the inequality $\lambda 1_{\qty{f \geq \lambda}} \leq f$, which holds on $E$.
\end{proof}

In particular if $g \in L^p$, $p < \infty$ and $\lambda > 0$ then $\mu(\abs{g} \geq \lambda) = \mu(\abs{g}^p \geq \lambda^p) \leq \frac{\mu(|g|^p)}{\lambda^p} \leq \infty$ this gives the tail estimates as $\lambda \to \infty$.

\begin{definition}[Convex Function]
	Let $I \subseteq \mathbb{R}$ be an interval.
	Then we say a map $c \colon I \to \mathbb R$ is \vocab{convex} if for all $x, y \in I$ and $t \in [0,1]$, we have $c(tx + (1-t)y) \leq tc(x) + (1-t)c(y)$.
	Equivalently, for all $x < t < y$ and $x, y \in I$, we have $\frac{c(t) - c(x)}{t-x} \leq \frac{c(y) - c(t)}{y-t}$.
\end{definition}

Thus a convex function is continuous on the interior of the interval and so is Borel measurable.

\begin{lemma}
	Let $I \subseteq R$ be an interval and $c : I \to \mathbb{R}$, and let $m \in$ the interior of $I$.
	If $c$ is convex on $I$, $\exists a, b$ s.t. $c(x) \geq ax + b \ \forall x \in I$, and $c(m) = am + b$.
\end{lemma}

\begin{proof}
	Define $a = \sup \qty{\frac{c(m) - c(x)}{m - x} : x < m, x \in I}$.
	This exists in $\mathbb R$ by the second definition of convexity.
	Let $x, y \in I$, and $y > m > x$.
	Then $\frac{c(m) - c(x)}{m - x} \leq a \leq \frac{c(y) - c(m)}{y - m}$, so $c(y) \geq ay - am + c(m) = ay + b$ where we define $b = c(m) - am$.
	Similarly, for $x$, we have $c(x) \geq ax + b$.
\end{proof}

\begin{theorem}[Jensen's inequality]
	Let $X$ be a integrable\footnote{$\expect{\abs{X}} < \infty$} r.v. taking values in an interval $I \subseteq \mathbb R$.
	Let $c \colon I \to \mathbb R$ be a convex function.
	Then $\expect{c(X)}$ well-defined and
	\begin{align*}
		c(\expect{X}) \leq \expect{c(X)}.
	\end{align*}
\end{theorem}

% Note that the integral $\expect{c(X)}$ is defined as $\expect{c^+(X)} - \expect{c^-(X)}$, and this is well-defined and takes values in $(-\infty, \infty]$.

\begin{proof}
	If $X$ is a constant a.s., then done.

	Otherwise, then $m = \mathbb{E}[X] \in \operatorname{int}\footnote{Interior of} I$

	Using the previous lemma, $\exists a, b$ s.t. $c(X) \geq aX + b$.
	In particular, $(c(X))^- \leq |a| |X| + |b|$\footnote{$f \geq g$ gives $-f \leq -g$ so $f^- \leq g^- \leq |g|$.}.
	Hence, $\expect{c^-(X)} \leq \abs{a} \expect{\abs{X}} + \abs{b} < \infty$, and $\expect{c(X)} = \expect{c^+(X)} - \expect{c^-(X)}$ is well-defined in $(-\infty,\infty]$.

	Integrating $c(X) \geq aX + b$\footnote{If $\mathbb{E}[c(X)] = \infty$ done.},
	\[ \expect{c(X)} \geq a \expect{X} + b\footnote{Expectation of $\Omega$ is 1, this is why we need a probability measure.} = am + b = c(m) = c(\expect{X}) \]
\end{proof}

\begin{example}
	$(\Omega, \mathcal{F}, \mathbb{P})$ and $1 \leq p \leq \infty$.
	If $X \in L^\infty(\mathbb{P})$ then $X \in L^p(\mathbb{P})$ as $\norm{X}_p \leq \norm{X}_\infty$ as $\mathbb{P}(\Omega)$ finite.
\end{example}

\begin{example}
	If $1 \leq p < q < \infty$, $c(x) = \abs{x}^{\frac{q}{p}}$ is a convex function.
	If $X \in L^q(\mathbb P)$, we then have
	\[ \norm{X}_p = \expect{\abs{X^p}}^{\frac{1}{p}} = c(\expect{\abs{X}^p})^{\frac{1}{q}} \mathcolor{red}{\leq}\footnote{By Jensen} \expect{c(\abs{X}^p)}^{\frac{1}{q}} = \norm{X}_q \]

	So $X \in L^q \implies X \in L^p$ so $L^q(\mathbb P) \subseteq L^p(\mathbb P)$ for all $1 \leq p \leq q \leq \infty$.
\end{example}

\begin{theorem}[H\"older's Inequality]
	Let $f, g$ be measurable functions on $(E,\mathcal E,\mu)$.
	If $p, q$ are \vocab{conjugate}, so $\frac{1}{p} + \frac{1}{q} = 1$ and $1 \leq p \leq q \leq \infty$, we have
	\[ \mu(\abs{fg}) = \int_E \abs{f(x)g(x)} \dd{\mu} \leq \norm{f}_p \cdot \norm{g}_q \]
\end{theorem}

\begin{remark}
	For $p = q = 2$, this is exactly the Cauchy--Schwarz inequality on $L^2$ (Simpler proof on Sheet 3 by considering $\int (f+g)^2 \geq 0$.).
\end{remark}

\begin{proof}
	The cases $p = 1$ or $p = \infty$ are obvious.
	We can assume $f \in L^p$ and $g \in L^q$ wlog since the right hand side would otherwise be infinite.
	We can also assume $f$ is not equal to zero a.e., otherwise this reduces to $0 \leq 0$.

	Hence, $\norm{f}_p > 0$.
	Then, we can divide both sides by $\norm{f}_p$ and then assume $\norm{f}_p = 1$.

	Define a prob measure $\mathbb{P}$ on $\mathcal{E}$ by $\mathbb{P}(A) = \int_A |f|^p \dd{\mu}$ ($\mathbb{P}$ has prob density $|f|^p$ wrt $\mu$).
	Note, for $h \geq 0$ $\int h \dd{\mathbb{P}} = \int h |f|^p \dd{\mu}$.

	The
	\begin{align*}
		\mu(|fg|) &= \mu(|fg| 1_{|f| > 0}) \\
		&= \int |f| |g| 1_{|f| > 0} \dd{\mu} \\
		&= \int \frac{|f|^p}{|f|^{p-1}} |g| 1_{|f| > 0} \dd{\mu} \\
		&= \int \frac{|g|}{|f|^{p-1}} 1_{|f| > 0} |f|^p \dd{\mu} \\
		&= \int \frac{|g|}{|f|^{p-1}} 1_{|f| > 0} \dd{\mathbb{P}} \\
		&= \mathbb{E}\qty[\frac{|g|}{|f|^{p-1}} 1_{|f| > 0}] \\
		&\leq \mathbb{E}\qty[\qty(\frac{|g|}{|f|^{p-1}} 1_{|f| > 0})^q]^{\frac{1}{q}}\footnote{Proven earlier by Jensen's that $\norm{X}_p \leq \norm{X}_q$ for $1 \leq p \leq q$.} \\
		&= \mathbb{E}\qty[\frac{|g|^q}{|f|^p} 1_{|f| > 0}]^{\frac{1}{q}} \\
		&= \qty(\int \frac{|g|^q}{|f|^p} 1_{|f| > 0} \dd{\mathbb{P}})^{\frac{1}{q}} \\
		&= \qty(\int |g|^q 1_{|f| > 0} \dd{\mu})^{\frac{1}{q}} \\
		&\leq \qty(\int |g|^q \dd{\mu})^{\frac{1}{q}} \\
		&= \norm{g}_q
	\end{align*}
\end{proof}

\begin{theorem}[Minkowski's inequality]
	Let $f, g \colon (E, \mathcal E, \mu) \to \mathbb R$ be measurable functions.
	Then for all $1 \leq p \leq \infty$, we have $\norm{f + g}_p \leq \norm{f}_p + \norm{g}_p$.
\end{theorem}

\begin{proof}
	The results for $p = 1, \infty$ are clear.
	Suppose $1 < p < \infty$.
	We can assume wlog that $f, g \in L^p$.

	We can integrate the pointwise inequality $\abs{f + g}^p \leq 2^p (\abs{f}^p + \abs{g}^p)$ to deduce that $\mu(|f+g|^p) \leq 2^p \qty[\mu(|f|^p) + \mu(|g|^p)] < \infty$ so $f + g \in L^p$.
	We assume that $0 < \norm{f+g}_p$, otherwise the result is trivial.
	Now, using H\"older's inequality with $q$ conjugate to $p$,
	\begin{align*}
		\norm{f+g}_p^p = \int \abs{f + g}^p \dd{\mu} &= \int \abs{f + g}^{p-1} \abs{f + g} \dd{\mu} \\
		&\leq \int \abs{f + g}^{p-1} \abs{f} \dd{\mu} + \int \abs{f + g}^{p-1} \abs{g} \dd{\mu} \\
		&\leq\footnote{By Holder's Inequality} \norm{f}_p \norm{(f+g)^{p-1}}_q + \norm{g}_p \norm{(f+g)^{p-1}}_q \\
		&\leq \qty(\int \abs{f + g}^{q(p-1)} \dd{\mu})^{\frac{1}{q}} \qty(\norm{f}_p + \norm{g}_p) \\
		&\leq \qty(\int \abs{f + g}^p \dd{\mu})^{\frac{1}{q}} \qty(\norm{f}_p + \norm{g}_p) \\
		&\leq \norm{f+g}_p^{\frac{p}{q}} \qty(\norm{f}_p + \norm{g}_p)
	\end{align*}
	Dividing both sides by $\norm{f+g}_p^{\frac{p}{q}}$ noting $\frac{p}{q} = p-1$, we obtain $\norm{f+g}_p \leq \norm{f}_p + \norm{g}_p$.
\end{proof}

So the $L^p$ spaces are indeed normed spaces.

\subsection{Banach spaces}

\begin{definition}[Banach Space]
	A \vocab{Banach space} is a complete normed vector space.
\end{definition}

\begin{theorem}[$\mathcal L^p$ is a Banach space]
	Let $1 \leq p \leq \infty$, and let $f_n \in L^p$ be a Cauchy sequence, so $\forall \varepsilon > 0 \ \exists N$ s.t. $\forall m, n \geq N$, we have $\norm{f_m - f_n}_p < \varepsilon$.
	Then $\exists f \in L^p$ s.t. $f_n \to f$ in $L^p$, so $\norm{f_n - f}_p \to 0$ as $n \to \infty$.
\end{theorem}

\begin{proof}
	For this proof, we assume $p < \infty$; the other case is already proven in IB Analysis and Topology.

	Since $f_n$ is Cauchy, using $\varepsilon = 2^{-k}$ we extract a subsequence $f_{N_k}$ of $L^p$ functions s.t.
	\[ S = \sum_{k=1}^\infty \norm{f_{N_{k+1}} - f_{N_k}}_p \leq \sum_{k=1}^\infty 2^{-k} < \infty \]
	By Minkowski's inequality, for any $K$, we have
	\[ \norm{\sum_{k=1}^K \abs{f_{N_{k+1}} - f_{N_k}}}_p \leq \sum_{k=1}^K \norm{f_{N_{k+1}} - f_{N_k}}_p \leq S < \infty. \]
	So $\int \abs{\sum_{k=1}^K \abs{f_{N_{k+1}} - f_{N_k}}}^p \dd{\mu} \leq S^p < \infty$.

	By the monotone convergence theorem applied to $\abs{\sum_{k=1}^K \abs{f_{N_{k+1}} - f_{N_k}}}^p$ which increases to $\abs{\sum_{k=1}^\infty \abs{f_{N_{k+1}} - f_{N_k}}}^p$, we find
	\[ \norm{\sum_{k=1}^\infty \abs{f_{N_{k+1}} - f_{N_k}}}_p \leq S < \infty \]
	Since the integral is finite, we see that $\sum_{k=1}^\infty \abs{f_{N_{k+1}} - f_{N_k}}$ is finite $\mu$-a.e..
	Let $A$ be the set where this sum is finite, then $\mu(A^c) = 0$.
	For any $x \in A$, $(f_{N_k}(x))$ is Cauchy as sum finite, and since $\mathbb{R}$ complete it converges.
	Define,
	\[ f(x) = \begin{cases}
		\lim_{k \to \infty} f_{N_k}(x) & x \in A \\
		0 & x \in A^c
	\end{cases} \]
	so $f_{N_k} \to f$ as $k \to \infty$ $\mu$-a.e. and $f$ measurable as the limit of measurable fcns.

	Now, by Fatou's lemma,
	\begin{align*}
		\norm{f_n - f}_p^p &= \mu(\abs{f_n - f}^p) \\
		&= \mu(\lim_k \abs{f_n - f_{N_k}}^p) \\
		&= \mu(\liminf_k \abs{f_n - f_{N_k}}^p) \\
		&\leq \liminf_k \mu(\abs{f_n - f_{N_k}}^p) \\
		&\leq \epsilon^p \quad \forall n \geq N\footnote{This is the $N$ defined in the statement of the theorem.}.
	\end{align*}

	Since the $f_n$ are Cauchy,
	\[ \norm{f}_p \leq \underbrace{\norm{f - f_N}_p}_{\leq \varepsilon} + \underbrace{\norm{f_N}_p}_{< \infty} < \infty \]
	so $f \in L^p$ and so $f_n \to f$ in $L^p$.
\end{proof}

\begin{remark}
	If $V$ is any of the spaces
	\[ C([0, 1]);\quad\qty{f \text{ simple}};\quad\qty{f \text{ a finite linear combination of indicators of intervals}} \]
	then $V$ is dense in $L^p((0, 1), \mathcal{B}, \lambda)$.
	So the completion $\overline{(V,\norm{\wildcard}_1)}$ is exactly $L^1(\lambda)$ (Proof on Sheet 3, first prove for finite linear combinations, use monotone class theorem, approximate continuous fcns by indicators of intervals so done).
\end{remark}

\subsection{Hilbert spaces}

\begin{definition}[Inner Product]
	A symmetric bilinear form $\inner{\wildcard, \wildcard} \colon V \times V \to \mathbb R$ on a real vector space $V$ is called an \vocab{inner product} if $\inner{v,v} \geq 0$ and $\inner{v,v} = 0 \iff v = 0$. \\
	In this case, we can define a norm\footnote{Cauchy-Schwarz gives triangle inequality} $\norm{v} = \sqrt{\inner{v,v}}$.
\end{definition}

\begin{definition}[Hilbert Space]
	If $(V,\inner{\wildcard,\wildcard})$ is complete, we say that it is a \vocab{Hilbert space}.
\end{definition}

\begin{corollary}
	The space $\mathcal L^2$ is a Hilbert space for the inner product $\inner{f,g} = \int_E fg \dd{\mu}$.
\end{corollary}

\begin{example}
	An analog of the Pythagorean theorem holds.
	Let $f, g \in L^2$, then $\norm{f + g}_2^2 = \norm{f}_2^2 + 2\inner{f,g} + \norm{g}_2^2$.
\end{example}

\begin{example}
	The parallelogram identity holds: $\norm{f+g}_2^2 + \norm{f-g}_2^2 = 2 \qty(\norm{f}_2^2 + \norm{g}_2^2)$
\end{example}

\begin{definition}[Orthogonal]
	We say $f$ is \vocab{orthogonal} to $g$ if $\inner{f,g} = 0$.
\end{definition}

\begin{remark}
	$f$ and $g$ are orthogonal iff $\norm{f + g}_2^2 = \norm{f}_2^2 + \norm{g}_2^2$.

	For centred (mean zero) r.v.s $X, Y$, we have $\inner{X,Y} = \expect{XY} = \mathbb{E}[(X - \mathbb{E}[X])(Y - \mathbb{E}[Y])] = \Cov{X,Y}$ which vanishes when $X$ and $Y$ are orthogonal.
\end{remark}

\begin{definition}[Orthogonal Complement]
	Let $V \subseteq L^2(\mu)$.
	We define its \vocab{orthogonal complement} to be
	\[ V^\perp = \qty{f \in L^2(\mu) : \inner{f,g} = 0 \quad \forall g \in V} \]
\end{definition}

\begin{definition}[Closed Set]
	We say that a subset $V$ of $L^2$ is \vocab{closed} if for any sequence $f_n \in V$ that converges in $L^2$, its limit $f$ coincides a.e. with some $v \in V$.
\end{definition}

\begin{theorem}[Orthogonal Projection]
	Let $V$ be a \underline{closed linear subspace} of $L^2(\mu)$.
	Then $\forall f \in L^2$, $\exists$ an orthogonal decomposition $f = v + u$ where $v \in V$ and $u \in V^\perp$. \\
	Moreover, $\norm{f - v}_2 \leq \norm{f - g}_2 \ \forall g \in V$ with equality iff $v = g$ a.e..
\end{theorem}

\begin{definition}[Projection]
	We call $v$ the \vocab{projection} of $f$ onto $V$.
\end{definition}

\begin{proof}
	In this proof, we use $p = 2$ for all norms.
	We define $d(f,V) = \inf_{g \in V} \norm{g - f}$, and let $g_n \in V$ be a sequence of functions s.t. $\norm{g_n - f} \to d(f,V)$. \\
	By the parallelogram law,
	\begin{align*}
		2\norm{f - g_n}^2 + 2\norm{f - g_m}^2 &= \norm{2f - (g_n + g_m)}^2 + \norm{g_n - g_m}^2 \\
		&= 4 \norm{f - \underbrace{\frac{g_n + g_m}{2}}_{\in V}}^2 + \norm{g_n - g_m}^2 \\
		&\geq 4 d(f,V)^2 + \norm{g_n - g_m}^2
	\end{align*}
	Thus $\lim_{n,m \to \infty} \norm{g_n - g_m}^2 \to 0$, i.e. $g_n$ is Cauchy in $L^2$, so by completeness, it converges to some $v \in L^2$.
	Since $V$ is closed, $v \in V$.
	In particular, $d(f,V) = \inf_{g \in V} \norm{g - f} = \norm{v - f}$.

	Note that $d(f,V)^2 \leq F(t) = \norm{f - (v+th)}^2 = d(f, V)^2 - 2t \inner{f-v, h} + t^2 \norm{h}^2$ where $t \in \mathbb R$ and $h \in V$.
	Letting $t \downarrow 0$ and $t \uparrow 0$, we obtain $\inner{f - v, h} = 0$ for all $h$.
	Defining $f - v = u$, we have $f = u + v$ and $u \in V^\perp$ since $h$ was arbitrary.

	For any $g \in V$, $f - g = \underbrace{f-v}_{\in V^\perp} + \underbrace{v - g}_{\in V}$.
	So $\norm{f - g}^2 = \norm{f - v}^2 + \norm{v - g}^2$ hence $\norm{f - g} \geq \norm{f - v}$ with equality iff $\norm{v - g} = 0$, i.e. $v = g$ a.e..

	% For uniqueness, suppose $f = w + z$ with $w \in V$ and $z \in V^\perp$.
	% Then $v - w + u - z = f - f = 0$, so taking norms, $0 = \norm{v - w + u - z}^2 = \norm{v - w}^2 + \norm{u - z}^2$ so $v = w$ and $u = z$ (a.e.) by orthogonality.
\end{proof}

\subsection{Conditional Expectation}

% \begin{definition}[Sub-$\sigma$ algebra]

% \end{definition}

If $\mathcal{G}$ a sub-$\sigma$ algebra of $\mathcal{F}$ (i.e. $\mathcal{G} \subseteq \mathcal{F}$), then $L^2(\Omega, \mathcal{G}, \mathbb{P})$ is a closed subspace of $L^2(\Omega, \mathcal{F}, \mathbb{P})$.

\begin{definition}[Conditional Expectation]
	For $X \in L^2(\Omega, \mathcal{F}, \mathbb{P})$ s.t. $X$ measurable wrt $\mathcal{G}$, $\norm{X - Y}_2 \geq \norm{X - \mathbb{E}[X \mid \mathcal{G}]}_2 \quad \forall Y$ that are $\mathcal{G}$ measurable.\\
	The\footnote{This is actually only a variant of the conditional expectation.} \vocab{conditional expectation of $X$ given $\mathcal{G}$}, $\mathbb{E}[X \mid \mathcal{G}]$ is defined as the orthogonal projection of $X$ on $L^2(\Omega, \mathcal{G}, \mathbb{P})$.
\end{definition}

\begin{question}
	How to define $\mathcal{E}[X \mid \mathcal{G}]$ if $X \in L^1(\Omega, \mathcal{F}, \mathbb{P})$, see Advanced Probability.
\end{question}

\begin{example}
	Let $(G_i)_{i \in I}$ be a countable family of disjoint events whose union is $\Omega$ and set $\mathcal{G} = \sigma(G_i : i \in I)$.
	Let $X$ be integrable.
	Then the conditional expectation of $X$ given $\mathcal{G}$ is given by:
	\begin{align*}
		\mathbb{E}[X \mid G_i] = \frac{\mathbb{E}[X 1_{G_i}]}{\mathbb{P}(G_i)} \quad \forall i \in I.
	\end{align*}
	Let $Y = \sum_i \mathbb{E}[X \mid G_i] 1_{G_i}$ (i.e. if $\omega \in G_i$, $Y(\omega) = \mathbb{E}[X \mid G_i]$).
	Check that $Y$ is $\mathcal{G}$-measurable; $Y \in L^2(\Omega, \mathcal{G}, \mathbb{P})$; and that $Y$ is ``the'' orthogonal projection of $X$ onto $L^2(\Omega, \mathcal{G}, \mathbb{P})$ if $X \in L^2(\Omega, \mathcal{F}, \mathbb{P})$.
\end{example}

\subsection{$L^p$ Convergence and Uniform Integrability}

For $(\Omega, \mathcal{F}, \mathbb{P})$, what are the implications between convergence: a.s., in $L^p$ for $1 \leq p < \infty$, in $\mathbb{P}$ and in distribution.

Let $f_n = n 1_{(0, 1/n)}$ on $\qty((0, 1), \mathcal{B}, \lambda)$.
$f_n \to 0$ a.s. but $\mathbb{E}\abs{f_n} = \mathbb{E}[f_n] = 1 \ \forall n$ so a.s. $\centernot\implies$ $L^p$ convergence.

$\mathbb{P}(|X_n - X| > \epsilon) \leq \frac{\mathbb{E}|X_n - X|^p}{\epsilon^p}$ by Markov's Inequality, so convergence in $L^p$ for $1 \leq p < \infty \implies$ convergence in $\mathbb{P}$.

\begin{theorem}[Dominated Convergence Theorem]
	Let $X_n$ be r.v.s on $(\Omega, \mathcal F, \mathbb P)$ s.t. $\abs{X_n} \leq Y$ for integrable r.v. $Y$ and they converge in $\mathbb{P}$ to $X$.
	Then $X_n \to X$ in $L^1(\mathbb P)$, i.e. $\mathbb{E}|X_n - X| \to 0$.
\end{theorem}

\begin{question}
	What is the ``minimum condition'' on $(X_n)$ under which $X_n \to X$ in $\mathbb{P}$ implies $X_n \to X$ in $L^1(\mathbb{P})$.
\end{question}

\begin{answer}
	Uniformly Integrable
\end{answer}

% \begin{proof}
% 	We know that $X_{n_k} \to X$ almost surely along a subsequence $n_k$.
% 	So $\abs{X} = \lim_k \abs{X_{n_k}} \leq C < \infty$ almost surely.
% 	Then
% 	\begin{align*}
% 		\expect{\abs{X_n - X}} &= \expect{\abs{X_n - X} \qty(1_{\qty{\abs{X_n - X} > \frac \varepsilon 2}} + 1_{\qty{\abs{X_n - x} \leq \frac \varepsilon 2}})} \\
% 		&\leq 2 C \prob{\abs{X_n - X} \geq \frac{\varepsilon}{2}} + \frac \varepsilon 2 \\
% 		&< \varepsilon
% 	\end{align*}
% 	for sufficiently large $n$.
% \end{proof}

For $X \in L^1(\mathbb P)$, then as $\delta \to 0$,
\[ I_X(\delta) = \sup \qty{ \expect{\abs{X} 1_A} : \prob{A} \leq \delta, A \in \mathcal{F}} \to 0 \]
If not, $\exists \varepsilon > 0$ and $A_n \in \mathcal F$ s.t. $\prob{A_n} \leq 2^{-n}$ but $\expect{\abs{X} 1_{A_n}} \geq \varepsilon$.
Since $\sum_n \prob{A_n} < \infty$, by the first Borel--Cantelli lemma, we have $\prob{\bigcap_n \bigcup_{m \geq n} A_m} = 0$.
But $\expect{\abs{X} 1_{A_n}} \leq \expect{\abs{X} 1_{\bigcup_{m \geq n} A_m}}$.
Note that $1_{\bigcup_{m \geq n} A_m} \to 1_{\bigcap_n \bigcup_{m \geq n} A_n} = 0$ a.s., so $\expect{\abs{X} 1_{\bigcup_{m \geq n} A_m}} \to \expect{\abs{X} 1_{\bigcap_n \bigcup_{m \geq n}}} = 0$ by DCT \Lightning.

\begin{definition}[Uniformly Integrable]
	For a collection $\mathcal X \subseteq L^1(\mathbb P)$ of r.v.s, we say $\mathcal X$ is \vocab{uniformly integrable (UI)} if it is bounded in $L^1(\mathbb P)$\footnote{I.e. $\sup_{x \in \mathcal{X}} \norm{X}_1 = \sup_{x \in \mathcal{X}} \mathbb{E}[|X|] = I_{\mathcal{X}}(1) < \infty$.}, and
	\[ I_{\mathcal X}(\delta) = \sup \qty{ \expect{\abs{X}1_A} : \prob{A} \leq \delta, X \in \mathcal X} \to 0 \text{ as } \delta \to 0.\]
\end{definition}

\begin{remark}
	\begin{enumerate}
		\item Any single integrable r.v. is UI.
		Also, true for any finite collection of integrable r.v.s.
		Also, if $\mathcal{X} = \qty{X : X \text{ a r.v. s.t. } |X| \leq Y \text{ for some } Y \in L^1}$ as $\sup_{X \in \mathcal{X}} \mathbb{E}[|X| 1_A] \leq \mathbb{E}[Y 1_A]$ implies $I_{\mathcal{X}}(\delta) \leq I_Y(\lambda) \to 0$ as $\delta \to 0$.
		\item If $\mathcal X$ is bounded in $L^p(\mathbb P)$ for $p > 1$, then by H\"older's inequality,
		\[ \expect{\abs{X}1_A} \leq \underbrace{\norm{X}_p}_{\text{bounded}} \cdot \underbrace{\prob{A}^{\frac 1 q}}_{\leq \delta^{\frac 1 q} \to 0} \]
	\end{enumerate}
\end{remark}

% \begin{remark}
% 	Note that $X_n = n1_{\qty[0,\frac{1}{n}]}$ for the Lebesgue measure $\mu$ on $[0,1]$ is bounded in $L^1(\mathbb P)$ but not uniformly integrable.
% \end{remark}

\begin{lemma}
	$\mathcal X \subseteq L^1(\mathbb P)$ is UI $\iff \sup_{X \in \mathcal X} \expect{\abs{X} 1_{\qty{\abs{X} > K}}} \to 0$ as $K \to \infty$.
\end{lemma}

\begin{proof}
	$(\implies)$: Applying Markov's inequality, as $K \to \infty$,
	\[ \prob{\abs{X} > K} \leq \frac{\expect{\abs{X}}}{K} = \frac{\expect{\abs{X}1_{\Omega}}}{K} \leq \frac{I_{\mathcal X}(1)}{K} \to 0 \]
	Using the uniform integrability property using $A = \qty{\abs{X} > K}$, we obtain the required limit.

	$(\Longleftarrow)$:
	\[ \expect{\abs{X}} = \expect{\abs{X}\qty(1_{\qty{\abs{X} \leq K}} + 1_{\qty{\abs{X} > K}})} \leq K + \frac{\varepsilon}{2} \]
	for sufficiently large $K$.
	So $\mathcal X$ is bounded in $L^1(\mathbb P)$ as required.
	Then for $A$ s.t. $\prob{A} \leq \delta$,
	\[ \expect{\abs{X}1_A\qty(1_{\qty{\abs{X} \leq K}} + 1_{\qty{\abs{X} > K}})} \leq K\prob{A} + \expect{\abs{X}1_{\qty{\abs{X} > K}}} \leq K\delta + \frac{\varepsilon}{2} < \varepsilon \]
	for sufficiently small $\delta$.
\end{proof}

\begin{theorem}
	Let $X_n, X$ be r.v.s on $(\Omega, \mathcal F, \mathbb P)$.
	Then the following are equivalent.
	\begin{enumerate}
		\item $X_n, X \in L^1(\mathbb P)$ and $X_n \to X$ in $L^1(\mathbb P)$.
		\item $\qty{X_n : n \in \mathbb N}$ is uniformly integrable, and $X_n \to X$ in $\mathbb{P}$.
	\end{enumerate}
\end{theorem}

\begin{proof}
	(1) $\implies$ (2i):
	Using Markov's inequality,
	\[ \prob{\abs{X_n - X} > \varepsilon} \leq \frac{\expect{\abs{X_n - X}}}{\varepsilon} \to 0 \]
	so $X_n \to X$ in $\mathbb{P}$.

	Choose $N$ s.t. $\mathbb{E}|X_n - X| < \frac{\epsilon}{2} \ \forall n \geq N$.
	Choose $\delta$ s.t. $\mathbb{E}[|X| 1_A] \leq \frac{\epsilon}{2}$ and $\mathbb{E}[|X_n|1_A] \leq \epsilon \quad \forall n = 1, \dots, N-1$ when $\mathbb{P}(A) < \delta$.
	\begin{align*}
		\expect{\abs{X_n} 1_A} \leq \expect{\abs{X_n - X} 1_A} + \expect{\abs{X} 1_A} \leq \frac{\varepsilon}{2} + \frac{\varepsilon}{2}
	\end{align*}
	$\{X_1, \dots, X_{N-1}, X\}$ is finite so UI.
	So $\mathcal{X}$ is UI.

	(2) $\implies$ (1):
	$X_n \to X$ in $\mathbb{P}$, so take a subsequence $n_k$ s.t. $X_{n_k} \to X$ a.s..
	Then,
	\begin{align*}
		\expect{\abs{X}} = \expect{\liminf_k \abs{X_{n_k}}} \leq\footnote{Fatou's lemma} \liminf_k \expect{\abs{X_{n_k}}} \leq I_{\mathcal X}(1) <\footnote{As $\mathcal{X}$ is UI, hence $L^1$ bounded.} \infty,
	\end{align*}
	so $X \in L^1(\mathbb P)$.

	Next, we define truncated r.v.s $X_n^K = \max(-K, \min(K, X_n))$ and $X^K = \max(-K, \min(K, X))$.
	Then $X_n^K \to X^K$ in $\mathbb{P}$ (as $\mathbb{P}(|X_n^K - X^K| > \epsilon) \leq \mathbb{P}(|X_n - X| < \epsilon)$)\footnote{Aside: If $X_n \to X$ in $\mathbb{P}$ and $f$ cts, then $f(X_n) \to f(X)$ in $\mathbb{P}$.}.
	And $|X_n^K| \leq K \quad \forall n$ so by BCT, $X_n^K \to X^K$ in $L^1$.
	Now,
	\begin{align*}
		\expect{\abs{X_n - X}} &\leq \expect{\abs{X_n - X_n^K}} + \expect{\abs{X_n^K - X^K}} + \expect{\abs{X^K - X}} \\
		&= \expect{\abs{X_n} 1_{\qty{\abs{X_n} > k}}} + \expect{\abs{X_n^K - X^K}} + \expect{\abs{X}1_{\qty{\abs{X} > K}}} \\
		&< \varepsilon
	\end{align*}
	by choosing sufficiently large $K$ (by UI) and $n$.
\end{proof}
    \section{Fourier transforms}
\subsection{Fourier transforms}
In this section, we will write $L^p = L^p(\mathbb R^d)$ for the space of \underline{complex valued} Borel measurable fcns on $\mathbb{R}^d$, i.e. $f \colon \mathbb R^d \to \mathbb C$ s.t. $\norm{f}_p = \qty(\int_{\mathbb R^d} \abs{f(x)}^p \dd{\mu(x)})^{\frac 1p} < \infty$ for $1 \leq p < \infty$.

\begin{remark}
	For $g$ measurable s.t. $\int |g| < \infty$, define $\int g(x) \dd{\mu(x)} = \int \Re(g(x)) \dd{\mu(x)} + i \int \Im(g(x)) \dd{\mu(x)}$.

	% Note that for some $u + iv = \alpha \in \mathbb C$ with $\abs{\alpha} = 1$,
	% \[ \abs{\int_{\mathbb R^d} f(x) \dd{x}} = \int_{\mathbb R^d} \alpha f(x) \dd{x} = \int_{\mathbb R^d} u(x) \dd{x} + i \int_{\mathbb R^d} v(x) \dd{x} \]
	% But since the left hand side is real-valued, the $i \int_{\mathbb R^d} v(x) \dd{x}$ term vanishes.
	% So
	% \[ \abs{\int_{\mathbb R^d} f(x) \dd{x}} = \int_{\mathbb R^d} u(x) \dd{x} \leq \int_{\mathbb R^d} \abs{f(x)} \dd{x} \]
\end{remark}

For $f, g \in L^2$, $\inner{f, g} = \int f(x) \overline{g(x)} \dd{\mu(x)}$ is an inner product on $L^2(\mu)$.

For any $y \in \mathbb{R}^d$,
\begin{align*}
	\int f(x - y) \dd{x} &= \int f(y - x) \dd{x} = \int f(x) \dd{x} \\
	&= \int f(-x) \dd{x}.
\end{align*}
This is by the translation invariance and $x \mapsto -x$ symmetry of $\lambda$, proved in Sheet 3.
Also, for $a \in \mathbb{R}$ with $a \neq 0$, $\int f(ax) \dd{x} = \frac{1}{a^d} \int f(x) \dd{x}$.

\begin{definition}[Fourier Transform]
	Let $f \in L^1(\mathbb R^d)$.
	We define the \vocab{Fourier transform} $\hat f$ by
	\[ \hat f(u) = \int_{\mathbb R^d} f(x) e^{i\inner{u,x}} \dd{x} \]
	where $u \in \mathbb{R}^d$ and $\inner{u,x} = \sum_{i=1}^d u_i x_i$.
\end{definition}

\begin{remark}
	Note that $\abs{\hat f(u)} \leq \norm{f}_1 \quad \forall u \in \mathbb{R}^d$, i.e. $\hat{f} \in L^\infty$.

	Also, if $u_n \to u$, then $e^{i\inner{u_n,x}} \to e^{i\inner{u,x}}$ so $f(x) e^{i\inner{u_n,x}} \to f(x) e^{i\inner{u,x}}$; $|f(x) e^{i\inner{u_n,x}}| \leq |f(x)|$ and $f \in L^1$.
	By the DCT $\hat f(u_n) \to \hat f(u)$.
	Moreover, $\lim_{\norm{u} \to \infty} \hat{f}(u) = 0$ (Riemann-Lebesgue Lemma, Sheet 3).
	Thus $\hat f \in C_0(\mathbb{R}^d) = \qty{f \text{ bounded cts and vanishing at } \pm \infty}$.

	The map is $1 - 1$ (but not onto), its injective but not surjective.
\end{remark}

\begin{definition}[Fourier Transform]
	Let $\mu$ be a finite Borel measure on $\mathbb R^d$.
	We define the \vocab{Fourier transform} of the measure for $u \in \mathbb{R}^d$ by
	\[ \hat\mu(u) = \int_{\mathbb R^d} e^{i\inner{u,x}} \dd{\mu(x)} \]
\end{definition}

Note that $\abs{\hat \mu(u)} \leq \mu(\mathbb R^d)$, and $\hat \mu$ a bounded cts fcn on $\mathbb{R}^d$.
If $\mu$ has a density $f$ (wrt $\lambda$), $\hat\mu = \int_{\mathbb R^d} e^{i\inner{u,x}} f(x) \dd{x} = \hat f$.

\begin{definition}[Characteristic Function]
	Let $X$ be an $\mathbb R^d$-valued r.v..
	The \vocab{characteristic function (c.f.)} $\varphi_X$ of $X$ is the Fourier transform of its law $\mu_X = \mathbb{P} \circ X\inv$.
	So,
	\begin{align*}
		\varphi_X(u) = \hat \mu_X(u) = \int e^{i \inner{u, x}} \underbrace{\dd{\mu_X(x)}}_{\dd{\mathbb{P}} \circ X\inv(x)}\footnote{Note that $\nu \circ f\inv(g) = \nu(f \circ g)$.} = \int e^{i\inner{u, x}} \dd{\mathbb{P}} = \expect{e^{i\inner{u,X}}}.
	\end{align*}
\end{definition}

In particular if $X$ has pdf $f$, then $\phi_X(u) = \hat{f}(u)$.

\begin{definition}[Fourier Inversion Formula]
	Let $f \in L^1(\mathbb R^d)$ s.t. $\hat f \in L^1(\mathbb R^d)$.
	Then we say that the \vocab{Fourier inversion formula} holds for $f$ if
	\[ f(x) = \frac{1}{(2\pi)^d} \int_{\mathbb R^d} \hat f(u) e^{-i\inner{u,x}} \dd{u} \]
	a.e. in $\mathbb R^d$.
\end{definition}

\begin{remark}
	The RHS is cts by DCT, so for $f$ cts the equality is everywhere.
\end{remark}

\begin{remark}
	The map from $L^1 \to C_0$ by $f \mapsto \hat{f}$ is $1-1$ (for $f, g \in L^1$ with $\hat{f} = \hat{g}$, then $f - g \in L^1$ and $\widehat{f - g} = \hat{f} - \hat{g} = 0$. So by Fourier Inversion $f - g = 0$ a.e.)
\end{remark}

% \begin{definition}[Plancherel identity]
% 	Let $f \in L^1(\mathbb R^d) \cap L^2(\mathbb R^d)$.
% 	Then the \vocab{Plancherel identity} holds for $f$ if
% 	\[ \norm{\hat f}_2 = (2\pi)^{\frac d2} \norm{f}_2 \]
% \end{definition}

% We will show that the Fourier inversion formula holds whenever $\hat f \in L^1(\mathbb R^d)$, and the Plancherel identity holds for all $f \in L^1(\mathbb R^d) \cap L^2(\mathbb R^d)$.

% \begin{remark}
% 	Given the Plancherel identity, the Fourier transform is a linear isometry of $L^2(\mathbb R^d)$, by approximating any function in $L^2(\mathbb R^d)$ by integrable functions.
% \end{remark}

\subsection{Convolutions}

A key concept in Fourier analysis is convolutions.

\begin{definition}[Convolution]
	Let $f \in L^p(\mathbb R^d)$, $1 \leq p < \infty$ and $\nu$ be a probability measure on $\mathbb R^d$.
	We define their \vocab{convolution} $f \ast \nu$ by
	\[ (f \ast \nu)(x) = \begin{cases}
		\int_{\mathbb R^d} f(x-y) \dd{\nu(y)} & \text{if integral exists;} \\
		0 & \text{else.}
	\end{cases} \]
\end{definition}

\begin{remark}
	If $1 \leq p < \infty$, by Jensen's inequality,
	\begin{align*}
		\int_{\mathbb{R}^d} \abs{f \ast \nu(x)}^p \dd{x} &\leq \int_{\mathbb R^d} \qty( \int_{\mathbb R^d} \abs{f(x-y)} \dd{\nu(y)} )^p \dd{x} \\
		&\leq \int_{\mathbb R^d} \int_{\mathbb R^d} \abs{f(x-y)}^p \dd{\nu(y)} \dd{x} \quad \text{as $p \geq 1$}\\
		&= \int_{\mathbb R^d} \int_{\mathbb R^d} \abs{f(x-y)}^p \dd{x} \dd{\nu(y)} \\
		&= \int_{\mathbb R^d} \int_{\mathbb R^d} \abs{f(x)}^p \dd{x} \dd{\nu(y)} \text{ as $\lambda$ translation invariant} \\
		&= \norm{f}_p^p < \infty \text{ as $\nu$ a prob measure.}
	\end{align*}
	Hence $f \ast v$ is defined a.e., and $\norm{f \ast v}_p \leq \norm{f}_p < \infty$.
	When $\nu$ has pdf $g \in L^1$\footnote{wrt Lebesgue measure}, $f \ast v(x) = \int f(x-y) g(y) \dd{y} = f \ast g(x)$.

	% For two probability measures $\mu, \nu$ on $\mathbb{R}^d$, the convolution $\mu \ast \nu$ is a new prob measure defined as
	% \begin{align*}
	% 	\mu \ast \nu(A) &= \int \int 1_A(x+y) \dd{\mu(x)} \dd{\nu(x)} = \mu \otimes \nu (x + y \in A) \\
	% 	&= \mathbb{P}(X + Y \in A)
	% \end{align*} where $X, Y$ independent and $X \sim \mu$ and $Y \sim \nu$.

	% If $\mu$ has pdf $f \in L^1$ then
	% \begin{align*}
	% 	\mu \ast \nu(A) &= \int \left(  \right)
	% \end{align*}


	% So $f \in L^p(\mathbb R^d)$, we have $(y \mapsto f(x-y)) \in L^p(\nu)$ almost everywhere, and again by Jensen's inequality,
	% \[ \norm{f \ast \nu}_p^p = \int_{\mathbb R^d} \abs{ \int_{\mathbb R^d} f(x-y)\dd{\nu(y)} }^p \dd{x} \leq \int_{\mathbb R^d} \qty( \int_{\mathbb R^d} \abs{f(x-y)} \dd{\nu(y)} )^p \dd{x} \leq \norm{f}_p^p \]
	% Hence $f \mapsto f \ast \nu$ is a contraction on $L^p(\mathbb R^d)$.
\end{remark}

In the case where $\nu$ has a density $g$ with respect to the Lebesgue measure, we write $f \ast g = f \ast \nu$.

\begin{definition}[Convolution]
	For probability measures $\mu, \nu$ on $\mathbb R^d$, their \vocab{convolution} $\mu \ast \nu$ is a probability measure on $\mathbb R^d$ given by the law of $X + Y$ where $X, Y$ are independent r.v.s with laws $\mu$ and $\nu$, so
	\begin{align*}
		(\mu \ast \nu)(A) &= \prob{X+Y \in A} \\
		&= \int_{\mathbb R^d \times \mathbb R^d} 1_A(x+y) \dd{(\mu \otimes \nu)(x, y)} \\
		&= \int_{\mathbb R^d} \int_{\mathbb R^d} 1_A(x+y) \dd{\nu(y)} \dd{\mu(x)}
	\end{align*}
\end{definition}

If $\mu$ has density $f \in L^1$ wrt the Lebesgue measure, $\mu \ast \nu$ has density $f \ast \nu$ wrt the Lebesgue measure.
Indeed,
\begin{align*}
	(\mu \ast \nu)(A) &= \int_{\mathbb R^d} \int_{\mathbb R^d} 1_A(x+y) f(x) \dd{x} \dd{\nu(y)} \\
	&= \int_{\mathbb R^d} \int_{\mathbb R^d} 1_A(x) f(x-y) \dd{x} \dd{\nu(y)} \\
	&= \int_{\mathbb R^d} 1_A(x) \int_{\mathbb R^d}f(x-y) \dd{\nu(y)} \dd{x} \\
	&= \int_{\mathbb R^d} 1_A(x) (f \ast \nu)(x) \dd{x}
\end{align*}

\begin{proposition}
	$\widehat{f \ast \nu}(u) = \hat f(u) \hat \nu(u)$ for all $f \in L^1$ and $\nu$ a prob measure.
\end{proposition}

\begin{proposition}
	$\widehat{\mu \ast \nu}(u) = \expect{e^{i\inner{u,X+Y}}} = \expect{e^{i\inner{u,X}}e^{i\inner{u,Y}}} = \hat \mu(u) \hat \nu(u)$ for all $\mu, \nu$ prob measures.
\end{proposition}

\subsection{Fourier transforms of Gaussians}

\begin{definition}[Normal Distribution]
	The \vocab{normal distribution} $N(0,t)$ is given by the probability density function
	\[ g_t(x) = \frac{1}{\sqrt{2\pi t}} e^{-\frac{x^2}{2t}} \]
\end{definition}

If $\varphi_Z$ is the characteristic function of $Z \sim N(0, 1)$, i.e. $\phi_Z(u) = \int \frac{1}{\sqrt{2 \pi}} e^{-z^2 / 2} e^{iuz} \dd{z}$ then by a previous thm, $\phi_z$ is differentiable and can use DUTIS.
\begin{align*}
	\dv{u} \varphi_Z(u) &= \frac{1}{\sqrt{2 \pi}} \int \dv{u} \left(e^{-z^2 / 2} e^{iuz}\right) \dd{z} \\
	&= \frac{1}{\sqrt{2 \pi}} \int iz e^{-z^2 / 2} e^{iuz} \dd{z} \\
	&= \frac{i}{\sqrt{2\pi}} \int \underbrace{e^{iuz}}_{v} \underbrace{z e^{-\frac{z^2}{2}}}_{w'} \dd{z} \\
	&= \frac{i}{\sqrt{2\pi}} \int iu e^{iuz} e^{-\frac{z^2}{2}} \dd{z} \\
	&= -u \varphi_Z(u)
\end{align*}

Hence,
\[ \dv{u}\qty(e^{\frac{u^2}{2}} \varphi_Z(u)) = ue^{\frac{u^2}{2}} \varphi_Z(u) - e^{\frac{u^2}{2}} u \varphi_Z(u) = 0 \]
In particular, $\varphi_Z(u) = \varphi_Z(0) e^{-\frac{u^2}{2}} = e^{-\frac{u^2}{2}}$.
In other words, $\hat g_1(u) = \sqrt{2\pi} g_1(u)$.

In $\mathbb R^d$, consider a Gaussian r.v. $Z = (Z_1, \dots, Z_d)$ with iid entries $N(0,1)$.
Then, the joint pdf (wrt $\lambda^d$) of $\sqrt{t}Z$ is
\[ g_t(x) = \prod_{j=1}^d \frac{1}{\sqrt{2\pi t}} e^{-\frac{x_j^2}{2t}} = (2\pi t)^{-\frac{d}{2}} e^{-\frac{\norm{x}^2}{2t}} \]
The Fourier transform of $g_t$ is
\[ \hat g_t(u) = \expect{e^{i\inner{u,\sqrt{t}Z}}} = \expect{\prod_{j=1}^d e^{iu_j \sqrt{t} z_j}} = \prod_{j=1}^d \underbrace{\expect{e^{iu_j \sqrt{t} z_j}}}_{\phi_{Z_i}(\sqrt{t}u_i)} = \prod_{j=1}^d e^{-u_j^2 \frac{t}{2}} = e^{-\frac{\norm{u}^2 t}{2}} \]
which implies that in general, $\hat g_t(u) = \frac{(2\pi)^{\frac{d}{2}}}{t^{\frac{d}{2}}} \frac{t^{\frac{d}{2}}}{(2\pi)^{\frac{d}{2}}} e^{-\frac{\norm{u}^2 t}{2}} = (2\pi)^{\frac{d}{2}} t^{-\frac{d}{2}} g_{\frac{1}{t}}(u)$.
Taking the Fourier transform with respect to $u$, $\hhat g_t = (2\pi)^{\frac{d}{2}} t^{-\frac{d}{2}} \hat{g}_{\frac{1}{t}}(u) = (2\pi)^d g_t$.
Since $g_t(-x) = g_t(x)$ and the Lebesgue measure is translation invariant, we have
\begin{align*}
	g_t(x) = \frac{1}{(2\pi)^d} \hhat g_t(x) = \frac{1}{(2\pi)^d} \int_{\mathbb R^d} e^{-i\inner{u,x}} \hat g_t(u) \dd{u}
\end{align*}
so the Fourier inversion theorem holds for $g_t$.

\begin{definition}[Gaussian Convolution]
	We say that a function on $\mathbb R^d$ is a \vocab{Gaussian convolution} if it is of the form
	\[ f \ast g_t(x) = \int_{\mathbb R^d} f(x-y) g_t(y) \dd{y} \]
	where $x \in \mathbb R^d, t > 0, f \in L^1(\mathbb R^d)$.
\end{definition}

\begin{enumerate}
	\item $f \ast g_t \in L^1$ as $f \in L^1$ (proved earlier) and $\norm{f \ast g_t}_1 \leq \norm{f}_1 < \infty$.
	\item $f \ast g_t$ is continuous on $\mathbb R^d$ by noting $f \ast g_t(x) = \int_{\mathbb R^d} f(y) g_t(x-y) \dd{y}$ by translation invariance, then using DCT noting $g$ bounded as cts.
	\item $f \ast g_t$ is bdd.
	\item  $\widehat{f \ast g_t}(u) = \hat{f}(u) \hat{g}_t(u) = \hat{f}(u) e^{-\frac{\norm{u}^2 t}{2}}$.
	\item $\widehat{f \ast g_t}$ is bdd cts as $f \ast g_t \in L_1$.
	\item $\norm{\widehat{f \ast g_t}} \leq c_t \norm{\hat{f}}_\infty \leq c_t \norm{\hat{f}}_1$.
	\item For $\mu$ a prob measure and any $t > 0$, $\mu \ast g_t$ is a Gaussian convolution. As, $g_t = g_{t / 2} \ast g_{t / 2}$ as $g_t$ is the density of a $N(0, t)$ r.v.. Then $\mu \ast g_t = \underbracket{(\mu \ast g_{t / 2})}_{L_1} \ast g_{t / 2}$.
\end{enumerate}

\begin{lemma}
	The Fourier inversion theorem holds for all Gaussian convolutions.
\end{lemma}

\begin{proof}
	Let $f \in L_1$ and $t > 0$.
	We can use the Fourier inversion theorem for $g_t(y)$ to see that
	\begin{align*}
		(2\pi)^d f \ast g_t(x) &= (2\pi)^d \int_{\mathbb R^d} f(x-y) g_t(y) \dd{y} \\
		&= \int_{\mathbb R^d} f(x-y) \int_{\mathbb R^d} e^{-i\inner{u,y}} \hat g_t(u) \dd{u} \dd{y} \\
		&= \int_{\mathbb R^d} e^{-i\inner{u,x}} \int_{\mathbb R^d} f(x-y) e^{i\inner{u,x-y}} \dd{y} \hat g_t(u) \dd{u} \\
		&= \int_{\mathbb R^d} e^{-i\inner{u,x}} \int_{\mathbb R^d} f(z) e^{i\inner{u,z}} \dd{z} \hat g_t(u) \dd{u} \\
		&= \int_{\mathbb R^d} e^{-i\inner{u,x}} \hat f(u) \hat g_t(u) \dd{u} \\
		&= \int_{\mathbb R^d} e^{-i\inner{u,x}} \widehat{f \ast g_t}(u) \dd{u}
	\end{align*}
\end{proof}

% \begin{remark}
% 	If $\mu$ is a finite measure, then $\mu \ast g_t = \mu \ast g_{\frac{t}{2}} \ast g_{\frac{t}{2}}$ with $\mu \ast g_{\frac{t}{2}} \in L^1$, so is also a Gaussian convolution.
% \end{remark}

\begin{lemma}[Gaussian convolutions are dense in $L^p$]
	Let $f \in L^p(\mathbb{R}^d)$ where $1 \leq p < \infty$.
	Then $\norm{f \ast g_t - f}_p \to 0$ as $t \to 0$.
\end{lemma}

\begin{proof}
	One can easily show that the space $C_c(\mathbb R^d)$ of continuous functions of compact support is dense in $L^p$.
	Hence, given $\varepsilon > 0$, $\exists h \in C_c(\mathbb R^d)$ s.t. $\norm{f - h}_p < \frac{\varepsilon}{3}$.
	Then by linearity of convolution,
	\[ \norm{f \ast g_t - h \ast g_t}_p = \norm{(f - h) \ast g_t}_p \leq \norm{f - h}_p < \frac{\varepsilon}{3} \]
	So by Minkowski's inequality,
	\begin{align*}
		\norm{f \ast g_t - f}_p \leq \underbracket{\norm{f \ast g_t - h \ast g_t}_p}_{\leq \epsilon / 3} + \underbracket{\norm{h - f}_p}_{\leq \epsilon / 3} + \norm{h \ast g_t - h}_p
		\leq \frac{2\varepsilon}{3} + \norm{h \ast g_t - h}_p
	\end{align*}
	so it suffices to prove the result for $f = h \in C_c(\mathbb R^d)$.
	We define a new map
	\[ e(y) = \int_{\mathbb R^d} \abs{h(x-y) - h(x)}^p \dd{x} \]
	Since $h$ is bdd (cts on compact support) and supported on $[-M, M]$ say, for some $M > 0$.
	As $y \to 0$, $|h(x-y) - h(x)|^p \to 0$ as $h$ cts.
	Also for $|y| < 1$, $|h(x-y) - h(x)|^p \leq 2^p \norm{h(x)}^p_\infty 1_{|x| \leq M + 1}$, with the RHS being integrable.
	Hence by DCT, $e(y) \to 0$ as $y \to 0$.

	Hence, by Jensen's inequality,
	\begin{align*}
		\norm{h \ast g_t - h}_p^p &= \int_{\mathbb R^d} \abs{ \int_{\mathbb R^d} h(x-y) g_t(y) \dd{y} - h(x)}^p \dd{x} \\
		&= \int_{\mathbb R^d} \abs{ \int_{\mathbb R^d} (h(x-y) - h(x)) g_t(y) \dd{y} }^p \dd{x} \\
		&\leq \int_{\mathbb R^d} \int_{\mathbb R^d} \abs{h(x-y) - h(x)}^p \dd{x} g_t(y)\footnote{$g_t$ the measure for our expectation in Jensen's so unaffected.} \dd{y} \\
		&= \int_{\mathbb R^d} e(y) g_t(y) \dd{y} \\
		&= \int_{\mathbb R^d} e(y) \frac{1}{t^{d/2}} g_1 \qty(\frac{y}{\sqrt{t}})\footnote{Note that $g_t(u) = \frac{1}{t^{d/2}} g_1 \qty(\frac{u}{\sqrt{t}})$} \dd{y} \\
		&= \int_{\mathbb R^d} \underbracket{e(\sqrt{t} z)}_{\to e(0) = 0 \text{ as } t \to 0} g_1(z) \dd{z} \\
		&\to 0 \text{ by DCT.}
	\end{align*}
\end{proof}
% TODO: Convert random instances of g into g_t
\begin{theorem}[Fourier Inversion]
	Let $f \in L^1(\mathbb R^d)$ be s.t. $\hat f \in L^1(\mathbb R^d)$.
	Then a.e. in $\mathbb{R}^d$,
	\[ f(x) = \frac{1}{(2\pi)^d} \int_{\mathbb R^d} e^{-i\inner{u,x}} \hat f(u) \dd{u} \]
\end{theorem}

\begin{remark}
	This proves that the Fourier transform is injective; $\hat f = \hat g$ implies $\widehat{f - g} = 0$ so by Fourier inversion, $f = g$ almost everywhere.
	The identity holds everywhere on $\mathbb R^d$ for the (unique) continuous representative $f$ in its equivalence class.
\end{remark}

\begin{proof}
	Consider $f \ast g_t$ and
	\begin{align*}
		f_t(x) = \frac{1}{(2\pi)^d} \int_{\mathbb R^d} e^{-i\inner{u,x}} \hat f(u) \underbracket{e^{\frac{-\abs{u}^2 t}{2}}}_{\hat{g_t}(u)} \dd{u}
	\end{align*}
	As FI holds for $f \ast g_t$, $f \ast g_t = f_t$.

	So, $\norm{f_t - f}_1 \overset{t \to 0}{\longrightarrow} 0$ by density of Gaussian convolutions and as $f \in L^1$.
	So $f_t \to f$ in $\mathbb{P}$ and thus $\exists$ a subsequence s.t. $f_{t_n} \to f$ a.e.

	Also, $e^{-i\inner{u,x}} \hat f(u) e^{\frac{-\abs{u}^2 t}{2}}$ is bounded by $\abs{\hat f(u)}$, which is integrable, and $\to e^{-i\inner{u,x}} \hat f(u)$ as $t \to 0$.
	So by DCT, $f_t(x) \to \frac{1}{(2\pi)^d} \int e^{-i\inner{u,x}} \hat f(u) \dd{u}$ as $t \to 0$ a.e.

	Hence $f = \frac{1}{(2\pi)^d} \int e^{-i\inner{u,x}} \hat f(u) \dd{u}$ a.e. as $f_t$ converges to it a.e. and $f_{t_n} \to f$.
\end{proof}

\begin{theorem}[Plancherel]
	Let $f \in L^1(\mathbb R^d) \cap L^2(\mathbb R^d)$.
	Then $\norm{f}_2 = (2\pi)^{-\frac{d}{2}} \norm{\hat f}_2$.
\end{theorem}

\begin{remark}
	By the Pythagorean identity, $\inner{f, g} = (2\pi)^{-d} \inner{\hat f, \hat g}$.
\end{remark}

\begin{proof}
	Initially, we assume $f, \hat f \in L^1$.
	In this case, $f, \hat f \in L^\infty$, and $(x,u) \mapsto f(x)\hat f(u)$ is integrable for the product Lebesgue measure $\dd{x} \otimes \dd{u}$ on $\mathbb R^d \times \mathbb R^d$, so Fubini's theorem for bounded functions applies.
	\begin{align*}
		(2\pi)^d \norm{f}_2^2 &= (2\pi)^d \int_{\mathbb R^d} f(x) \overline{f(x)} \dd{x} \\
		&= \int_{\mathbb R^d} \qty(\int_{\mathbb R^d} e^{-i\inner{u,x}} \hat f(u) \dd{u}) \overline{f(x)} \dd{x} < \infty \ (\text{and } f \in L^2) \\
		&= \int_{\mathbb R^d} \hat f(u) \overline{\int_{\mathbb R^d} e^{i\inner{u,x}} f(x) \dd{x}} \dd{u} \\
		&= \int_{\mathbb R^d} \hat f(u) \overline{\hat f(u)} \dd{u} \\
		&= \norm{\hat f}_2^2
	\end{align*}

	Now, let $f \in L^1 \cap L^2$.
	For $t > 0$, take $f_t = f \ast g_t \underset{t \to 0}{\longrightarrow} f$ in $L^2$ and so $\norm{f_t}_2 \underset{t \to 0}{\longrightarrow} \norm{f}_2$ continuity of the norm.
	Also, $\hat{f_t}(u) = \hat{f}(u) \hat{g_t}(u) = \hat f(u) e^{-\frac{\abs{u}^2 t}{2}}$.
	So $\abs{\hat{f_t}(u)} \uparrow \abs{\hat{f}(u)}$ as $t \to 0$.
	Thus $\norm{\hat{f_t}(u)}_2^2 = \int |\hat{f_t}(u)|^2 \dd{u} \underset{t \to 0}{\longrightarrow} \int |\hat{f}(u)|^2 \dd{u} = \norm{\hat{f}}_2^2$ by MCT.

	But, $f_t = f \ast g_t \in L^1$, and $\hat{f_t} \in L^1$.
	So by the first part of the proof, $(2\pi)^d \norm{f_t}_2^2 = \norm{\hat{f_t}}_2^2$.
	Letting $t \to 0$, we get $(2\pi)^d \norm{f}_2^2 = \norm{\hat{f}}_2^2$.
\end{proof}

\begin{remark}
	Since $L_1 \cap L_2$ is dense in $L^2$, we can extend the linear operator $F_0(f) = (2\pi)^{-\frac{d}{2}} \hat f$ to $L^2$ by continuity to a linear isometry $F \colon L^2 \to L^2$ known as the \emph{Fourier--Plancherel transform}.
	One can show that $F$ is surjective with inverse $F^{-1} \colon L^2 \to L^2$.
\end{remark}

\subsection{Characteristic fcns, Weak Convergence and the CLT}

\begin{definition}[Characteristic Function]
	For a r.v. $X$, its \vocab{characteristic function} is
	\begin{align*}
		\phi_X(t) = \mathbb{E}[e^{itX}] = \widehat{\mu_X} = \int e^{i \inner{t, x}} \dd{\mu_X(x)}
	\end{align*}
\end{definition}

\begin{example}
	Consider the Dirac measure $\delta_0$ on $\mathbb R$, so $\hat \delta_0(u) = \int_{\mathbb R} e^{iux} \dd{\delta_0(x)} = 1$.
	But the inverse Fourier transform would be $\frac{1}{2\pi} \int_{\mathbb R} e^{iux} \dd{u}$ which is not a Lebesgue integrable function.
\end{example}

To circumvent this, we test `$\mu$' on nice test fcns $f$.

\begin{remark}
	2 p.m.s $\mu, \nu$ on $\mathbb{R}^d$ coincide $\iff$ $\int f \dd{\mu} = \int f \dd{\nu} \quad \forall f : \mathbb{R}^d \to \mathbb{R}$ bdd cts\footnote{RHS implies LHS, as if true for $f = 1_{[a, b]}$ then true on a $\pi$-system so done. We can approximate such $f$ with bdd cts fcns and so done.}.
	In fact, enough to have condition holding $\forall f \in C_c^\infty$ (space of infinitely differentiable fcns with compact support).
\end{remark}

\begin{aside}{Aside}
	($\mu : f \mapsto \mu(f)$ mapping from $C_c^\infty \to \mathbb{R}$ is a linear, cts (on $L_f$ topology), hence $\mu$ is ``Schwartz distribution'').

\end{aside}

\begin{definition}[Weak Convergence]
	Let $(\mu_n), \mu$ be Borel prob measures on $\mathbb{R}^d$.
	Then \vocab{$\mu_n$ converges to $\mu$ weakly} if $\int f \dd{\mu_n} \to \int f \dd{\mu}$ as $n \to \infty$ for all $f : \mathbb{R}^d \to \mathbb{R}$ bdd cts.
\end{definition}

\begin{remark} \
	\begin{enumerate}
		\item For a sequence of r.v.s $X_n$ and $X$ another r.v., $X_n \to X$ weakly if $\mu_{X_n} \to \mu_X$ weakly.
		\item A sequence of prob measures, $\mu_n$, can have at most one weak limit by previous remark (the one about 2 p.m.s. coinciding).
		\item If $X_n \to X$ weakly, and $h : \mathbb{R}^d \to \mathbb{R}^k$ cts, then $h(X_n) \to h(X)$ weakly (as r.v. in $\mathbb{R}^k$). (Continuous Mapping Theorem) (from definition as $f \circ h$ bdd cts if $f$ bdd cts).
		\item Suffices to check condition in definition for all $f \in C_c^\infty$. (``tightness'' argument, i.e. $\exists K$ compact s.t. $\mu_n(K^c) < \epsilon \; \forall n$ if $\mu_n \to \mu$ weakly. Sheet 4)
		\item When $d = 1$, this is equivalent to $X_n \to X$ in distribution (i.e., $F_{X_n}(x) \to F_X(x)$ at all points where $x \mapsto F_X(x)$ is cts). Sheet 4 Q1, approximate indictators.
	\end{enumerate}
\end{remark}

\begin{theorem}
	Let $X$ be a r.v. in $\mathbb R^d$ with law $\mu_X$.
	Then the characteristic function $\varphi_X = \hat \mu_X$ uniquely determines $\mu_X$.
	In addition, if $\varphi_X \in L^1$, then $\mu_X$ has a bdd cts pdf $f_X(x) = \frac{1}{(2\pi)^d} \int_{\mathbb R^d} e^{-i\inner{u,x}} \varphi_X(u) \dd{u}$ a.e..
\end{theorem}

\begin{proof}
	Let $Z = (Z_1, \dots, Z_d)$ be a vector of independent and identically distributed r.v.s, independent of $X$, with $Z_j \sim N(0,1)$.
	Then $\sqrt{t} Z$ has pdf $g_t$ and $X + \sqrt{t} Z$ has pdf $f_t = \mu_X \ast g_t$.
	Then, $\hat{f_t}(u) = \hat{\mu_X}(u) \hat{g_t}(u) = \phi_X(u) e^{-\frac{\abs{u}^2 t}{2}}$.
	So by F.I. of Gaussian convolutions,
	\[ f_t(x) = \frac{1}{(2\pi)^d} \int_{\mathbb R^d} e^{-i\inner{u,x}} \varphi_X(u) e^{-\frac{\abs{u}^2 t}{2}} \dd{u} \]
	which is uniquely determined by $\varphi_X$.

	We show on an example sheet that two Borel prob measures $\mu, \nu$ on $\mathbb R^d$ coincide iff $\mu(g) = \nu(g) \quad \forall g \colon \mathbb R^d \to \mathbb R$ bdd, cts with compact support.
	Now,
	\[ \int_{\mathbb R^d} g(x) f_t(x) \dd{x} = \expect{\underbrace{g(X + \sqrt{t} Z)}_{\to X \text{ a.s.}}} \]
	Since $\abs{g(X + \sqrt{t}Z)} \leq \norm{g}_\infty < \infty$, by BCT, this converges as $t \to 0$ to $\expect{g(X)} = \int_{\mathbb R^d} g(x) \dd{\mu_X(x)} (\ast)$.
	So by uniqueness of limits, $\varphi_X$ determines $\mu_X$.

	If $\varphi_X \in L^1$, then $ e^{-i\inner{u,x}} \varphi_X(u) e^{-\frac{\abs{u}^2 t}{2}} \underset{t \to 0}{\longrightarrow} e^{-i\inner{u,x}} \phi_X(u)$.
	By DCT, $f_t(x) \to f_X(x)$ as $t \to 0 \; \forall x$.

	In particular, since $\mu_X \ast g_t \geq 0$, $f_X \geq$ on $\mathbb R^d$.
	Then, for any bdd cts $g$ with compact support,

	\begin{align*}
		\int \underbracket{g(x) f_t(x)}_{\text{As } t \to 0, \to g(x) f_X(x)} \dd{x} &\to \int g(x) f_X(x) \dd{x} \text{ by DCT as } |f_t(x)| \leq \frac{1}{(2\pi)^d} \norm{\phi_X}_1
	\end{align*}
	By $(\ast)$, $\int g(x) \dd{\mu_X(x)} = \int g(x) f_X(x) \dd{x} \quad \forall g$ bdd cts with compact support.

	Thus $\mu_X$ has density $f_X$.
\end{proof}

\begin{theorem}[L\'evy's continuity theorem]
	Let $X_n, X$ be r.v.s on $\mathbb R^d$, s.t. $\varphi_{X_n}(u) \to \varphi_X(u) \; \forall u$, as $n \to \infty$.
	Then $X_n \to X$ weakly.
\end{theorem}

\begin{remark}
	\begin{enumerate}
		\item A stronger version of this theorem is that if $\phi_{X_n}(u) \to \phi(u) \; \forall u$ for some fcn $\phi$ that is cts in a nbd of $0$, then $\phi$ is the c.f. of some r.v. $X$ and $X_n \to X$ weakly.
		\item \underline{Cramer' Wold device}: Let $(X_n), X$ be r.v.s on $\mathbb{R}^d$, then $X_n \to X$ weakly iff $\inner{u, X_n} \to \inner{u, X} \; \forall u \in \mathbb{R}^d$ weakly or in distribution in $\mathbb{R}$. LHS $\implies RHS$ by continuous mapping theorem, the converse holds as $e^{i \inner{u, x}}$ is bdd cts so $\mathbb{E}[f(X_n)] = \phi_{X_n}(u) \to \mathbb{E}[f(x)] = \phi_X(u) \; \forall u$. So done by L\'evy's.
		\item The converse holds by definition of weak convergence, testing against the complex exponentials in the Fourier transform.
	\end{enumerate}
\end{remark}

\begin{proof}
	Let $g : \mathbb{R}^d \to \mathbb{R}$ be compactly suppored and Lipschitz cts, i.e. $|g(x) - g(y)| \leq C_g |x-y| \; \forall x, y \in \mathbb{R}^d$.
	This includes all $g \in C_c^\infty$ as any fcn with bounded derivative is Lipschitz.
	Enough to show, $\mathbb{E}[g(X_n)] \to \mathbb{E}[g(X)]$.

	Let $Z \sim N(0, I_d)$ indep of $(X_n), X$.
	Then for fixed $\epsilon > 0$, choose $t > 0$ small enough s.t. $C_g \sqrt t \expect{\abs{Z}} \leq \frac{\varepsilon}{3}$.
	Then,
	\begin{align*}
		\abs{\mu_{X_n}(g) - \mu_X(g)} &= \abs{\expect{g(X_n)} - \expect{g(X)}} \\
		&\leq \abs{\expect{g(X_n) - g(X_n + \sqrt t Z)}} + \abs{\expect{g(X) - g(X + \sqrt t Z)}} \\
		&+ \abs{\expect{g(X_n + \sqrt t Z) - g(X + \sqrt t Z)}} \\
		&\leq \expect{\abs{g(X_n) - g(X_n + \sqrt t Z)}} + \expect{\abs{g(X) - g(X + \sqrt t Z)}} \\
		&+ \abs{\expect{g(X_n + \sqrt t Z) - g(X + \sqrt t Z)}} \\
		&\leq 2 C_g \sqrt t \expect{\abs{Z}} + \abs{\expect{g(X_n + \sqrt t Z) - g(X + \sqrt t Z)}} \\
		&\leq \frac{2\varepsilon}{3} + \abs{\expect{g(X_n + \sqrt t Z) - g(X + \sqrt t Z)}}
	\end{align*}
	We show that the remaining term can be made less than $\frac{\varepsilon}{3}$ as $n \to \infty$.
	Let $f_{t,n}(x) = g_t \ast \mu_{X_n}$.
	Then, by Fourier inversion for Gaussian convolutions,
	\begin{align*}
		\expect{g(X_n + \sqrt t Z)} &= \int_{\mathbb R^d} g(x) f_{t,n}(x) \dd{x} \\
		&= \frac{1}{(2\pi)^d} \int_{\mathbb R^d} g(x) \int_{\mathbb R^d} e^{-i\inner{u,x}} \varphi_{X_n}(u) e^{-\frac{\abs{u}^2 t}{2}} \dd{u} \dd{x}
	\end{align*}
	Since characteristic functions are bounded by 1, we can use DCT with dominating function $\abs{g(x)} e^{-\frac{\abs{u}^2 t}{2}}$ to find
	\begin{align*}
		\expect{g(X_n + \sqrt t Z)} &\to \frac{1}{(2\pi)^d} \int_{\mathbb R^d} g(x) \int_{\mathbb R^d} e^{-i\inner{u,x}} \varphi_X(u) e^{-\frac{\abs{u}^2 t}{2}} \dd{u} \dd{x} \\
		&= \int_{\mathbb R^d} g(x) f_t(x) \dd{x} \\
		&= \expect{g(X + \sqrt t Z)}
	\end{align*}
	where $f_t = g_t \ast \mu_X$.
	So as $n \to \infty$, the difference between these two terms can be made less than $\frac{\varepsilon}{3}$ as required.
\end{proof}

\begin{note}
	We like adding Gaussians, as pdf of $X + Z$ exists due to $Z$ having a pdf. Also, pdf is a Gaussian convolution, which is nice.
\end{note}

\begin{theorem}[Central Limit Theorem]
	Let $X_1, \dots, X_n$ be iid r.v.s on $\mathbb{R}$ with $\expect{X_i} = 0$ and $\Var{X_i} = 1$.
	Let $S_n = \sum_{i=1}^n X_n$.
	Then
	\[ \frac{1}{\sqrt{n}} S_n \xrightarrow{\text{weakly}}\footnote{As $d = 1$ equiv to in distribution convergence} Z \sim N(0,1) \]
	In particular,
	\[ \prob{\frac{1}{\sqrt{n}} S_n \leq x} \to \prob{Z \leq x} \]
\end{theorem}

\begin{proof}
	Let $X = X_1$.
	The characteristic function $\varphi(u) = \varphi_X(u) = \expect{e^{iuX}}$ satisfies $\varphi(0) = 1$.
	As $\mathbb{E}[X^2] < \infty$ by DUTIS $\varphi'(u) = i \expect{X e^{iuX}}$,    $\varphi''(u) = i^2 \expect{X^2 e^{iuX}}$ (Sheet 3).
	We can find $\varphi'(0) = i\expect{X} = 0$ and $\varphi''(0) = -\expect{X^2} = -\Var X = -1$.
	By Taylor's theorem, $\varphi(v) = 1 - \frac{v^2}{2} + o(v^2)$ as $v \to 0$.
	Now, denoting $\varphi_n(u) = \varphi_{\frac{1}{\sqrt n} S_n}(u)$, we can write
	\begin{align*}
		\varphi_n(u) &= \expect{e^{i\frac{u}{\sqrt n} (X_1 + \dots + X_n)}} \\
		&= \prod_{j=1}^n \expect{e^{i\frac{u}{\sqrt n} X_j}} \\
		&= \qty[\varphi\qty(\frac{u}{\sqrt n})]^n \\
		&= \qty[1 - \frac{u^2}{2n} + o\qty(\frac{u^2}{n})]^n \\
		&= \qty[1 - \frac{u^2}{2n} + o\qty(\frac{1}{n})]^n \text{ fixing $u$ and letting $n \to \infty$}
	\end{align*}
	The complex logarithm satisfies $\log(1 + z) = z + o(z)$, so by taking logarithms, we find
	\[ \log \varphi_n(u) = n \log\qty(1 - \frac{u^2}{2n} + o\qty(\frac{1}{n})) = -\frac{u^2}{2} \]
	Hence, $\varphi_n(u) \to e^{-\frac{\abs{u}^2}{2}} = \varphi_Z(u)$.
	So by L\'evy's continuity theorem, the result follows.
\end{proof}

\begin{remark}
	The CLT in $\mathbb{R}^d$ can be proved similarly using the Cramer-Wold device and properties of multi-variate Guassians.

	% This theorem extends to $\mathbb R^d$ by using the next proposition, using the fact that $X_n \to X$ weakly in $\mathbb R^d$ if and only if $\inner{X_n, v} \to \inner{X, v}$ weakly in $\mathbb R$ for all $v \in \mathbb R^d$.
\end{remark}

\begin{definition}[Gaussian]
	A r.v. on $\mathbb{R}$ is \vocab{Gaussian} ($N(\mu, \sigma)$), if it has density
	\begin{align*}
		\frac{1}{\sqrt{2\pi} \sigma} e^{- \frac{(x - \mu)^2}{2\sigma^2}}
	\end{align*}
	for $\mu \in \mathbb{R}$, $\sigma > 0$.
\end{definition}

\begin{definition}[Gaussian]
	A r.v. $X$ in $\mathbb R^d$ is \vocab{Gaussian} if $\inner{X_n, v}$ are Gaussian for each $v \in \mathbb R^d$.
\end{definition}

\begin{example}
	If $X_1, \dots, X_n \overset{\text{iid}}{\sim} N(0, 1)$, then $X = (X_1, \dots, X_n)$ is Gaussian in $\mathbb{R}^n$.
	Check the c.f. of $\inner{X, v}$
\end{example}

\begin{proposition}
	Let $X$ be Gaussian in $\mathbb R^n$, $A$ is an $m \times n$ matrix and $b \in \mathbb R^m$.
	Then
	\begin{enumerate}
		\item $AX + b$ is Guassian in $\mathbb{R}^m$.
		\item $X \in L^2(\mathbb R^d)$, and $\mu = \expect{X}$ and $V = \Cov{(X_i, X_j)}$ exist and determine $\mu_X$.
		\item $\varphi_X(u) = e^{i\inner{\mu,u} - \frac{\inner{u,Vu}}{2}}$
		\item If $V$ is invertible, then $\mu_X$ has pdf
		\[ f_X(x) = (2\pi)^{-\frac{d}{2}} (\det V)^{-\frac{1}{2}} \exp{-\frac{1}{2} \inner{x-\mu, V^{-1}(x - \mu)}} \]
		\item Subvectors $X_{(1)}, X_{(2)}$\footnote{$X_{(1)}, X_{(2)}$ disjoint $X$} of $X$ are indep iff $\Cov{(X_{(1)}, X_{(2)})} = 0$.
	\end{enumerate}
\end{proposition}

\begin{proof}
	Proofs are easy, and in examples sheets and James Norris' notes.
\end{proof}

% \begin{proposition}
% 	Let $X_n \to X$ weakly in $\mathbb R^d$ as $n \to \infty$.
% 	Then,
% 	\begin{enumerate}
% 		\item if $h \colon \mathbb R^d \to \mathbb R^k$ is continuous, then $h(X_n) \to h(X)$ weakly;
% 		\item if $\abs{X_n - Y_n} \to 0$ in probability, then $Y_n \to X$ weakly;
% 		\item if $Y_n \to c$ in probability where $c$ is constant on $\Omega$, then $(X_n, Y_n) \to (X, c)$ weakly in $\mathbb R^d \times \mathbb R^d$.
% 	\end{enumerate}
% \end{proposition}

% \begin{remark}
% 	Combining parts (iii) and (i), $X_n + Y_n \to X + c$ weakly if $Y_n \to c$ in probability.
% 	If $d = 1$, then in addition $X_n Y_n \to c X$ weakly.
% \end{remark}

% \begin{proof}
% 	\emph{Part (i).}
% 	This follows from the fact that $gh$ is continuous for any test function $g$.

% 	\emph{Part (ii).}
% 	Let $g \colon \mathbb R^d \to \mathbb R$ be bounded and Lipschitz continuous.
% 	Then
% 	\[ \abs{\expect{g(Y_n)} - \expect{g(X)}} \leq \underbrace{\abs{\expect{g(X_n)} - \expect{g(X)}}}_{< \frac{\varepsilon}{3}} + \expect{\abs{g(X_n) - g(Y_n)}} \]
% 	where the bound on $\expect{g(X_n)} - \expect{g(X)}$ holds for sufficiently large $n$.
% 	Then the remaining term is upper bounded by
% 	\[ \expect{\abs{g(X_n) - g(Y_n)}} \qty(1_{\qty{\abs{X_n - Y_n} \leq \frac{\varepsilon}{3\norm{g}_{\mathrm{Lip}}}}} + 1_{\qty{\abs{X_n - Y_n} > \frac{\varepsilon}{3\norm{g}_{\mathrm{Lip}}}}} ) \]
% 	\[ \leq \norm{g}_{\mathrm{Lip}} \frac{\varepsilon}{3\norm{g}_{\mathrm{Lip}}} + 2\norm{g}_\infty \prob{\abs{X_n - Y_n} > \frac{\varepsilon}{3\norm{g}_{\mathrm{Lip}}}} < \frac{2\varepsilon}{3} \]
% 	for sufficiently large $n$.

% 	\emph{Part (iii).}
% 	$\abs{(X_n, c) - (X_n, Y_n)} = \abs{Y_n - c} \to 0$ in probability.
% 	Also, $\expect{g(X_n, c)} \to \expect{g(X, c)}$ for all bounded continuous maps $g \colon \mathbb R^d \times \mathbb R^d \to \mathbb R$, so $(X_n, c) \to (X, c)$ weakly.
% 	Hence, by (ii), $(X_n, Y_n) \to (X, c)$ weakly.
% \end{proof}

    \section{Ergodic theory}

\subsection{Laws of Large Numbers}

\begin{proposition}[Weak Law of Large Numbers]
	Let $(X_n)_{n \in \mathbb N}$ be iid s.t. $\expect{X_n} = \mu$ and $\Var{X_n} = \sigma^2 < \infty$.
	Then $\frac{1}{n} \sum_{i=1}^n X_i \to \mu$ in probability as $n \to \infty$.
\end{proposition}

\begin{proof}
	By Chebyshev's inequality,
	\begin{align*}
        \prob{\abs{\frac{1}{n}\sum_{i=1}^n X_i - \mu} > \varepsilon} \leq \frac{1}{n^2 \varepsilon^2} \Var{\sum_{i=1}^n X_i} = \frac{\sigma^2}{n\varepsilon^2} \xrightarrow{n \to \infty} 0
    \end{align*}
	So $\frac{1}{n} \sum_{i=1}^n X_i \to \mu$ in probability.
\end{proof}

This result has several weaknesses, and we can provide stronger results.

\begin{proposition}[Strong Law of Large Numbers]
	Let $(X_n)_{n \in \mathbb N}$ be iid s.t. $\expect{X_n} = \mu < \infty$.
	Then $\frac{1}{n} \sum_{i=1}^n X_i \to \mu$ a.s. as $n \to \infty$.
\end{proposition}

\begin{proposition}
	Let $(X_n)_{n \in \mathbb N}$ be indep with $\expect{X_n} = \mu$ and $\expect{X_n^4} \leq M \ \forall n$.
	Then $\frac{1}{n} \sum_{i=1}^n X_i \to \mu$ a.s. as $n \to \infty$.
\end{proposition}

\begin{proof}
	Let $Y_n = X_n - \mu$.
	Then $\expect{Y_n} = 0$, and $\expect{Y_n^4} \leq 2^4 \qty(\expect{X_n^4} + \mu^4) < \infty$.
	So we can assume $\mu = 0$.
	For distinct indices $i, j, k, \ell$, by independence and the Cauchy--Schwarz inequality, we have
	\begin{align*}
        0 = \expect{X_i X_j X_k X_\ell} = \expect{X_i^2 X_j X_k} = \expect{X_i^3 X_j};\quad \expect{X_i^2 X_j^2} \leq \sqrt{\expect{X_i^4}}\sqrt{\expect{X_j^4}} \leq M
    \end{align*}
	So we can compute
	\begin{align*}
		\expect{\qty(\sum_{i=1}^n X_i)^4} = \expect{\sum_{i=1}^n X_i^4} + 6\expect{\sum_{i < j} X_i^2 X_j^2} \leq nM + 3n(n-1)M \leq 3n^2 M
    \end{align*}
	Let $S_n = \sum_{i=1}^n X_i$.
	Then,
	\begin{align*}
		\expect{\qty(\frac{S_n}{n})^4} &\leq \frac{1}{n^4} 3n^2 M \\
        \expect{\sum_{n=1}^\infty \qty(\frac{S_n}{n})^4} &= \sum_{i=1}^{\infty} \expect{\qty(\frac{S_n}{n})} \text{ by MCT} \\
		&\leq \sum_{n=1}^\infty \frac{3M}{n^2} < \infty
    \end{align*}
	Hence $\sum_{n=1}^\infty \qty(\frac{S_n}{n})^4 < \infty$ almost surely.
	But then $\qty(\frac{S_n}{n})^4 \to 0$ almost surely, so $\frac{S_n}{n} \to 0$ almost surely.
\end{proof}

\subsection{Invariants - Measure Preserving Transformations}

Let $(E, \mathcal E, \mu)$ be a $\sigma$-finite measure space.

\begin{definition}[Measure Preserving]
	A measurable map $\Theta \colon E \to E$ is \vocab{measure-preserving} (m.p.) if $\mu \circ \Theta\inv = \mu$, i.e. $\mu(\Theta^{-1}(A)) = \mu(A) \; \forall A \in \mathcal E$.
\end{definition}

In this case, for any integrable function $f \in L^1(\mu)$, we have $\int_E f \dd{\mu} = \int_E f \circ \Theta \dd{\mu}$ as $\int_E f \circ \Theta \dd{\mu} = \int_E f \dd{\mu \circ \Theta\inv}$.

\begin{definition}[$\Theta$-Invariant]
	A measurable map $f \colon E \to \mathbb R$ is called \vocab{$\Theta$-invariant} if $f \circ \Theta = f$. \\
	A set $A \in \mathcal E$ is \vocab{$\Theta$-invariant} if $\Theta^{-1}(A) = A$, or equivalently, $1_A$ is $\Theta$-invariant.
\end{definition}

The collection $\mathcal E_\Theta$ of $\Theta$-invariant sets forms a $\sigma$-algebra over $E$.
A function $f \colon E \to \mathbb R$ is invariant iff $f$ is $\mathcal E_\Theta$-measurable (Sheet 4).

\begin{definition}[Ergodic]
	$\Theta$ is called \vocab{ergodic} if the $\mathcal{E}_\Theta$ is $\mu$-trivial, i.e. $\forall A \in \mathcal{E}_\Theta$ $\mu(A) = 0$ or $\mu(A^c) = 0$.
\end{definition}

The point is an ergodic transformation mixes the sets well.
Boltzman (1880), Ergodic hypothesis - over long times, a gas particle in some space will ``fill the whole space'', it will be arbitrarily close to any point in the space.
You could imagine a trajectory, $x, \Theta(x), \Theta^2(x), \dots$


For Markov Chains, ergodicity $\iff$ irreducibility.

\begin{fact}
	If $f$ is $\Theta$-invariant, $\Theta$ is ergodic iff $f$ is constant a.s. on $E$ (Sheet 4).
\end{fact}

\begin{proof}
	See Sheet 4.
	We want to show distribution of $f$ is a step function.
	We know $f \in \mathcal{E}_\Theta$ so $\mu(f\inv(-\infty, x)) = 0$ or $\mu(f\inv[x, \infty)) = 0$.
	Hence $f$ is constant. % with the constant being the sup of the $x$ s.t. $\mu(f\inv(-\infty, x)) = 0$.
\end{proof}

\begin{example}
	Consider $(E, \mathcal E) = ((0,1], \mathcal B)$ with the Lebesgue measure $\mu$.
	The maps $\Theta_a(x) = x + a$ modulo 1 and $\Theta(x) = 2x$ modulo 1 are both m.p., and ergodic unless $a \in \mathbb Q$ (Sheet 4).
\end{example}

\subsection{Ergodic theorems}
\begin{lemma}[Maximal Ergodic Lemma] \label{lem:max}
    Let $(E, \mathcal E, \mu)$ be a $\sigma$-finite measure space.
	Let $\Theta \colon E \to E$ be m.p..
	For $f \in L^1(\mu)$, we define $S_0(f) = 0$ and $S_n(f) = \sum_{k=0}^{n-1} f \circ \Theta^k$.
    Let $S^\star = S^\star(f) = \sup_{n \geq 0} S_n(f)$.
    Then $\int_{\qty{S^\star > 0}} f \dd{\mu} \geq 0$.
\end{lemma}

\begin{proof}[Proof (non-examinable)]
    Define $S_n^\star = \max_{k \leq n} S_k$.
    Then clearly $S_n^\star \uparrow S^\star$, and $S_k \leq S_n^\star$ for all $k \leq n$.
    Note that for $m \leq n+1$, $S_m = S_{m-1} \circ \Theta + f \leq S_n^\star \circ \Theta + f$.

    Define $A_n = \qty{S_n^\star > 0}$.
    On $A_n$, we have
    \begin{align*}
        S_n^\star = \max_{1 \leq k \leq n} S_k \leq \max_{1 \leq k \leq n+1} S_k \leq S_n^\star \circ \Theta + f
    \end{align*}
    since $S_0 = 0$.
    We can integrate this inequality to find
    \begin{align*}
        \int_{A_n} S_n^\star \dd{\mu} \leq \int_{A_n} S_n^\star \circ \Theta \dd{\mu} + \int_{A_n} f \dd{\mu}
    \end{align*}
    On $A_n^c$, we must have $S_n^\star = 0 \leq S_n^\star \circ \Theta$.
    Hence,
    \begin{align*}
        \int_E S_n^\star \dd{\mu} \leq \int_E S_n^\star \circ \Theta \dd{\mu} + \int_{A_n} f \dd{\mu}
    \end{align*}
    Since $\Theta$ is m.p.,
    \begin{align*}
        \int_E S_n^\star \dd{\mu} \leq \int_E S_n^\star \dd{\mu} + \int_{A_n} f \dd{\mu}
    \end{align*}
    so we obtain (as $S_n^\star \in L_1$)
    \begin{align*}
        \int_{A_n} f \dd{\mu} \geq 0 \quad \forall n.
    \end{align*}

	$A_n = \qty{S_n^\star > 0} = \qty{\max_{0 \leq m \leq n} S_m > 0} = \bigcup_{m = 0}^n \qty{S_m > 0} \uparrow \bigcup_{m = 0}^\infty \qty{S_m > 0} = \qty{\sup S_m > 0} = S^\star$. \\
	Hence, $1_{A_n} \to 1_{S^\star > 0}$ and so $f 1_{A_n} \to f 1_{\qty{S^\star > 0}}$ pointwise, and $\abs{f 1_{A_n}} \leq \abs{f} \in L^1(\mu)$, by DCT
    \begin{align*}
        \int_{\qty{S^\star > 0}} f \dd{\mu} = \lim_{n \to \infty} \int_{A_n} f \dd{\mu} \geq 0
    \end{align*}
    as required.
\end{proof}

\begin{remark}
	Let $\mu$ be a finite measure.
	Then for $f \in L^1$ and any $\alpha > 0$, define $\overline S_k = \frac{S_k(f)}{k}$ and $\overline S^\star = \sup_{k \geq 0} \overline S_k$, then
	\begin{align*}
		\mu\qty(\overline S^\star > \alpha) \leq \frac{1}{\alpha} \int_{\overline S^\star > \alpha} f \dd{\mu} \leq \frac{1}{\alpha} \int_E \abs{f} \dd{\mu}.
	\end{align*}
\end{remark}

\begin{proof}
	Proof is left as an exercise, follows from shifting $f$ by $\alpha$ and then applying the maximal ergodic theorem.
\end{proof}

\begin{exercise}
	For $\mu$ a prob measure and $f \in L^1(\mu)$, show that $\qty{\frac{S_n(f)}{n} : n \in \mathbb{N}}$ is UI.
	Hence $\frac{S_n(f)}{n} \to \overline f$ in $L^1$ by Birkhoff's.
	If $\Theta$ is ergodic, then $\overline f = \int f \dd{\mu}$ a.s..
\end{exercise}

\begin{theorem}[Birkhoff's Ergodic Theorem]
	Let $(E, \mathcal E, \mu)$ be a $\sigma$-finite measure space.
	Let $\Theta \colon E \to E$ be m.p..
	For $f \in L^1(E, \mathcal{E}, \mu)$, we define $S_0 = 0$ and $S_n = S_n(f) = \sum_{k=0}^{n-1} f \circ \Theta^k$.
	Then $\exists$ a $\Theta$-invariant integrable fcn $\overline f \in L^1(E, \mathcal{E}, \mu)$ with $\mu\qty(\abs{\overline f}) \leq \mu(\abs{f})$ s.t. $\frac{S_n(f)}{n} \to \overline f$ a.e. as $n \to \infty$.
\end{theorem}

\begin{remark}
	If $\Theta$ ergodic, $\hat{f}$ is constant a.e..

	Relating back to the gas example, $x, \Theta(x), \dots$ is the trajectory of a gas particle.
	Then $\frac{S_n(f)}{n}$ is the average of $f$ along the trajectory (time average).
	Then $\overline f$ is the average of $f$ over the whole space.
\end{remark}

\begin{proof}[Proof (non-examinable)]
	Since $\mu(\abs{f \circ \Theta^{n-1}}) = \mu(\abs{f})$, we have $\mu(\abs{S_n}) \leq n \mu(\abs{f})$ and thus by Fatou's
    \begin{align*}
        \mu\qty(\abs{\overline f}) = \mu\qty(\liminf_n \abs{\frac{S_n}{n}}) \leq \liminf_n \mu\qty(\abs{\frac{S_n}{n}}) \leq \mu(\abs{f}).
    \end{align*}

    Note that
    \begin{align*}
		\frac{S_n \circ \Theta}{n} &= \frac{S_{n+1} - f}{n + 1} \frac{n + 1}{n} \\
        \text{So, } \limsup_n \frac{S_n \circ \Theta}{n} &= \limsup_n \frac{S_{n+1}}{n+1} = \limsup_n \frac{S_n}{n}
    \end{align*}
    and the same holds for $\liminf_n$.
    Hence $\limsup_n \frac{S_n}{n}$ and $\liminf_n \frac{S_n}{n}$ are invariant functions.
    So they are $\mathcal E_\Theta$-measurable.
    Hence
    \begin{align*}
        D = D_{a,b} = \qty{\liminf_n \frac{S_n}{n} < a < b < \limsup_n \frac{S_n}{n}}
    \end{align*}
    are measurable and invariant sets.

	It suffices to show that $\mu(D) = 0$.
	As letting $\Delta = \qty{\liminf \qty(\frac{S_n}{n}) < \limsup \qty(\frac{S_n}{n})} = \bigcup_{a < b \in \mathbb{Q}} D_{a, b}$.
	Hence if $\mu(D) = 0 \implies \mu\qty(\bigcup_{a < b \in \mathbb{Q}} D_{a, b}) = 0 \implies \mu(\Delta) = 0$.
	Define,
	\begin{align*}
		\overline f = \begin{cases}
			\liminf \frac{S_n}{n} = \limsup \frac{S_n}{n} & x \in \Delta^c \\
			0 & x \in \Delta
		\end{cases}
	\end{align*}
	then $S_n/n \to \overline f$ $\mu$-a.e. and $\overline f$ is $\Theta$ invariant (as $\liminf \frac{S_n}{n}$ and $\Delta$ are $\Theta$-invariant).

	Fix $a < b$.
	Note that $\Theta : D \to D$ by invariance and $\Theta$ is $\mu \mid_D$-measure preserving as
	\begin{align*}
        \eval{\mu}_D(A) &= \mu(A \cap D) = \mu(\Theta^{-1}(A \cap D)) = \mu(\Theta^{-1}(A) \cap \Theta^{-1}(D)) \\
		&= \mu(\Theta^{-1}(A) \cap D) = \eval{\mu}_D(\Theta^{-1}(A))
    \end{align*}
	Also either $b > 0$ or $a < 0$ (if $a < 0$ change $f$ to $-f$ and $a$ to $-b$, then $b = -a > 0$).
	So assume $b > 0$ wlog.

	We will apply the \nameref{lem:max} with $E = D$ and $\mu = \eval{\mu}_D$.
    Let $B \in \mathcal E$, where $B \subseteq D$ s.t. $\mu(B) < \infty$.
    Let $g = f - b 1_B \in L^1(\mu)$.
    Then,
    \begin{align*}
        S_n(g) = S_n(f) - bS_n(1_B) \geq S_n - bn
    \end{align*}
    which is positive on $D$ for some $n$ by the definition of $\limsup_n$.
	Hence, $S^\star(g) = \sup_{n \geq 0} S_n(g) > 0$ on $D$.

    % Note that $\qty{S^\star > 0} \subseteq D$ as we restrict our measure space to $D$, but by the previous inequality, $S^\star > 0$ on $D$.
    So $D = \qty{S^\star > 0} \cap D$.
    Then by the \nameref{lem:max} on $D$,
    \begin{align*}
        0 \leq \int_{D \cap \qty{S^\star > 0}} g \dd{\mu} = \int_D g \dd{\mu} = \int_D f \dd{\mu} - b \mu(B)
    \end{align*}
    Hence, $b \mu(B) \leq \int_D f \dd{\mu}$.

    By $\sigma$-finiteness, this inequality extends to $D$; $\exists$ measurable sets $B_n \uparrow D$ where $\mu(B_n) < \infty$.
	Hence taking limits, $b\mu(D) = b \lim_n \mu(B_n) \leq \int_D f \dd{\mu}$.
	Thus $\mu(D) < \infty$ as $f \in L^1$. \\
    Repeating a similar arguments to above\footnote{Now however we take $D$ instead of $B$ and $g' = -f - (-a) 1_D$. On $D$, $S_n(g') = S_n(-f) - (-a) S_(1_D) = S_n(-f) - (-a) n 1_D = S_n(-f) - (-a)n$ as $D$ is $\Theta$ invariant, i.e. $S_n(1_D) = 1_D$.} for $-f$ and $-a$, we obtain $-a\mu(D) \leq \int_D -f \dd{\mu}$.
    Combining these two inequalities gives
    \begin{align*}
        b\mu(D) \leq \int_D f \dd{\mu} \leq a\mu(D)
    \end{align*}
    But $a < b$ and $\mu(D) < \infty$, so $\mu(D) = 0$.
\end{proof}

\begin{theorem}[von Neumann's $L^p$ ergodic theorem]
	Let $\mu(E) < \infty$ and $1 \leq p < \infty$.
	Then for $f \in L^p(\mu)$, $\frac{S_n(f)}{n} \to \overline f$ in $L^p$ as $n \to \infty$.
\end{theorem}

\begin{proof}
	Since $\Theta$ is m.p., we have
	\begin{align*}
        \norm{f \circ \Theta^n}_p^p = \int_E \abs{f}^p \circ \Theta^n \dd{\mu} = \int_E \abs{f}^p \dd{\mu} = \int_E \abs{f}^p \dd{\mu} = \norm{f}_p^p
    \end{align*}
	Thus, by Minkowski's inequality, for all $f \in L^p$ we have
	\begin{align*}
        \norm{\frac{S_n(f)}{n}}_p \leq \frac{1}{n} \sum_{i=0}^{n-1} \norm{f \circ \Theta^i}_p = \norm{f}_p
    \end{align*}
	So $\frac{S_n(f)}{n}$ is a contraction in $L^p$.
	For each $K > 0$, we define $f_K = \max(\min(f, K), -K)$.
	Then
	\begin{align*}
        \norm{f - f_K}_p^p = \int_E \abs{f - f_K}^p 1_{\abs{f} > K} \dd{\mu}
    \end{align*}
	Since $1_{\abs{f} > K} \to 0$ pointwise, and $\abs{f - f_K} \leq 2\abs{f}^p \in L^1$, we find $\norm{f - f_K}_p < \frac{\varepsilon}{3}$ by DCT, for sufficiently large $K = K_\varepsilon$.

	As $\abs{f_K} \leq K$, we have $\abs{\frac{S_n(f_K)}{n}} \leq K$.
	Since $\mu$ is finite, $f_K \in L^1(\mu)$, so by Birkhoff's ergodic theorem, $\frac{S_n(f_K)}{n} \to \overline f_K$ a.e. for some invariant $\overline f_K \in L^1$.
	Note that $\overline f_k$ is bounded by $K$ as $\frac{S_n(f_K)}{n}$ is bounded by $K$.
	By the bounded convergence theorem, we deduce that $\norm{\frac{S_n(f_K)}{n} - \overline f_K} \to 0$ as $n \to \infty$.
	Further, this holds in $L^p$ since
	\begin{align*}
        \norm{\frac{S_n(f_K)}{n} - \overline f_K}_p \leq (2K)^{\frac{p-1}{p}} \norm{\frac{S_n(f_K)}{n} - \overline f_K}_1 < \frac{\varepsilon}{3}
    \end{align*}
	where the last inequality holds for sufficiently large $n$.
	Since $\mu$ is a finite measure, $L^p(\mu) \subseteq L^1(\mu)$, hence by Birkhoff's ergodic theorem, $\frac{S_n(f)}{n} \to \overline f$ a.e. as $f \to \infty$.
	Then, by the contraction property applied to $f - f_K$,
	\begin{align*}
		\norm{\overline f - \overline f_K}_p^p &= \int_E \abs{\overline f - \overline f_K}^p \dd{\mu} \\
		&= \int_E \liminf_n \abs{\frac{S_n(f) - S_n(f_K)}{n}}^p \dd{\mu} \\
		&\leq \liminf_n \int_E \abs{\frac{S_n(f) - S_n(f_K)}{n}}^p \dd{\mu} \quad \text{by Fatou} \\
		&= \liminf_n \int_E \abs{\frac{S_n(f - f_K)}{n}}^p \dd{\mu} \\
		&= \liminf_n \norm{\frac{S_n(f - f_K)}{n}}_p^p \\
		&\leq \liminf_n \norm{f - f_K}_p^p \quad \text{shown earlier by Minkowski}\\
		&= \norm{f - f_K}_p^p < \qty(\frac{\varepsilon}{3})^p
	\end{align*}
	So in particular, $\overline f \in L^p$.
	Then by Minkowski,
	\begin{align*}
			\norm{\frac{S_n(f)}{n} - \overline f}_p &\leq \norm{\frac{S_n(f) - S_n(f_K)}{n}}_p + \norm{\frac{S_n(f_K)}{n} - \overline f_K}_p + \norm{\overline f - \overline f_K}_p \\
			&< \norm{\frac{S_n(f) - S_n(f_K)}{n}}_p + \frac{2\varepsilon}{3} \\
			&\leq \norm{f - f_K}_p + \frac{2\varepsilon}{3} = \varepsilon
	\end{align*}
	for sufficiently large $n$.
\end{proof}

\begin{remark} \label{rem:ergodic} \
	\begin{enumerate}
		\item If $\mu$ a prob measure and $\Theta$ ergodic, then $\bar f$ is a constant a.s., so $\bar f = \int \bar f \dd{\mu}$.
		Also, $\int f \dd{\mu} = \int \frac{S_n(f)}{n} \dd{\mu} \to \int \bar f \dd{\mu} \; \forall n$.
		Hence $\int \bar f \dd{\mu} = \int f \dd{\mu}$. \\
		Then, $\frac{S_n(f)}{n} \xrightarrow{n \to \infty} \mathbb{E}[f]$ $\mu$ a.s. and in $L^1$.
		\item For $\Theta$ m.p. and $f \in L^1$, $\frac{S_n(f)}{n} \xrightarrow{\mu \text{ a.s and } L^1} \mathbb{E}[f \mid \mathcal{E}_\Theta]$.
		For $f \in L^2$, $\mathbb{E}[f \mid \mathcal{E}_\Theta]$ is (a version of) the projection of $f$ on $L^2(\mathcal{E}_\Theta)$.
	\end{enumerate}
\end{remark}

\subsection{Infinite product spaces}
Let $E = \mathbb R^{\mathbb N} = \qty{x = (x_n)_{n \in \mathbb N}}$ be the space of real sequences.
Consider
\begin{align*}
        \mathcal C = \qty{A = \prod_{n=1}^\infty A_n : A_n \in \mathcal B, \exists N \in \mathbb N, \forall n > N, A_n = \mathbb R}
    \end{align*}
This forms a $\pi$-system, which generates the \vocab{cylindrical $\sigma$-algebra} $\sigma(\mathcal C)$.
We can show that $\sigma(\mathcal C) = \sigma(\qty{f_n : n \in \mathbb N})$ where $f_n(x) = x_n$ are the coordinate projection functions on $E$.
We can also show $\sigma(\mathcal C) = \mathcal B(\mathbb R^{\mathbb N})$ for the product topology or the topology of pointwise convergence.

Let $(X_n)_{n \in \mathbb N}$ be a sequence of iid r.v.s defined on $(\Omega, \mathcal F, \mathbb P)$ with common law $\mu_{X_n} = m$ for all $n$; this exists by an earlier theorem.
We define a map $X \colon \Omega \to E$ by $X(\omega) = (X_1(\omega), X_2(\omega), \dots)$.
This is $\mathcal F$--$\sigma(\mathcal C)$ measurable, since for all $A \in \mathcal C$, we have
\begin{align*}
        X^{-1}(A) = \qty{\omega : X_1(\omega) \in A_1, \dots, X_N(\omega) \in A_N} = \bigcap_{n=1}^N X_n^{-1}(A_n) \in \mathcal F
    \end{align*}
We denote $\mu = \mathbb P \circ X^{-1}$, which is the unique product prob measure in $E$ satisfying
\begin{align*}
		\mu\qty(\prod_{n=1}^\infty A_n) &= \lim_{N \to \infty} \mu\qty(\prod_{n=1}^N A_n) \\
		&= \lim_{N \to \infty} \prob{X_1 \in A_1, \dots, X_N \in A_N} \\
		&= \lim_{N \to \infty} \prob{X_1 \in A_1} \cdots \prob{X_N \in A_N} \\
		&= \prod_{n=1}^\infty \prob{X_n \in A_n} \\
		&= \prod_{n=1}^\infty m(A_n)
\end{align*}
Note that we need to use finiteness of $N$ to exploit independence of the $X_i$.
We call $(E, \mathcal E, \mu) = (\mathbb R^{\mathbb N}, \sigma(\mathcal C), m^{\mathbb N})$ the \vocab{canonical model} for an infinite sequence of iid r.v.s of law $m$.

\begin{definition}[Shift Map]
	The shift map $\Theta \colon E \to E$ is defined by $\Theta(x_1, x_2, \dots) = (x_2, x_3, \dots)$.
\end{definition}

\begin{theorem}
	On $(E, \mathcal{E}, \mu)$, the shift map $\Theta$ is measurable, m.p. and ergodic.
\end{theorem}

\begin{proof}
	Measurability is obvious.

	For $A \in \mathcal C$,
	\begin{align*}
			\mu(A) &= \prob{X_1 \in A_1, \dots, X_N \in A_N} \\
			&= \prob{X_1 \in A_1} \cdots \prob{X_N \in A_N} \\
			&= \prod_{n=1}^N m(A_n) \\
			&= \prob{X_2 \in A_1} \cdots \prob{X_{N+1} \in A_N} \\
			&= \mu(\Theta^{-1}(A))
	\end{align*}
	so $\Theta$ is m.p. as m.p. on $\pi$-system $\mathcal{C}$.

	Recall that the tail $\sigma$-algebra is defined by $\mathcal T = \bigcap_n \mathcal T_n$ where $\mathcal T_n = \sigma(X_{n+1}, X_{n+2}, \dots) = \sigma(f_{n+1}, f_{n+2}, \dots)$.
	Note that for all $A \in \mathcal C$, we have
	\begin{align*}
        \Theta^{-n}(A) &= \mathbb{R}^n \times A_1 \times A_2 \times \dots \\
		&= \qty{x \in \mathbb R^{\mathbb N} : (x_{n+1}, x_{n+2}, \dots) \in A} \in \mathcal T_n
    \end{align*}
	Now, if $A$ is invariant, $A = \Theta^{-n}(A) \in \mathcal T_n \; \forall n$, so $A \in \mathcal T$.
	By Kolmogorov's zero-one law as $(X_i)$ iid, $\mu(A) = 0$ or $\mu(A) = 1$ as required for ergodicity.
\end{proof}
We can apply Birkhoff's ergodic theorem to $\Theta$.
If $f \in L^1(\mu)$, then $\frac{S_n(f)}{n} \to \overline f \in L^1(\mu)$ almost surely.
Since $\overline f$ is invariant and $\Theta$ is ergodic, $\overline f$ is almost surely constant.
By von Neumann's $L^p$-ergodic theorem, convergence holds in fact in $L^1$.

\subsection{Strong law of large numbers}

\begin{theorem} \label{thm:canonical}
	Let $m$ be a prob measure on $\mathbb{R}$ s.t. $\int_{\mathbb R} \abs{x} \dd{m(x)} < \infty$ and $\int_{\mathbb R} x \dd{m(x)} = \nu$.
	Let $(E, \mathcal{E}, \mu)$ be the canonical model where the coordinate maps $f_n(x) = x_n$ are iid with law $m$.
	Then
	\begin{align*}
        \mu\qty(\qty{x \in \mathbb R^{\mathbb N} : \frac{x_1 + x_2 + \dots + x_n}{n} \to \nu}) = 1
    \end{align*}
\end{theorem}

\begin{proof}
	Let $\Theta : E \to E$ be the shift map $\Theta(x_1, x_2, \dots) = (x_2, x_3, \dots)$.
	It is m.p. and ergodic by previous thm.
	Consider $f : E \to \mathbb{R}$ as $f(x) = x_1$.
	Then $f \in L^1(\mu)$, since $\int_E \abs{f} \dd{\mu} = \int_{\mathbb{R}} \dd{\mu \circ f\inv(x)} = \int_{\mathbb R} \abs{x_1} \dd{m(x_1)} < \infty$.
	So by Birkhoff and von-Neumann, as $\Theta$ ergodic, by \cref{rem:ergodic}, $\frac{S_n(f)}{n} = \frac{x_1 + \dots + x_n}{n} \to \bar f = \int f \dd{\mu} = \int_{\mathbb R} x_1 \dd{m(x_1)} = \nu$ $\mu$ a.s.
\end{proof}

\begin{theorem}[Kolmogorov SLLN (1930)]
	Let $(X_n)_{n \in \mathbb N}$ be iid r.v.s s.t. $\expect{\abs{X_1}} < \infty$.
	Then $\frac{1}{n} \sum_{i=1}^n X_i \to \expect{X_1}$ almost surely.
\end{theorem}

\begin{proof}
	Let $m$ be the law of $X_n$, $\nu = \mathbb{E}[X_1]$ and $\mu = \mathbb{P} \circ X\inv$ where $X : \Omega \to E = \mathbb{R}^\mathbb{N}$ is $X(\omega) = (X_1(\omega), X_2(\omega), \dots)$.
	Then apply \cref{thm:canonical},
	\begin{align*}
        \prob{\frac{1}{n} \sum_{i=1}^n X_i \to \expect{X_1}} = \mu\qty(\qty{ x : \frac{x_1 + \dots + x_n}{n} \to \nu }) = 1
    \end{align*}
\end{proof}

\begin{remark}
	The hypothesis $\expect{\abs{X}} < \infty$ cannot be weakened; we see on an example sheet that $\frac{1}{n} \sum_{i=1}^n X_i$ can exhibit various behaviours.
	Note that this notion of convergence is stronger than the weak convergence seen in the central limit theorem.
	The law of the iterated logarithm is that
	\begin{align*}
        \limsup_n \frac{X_1 + \dots + X_n}{\sqrt{2 n \log \log n}} = 1
    \end{align*}
	almost surely, and $-1$ for the limit inferior.
	In particular, the central limit theorem does not hold almost surely.
\end{remark}

\begin{corollary}
	By von Neumann's ergodic theorem, in the strong law of large numbers, we have $\expect{\abs{\frac{1}{n} \sum_{i=1}^n X_i - \expect{X}}} \to 0$ as $n \to \infty$.
\end{corollary}

\begin{aside}{Aside}
	\begin{enumerate}
		\item If $(\mu_n)$ is a sequence in $\mathbb{R}^n$ of prob measures that converges weakly to $\mu$, then $(\mu_n)$ is ``tight'', i.e. $\forall \epsilon > 0$, $\exists$ a compact set $K$ s.t. $\mu_n(K^c) < \epsilon \; \forall n$.
		\item If $(\mu_n)$ a sequence of prob measures that are tight, then $\exists$ a subsequence $(n_k)$ and a prob measure $\mu$ s.t. $\qty(\mu_{n_k}) \to \mu$ weakly (Prokhorov's  thm or Banach-Alaoglu thm).
		\item If distributions $F_n \xrightarrow{d} F$, then $\exists$ a prob space s.t. $\mu_{X_n}$ is $F_n$ and $X_n \to X$ a.s..
		\item If $X_n \xrightarrow{d} X$ and $Y_n \xrightarrow{d} Y$ then $X_n + Y_n \not\to X + Y$, in fact $X, Y$ are not necessarily even defined in the same prob space so $X + Y$ doesn't even make sense. \\
		However, if $X_n \xrightarrow{d} X$ and $Y_n \xrightarrow{\mathbb{P}} c$, where $c$ is a constant.
		Then $(X_n, Y_n) \to (X, c)$ so by continuous mapping thm $X_n + Y_n \xrightarrow{d} X + c$ (Slutsky's thm). \\
		This is quite useful in stats where with CLT if variance is unknown we can replace it with std deviation which converges to the true value a.s. and hence in $\mathbb{P}$.
	\end{enumerate}
\end{aside}
\end{document}