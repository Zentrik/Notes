\section{Integration}
\subsection{Notation}
Let $f \colon (E, \mathcal E, \mu) \to \mathbb R$ be measurable and $f \geq 0$\footnote{$f$ is measurable when mapped to $\mathbb{R}$ and $f \geq 0$, this is different from saying $f$ non-negative, measurable.}.

\begin{notation}
	We will then define the integral with respect to $\mu$, either written $\mu(f)$ or $\int_E f \dd{\mu} = \int_E f(x) \dd{\mu(x)}$.

	When $(E, \mathcal{E}, \mu) = (\mathbb{R}, \mathcal{B}, \lambda)$, we write it as $\int f(x) dx$.
\end{notation}
\begin{notation}
	If $X$ is a random variable, we will define its expectation $\expect{X} = \int_\Omega X \dd{\mathbb P} = \int_\Omega X(\omega) \dd{\mathbb P(\omega)}$.
\end{notation}

\subsection{Definition}
\begin{definition}[Simple]
	We say that a function $f \colon (E,\mathcal E,\mu) \to \mathbb R$ is \vocab{simple} if it is of the form
	\begin{align*}
		f = \sum_{k=1}^m a_k 1_{A_k};\quad a_k \geq 0;\quad A_k \in \mathcal E;\quad m \in \mathbb N
	\end{align*}
\end{definition}

\begin{definition}[$\mu$-integral]
	The \vocab{$\mu$-integral} of a simple function $f$ defined as above is
	\[ \mu(f) = \sum_{k=1}^m a_k \mu(A_k)\footnote{Note we take $0 \cdot \infty = 0.$} \]
	which is independent of the choice of representation of the simple function, i.e. well-defined.
\end{definition}

\begin{remark} \
	\begin{itemize}
		\item We have $\mu(\alpha f + \beta g) = \alpha \mu(f) + \beta \mu(g)$ for all nonnegative coefficients $\alpha, \beta$ and simple functions $f, g$.
		\item If $g \leq f$, $\mu(g) \leq \mu(f)$, so $\mu$ is increasing.
		\item $f = 0$ a.e. $\iff \mu(f) = 0$.
	\end{itemize}
\end{remark}

\begin{definition}[$\mu$-integral]
	For a general non-negative function $f \colon (E,\mathcal E,\mu) \to \mathbb R$, we define its \vocab{$\mu$-integral} to be
	\begin{align*}
		\mu(f) = \sup\qty{\mu(g) : g \leq f, g \text{ simple}}
	\end{align*}
	which agrees with the above definition for simple functions.
\end{definition}

Clearly if $0 \leq f_1 \leq f_2$ then $\mu(f_1) \leq \mu(f_2)$.

Now, for $f \colon (E,\mathcal E,\mu) \to \mathbb R$ measurable but not necessarily non-negative, we define $f^+ = \max(f,0)$ and $f^- = \max(-f,0)$, so that $f = f^+ - f^-$ and $\abs{f} = f^+ + f^-$.
\begin{definition}[$\mu$-integrable]
	A measurable function $f \colon (E,\mathcal E,\mu) \to \mathbb R$ is \vocab{$\mu$-integrable} if $\mu(\abs{f}) < \infty$.
	In this case, we define its integral to be
	\[ \mu(f) = \mu(f^+) - \mu(f^-) \]
	which is a well-defined real number.
\end{definition}
Later we shall prove that $\mu(|f|) = \mu(f^+) + \mu(f^-)$ hence $|\mu(f)| \leq \mu(|f|)$.

If one of $\mu(f^+)$ or $\mu(f^-)$ is $\infty$ and the other finite, we defined $\mu(f)$ to be $\infty$ or $-\infty$ respectively (though $f$ i sno t integrable).

\subsection{Monotone Convergence Theorem}
\begin{notation} \
	\begin{itemize}
		\item 	We say $x_n \uparrow x$ to mean $x_n \leq x_{n+1} \; \forall \; n$ and $x_n \to x$.
		\item 	We say $f_n \uparrow f$ to mean $f_n(x) \leq f_{n+1}(x) \; \forall \; n$ and $f_n(x) \to f$.
	\end{itemize}
\end{notation}

\begin{theorem}[Monotone Convergence Theorem] \label{thm:mct}
	Let $f_n, f \colon (E,\mathcal E,\mu) \to \mathbb R$ be measurable and non-negative s.t. $f_n \uparrow f$.
	Then, $\mu(f_n) \uparrow \mu(f)$.
\end{theorem}

\begin{remark}
	This is a theorem that allows us to interchange a pair of limits, $\mu(f) = \mu\qty(\lim_n f_n) = \lim_n \mu(f_n)$, i.e. $\lim_n \int f_n \dd{\mu} = \int \lim_n f_n \dd{\mu}$ for $f_n \geq 0$ and $f_n \uparrow f$. \\
	If $g_n \geq 0$, letting $f_n = \sum_{k=1}^{n} g_k$ and $f_n \uparrow f = \sum_{k=1}^{\infty} g_k$ we get $\lim_n \int \sum_{k=1}^{n} g_k \dd{\mu} = \int \sum_{k=1}^\infty g_k \dd{\mu} \implies \sum_{k=\infty}^{n} \int g_k \dd{\mu} = \int \sum_k g_k \dd{\mu}$ or equivalently $\mu\qty(\sum_k g_k) = \sum_k \mu(g_k)$.
	This generalises the countable additivity of $\mu$ to integrals of non-negative functions.

	If we consider the approximating sequence $\widetilde f_n = 2^{-n} \floor*{2^n f}$, as defined in the monotone class theorem, then this is a non-negative sequence converging to $f$.
	So in particular, $\mu(f)$ is equal to the limit of the integrals of these simple functions.

	It suffices to require convergence of $f_n \to f$ a.e., the general argument does not need to change.
	The non-negativity constraint is not required if the first term in the sequence $f_0$ is integrable, by subtracting $f_0$ from every term.
\end{remark}

\begin{proof}
	Recall that $\mu(f) = \sup\qty{\mu(g) : g \leq f, g \text{ simple}}$.
	Let $M = \sup_n \mu(f_n)$, then $\mu(f_n) \uparrow M$.

	We now show $M = \mu(f)$.

	Since $f_n \leq f$, $\mu(f_n) \leq \mu(f)$, so taking suprema, $M \leq \mu(f)$.

	Now, we need to show $\mu(f) \leq M$, or equivalently, $\mu(g) \leq M$ for all simple $g$ s.t. $g \leq f$, so by taking suprema, $\mu(f) = \sup_g \mu(g) \leq M$. \\
	Now let $g = \sum_{k=1}^m a_k 1_{A_k}$ where $a_k \geq 0$ and wlog the $A_k \in \mathcal E$ are disjoint.
	We define $g_n = \min (\overline f_n, g)$, where $\overline f_n$ is the $n$th approximation of $f_n$ by simple functions as in the \nameref{thm:monclass}.
	So $g_n$ is simple, $g_n \leq \overline{f}_n \leq f_n \uparrow f$, so $g_n \uparrow \min(f, g) = g$.
	I.e. $g \uparrow g$ and $g_n$ simple with $g_n \leq f$.

	Fix $\epsilon \in (0, 1)$, and define sets $A_k(n) = \qty{x \in A_k : g_n(x) \geq (1-\varepsilon) a_k}$.
	Since $g = a_k$ on $A_k$, and since $g_n \uparrow g$, $A_k(n) \uparrow A_k$ for all $k$.
	Since $\mu$ is a measure, $\mu(A_k(n)) \uparrow \mu(A_k)$ by countable additivity.

	Also, we have $g_n 1_{A_k} \geq g_n 1_{A_k(n)} \geq (1-\epsilon)a_k 1_{A_k(n)}$ as $A_k(n) \subseteq A_k$.
	So as $\mu(f)$ is increasing, we have $\mu(g_n 1_{A_k}) \geq \mu\qty((1-\epsilon)a_k 1_{A_k(n)})$ and so $\mu(g_n 1_{A_k}) \geq (1-\epsilon)a_k \mu(1_{A_k(n)})$ as they are simple functions.

	Finally, $g_n = \sum_{k=1}^n g_n 1_{A_k}$ as $g_n \leq g$ and $g$ supported on $\bigcup_{k=1}^n A_k$ and $A_k$ disjoint.
	So
	So as $g_n 1_{A_k}$ is simple,
	\begin{align*}
		\mu(g_n) &= \mu\qty(\sum_{k=1}^{n} g_n 1_{A_k}) \\
		&= \sum_{k=1}^{n} \mu(g_n 1_{A_k}) \\
		&\geq \sum_{k=1}^{n} (1-\epsilon)a_k \mu(A_k(n)) \\
		&\uparrow \sum_{k=1}^{n} (1-\epsilon) a_k \mu(A_k) \\
		&= (1 - \epsilon) \mu(g).
	\end{align*}
	Then,
	\begin{align*}
		(1-\epsilon)\mu(g) \leq \lim_n \mu(g_n) \leq\footnote{As $g_n \leq f_n$} \lim_n \mu(f_n) \leq M
	\end{align*} so $\mu(g) \leq \frac{M}{1 - \epsilon} \; \forall \; \epsilon \in (0, 1)$ hence $\mu(g) \leq M$.
	Since $\epsilon$ was arbitrary, this completes the proof.
\end{proof}

\subsection{Linearity of Integral}
\begin{theorem}[Linearity of Integral]
	Let $f, g \colon (E, \mathcal E, \mu) \to \mathbb R$ be nonnegative measurable functions.
	Then $\forall \; \alpha, \beta \geq 0$,
	\begin{itemize}
		\item $\mu(\alpha f + \beta g) = \alpha \mu(f) + \beta \mu(g)$;
		\item $f \leq g \implies \mu(f) \leq \mu(g)$;
		\item $f = 0$ a.e. $\iff \mu(f) = 0$.
	\end{itemize}
\end{theorem}

\begin{proof}
	If $\widetilde f_n, \widetilde g_n$ are the approximations of $f$ and $g$ by simple functions from the \nameref{thm:monclass} let $f_n = \min(\widetilde f_n, n)$ and $g_n = \min(\widetilde g_n, n)$.
	Then $f_n, g_n$ are simple and $f_n \uparrow f$ and $g_n \uparrow g$.
	Then $\alpha f_n + \beta g_n \uparrow \alpha f + \beta g$, so by MCT\footnote{\nameref{thm:mct}}, $\mu(f_n) \uparrow \mu(f)$, $\mu(g_n) \uparrow \mu(g)$ and $\mu(\alpha f_n + \beta g_n) \uparrow \mu(\alpha f + \beta g)$.
	As $f_n$, $g_n$ simple $\mu(\alpha f_n + \beta g_n) = \alpha \mu(f_n) + \beta \mu(g_n) \uparrow \alpha \mu(f) + \beta \mu(g)$.
	So $\alpha \mu(f) + \beta \mu(g) = \mu(\alpha f + \beta g)$.

	The second part is obvious from definition.

	If $f = 0$ a.e, then $0 \leq f_n \leq f$, so $f_n = 0$ a.e. but $f_n$ simple $\implies \mu(f_n) = 0$.
	As $\mu(f_n) \uparrow \mu(f)$ so $\mu(f) = 0$. \\
	Conversely, if $\mu(f) = 0$, then $0 \leq \mu(f_n) \uparrow \mu(f)$ so $\mu(f_n) = 0 \; \forall \; n \implies f_n = 0$ a.e.
	But $f_n \uparrow f \implies f = 0$ a.e.
\end{proof}

\begin{remark}
	Functions such as $1_{\mathbb Q}$ are integrable and have integral zero.
	They are `identified' with the zero element in the theory of integration.
\end{remark}

\begin{theorem}[Linearity of Integral]
	Let $f, g \colon (E, \mathcal E, \mu) \to \mathbb R$ be integrable.
	Then $\forall \; \alpha, \beta \in \mathbb{R}$,
	\begin{itemize}
		\item $\mu(\alpha f + \beta g) = \alpha \mu(f) + \beta \mu(g)$;
		\item $f \leq g \implies \mu(f) \leq \mu(g)$;
		\item $f = 0$ a.e. $\implies \mu(f) = 0$.
	\end{itemize}
\end{theorem}

\begin{proof}
	Left as an exercise, just use $f = f^+ - f^-$ and use definitions and $\mu(f) = \mu(f^+) - \mu(f^-)$ etc.
\end{proof}

\subsection{Fatou's lemma}
\begin{lemma}
	Let $f_n \colon (E, \mathcal E, \mu) \to \mathbb R$ be nonnegative measurable functions.
	Then $\mu (\liminf_n f_n) \leq \liminf_n \mu(f_n)$.
\end{lemma}
\begin{remark}
	Recall that $\liminf_n x_n = \sup_n \inf_{m \geq n} x_m$ and $\limsup_n x_n = \inf_n \sup_{m \geq n} x_m$.
	In particular, $\limsup_n x_n = \liminf_n x_n$ implies that $\lim_n x_n$ exists and is equal to $\limsup_n x_n$ and $\liminf_n x_n$.
	Hence, if the $f_n$ converge to some measurable function $f$, we must have $\mu(f) \leq \liminf_n \mu(f_n)$.
\end{remark}
\begin{proof}
	We have $\inf_{m \geq n} f_m \leq f_k$ for all $k \geq n$, so by taking integrals, $\mu\qty(\inf_{m \geq n} f_m) \leq \mu(f_k)$.
	Thus,
	\[ \mu\qty(\inf_{m \geq n} f_m) \leq \inf_{k \geq n} \mu(f_k) \leq \sup_n \inf_{k \geq n} \mu(f_k) = \liminf_n \mu(f_n) \]
	Note that $\inf_{m \geq n} f_m$ increases to $\sup_n \inf_{m \geq n} f_m = \liminf_n f_n$.
	By the monotone convergence theorem,
	\[ \mu\qty(\liminf_n f_n) = \lim_n \mu\qty(\inf_{m \geq n} f_m) \leq \liminf_n \mu(f_n) \]
	as required.
\end{proof}

\subsection{Dominated convergence theorem}
\begin{theorem}
	Let $f_n, f \colon (E, \mathcal E, \mu)$ be measurable functions such that $\abs{f_n} \leq g$ almost everywhere on $E$, and the dominating function $g$ is $\mu$-integrable, so $\mu(g) < \infty$.
	Suppose $f_n \to f$ pointwise (or almost everywhere) on $E$.
	Then $f_n$ and $f$ are also integrable, and $\mu(f_n) \to \mu(f)$ as $n \to \infty$.
\end{theorem}
\begin{proof}
	Clearly $\mu(\abs{f_n}) \leq \mu(g) < \infty$, so the $f_n$ are integrable.
	Taking limits in $\abs{f_n} \leq g$, we have $\abs{f} \leq g$, so $f$ is also integrable by the same argument.
	Now, $g \pm f_n$ is a nonnegative function, and converges pointwise to $g \pm f$.
	Since limits are equal to the limit inferior when they exist, by Fatou's lemma, we have
	\[ \mu(g) + \mu(f) = \mu(g + f) = \mu\qty(\liminf_n (g + f_n)) \leq \liminf_n \mu(g + f_n) = \mu(g) + \liminf_n \mu(f_n) \]
	Hence $\mu(f) \leq \liminf_n \mu(f_n)$.
	Likewise, $\mu(g) - \mu(f) \leq \mu(g) - \liminf_n \mu(f_n)$, so $\mu(f) \geq \limsup_n \mu(f_n)$, so
	\[ \limsup_n \mu(f_n) \leq \mu(f) \leq \liminf_n \mu(f_n) \]
	But since $\liminf_n \mu(f_n) \leq \limsup_n \mu(f_n)$, the result follows.
\end{proof}
\begin{example}
	Let $E = [0,1]$ with the Lebesgue measure.
	Let $f_n \to f$ pointwise and the $f_n$ are uniformly bounded, so $\sup_n \norm{f_n}_\infty \leq g$ for some $g \in \mathbb R$.
	Then since $\mu(g) = g < \infty$, the dominated convergence theorem implies that $f_n, f$ are integrable and $\mu(f_n) \to \mu(f)$ as $n \to \infty$.
	In particular, no notion of uniform convergence of the $f_n$ is required.
\end{example}
\begin{remark}
	The proof of the fundamental theorem of calculus requires only the fact that
	\[ \int_x^{x + h} \dd{t} = h \]
	This is a fact which is obviously true of the Riemann integral and also of the Lebesgue integral.
	Therefore, for any continuous function $f \colon [0,1] \to \mathbb R$, we have
	\[ \underbrace{\int_0^x f(t) \dd{t}}_{\text{Riemann integral}} = F(x) = \underbrace{\int_0^x f(t) \dd{\mu(t)}}_{\text{Lebesgue integral}} \]
	So these integrals coincide for continuous functions.
	We can show that all Riemann integrable functions are $\mu^\star$-measurable, where $\mu^\star$ is the outer measure of the Lebesgue measure, as defined in the proof of Carath\'eodory's theorem.
	However, there exist certain Riemann integrable functions that are not Borel measurable.
	We can find that a bounded $\mu^\star$-measurable function is Riemann integrable if and only if
	\[ \mu\qty(\qty{x \in [0,1] \mid f \text{ is discontinuous at } x}) = 0 \]
	The standard techniques of Riemann integration, such as substitution and integration by parts, extend to all bounded measurable functions by the monotone class theorem.
\end{remark}
\begin{theorem}
	Let $U \subseteq \mathbb R$ be an open set and $(E, \mathcal E, \mu)$ be a measure space.
	Let $f \colon U \times E \to \mathbb R$ be a map such that $x \mapsto f(t, x)$ is measurable, and $t \mapsto f(t,x)$ is differentiable where $\abs{\pdv{f}{t}} < g(x)$ for all $t \in U$, and $g$ is $\mu$-integrable.
	Then
	\[ F(t) = \int_E f(t,x) \dd{\mu(x)} \implies F'(t) = \int_E \pdv{f}{t}\qty(t,x) \dd{\mu(x)} \]
\end{theorem}
\begin{proof}
	By the mean value theorem,
	\[ g_h(x) = \frac{f(t + h, x) - f(t, x)}{h} - \pdv{f}{t}\qty(t,x) \implies \abs{g_h(x)} = \abs{\pdv{f}{t}\qty(\widetilde t, x) - \pdv{f}{t}\qty(t, x)} \leq 2g(x) \]
	Note that $g$ is $\mu$-integrable.
	By differentiability of $f$, we have $g_h \to 0$ as $h \to 0$, so applying the dominated convergence theorem, $\mu(g_h) \to \mu(0) = 0$.
	By linearity of the integral,
	\[ \mu(g_h) = \frac{\int_E f(t + h, x) - f(t, x) \dd{\mu(x)}}{h} - \int_E \pdv{f}{t}\qty(t,x) \dd{\mu(x)} \]
	Hence, $\frac{F(t+h) - F(t)}{h} - F'(t) \to 0$.
\end{proof}
\begin{example}
	For a measurable function $f \colon (E, \mathcal E, \mu) \to (G, \mathcal G)$, if $g \colon G \to \mathbb R$ is a nonnegative function, we show on an example sheet that
	\[ \mu \circ f^{-1}(g) = \int_G g \dd{\mu\circ f^{-1}} = \int_E g(f(x)) \dd{\mu(x)} = \mu(g \circ f) \]
	On a probability space $(\Omega, \mathcal F, \mathbb P)$ and a $G$-valued random variable $X$, we then compute
	\[ \expect{g(X)} = \mu_X(g) = \int_\Omega g(X(\omega)) \dd{\mathbb P(\omega)} = \int_\Omega g \dd{\mathbb {P}} \]
\end{example}
\begin{example}[measures with densities]
	If $f \colon (E, \mathcal E, \mu) \to \mathbb R$ is a nonnegative measurable function, we can define $\nu_f(A) = \mu(f 1_A)$ for any measurable set $A$, which is again a measure on $(E, \mathcal E)$ by the monotone convergence theorem.
	In particular, if $g \colon (E, \mathcal E) \to \mathbb R$ is measurable, $\nu_f(g) = \int_E g(x) f(x) \dd{\mu(x)} = \int_E g \dd{\nu(f)}$.
	We call $f$ the \emph{density} of $\nu_f$ with respect to $\mu$.
	If its integral is one, it is called a \emph{probability density function}.
\end{example}
