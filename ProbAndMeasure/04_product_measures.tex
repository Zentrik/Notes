\section{Product Measures}
\subsection{Integration in product spaces}
Let $(E_1, \mathcal E_1, \mu_1), (E_2, \mathcal E_2, \mu_2)$ be finite measure spaces.
On $E = E_1 \times E_2$, we can consider the $\pi$-system of `rectangles' $\mathcal A = \qty{A_1 \times A_2 : A_1 \in \mathcal E_1, A_2 \in \mathcal E_2}$.
Then we define the $\sigma$-algebra $\mathcal{E} = \mathcal E_1 \otimes \mathcal E_2 = \sigma(\mathcal A)$ on the product space.

If the $E_i$ are topological spaces with a countable basis, then $\mathcal B(E_1 \times E_2) = \mathcal B(E_1) \otimes \mathcal B(E_2)$ where we take the product topology.

\begin{lemma}
	Let $f \colon (E, \mathcal E) \to \mathbb R$ be measurable.
	Then $\forall \; x_1 \in E_1$, the fcn $(x_2 \mapsto f(x_1, x_2)) \colon (E_2, \mathcal E_2) \to \mathbb R$ is $\mathcal E_2$-measurable.
\end{lemma}

\begin{proof}
	Let
	\[ \mathcal V = \qty{f \colon (E,\mathcal E) \to \mathbb R : f \text{ bounded, measurable, conclusion of the lemma holds}} \]
	This is a $\mathbb R$-vector space, and $1_E, 1_A \in \mathcal{V} \ \forall \; A = A_1 \times A_2 \in \mathcal A$, since $1_A(x_1, x_2) = 1_{A_1}(x_1) 1_{A_2}(x_2)$ thus fixing $x_1$ gives $0$ or $1_{A_2}$.

	Now, let $0 \leq f_n$ increase to $f$, $f_n \in \mathcal V$.
	Then $(x_2 \mapsto f(x_1, x_2)) = \lim_n (x_2 \mapsto f_n(x_1, x_2))$, so it is $\mathcal E_2$-measurable as it's a limit of a sequence of measurable functions.
	Then by the \nameref{thm:monclass}, $\mathcal V$ contains all bounded measurable functions.
	This extends to all measurable functions by truncating the absolute value of $f$ to $n \in \mathbb N$, then the sequence of such bounded truncations converges pointwise to $f$.
\end{proof}

\begin{lemma} \label{lem:4-2}
	Let $f \colon (E, \mathcal E) \to \mathbb R$ be measurable s.t.
	\begin{enumerate}
		\item $f$ is bounded; or
		\item $f$ is nonnegative.
	\end{enumerate}
	Then the map $x_1 \mapsto \int_{E_2} f(x_1,x_2) \dd{\mu_2(x_2)}$ is $\mu_1$-measurable and is bounded\footnote{As $\mu_2$ is a finite measure.} or nonnegative respectively.
\end{lemma}

\begin{remark}
	In case (ii), the map on $x_1$ may evaluate to infinity, but the set of values
	\[ \qty{x_1 \in E_1 : \int_{E_2} f(x_1,x_2) \dd{\mu_2(x_2)} = \infty} \]
	lies in $\mathcal E_1$.

	Generally, a fcn $f$ taking values in $[0, \infty]$ is measurable means $f\inv(\{\infty\}) \in \mathcal{E}_1$ and $f\inv(A) \in \mathcal{E}_1 \ \forall \; A \in \mathcal{B}$.
\end{remark}

\begin{proof}
	Let
	\[ \mathcal V = \qty{f \colon (E,\mathcal E) \to \mathbb R : f \text{ bounded, measurable, conclusion of the lemma holds}} \]
	This is a vector space by linearity of the integral.
	$1_E \in \mathcal V$, since $\int_{E_2} 1_E(x_1,x_2) \dd{\mu_2(x_2)} = 1_{E_1} \mu_2(E_2)$ is non-negative and bounded.
	$1_A \in \mathcal V \ \ \forall \; A \in \mathcal A$, because $1_{A_1}(x_1) \mu_2(A_2)$ is $\mathcal E_1$-measurable, non-negative, and bounded since it is at most $\mu_2(E_2) < \infty$.

	Now let $f_n$ be a sequence of non-negative functions that increase to $f$, where $f_n \in \mathcal V$.
	Then by the \nameref{thm:mct},
	\[ \int_{E_2} \lim_{n \to \infty} f_n(x_1, x_2) \dd{\mu_2(x_2)} = \lim_{n \to \infty} \int_{E_2} f_n(x_1, x_2) \dd{\mu_2(x_2)} \]
	is an increasing limit of $\mathcal E_1$-measurable functions, so is $\mathcal E_1$-measurable.
	It is bounded by $\mu_2(E_2) \norm{f}_\infty$, or non-negative as required.
	So $f \in \mathcal V$.
	By the \nameref{thm:monclass}, the result for bounded functions holds.

	% In case (ii), we can take a bounded approximation in $\mathcal V$ of an arbitrary measurable function $f$ to conclude the proof.
\end{proof}

\begin{theorem}[Product Measure]
	There $\exists$ a unique measure $\mu = \mu_1 \otimes \mu_2$ on $(E, \mathcal E)$ such that $\mu(A_1 \times A_2) = \mu_1(A_1) \mu_2(A_2)$ for all $A_1 \in \mathcal E_1$, $A_2 \in \mathcal E_2$.
\end{theorem}

\begin{proof}
	$\mathcal A$ is a $\pi$-system generating $\mathcal E$ and $\mu$ a finite measure, so by the \nameref{thm:uni}, $\mu$ unique.

	We define for $A \in \mathcal{E}$,
	\[ \mu(A) = \int_{E_1} \qty( \int_{E_2} 1_A(x_1,x_2) \dd{\mu_2(x_2)} ) \dd{\mu_1(x_1)}. \]
	This is well-defined by the two previous lemmas.

	We have
	\begin{align*}
		\mu(A_1 \times A_2) &= \int_{E_1} \qty( \int_{E_2} 1_{A_1}(x_1) 1_{A_2}(x_2) \dd{\mu_2(x_2)} ) \dd{\mu_1(x_1)} \\
		&= \int_{E_1} 1_{A_1}(x_1) \mu_2(A_2) \dd{\mu_1(x_1)} \\
		&= \mu_1(A_1) \mu_2(A_2)
	\end{align*}
	Clearly $\mu(\varnothing) = 0$, so it suffices to show countable additivity.
	Let $A_n$ be disjoint sets in $\mathcal E$.
	Then
	\[ 1_{\qty(\bigcup_n A_n)} = \sum_n 1_{A_n} = \lim_{n \to \infty} \sum_{i=1}^n 1_{A_n} \]
	Then by the \nameref{thm:mct} and the previous lemmas,
	\begin{align*}
		\mu\qty(\bigcup_n A_n) &= \int_{E_1} \qty( \int_{E_2} \lim_{n \to \infty} \sum_{i=1}^n 1_{A_i} \dd{\mu_2(x_2)} ) \dd{\mu_1(x_1)} \\
		&= \int_{E_1} \qty( \lim_{n \to \infty} \int_{E_2} \sum_{i=1}^n 1_{A_i} \dd{\mu_2(x_2)} ) \dd{\mu_1(x_1)} \\
		&= \lim_{n \to \infty} \int_{E_1} \qty( \int_{E_2} \sum_{i=1}^n 1_{A_i} \dd{\mu_2(x_2)} ) \dd{\mu_1(x_1)} \\
		&= \lim_{n \to \infty} \sum_{i=1}^n \int_{E_1} \qty( \int_{E_2} 1_{A_i} \dd{\mu_2(x_2)} ) \dd{\mu_1(x_1)} \\
		&= \lim_{n \to \infty} \sum_{i=1}^n \mu(A_i) \\
		&= \sum_{n=1}^\infty \mu(A_n)
	\end{align*}
\end{proof}

\begin{remark}
	Note $\mu(A) = \int_{E_2} \qty(\int_{E_1} 1_A(x_1, x_2) \dd{\mu_1(x_1)}) \dd{\mu_2(x_2)}$ by just swapping the order of integration in the previous lemmas and proofs and then by \nameref{lem:dyn}.
\end{remark}

\subsection{Fubini's theorem}
\begin{theorem}[Fubini-Tonelli]
	Let $(E, \mathcal E, \mu) = (E_1 \times E_2, \mathcal E_1 \otimes \mathcal E_2, \mu_1 \otimes \mu_2)$ be a finite measure space.
	\begin{enumerate}
		\item Let $f \colon E \to \mathbb R$ be measurable, non-negative.
		Then
		\begin{align*}
			\mu(f) &= \int_E f \dd{\mu} \\
			&= \int_{E_1} \qty( \int_{E_2} f(x_1,x_2) \dd{\mu_2(x_2)} ) \dd{\mu_1(x_1)} \\
			&= \int_{E_2} \qty( \int_{E_1} f(x_1,x_2) \dd{\mu_1(x_1)} ) \dd{\mu_2(x_2)}
		\end{align*}
		\item Let $f \colon E \to \mathbb R$ be a $\mu$-integrable function (on the product measure).
		Let
		\[ A_1 = \qty{x_1 \in E_1 : \int_{E_2} \abs{f(x_1,x_2)} \dd{\mu_2(x_2)} < \infty}. \]
		Define $f_1 : E_1 \to \mathbb{R}$ by $f_1(x_1) = \int_{E_2} f(x_1,x_2) \dd{\mu_2(x_2)}$ on $A_1$ and 0 elsewhere. \\
		Then $\mu_1(A_1^c) = 0$, $f_1$ is $\mu_1$-integrable and $\mu(f) = \mu_1(f_1) = \mu_1(f_1 1_{A_1})$, and defining $A_2$ symmetrically, $\mu(f) = \mu_2(f_2) = \mu_2(f_2 1_{A_2})$.
	\end{enumerate}
\end{theorem}

\begin{remark}
	If $f$ is bounded, $A_1 = E_1$.
	Note, for $f(x_1,x_2) = \frac{x_1^2-x_2^2}{(x_1^2+x_2^2)^2}$ on $(0,1)^2$, we have $\mu_1(f_1) \neq \mu_2(f_2)$, but $f$ is not Lebesgue integrable on $(0,1)^2$.
\end{remark}

\begin{proof}
	By the definition of the product measure, first statement is true for $f = 1_A$ for $A \in \mathcal{E}$.
	Then, by linearity of the integral, this extends to simple functions.
	For general fcn $f \geq 0$ by \nameref{thm:mct} and the standard approximation by simple fcns $f_n = \min(2^{-n} \floor{2^n f}, n)$, the first statement follows.

	Now let $f$ be $\mu$-integrable.
	Define $h : E_1 \to [0, \infty]$ as $h(x_1) = \int_{E_2} \abs{f(x_1,x_2)} \dd{\mu_2(x_2)}$.
	By \Cref{lem:4-2}, $h$ is measurable (as $\abs{f} \geq 0$), is non-negative, so $A_1 \in \mathcal{E}_1$\footnote{$h$ measurable $\implies h\inv(\{\infty\}) \in \mathcal{E}_1$. $A_1 = h\inv(\{\infty\})^c$ thus in $\mathcal{E}_1$.}. \\
	Then by the first part, $\mu_1(h) \leq \mu(\abs{f}) < \infty$.
	So $f_1$ is $\mu_1$-integrable.
	We have $\mu_1(A_1^c) = 0$, otherwise $\mu_1(h) \geq \mu_1(h 1_{A_1^c}) = \infty$ \Lightning.

	Setting, $f_1^\pm = \int_{E_2} f^\pm(x_1,x_2) \dd{\mu_2(x_2)}$ we see than $f_1 = (f_1^+ - f_1^-) 1_{A_1}$.
	Also by the first part, $\mu_1(f_1^+) = \mu(f^+) < \infty$ and $\mu_1(f_1^-) = \mu(f^-) < \infty$.
	Hence, $\mu(f) =\footnote{As $f$ integrable} \mu(f^+) - \mu(f^-) = \mu_1(f_1^+) - \mu_1(f_1^-) =\footnote{As $f_1$ integrable due to $\mu_1(A_1^c) = 0$.} \mu_1(f_1)$ as required.
\end{proof}

\begin{remark}
	The proofs above extend to $\sigma$-finite measures $\mu$.

	Let $(E_i, \mathcal E_i, \mu_i)$ be measure spaces with $\sigma$-finite measures.
	Note that $(\mathcal E_1 \otimes \mathcal E_2) \otimes \mathcal E_3 = \mathcal E_1 \otimes (\mathcal E_2 \otimes \mathcal E_3)$, by a $\pi$-system argument using Dynkin's lemma.
	So we can iterate the construction of the product measure to obtain a measure $\mu_1 \otimes \dots \otimes \mu_n$\footnote{This is associative.}, which is a unique measure on $\qty(\prod_{i=1}^n E_i, \bigotimes_{i=1}^n \mathcal E_i)$ with the property that the measure of a hypercube $\mu(A_1 \times A_n)$ is the product of the measures of its sides $\mu_i(A_i)$.

	In particular, we have constructed the Lebesgue measure $\mu^n = \bigotimes_{i=1}^n \mu$ on $\mathbb R^n$.
	Applying Fubini's theorem, for functions $f$ that are either non-negative and measurable or $\mu^n$-integrable, we have
	\[ \int_{\mathbb R^n} f \dd{\mu^n} = \idotsint_{\mathbb R \dots \mathbb R} f(x_1, \dots, x_n) \dd{\mu(x_1)} \dots \dd{\mu(x_n)} \]
\end{remark}

\subsection{Product probability spaces and independence}
\begin{proposition}
	Let $X_1, \dots, X_n$ be r.v.s, $X_i : (\Omega, \mathcal F, \mathbb P) \to (E_i, \mathcal{E}_i)$.
	Set $(E, \mathcal E) = \qty(\prod_{i=1}^n E_i, \bigotimes_{i=1}^n \mathcal E_i)$.
	Consider $X \colon (\Omega, \mathcal F, \mathbb{P}) \to (E, \mathcal E)$ given by $X(\omega) = (X_1(\omega), X_2(\omega), \dots, X_n(\omega))$.
	Then $X$ is $\mathcal{E}$-measurable and the following are equivalent.
	\begin{enumerate}
		\item $X_1, \dots, X_n$ are independent random variables;
		\item $\mu_X = \bigotimes_{i=1}^n \mu_{X_i}$;
		\item for all bounded and measurable $f_i \colon E_i \to \mathbb R$, $\expect{\prod_{i=1}^n f_i(X_i)} = \prod_{i=1}^n \expect{f_i(X_i)}$.
	\end{enumerate}
\end{proposition}
\begin{proof}
	To show $X$ measurable suffices to check $X\inv(A_1 \times \dots \times A_n) \in \mathcal{F}$, where $A_i \in \mathcal{E}_i \ \forall \; i$ as this is a $\pi$-system generating $\mathcal{E}$.
	\begin{align*}
		X\inv(A_1 \times \dots \times A_n) &= \qty{\omega : X_1(\omega) \in A_1, \dots, X_n(\omega) \in A_n} \\
		&= \bigcap_{i = 1}^n X_i\inv(A_i).
	\end{align*}
	$X_i$ measurable so $X_i\inv(A_i) \in \mathcal{F}$ and so the intersection is in $\mathcal{F}$.

	(1) $\implies$ (2):
	Consider the $\pi$-system $\mathcal A$ of rectangles $A = \prod_{i=1}^n A_i$ for $A_i \in \mathcal E_i$, as this generates $\mathcal{E}$ suffices to check equality on it.

	Since $\mu_X$ is an image measure, then
	\begin{align*}
		\mu_X(A_1 \times \dots \times A_n) = \prob{X_1 \in A_1, \dots, X_n \in A_n} = \prob{X_1} \dots \prob{A_n} &= \prod_{i=1}^n \mu_{X_i}(A_i) \\
		&= \qty(\bigotimes_{i=1}^n \mu_{X_i})(A).
	\end{align*}

	(2) $\implies$ (3):
	By Fubini's theorem,
	\begin{align*}
		\expect{\prod_{i=1}^n f_i(X_i)} &= \mu_X\qty(\prod_{i=1}^n f_i(x_i)) \\
		&= \int_E f(x) \dd{\mu_X(x)} \\
		&= \idotsint_{E_i} \qty(\prod_{i=1}^n f_i(x_i)) \dd{\mu_{X_1}(x_1)} \dots \dd{\mu_{X_2}(x_2)} \\
		&= \prod_{i=1}^n \int_{E_i} f_i(x_i) \dd{\mu_{X_i}(x_i)} \\
		&= \prod_{i=1}^n \expect{f_i(X_i)}
	\end{align*}

	(3) $\implies$ (1):
	Let $f_i = 1_{A_i}$ for any $A_i \in \mathcal E_i$.
	These are bounded and measurable functions.
	Then
	\[ \prob{X_1 \in A_1, \dots, X_n \in A_n} = \expect{\prod_{i=1}^n 1_{A_i}(X_i)} = \prod_{i=1}^n \expect{1_{A_i}(X_i)} = \prod_{i=1}^n \prob{X_i \in A_i} \]
	So the $\sigma$-algebras generated by the $X_i$ are independent as required.
\end{proof}
