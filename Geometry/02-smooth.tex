\section{Smooth surfaces}

\subsection{Charts and atlases}
Recall that if $\Sigma$ is a topological surface, any point lies in an open neighbourhood homeomorphic to a disc.
\begin{definition}
	A pair $(U, \varphi)$, where $U$ is an open set in $\Sigma$ and $\varphi \colon U \to V$ is a homeomorphism to an open set $V \subseteq \mathbb R^2$, is called a \textit{chart} for $\Sigma$.
	If $p \in U$, we might say that $(U, \varphi)$ is a chart for $\Sigma$ \textit{at $p$}.
	A collection of charts whose domains cover $\Sigma$ is known as an \textit{atlas} for $\Sigma$.
	The inverse $\sigma = \varphi^{-1} \colon V \to U$ is known as a \textit{local parametrisation} for the surface.
\end{definition}
\begin{example}
	If $Z \subseteq \mathbb R^2$ is closed, $\mathbb R^2 \setminus Z$ is a topological surface with an atlas containing one chart, $(\mathbb R^2 \setminus Z, \phi = \id)$.

	For $S^2$, there is an atlas with two charts, which are the two stereographic projections from the poles.
	We could consider alternative charts, for instance the projection to the $yz$ plane, but this would be insufficient for describing the poles.
\end{example}
\begin{definition}
	Let $(U_i, \varphi_i)$ be charts containing the point $p \in \Sigma$, for $i = 1, 2$.
	Then the map
	\begin{align*}
		\ast \colon \varphi_1(U_1 \cap U_2) \to \varphi_2(U_1 \cap U_2);\quad \ast = \varphi_2 \circ \eval{\varphi_1^{-1}}_{\varphi_1(U_1 \cap U_2)}
	\end{align*}
	converts betwen the corresponding charts, and is called a \textit{transition map}.
	This is a homeomorphism of open sets in $\mathbb R^2$.
\end{definition}
Recall from Analysis and Topology that if $V \subseteq \mathbb R^n$ and $V' \subseteq \mathbb R^m$ are open, then a continuous map $f \colon V \to V'$ is called \textit{smooth} if it is infinitely differentiable.
Equivalently, it is smooth if partial derivatives of all orders in all variables exist at all points.
If $n = m$, then in particular the homeomorphism $f \colon V \to V'$ is called a \textit{diffeomorphism} if it is smooth and has smooth inverse.
\begin{definition}
	An \textit{abstract smooth surface} is a topological space $\Sigma$ together with an atlas of charts $(U_i, \varphi_i)$ such that all transition maps $\varphi_i \circ \varphi_j^{-1} \colon \varphi_j(U_i \cap U_j) \to \varphi_i(U_i \cap U_j)$ are diffeomorphisms.
\end{definition}
\begin{remark}
	We could not simply consider a smoothness condition for $\Sigma$ itself without appealing to atlases, since $\Sigma$ is an arbitrary topological space and could have almost any topology.
\end{remark}
\begin{example}
	The atlas of two charts with stereographic projections gives $S^2$ the structure of an abstract smooth surface.
\end{example}
\begin{example}
	For the torus $T^2 = \faktor{\mathbb R^2}{\mathbb Z^2}$, we can find charts of all points by choosing sufficiently small discs in $\mathbb R^2$ such that they do not intersect any of their non-trivial integer translates.
	The transition maps for this atlas are all translations of $\mathbb R^2$.
	Hence $T^2$ inherits the structure of an abstract smooth surface.
	Explicitly, let us define $e \colon \mathbb R^2 \to T^2$ by $(t,s) \mapsto \qty(e^{2\pi i t}, e^{2 \pi i s})$, then consider the atlas
	\begin{align*}
		\qty{(e\qty(D_\varepsilon(x,y)), e^{-1} \text{ on this image})}
	\end{align*}
	for $\varepsilon < \frac{1}{3}$.
	These are charts on $T^2$, and the transition maps are (restricted to appropriate domains) translations in $\mathbb R^2$.
	Hence $T^2$, via this atlas, has the structure of an abstract smooth surface.
\end{example}
\begin{remark}
	The definition of a topological surface is a notion of structure.
	One can observe a topological space and determine whether it is a topological surface.
	Conversely, to be an abstract smooth surface is to have a specific set of data; that is, we must provide charts for the surface in order to see that it is indeed an abstract smooth surface.
\end{remark}
\begin{definition}
	Let $\Sigma$ be an abstract smooth surface, and $f \colon \Sigma \to \mathbb R^n$ be a continuous map.
	We say that $f$ is \textit{smooth} at $p \in \Sigma$ if, for all charts $(U, \varphi)$ of $p$ belonging to the smooth atlas for $\Sigma$, the map
	\begin{align*}
		f \circ \varphi^{-1} \colon \varphi(U) \to \mathbb R^n
	\end{align*}
	is smooth at $\varphi(p) \in \mathbb R^2$.
\end{definition}
\begin{remark}
	Note that the choice of chart and atlas was arbitrary, but smoothness of $f$ at $p$ is independent of the choice of chart, since the transition maps between two such charts are diffeomorphisms.
\end{remark}
\begin{definition}
	Let $\Sigma_1, \Sigma_2$ be abstract smooth surfaces.
	Then a map $f \colon \Sigma_1 \to \Sigma_2$ is \textit{smooth} if it is `smooth in the local charts'.
	Given a chart $(U, \varphi)$ at $p$ and a chart $(U', \psi)$ at $f(p)$, both mapping to open subsets of $\mathbb R^2$, the map $\psi \circ f \circ \varphi^{-1}$ is smooth at $\varphi(p)$.
	Smoothness of $f$ does not depend on the choice of chart, provided that the charts all belong to the same atlas.
\end{definition}
\begin{definition}
	Two surfaces $\Sigma_1, \Sigma_2$ are \textit{diffeomorphic} if there exists a homeomorphism $f \colon \Sigma_1 \to \Sigma_2$ which is smooth and has smooth inverse.
\end{definition}
\begin{remark}
	Often, we convert from a given smooth atlas for an abstract smooth surface $\Sigma$ to the \textit{maximal compatible} smooth atlas.
	That is, we consider the atlas with the maximal possible set of charts, all of which have transition maps that are diffeomorphisms.
	This can be accomplished formally by use of Zorn's lemma.
\end{remark}
