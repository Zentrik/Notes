\section{Geodesics}

\subsection{Definitions}
Recall that we defined, for a smooth curve $\gamma \colon [a,b] \to \mathbb R^3$,
\begin{align*}
	\mathrm{length}(\gamma) = L(\gamma) = \int_a^b \norm{\gamma'(t)} \dd{t}
\end{align*}

\begin{definition}[Energy]
	The \vocab{energy} of $\gamma$ is given by
	\begin{align*}
		E(\gamma) = \int_a^b \norm{\gamma'(t)}^2 \dd{t}
	\end{align*}
\end{definition}

Consider $\Omega_{pq} = \{\text{all smooth curves } \gamma: [a, b] \to \mathbb{R}^3 : \gamma(a) = p, \gamma(b) = q\}$ then $E : \Omega_{pq} \to \mathbb{R}$.
In fact what we really want is given $\Sigma \subset \mathbb{R}^3$, $\gamma : [a, b] \to \Sigma$.
Then we want to find the critical points of $E$ exactly like in variational principles.

\begin{definition}[One-Parameter Variation]
	Let $\gamma \colon [a,b] \to \Sigma$, where $\Sigma$ is a smooth surface in $\mathbb R^3$.
	A \vocab{one-parameter variation} (with fixed endpoints) of $\gamma$ is a smooth map $\Gamma \colon (-\varepsilon, \varepsilon) \times [a,b] \to \Sigma$, such that if $\gamma_s = \Gamma(s,\wildcard)\footnote{I.e. $\gamma_s(t) = \Gamma(s, t)$}$, then
	$\gamma_0(t) = \gamma(t)$, and $\gamma_s(a)$ and $\gamma_s(b)$ are independent of $s$.
\end{definition}

\begin{definition}[Geodesic]
	A smooth curve $\gamma \colon [a,b] \to \Sigma$ is a \vocab{geodesic} if, for every variation $(\gamma_s)$ of $\gamma$ with fixed endpoints as above, we have $\eval{\dv{s}}_{s=0} E(\gamma_s) = 0$.
	In other words, $\gamma$ is a critical point of the energy functional on curves from $\gamma(a)$ to $\gamma(b)$.
\end{definition}

A geodesic is the path that a free particle would follow if the only force acting on it was the one that kept it on the surface.
E.g take a sphere to a place with no gravity, put a marble on it and give it a kick. 
The path it follows will be a geodesic.

\subsection{The geodesic equations}
Let $\gamma$ have image contained within the image of an allowable parametrisation $\sigma \colon V \to U$.
Then, for sufficiently small $s$, we can write $\gamma_s(t) = \sigma(u(s,t), v(s,t))$.
Suppose that the first fundamental form, with respect to $\sigma$, is
\begin{align*}
	E \dd{u}^2 + 2F \dd{u} \dd{v} + G \dd{v}^2
\end{align*}
Let
\begin{align*}
	R &= E(u(s, t), v(s, t)) \dot u^2 + 2F(u(s, t), v(s, t)) \dot u \dot v + G(u(s, t), v(s, t)) \dot v^2 \\
	&= E \dot u^2 + 2F \dot u \dot v + G \dot v^2
\end{align*}
where $\dot{u} = \frac{\partial u}{\partial t}$, $\dot{v} = \frac{\partial v}{\partial t}$.
By definition,
\begin{align*}
	E(\gamma_s) = \int_a^b R \dd{t}
\end{align*}
where $R$ depends on $s$.
Hence,
\begin{align*}
	\pdv{R}{s} & = \qty(E_u \dot u^2 + 2F_u \dot u \dot v + G_u \dot v^2) \pdv{u}{s} + \qty(E_v \dot v^2 + 2F_v \dot u \dot v + G_v \dot v^2) \pdv{v}{s} \\
	           & + 2 (E \dot u + F \dot v) \pdv{\dot u}{s} + 2(F \dot u + G \dot v) \pdv{\dot v}{s}
\end{align*}
This gives
\begin{align*}
	\dv{s} E(\gamma_s) = \int_a^b \pdv{R}{s} \dd{t}.
\end{align*}
Note $\frac{\partial \dot{u}}{\partial s} = \frac{\partial^2 u}{\partial s \partial t}$, $\frac{\partial \dot{v}}{\partial s} = \frac{\partial^2 v}{\partial s \partial t}$ and so we can integrate by parts.
Note that $\pdv{u}{s}$ and $\pdv{v}{s}$ vanish at $a,b$ as the endpoints are fixed. 
Hence,
\begin{align*}
	\eval{\dv{s}}_{s=0} E(\gamma_s) = \int_a^b \qty(A \pdv{u}{s} + B \pdv{v}{s}) \dd{t}
\end{align*}
where
\begin{align*}
	A & = E_u \dot u^2 + 2F_u \dot u \dot v + G_u \dot v^2 - 2 \pdv{t} \qty(E \dot u + F \dot v) \\
	B & = E_v \dot u^2 + 2F_v \dot u \dot v + G_v \dot v^2 - 2 \pdv{t} \qty(F \dot u + G \dot v)
\end{align*}

Note that we have absolute freedom for choosing the ``variational vector field''
\begin{align*}
	w(t) = \qty(\frac{\partial u}{\partial s}(0, t), \frac{\partial v}{\partial s}(0, t)),
\end{align*} which are the $\frac{\partial u}{\partial s}, \frac{\partial v}{\partial s}$ in $\dv{s} E(\gamma_s) = \int_a^b \pdv{R}{s} \dd{t}$.

\begin{corollary}
	A smooth curve $\gamma \colon [a,b] \to \Sigma$ with image in $\Im \sigma$ is a geodesic iff $A = B = 0$, i.e. it satisfies the \textit{geodesic equations}:
	\begin{align*}
		\dv{t} \qty(E \dot u + F \dot v) & = \frac{1}{2} \qty( E_u \dot u^2 + 2F_u \dot u \dot v + G_u \dot v^2 ) \\
		\dv{t} \qty(F \dot u + G \dot v) & = \frac{1}{2} \qty( E_v \dot u^2 + 2F_v \dot u \dot v + G_v \dot v^2 )
	\end{align*}
	Note that these equations are evaluated at $s = 0$, so no choice of variation is required.
\end{corollary}

\begin{remark}
	\begin{enumerate}
		\item If $w(t)$ with $w(a) = w(b) = 0$ then 
		\begin{align*}
			\gamma_s(t) = \sigma((u(t), v(t)) + sw(t))
		\end{align*} 
		for $s$ small enough is a variation of $\gamma$ with fixed endpoints and variational vector field $w$.
		\item Recall Q10, Sheet 4 of IA Analysis:
		\begin{gather*}
			\int_a^b \underbracket{f(x)}_{\text{cont}} g(x) \,dx = 0 \quad \forall \; g : [a, b] \to \mathbb{R} \text{ s.t. } g(a) = g(b). \\
			\implies f \equiv 0.
		\end{gather*} 
		This justifies why $A = B = 0$
		\item  The best way to think about the geodesic equations is via the Euler-Lagrange equations of the Lagrangian, $L(u, v, \dot{u}, \dot{v}) = \frac{1}{2} \qty(E \dot{u}^2 + 2 F \dot{u} \dot{v} + G \dot{v}^2)$ (purely kinetic energy).
		Recall from Variational Principles that the E-L eqns are $\frac{d }{d t} \frac{\partial L}{\partial \dot{q}_i} = \frac{\partial L}{\partial q_i}$ where $q_1 = u, q_2 = v$.
		These are the geodesic equations.
	\end{enumerate} 
\end{remark} 

\begin{remark}
	Solving a differential equation is a local procedure.
	The original definition of the geodesic seems to be a global property.
	However, we can always consider a sub-curve of $\gamma$ to also be a geodesic, since its variations are variations of $\gamma$.
	So the definition can be thought of as local.
\end{remark}

\subsection{Equivalent characterisation of geodesics}
We have so far restricted our analysis to the first fundamental form, without considering its embedding in $\mathbb R^3$.
Intuitively, we know that straight lines in $\mathbb R^2$ are not just locally shortest but also locally straightest.
We would expect this to hold for other surfaces as well.
We can charaterise this notion via stating that the change in the tangent vector to a curve is as small as it could be, subject to the constraint that it lies on the surface.

\begin{proposition} \label{prp:3.1}
	Let $\Sigma$ be a smooth surface in $\mathbb R^3$.
	A smooth curve $\gamma \colon [a,b] \to \Sigma$ is a geodesic iff $\ddot \gamma(t)$ is everywhere normal to the surface $\Sigma$.
\end{proposition}

\begin{remark}
	This proposition makes use of the tangent plane, a notion that exists only because we have an embedding in $\mathbb R^3$.
\end{remark}

\begin{proof}
	The property of being a geodesic as we previously defined is a local property, and so is the condition in the proposition.
	Hence, we may work entirely within an allowable parametrisation $\sigma \colon V \to U$.
	Suppose $\gamma(t) = \sigma(u(t), v(t))$.
	Hence,
	\begin{align*}
		\dot \gamma = \sigma_u \dot u + \sigma_v \dot v
	\end{align*}
	$\ddot \gamma$ is normal to $\Sigma$ when it is orthogonal to the tangent plane, which is spanned by $\sigma_u, \sigma_v$.
	This is true iff
	\begin{align*}
		\inner{ \dv{t} \qty(\sigma_u \dot u + \sigma_v \dot v), \sigma_u } = 0 = \inner{ \dv{t} \qty(\sigma_u \dot u + \sigma_v \dot v), \sigma_v }
	\end{align*}
	We will prove the first equality.
	This can be rewritten
	\begin{align*}
		\dv{t} \inner{\sigma_u \dot u + \sigma_v \dot v, \sigma_u} - \inner{\sigma_u \dot u + \sigma_v \dot v, \dv{t} \sigma_u} = 0
	\end{align*}
	Note that $\inner{\sigma_u, \sigma_u} = E$ and $\inner{\sigma_u, \sigma_v} = F$.
	\begin{align*}
		\dv{t} (E \dot u + F \dot v) - \inner{\sigma_u \dot u + \sigma_v \dot v, \sigma_{uu} \dot u + \sigma_{uv} \dot v} = 0
	\end{align*}
	Hence,
	\begin{align*}
		\dv{t} (E \dot u + F \dot v) - \qty[ \dot u^2 \inner{\sigma_u, \sigma_{uu}} + \dot u \dot v \qty(\inner{\sigma_u, \sigma_{uv}} + \inner{\sigma_v, \sigma_{uu}}) + \dot v^2 \inner{\sigma_v \sigma_{uv}} ] = 0
	\end{align*}
	Note that $E_u = 2 \inner{\sigma_u, \sigma_{uu}}$, $F_u = \inner{\sigma_u, \sigma_{uv}} + \inner{\sigma_v, \sigma_{uu}}$, and $G_u = 2\inner{\sigma_v, \sigma_{uv}}$.
	This gives
	\begin{align*}
		\dv{t} (E \dot u + F \dot v) = \frac{1}{2} \qty(E_u \dot u^2 + 2 F_u \dot u \dot v + G_u \dot v^2)
	\end{align*}
	which is the first of the geodesic equations.
	By symmetry, we find the second geodesic equation similarly.
\end{proof}

\begin{corollary} \label{cor:3.2}
	If $\gamma : [a, b] \to \Sigma$ is a geodesic, then $|\dot{\gamma}(t)|$ is a constant, so geodesics are parametrised proportional to arc length (i.e. has constant speed).
\end{corollary} 

\begin{proof}
	\begin{align*}
		\dv{t} \inner{\dot \gamma, \dot \gamma} = 2 \inner{\underbrace{\dot \gamma}_{\text{tangent to } \Sigma}, \underbrace{\ddot \gamma}_{\text{normal to } \Sigma}} = 0
	\end{align*}
\end{proof} 

\subsubsection{Length vs Energy}

Energy is sensitive to reparametrisation.
If $f, g \colon [a,b] \to \mathbb R$ are smooth, the Cauchy-Schwarz inequality gives that
\begin{align*}
	\qty(\int_a^b fg \dd{t})^2 \leq \int_a^b f^2 \dd{t} \cdot \int_a^b g^2 \dd{t}
\end{align*}
Let us apply this to $f = |\dot{\gamma}|$, $g = 1$ to find
\begin{align*}
	(L(\gamma))^2 = \qty(\int_{a}^{b} |\dot{\gamma}(t)| \cdot 1 \dd{t})^2 &\leq \qty(\int_{a}^{b} |\dot{\gamma}(t)|^2 \dd{t}) \qty(\int_{a}^{b} 1 \dd{t}) = E(\gamma)(b-a).
\end{align*}
Since equality holds only when the two functions are proportional, we must have that $\norm{\gamma'(t)}$ is constant for the equality to hold.
In other words, $\gamma$ must be parametrised proportional to arc length.

\begin{corollary} \label{cor:3.3}
	A smooth curve $\gamma: [a, b] \to \Sigma \subset \mathbb{R}^3$ that has constant speed and locally minimises length is a geodesic.
	% Further, if $\gamma$ globally minimises energy, then it must globally minimise length, and is parametrised with constant speed.
\end{corollary}

\begin{proof}
	Need to prove $\gamma$ is a critical point of $E$.

	Let $\tau : [a, b] \to \Sigma$ be any other curve connecting $\gamma(a)$ to $\gamma(b)$.
	\begin{align*}
		E(\gamma) = \frac{(L(\gamma))^2}{b - a} \leq \frac{(L(\tau))^2}{b - a} \leq E(\tau)
	\end{align*} 
	Thus $\gamma$ is a critical point of $E$ and hence a geodesic.
\end{proof} 

\begin{remark}
	We would like geodesics to be a local property, but not necessarily global length minimisers.
	For example, all arcs of great circles will be shown to be geodesics, even if large arcs are not global length minimisers between fixed endpoints.

	While geodesics might not be global minimisers they are always local minimisers (see Wilson's book for proof).
\end{remark}

\begin{example}[Geodesic on planes]
	The plane $\mathbb R^2$ has parametrisation $\sigma(u,v) = (u,v,0)$ and first fundamental form $\dd{u}^2 + \dd{v}^2$.
	The geodesic equations here are
	\begin{align*}
		\ddot u = 0;\quad \ddot v = 0
	\end{align*}
	In particular, the geodesics on the plane are given by
	\begin{align*}
		u(t) = \alpha t + \beta;\quad v(t) = \gamma t + \delta
	\end{align*}
	This is a straight line, parametrised at constant speed.
\end{example}

\begin{example}[Geodesics on unit sphere]
	Consider the unit sphere with parametrisation
	\begin{align*}
		\sigma(u,v) &= (\sin \theta \cos \phi, \sin \theta \sin \phi, \cos \theta) \\
		\sigma_\phi &= (-\sin \theta \sin \phi, \sin \theta \cos \phi, 0) \\
		\sigma_\theta &= (\cos \theta \cos \phi, -\cos \theta \sin \phi, -\sin \theta)
	\end{align*}
	This has first fundamental form
	\begin{align*}
		E = \sin^2 \theta, F = 0, G = 1
	\end{align*}
	We have Lagrangian
	\begin{align*}
		L(\theta, \phi, \dot{\theta}, \dot{\phi}) = \frac{1}{2} \qty(\dot{\phi}^2 \sin^2 \theta + \dot{\theta}^2)
	\end{align*} 
	Euler-Lagrange
	\begin{align*}
		\frac{\partial L}{\partial \dot{\theta}} &= \dot{\theta}, \frac{\partial L}{\partial \dot{\phi}} = \sin^2 \dot{\phi} \\
		\frac{\partial L}{\partial \phi} &= 0, \frac{\partial L}{\partial \theta} = \dot{\phi}^2\sin \theta \cos \theta \\
		\frac{d }{d t} \qty(\frac{\partial L}{\partial \dot{x}}) &= \frac{\partial L}{\partial x} \\
		\implies \frac{d}{dt} (\dot{\phi} \sin^2 \theta) &= 0, \ddot{\theta} = \dot{\phi}^2 \sin \theta \cos \theta \tag{$\dagger$}
	\end{align*} 
	This gives right away that the equator $t \mapsto (t, \frac{\pi}{2})$ is a geodesic with speed $1$.
	In fact all great circles parametrised with constant speed are geodesics.
	We can prove this by integrating $(\dagger)$, but we can see this by geometrically noticing that such curves have $\ddot{\gamma}$ normal to $T_{\gamma(t)} S^2$.

	Since geodesics solve a 2nd order ODE prescribing a point $p \in \Sigma$ and a direction $v \in T_p \Sigma$ determines the geodesic completely.
	Thus great circles are all possible geodesics, as there exists a great circle for all $p, v$.

	Note that $\gamma$ between $p, q$ as in the picture does \underline{not} minimise length.
\end{example} 

% \begin{example}[Geodesic on a Torus]
% 	Consider the surface of revolution of a circle in the $xz$-plane centred at $(a,0,0)$ about the $z$ axis, giving a torus.
% 	An allowable parametrisation for this surface is
% 	\begin{align*}
% 		\sigma(u,v) = ((a+\cos u)\cos v, (a+\cos u)\sin v, \sin u)
% 	\end{align*}
% 	The first fundamental form is
% 	\begin{align*}
% 		\dd{u}^2 + (a+\cos u)^2 \dd{v}^2 \implies E = 1;\;F = 0;\;G = (a+\cos u)^2
% 	\end{align*}
% 	Note that if we were to take $a = 0$, we would arrive at the unit sphere and its first fundamental form.
% 	We can follow the same procedure as above with the sphere, or formally replace $\cos u$ with $a+\cos u$ in the result.
% 	\begin{align*}
% 		\dv{v}{u} = \frac{C}{(a+\cos u)\sqrt{(a+\cos u)^2 - C^2}}
% 	\end{align*}
% 	which cannot be integrated using classical functions.
% 	This leads to the study of elliptic functions.
% \end{example} 

% \subsection{Planes of symmetry}
% Let $\Sigma$ be a smooth surface in $\mathbb R^3$ such that there exists a plane $\Pi \subset \mathbb R^3$ such that $\Pi \cap \Sigma$ is a smooth embedded curve $C \subset \Sigma$, and $\Sigma$ is setwise preserved by reflection in the plane $\Pi$.
% We will show that $C$ is a geodesic when parametrised at constant speed.
% Consider a point $p$ on $C$.
% We can think of $\mathbb R^3 = \Pi \oplus \Pi^\perp$, where we change coordinates such that $p$ is the origin.
% We can also write $\mathbb R^3 = T_p \Sigma \oplus \mathbb R n_p$, where $\mathbb R n_p$ is the vector subspace of $\mathbb R^3$ generated by $n_p$.
% Clearly, reflection in $\Pi$ acts on $\Pi$ by the identity, and on $\Pi^\perp$ by $-1$.
% Since reflection in $\Pi$ fixes $\Sigma$ setwise and fixes $p$, it must also preserve the subspace $T_p \Sigma$.
% Hence it also preserves $\mathbb R n_p$, so $\mathbb R n_p \subset \Pi$, since $\Pi$ is not the identity on $T_p \Sigma$.
% Now, let us parametrise $C$ locally near $p$ using $t \mapsto \gamma(t) \in C$ at constant speed.
% Since $\gamma(t) \subset \Pi$, we have $\dot \gamma(t), \ddot \gamma(t) \in \Pi$.
% $\gamma$ has constant speed, so $\inner{\dot \gamma, \ddot \gamma} = 0$.
% Hence $\dot \gamma$ lies in $\Pi \cap T_p \Sigma$ and $\ddot \gamma$ is orthogonal to this and lies in $\Pi$, so lies in $\mathbb R n_p \subset \Pi$.
% Hence $\gamma$ is indeed a geodesic.

% In particular, arcs of great circles are geodesics, since they lie in planes of symmetry.

\subsection{Surfaces of revolution}
This is an important example.

Consider the surface of revolution given by $\eta(u) = (f(u), 0, g(u))$ in the $xz$-plane rotated about the $z$ axis, where $\eta$ is smooth and injective, and $f(u) > 0$. \\

\begin{definition}
	A circle obtained by rotating a point of $\eta$ is called a \textit{parallel}.
	A curve optained by rotating $\eta$ itself by a fixed angle about the $z$ axis is called a \textit{meridian}.
\end{definition}

% A plane in $\mathbb R^3$ containing the $z$ axis is a plane of symmetry, hence meridians are geodesics by the previous discussion.
% Not all parallels are geodesics.

\begin{lemma}
	A parallel given by $u = u_0$ is a geodesic when parametrised at constant speed iff $f'(u_0) = 0$.
\end{lemma}

\begin{proof}
	Consider the allowable parametrisation
	\begin{align*}
		\sigma(u,v) = (f(u) \cos v, f(u) \sin v, g(u))
	\end{align*}
	where $u \in (a,b)$ and $v \in (0,2\pi)$.
	The first fundamental form is
	\begin{align*}
		\qty[ (f')^2 + (g')^2 ] \dd{u}^2 + f^2 \dd{v}^2
	\end{align*}
	If wlog we choose to parametrise $\eta$ by arc length, this becomes
	\begin{align*}
		\dd{u}^2 + f^2 \dd{v}^2
	\end{align*}
	The Lagrangian for the geodesic is
	\begin{align*}
		L &= \frac{1}{2} \qty(\dot{u}^2 + f^2 \dot{v}^2) \\
		\frac{\partial L}{\partial u} &= f f' \dot{v}^2, \frac{\partial L}{\partial \dot{u}} = \dot{u}, \frac{\partial L}{\partial v} = 0, \frac{\partial L}{\partial \dot{v}} = f^2 \dot{v} \\
		\text{E-L eqns} \implies \ddot{u} &= f f' \dot{v}^2, \frac{d }{d t} \qty(f^2 \dot{v}) = 0. \tag{$\dagger$}
	\end{align*} 
	We also know that geodesics travel with constant speed so $\dot{u}^2 + f^2 \dot{v}^2$ is a non-zero constant.
	This is an example of a ``completely integrable'' problem, it has the same number of degrees of freedom as conserved quantities, $2$.

	Meridians: Let $v = v_0$, if $u(t) = t + u_0$, the map $t \mapsto (t + u_0, v_0)$ is a geodesic with speed 1 through $(u_0, v_0)$ as it satisfies $(\dagger)$.
	As isometries map geodesics to geodesics (Sheet 3) all meridians are geodesics.

	Parallels: Let $u = u_0$ then as $\dot{u}^2 + f^2 \dot{v}^2 = a$ for some $a \neq 0$, $f^2 \dot{v}^2 = a - u_0^2$. 
	From $(\dagger)$ we need $f f' \dot{v}^2 = 0$ so $f'(u_0) = 0$.
\end{proof}

Let's look at the conserved quantity $f^2 \dot{v}$ in more detail.

\begin{proposition}[Clairaut's relation]
	Consider a curve $\gamma(t)$ on $\Sigma$, making angle $\theta$ with the parallel of radius $\rho = f$.
	If $\gamma$ is a geodesic, then $\rho \cos \theta$ is constant along $\gamma$.
\end{proposition}

\begin{proof}
	Let $\gamma(t) = \sigma(u(t),v(t))$, so $\dot\gamma = \sigma_u \dot u + \sigma_v \dot v$.
	The tangent vector to the parallel is $\sigma_v = (-f \sin v, f \cos v, 0)$.
	By the earlier discussion on angles in terms of the first fundamental form,
	\begin{align*}
		\cos \theta = \frac{\inner{\sigma_v, \sigma_u \dot u + \sigma_v \dot v}}{\norm{\sigma_v} \cdot \norm{\sigma_u \dot u + \sigma_v \dot v}}
	\end{align*}
	Assume $\gamma$ is parametrised by arc length, so $\norm{\dot \gamma} = 1$, so $\norm{\sigma_u \dot u + \sigma_v \dot v} = 1$.
	Using that $F = 0, G = f^2$ we get
	\begin{align*}
		\cos \theta = \frac{f^2 \dot{v}}{f} = f \dot{v}.
	\end{align*}
	So if $\gamma$ a geodesic then $\rho \cos \theta = f^2 \dot{v}$ is a constant.
\end{proof}

This is just another way to write the conservation law arising from $\frac{\partial L}{\partial v} = 0$.

\begin{example}[Ellisoid of revolution]
	Usually, for a surface of revolution, we take the assumption that $\eta$ never intersects the $z$-axis, or that $f$ is positive.
	This ensures that all points on the surface are locally smooth.
	However, we can allow $\eta$ to meet the $z$-axis orthogonally, as in the ellipsoid or sphere.

	Consider an ellipsoid of revolution.
	$\rho \cos \theta$ is constant along a geodesic $\gamma$.
	Suppose that at some point $\gamma$ intersects a parallel of radius $\rho_0$ at angle $\theta_0$, and that $\gamma$ is not a meridian (so $\cos \theta \neq 0$).
	Hence $\theta_0 \in \left[0, \frac{\pi}{2}\right)$.
	In particular, $0 < c = \rho \cos \theta \leq \rho$ for $c = \rho_0 \cos \theta_0$ so $\rho$ is bounded below by $c$.
	A geodesic which is not a meridian is therefore `trapped' between parallels with radius $c$.
	In particular, any geodesic through a pole is a meridian.
\end{example}

\subsection{Local existence of geodesics}
It is difficult to solve the geodesic equations globally.
We can often intead prove local results about any geodesics that may arise.

Recall Picard's theorem from Analysis and Topology.
Let $I = [t_0 - a, t_0 + a] \subset \mathbb R$, $B = \qty{x \colon \norm{x-x_0} \leq b} \subset \mathbb R^n$, and $f \colon I \times B \to \mathbb R^n$ that is continuous, and Lipschitz in the second variable.
\begin{align*}
	\norm{f(t,x_1) - f(t,x_2)} \leq N \norm{x_1-x_2}
\end{align*}
Then the differential equation $\dv{x}{t} = f(t,x)$ with $x(t_0) = x_0$ has a unique solution for some time interval $\abs{t-t_0} < h$, where $h = \min\qty{a,\frac{b}{s}}$ where $s = \sup \norm{f}$.
Further, if $f$ is smooth in all parameters, then the solution to the differential equation is smooth and depends smoothly on the initial condition.

Recall the geodesic equations:
\begin{align*}
	\dv{t} \qty(E \dot u + F \dot v) & = \frac{1}{2} \qty( E_u \dot u^2 + 2F_u \dot u \dot v + G_u \dot v^2 ) \\
	\dv{t} \qty(F \dot u + G \dot v) & = \frac{1}{2} \qty( E_v \dot u^2 + 2F_v \dot u \dot v + G_v \dot v^2 )
\end{align*}
We can write this as
\begin{align*}
	\begin{pmatrix}
		E & F \\
		F & G
	\end{pmatrix} \begin{pmatrix}
		\ddot u \\
		\ddot v
	\end{pmatrix}= R(u,v, \dot{u}, \dot{v})
\end{align*}
where $R$ is composed of smooth functions of $u,v, \dot{u}, \dot{v}$.
The matrix on the left hand side is invertible, and the inverse map $A \mapsto A^{-1}$ on matrices is smooth.
Hence, we can write the geodesic equations in the form
\begin{align*}
	\ddot u = A(u, v, \dot u, \dot v);\quad \ddot v = B(u, v, \dot u, \dot v)
\end{align*}
In the usual way we can turn second-order equations into first-order equations by introducing $p = \dot u, q = \dot v$, and we find
\begin{align*}
	\dot u = p;\quad \dot v = q;\quad \dot p = A(u,v,p,q);\quad \dot q = B(u,v,p,q)
\end{align*}
This is a system of first-order ordinary differential equations as governed by Picard's theorem.
Since $A, B$ are smooth, a local bound on $\norm{DA}$ and $\norm{DB}$ will give the required Lipschitz condition.

\begin{corollary} \label{cor:3.4}
	Let $\Sigma$ be a smooth surface in $\mathbb R^3$.
	For $p \in \Sigma$ and $v \in T_p \Sigma$, then there exists $\varepsilon > 0$ and a unique geodesic $\gamma \colon (-\varepsilon, \varepsilon) \to \Sigma$ such that
	\begin{align*}
		\gamma(0) = p;\quad \dot \gamma(0) = v
	\end{align*}
	Moreover, $\gamma$ depends smoothly on $p,v$.
\end{corollary}

The local existence of geodesics gives rise to allowable parametrisations of $\Sigma$ with `nice' properties in terms of the first fundamental form.
Let $p \in \Sigma$, and consider a geodesic arc $\gamma$ starting at $p$ and parametrised by arc length.
At each point $\gamma(t)$ for small $t$, we can consider a geodesic arc $\gamma_t$ starting at $\gamma(t)$, and $\gamma_t'(0)$ is orthogonal to $\gamma'(t)$, and also parametrised by arc length.
Now, we define $\sigma(u,v) = \gamma_v(u)$, which is defined for $u \in (-\varepsilon,\varepsilon)$ and $v \in (-\delta,\delta)$.

\begin{lemma} \label{lem:3.5}
	For $\varepsilon, \delta$ sufficiently small, $\sigma \colon (u,v) \mapsto \gamma_v(u)$ defines an allowable parametrisation of an open set in $\Sigma$.
\end{lemma}

\begin{proof}
	Smoothness follows from \Cref{cor:3.4}.
	At the origin $(0,0)$, by construction we have $\sigma_u, \sigma_v$ orthogonal and have norm 1.
	Thus $\eval{D \sigma}_0 : \mathbb{R}^2 \to T_p \Sigma$ is a linear isomorphism.
	Now we cann apply inverse function theorem as in Q9, Sheet 1 to deduce that $\sigma$ is a local diffeomorphism at $(0, 0)$ and hence for $\epsilon, \delta$ small enough it is an allowable parametrisation.
\end{proof}

\begin{proposition} \label{prp:3.6}
	Any smooth surface $\Sigma$ in $\mathbb R^3$ admits local parametrisations for which the first fundamental form has form $\dd{u}^2 + G(u,v) \dd{v}^2$, so $E = 1$ and $F = 0$.
\end{proposition}

\begin{proof}
	Consider the parametrisation $\sigma(u,v) = \gamma_v(u)$ as above.
	For $v_0$ fixed, the curve $u \mapsto \gamma_{v_0}(u)$ is a geodesic parametrised by arc-length, so $E = \inner{\sigma_u, \sigma_u} = 1$.
	One of the geodesic equations is
	\begin{align*}
		\dv{t} \qty(F \dot u + G \dot v) = \frac{1}{2} \qty(E_v \dot u^2 + 2F_v \dot u \dot v + G_v \dot v^2)
	\end{align*}
	and consider $v(t) = v_0$, $u(t) = t$.
	$E_v = 0, \dot v = 0$ and $\dot u = 1$, so
	\begin{align*}
		\dv{t} F = 0 \implies F_u \dot u = 0 \implies F_u = 0
	\end{align*}
	So $F$ is independent of $u$.
	At $u = 0$, then by construction of $\gamma_v$ as being orthogonal to $\gamma$ at $\gamma(v)$, we see $F = 0$.
\end{proof}

\begin{remark}
	\begin{enumerate}
		\item These coordinates are called \vocab{Fermi coordinates} (and sometimes geodesic normal coordinates).
		\item Note that by fixing $u$ and letting $v$ vary, the curve obtained is typically not a geodesic, except for $u = 0$ which is $\gamma$ itself.
		\item In these coordinates, we can also find
		\begin{align*}
			G(0,v) = 1;\quad G_u(0,v) = 0
		\end{align*}
		The first result holds since $\sigma_v$ has unit length at $u = 0$.
		The second result holds because $u = 0$ yields a geodesic with arc length parametrisation, and then we can use one of the geodesic equations to find
		\begin{align*}
			\dv{t} \qty(E \dot u + F \dot v) = \frac{1}{2}\qty(E_u \dot u^2 + 2F_u \dot u \dot v + G_u \dot v^2) \implies 0 = \frac{1}{2} G_u(0,v)
		\end{align*}
		\item Once can show that if $E = 1$ and $F = 0$, then the Gauss curvature is given by
		\begin{align*}
			\kappa = \frac{-\qty(\sqrt{G})_{uu}}{\sqrt{G}}
		\end{align*}
		This proves \Cref{thm:egregium}!
		Proving this is not too hard, but beyond the scope of this course.
	\end{enumerate} 
\end{remark} 

\subsection{Surfaces of constant curvature}
If $\Sigma \subset \mathbb{R}^3$ and $f \colon \mathbb R^3 \to \mathbb R^3$ is a dilation $f(x,y,z) = (\lambda x,\lambda y,\lambda z), \lambda \neq 0$, then
\begin{align*}
	\kappa_{f(\Sigma)} = \frac{1}{\lambda^2} \kappa_\Sigma
\end{align*}
since $E, F, G$ rescale by $\lambda^2$, and $L, N, M$ rescale by $\lambda$.
% This matches the results previously found for spheres of varying radii.

\begin{question}
	What do constant curvature surfaces look like?
\end{question} 

\begin{answer}
	By dilating, to understand surfaces of constant curvature it suffices to consider surfaces with constant curvature $\pm 1$ or 0.
\end{answer} 

\begin{proposition} \label{prp:3.7}
	Let $\Sigma$ be a smooth surface in $\mathbb R^3$.
	Then,
	\begin{enumerate}
		\item if $\kappa \equiv 0$, then $\Sigma$ is locally isometric to $(\mathbb R^2, \dd{u^2} + \dd{v}^2)$;
		\item if $\kappa \equiv 1$, then $\Sigma$ is locally isometric to $(S^2, \dd{u}^2 + \cos^2 u \dd{v}^2)$.
	\end{enumerate}
\end{proposition}

\begin{proof}
	$\Sigma$ admits an allowable parametrisation with $E = 1$, $F = 0$, $G(0,v) = 1$ and $G_u(0,v) = 0$ by using Fermi coordinates.
	Also
	\begin{align*}
		\kappa = \frac{-(\sqrt{G})_{uu}}{\sqrt{G}}
	\end{align*}
	If $\kappa \equiv 0$, we have $(\sqrt{G})_{uu} = 0$, so $\sqrt{G} = A(v) u + B(v)$, and the boundary conditions give $A \equiv 0, B \equiv 1$.
	In particular, $G \equiv 1$.
	The fundamental form then is $\dd{u}^2 + \dd{v}^2$, which is that of $\mathbb R^2$.

	If $\kappa \equiv 1$, we find $\qty(\sqrt{G})_{uu} + \sqrt{G} = 0$ so $\sqrt{G} = A(v) \sin u + B(v) \cos u$.
	The boundary conditions then imply that $A \equiv 0, B \equiv 1$ and hence the fundamental form is $\dd{u}^2 + \cos^2 u \dd{v}^2$.
	This matches the first fundamental form of a sphere with parametrisation
	\begin{align*}
		\sigma(u,v) = (\cos u \cos v, \cos u \sin v, \sin u)
	\end{align*}
\end{proof}
\begin{remark}
	If $\kappa \equiv -1$, we will find the first fundamental form $\dd{u}^2 + \cosh^2 u \dd{v}^2$.
	There exists an object known as the tractoid, which is a smooth surface in $\mathbb R^3$, and has this first fundamental form (Q5, Sheet 2).
	We could alternatively choose not to embed this surface in $\mathbb R^3$.

	In fact, the change of variables $v = e^v \tanh u, w = e^v \sech u$ turns the fundamental form $\dd{u}^2 + \cosh^2 u \dd{v}^2$ into $\frac{\dd{V}^2 + \dd{W}^2}{W^2}$, which is `the standard presentation' of the hyperbolic plane, which we will see more of later.
\end{remark}
